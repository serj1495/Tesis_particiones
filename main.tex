\documentclass[12pt]{book}
\usepackage[utf8]{inputenc}
\usepackage[spanish, mexico]{babel}
\usepackage[T1]{fontenc}
\usepackage[table]{xcolor}
\usepackage{graphicx}
\usepackage[export]{adjustbox}
\usepackage{amsmath}
\usepackage{natbib}
\usepackage{amsthm}
\usepackage{amssymb}
\usepackage{amsfonts}
\usepackage{subcaption}
\usepackage{array}
\usepackage{listings}
\lstset{basicstyle=\ttfamily}
\usepackage{setspace} %setstrech
\usepackage{breqn} %equation break
%\usepackage{algorithmic}
\usepackage[linesnumbered,ruled,spanish,onelanguage,vlined]{algorithm2e}

\usepackage{float}
\usepackage{tikz}
\usetikzlibrary{babel}
\usetikzlibrary{arrows,shapes,matrix,
                decorations.pathmorphing,
                shapes.geometric,calc}

% Tikz style
\tikzstyle{edge}=[>=stealth, line width=0.8pt]
\tikzstyle{vertex}=[circle, fill=white, draw,
                    minimum size=5pt,
                    inner sep=0pt, outer sep=0pt]
\tikzstyle{cotreenode}=[circle, fill=white, draw,
                    minimum size=5pt,
                    inner sep=2pt, outer sep=0pt]

\tikzstyle{cotreedescription}=[ellipse, fill=white, draw,
                    minimum size=5pt,
                    inner sep=2pt, outer sep=0pt]

\tikzstyle{minivertex}=[circle, fill=white, draw,
                    minimum size=3pt,
                    inner sep=0pt, outer sep=0pt]

\usepackage{subcaption}
\usepackage[shortlabels]{enumitem}

\usepackage{tabularx,ragged2e}
\newcolumntype{C}{>{\Centering\arraybackslash}X} % centered "X" column
\PassOptionsToPackage{hyphens}{url}
\usepackage{url}
%\usepackage{hyperref}
%\renewcommand{\UrlFont}{\ttfamily\small}
\usetikzlibrary{arrows.meta,
                calc, chains,
                positioning,
                shapes.multipart}

\usepackage{algpseudocode}
\usepackage{makecell}
\usepackage{multirow}
\usepackage{booktabs}
\usepackage{hyphenat}

%\hyphenation{geo-mé-tri-cas}
\graphicspath{ {images/} }

% User-defined environments
\newtheorem{theorem}{Teorema}[section]
\newtheorem{corollary}{Corolario}[theorem]
\newtheorem{lemma}[theorem]{Lema}

\theoremstyle{definition}
\newtheorem{definition}{Definición}
\algnewcommand\algorithmicforeach{\textbf{for each}}
\algdef{S}[FOR]{ForEach}[1]{\algorithmicforeach\ #1\ \algorithmicdo}

\newenvironment{dedication}
  {
   \thispagestyle{empty}% no header and footer
   \vspace*{\stretch{1}}% some space at the top
   \itshape             % the text is in italics
   \raggedleft          % flush to the right margin
  }
  {\par % end the paragraph
   \vspace{\stretch{3}} % space at bottom is three times that at the top
   \clearpage           % finish off the page
  }


%% Es para poner los márgenes en una impresión
%% a ambas caras de la hoja



\begin{document}

%\frontmatter
\thispagestyle{empty}

\begin{minipage}{0.15\textwidth}
	\includegraphics[width=0.85\textwidth]{imagen/cinvestavlogo.pdf}
\end{minipage}%
\begin{minipage}{0.90\textwidth}
\begin{center}
	 \sc Centro de Investigación y de Estudios Avanzados\newline
	           del Instituto Politécnico Nacional\newline
\end{center}
\end{minipage}

\centerline{\sc Unidad Zacatenco}
\centerline{\sc Departamento de computación}

\vspace*{\stretch{1}}
\begin{center}
\Large \bf
Particiones de cográficas en gráficas multipartitas completas
\end{center}
\vspace*{\stretch{1}}

\centerline{ \ TESIS QUE PRESENTA}
\vspace{0.3cm}
\centerline{\large \bf Sergio Eduardo Juárez Martínez}
\vspace{1cm}
\centerline{ \ PARA OBTENER EL GRADO DE}
\vspace{0.3cm}
\centerline{\large \bf Maestro en Ciencias en Computación}
\vspace{1.5cm}
\centerline{  DIRECTORES DE LA TESIS}
\vspace{0.3cm}
\centerline{\large \bf Dr. César Hernández Cruz}
\centerline{\large \bf Dra. Dolores Lara Cuevas}
 
\vspace{2cm}
{\large \bf México, Ciudad de México \hfill Octubre 2020}



%\setcounter{chapter}{0}
%\setcounter{page}{0}
%\setcounter{secnumdepth}{4}
%\pagenumbering{arabic}

\chapter*{Abstract}
Graph Theory is the branch of Discrete Mathematics that studies the mathematical structures known as graphs. A graph $G$ is an ordered pair of disjoint sets $(V,E)$ such that $E$ is a set of unordered pairs of elements of $V$. We call $V$ the set of vertices of $G$ and $E$ the set of edges of $G$. If an element of $E$ contains two vertices, we say that those vertices are adjacent. In this thesis, we work with a particular kind of graphs known as cographs, that can be characterized in several ways. One of this ways is that cographs are the graphs that does not contain the path of 4 vertices, $P_4$, as an induced subgraph.

A clasic problem in Graph Theory is graph coloring, which consits in deciding if it is possible to tag the vertices of a graph with a given number of different tags, also known as colors, in a way such that if two vertices are adjacent, then those two vertices have different tags. A generalization of the graph coloring problem are matrix partitions, that consist in deciding if the vertices of a graph can be tagged with a given number of tags in a way such that vertices with the same tag satisfy an homogeneity property. In this thesis we address a problem of matrix partitions. We study the classes of cographs defined in the following way. Given an integer $i$ greater than or equal to one, the class $M_i$ is the class of the cographs whose set of vertices accepts a partition in $i$ parts such that each parte induces a multipartite complete graph. The class $M_1$, which is the class of the multipartite complete cographs, has been widely studied. However, the classes $M_i$ for values of $i$ greater than one have not been previously studied. The base of our research is the study of the class $M_2$. For this study, we use the research done about polar cographs as guide. We characterize the class $M_2$ through its set of minimal obstructions, we present an algorithm to recognize its elements and a certifier algorithm. We also study a set of subclasses of the class $M_2$ that we call the classes $(\alpha,\beta)$-$M_2$. Similarly, we characterize the class $M_3$ through its set of minimal obstructions and present two families of minimal obstructions for any class $M_i$.
\chapter*{Resumen}
La Teoría de Gráficas es la rama de las Matemáticas Discretas encargada del estudio de los objetos matemáticos conocidos como gráficas. Una gráfica $G$ es una pareja ordenada de conjuntos ajenos $(V,E)$ tal que $E$ es un conjunto de parejas no ordenadas de elementos de $V$. Llamamos a $V$ el conjunto de vértices de $G$ y a $E$ el conjunto de aristas de $G$. Si dos elementos de $V$ forman una pareja que está en el conjunto $E$, decimos que estos son adyacentes. Con frecuencia, las gráficas son representadas con un dibujo en el que los vértices aparecen como puntos o pequeños círculos y las aristas como líneas que unen a los vértices adyacentes. En esta tesis trabajamos con un tipo particular de gráficas conocidas como cográficas, que pueden ser caracterizadas de múltiples maneras. Una de ellas es que son las gráficas que no tienen a la trayectoria de $4$ v\'ertices, $P_4$, como subgráfica inducida.

Un problema clásico en la Teoría de Gráficas es la coloración de gráficas que consiste en determinar si los vértices de una gráfica se pueden etiquetar con un número determinado de etiquetas diferentes, también llamadas colores, de forma tal que, si dos vértices son adyacentes, estos tienen etiquetas diferentes. Una generalización de las coloraciones de gráficas son las particiones matriciales, que consisten en determinar si los vértices de una gráfica se pueden etiquetar con un número determinado de etiquetas diferentes de manera tal que los vértices con la misma etiqueta cumplan con una propiedad de homogeneidad. En esta tesis abordamos un problema de particiones matriciales. Estudiamos a las clases de cográficas que se definen de la siguiente manera. Dado $i$, un entero mayor o igual a uno, la clase $M_i$ es la clase de las cográficas cuyo conjunto de vértices acepta una partición en $i$ partes tal que cada parte induce una gráfica multipartita completa. La clase $M_1$, que es la clase de las cográficas multipartitas completas, ha sido ampliamente estudiada. Sin embargo, las clases $M_i$ para valores de $i$ mayores a uno no han sido estudiadas con anterioridad. Nuestra investigación tiene como base el estudio de la clase $M_2$, para el cuál tomamos como guía la investigación realizada sobre las cográficas polares. Caracterizamos a la clase $M_2$ a través de su conjunto de obstrucciones mínimas, presentamos un algoritmo para reconocer a sus elementos y un algoritmo certificador y estudiamos a un conjunto de subclases de esta clase a las que llamamos clases $(\alpha,\beta)$-$M_2$. De igual manera, caracterizamos a la clase $M_3$ a través de su conjunto de obstrucciones mínimas y proporcionamos dos familias de obstrucciones mínimas para cualquier clase $M_i$.


\tableofcontents

\listoffigures

\listoftables

%\mainmatter
\chapter{Introducción}
Una gráfica $G$ es una pareja ordenada de conjuntos ajenos $(V,E)$ tal que $E$ es un conjunto de parejas no ordenadas de elementos de $V$. Llamamos a $V$ el conjunto de vértices de $G$ y a $E$ el conjunto de aristas de $G$. Decimos que dos vértices de $G$ son adyacentes si la pareja formada por estos está en $E$. Sea $V'$ un subconjunto de $V$, la subgráfica de $G$ inducida por $V'$ es la gráfica que tiene como conjunto de vértices a $V'$ y como conjunto de aristas al subconjunto de $E$ de las parejas no ordenadas de elementos de $V'$. Una gráfica multipartita completa es una gráfica cuyo conjunto de vértices acepta una partición $(A_1, A_2,\dots, A_n)$ tal que, para cualesquiera enteros diferentes $1\le i,j \le n$, los vértices en $A_i$ no son adyacentes entre sí pero sí son adyacentes a cada uno de los vértices en $A_j$. Un clan es un conjunto de vértices de $G$ tal que todos sus elementos son adyacentes entre sí.

Una clase hereditaria de gráficas $C$ es un conjunto de gráficas tal que, si $C$ contiene a $G=(V,E)$, entonces, para cualquier subconjunto $V'$ de $V$, $C$ contiene a la subgráfica de $G$ inducida por $V'$. Toda clase hereditaria de gráficas $C$ puede ser caracterizada a través de un conjunto de gráficas $S$ tal que toda gráfica en $C$ no tiene a ninguna gráfica de $S$ como subgráfica inducida. Llamamos a $S$ el conjunto de obstrucciones mínimas de $C$. Las cográficas son la clase hereditaria de gráficas definida recursivamente de la siguiente manera:

\begin{itemize}
    \item Una gráfica con un solo vértice es una cográfica.
    \item Si $G=(V_G,E_G)$ y $H=(V_H,E_H)$ son cográficas sin vértices en común, entonces $(V_G\cup V_H, E_G \cup E_H)$ es una cográfica.
    \item Si $G=(V,E)$ es una cográfica y $W$ el conjunto de todas las parejas no ordenadas de elementos de $V$, entonces $(V_G, W-E)$ es una cográfica.
\end{itemize}

En 1990, Damascke, P. \cite{Damaschke} mostró que cualquier clase hereditaria de cográficas puede ser caracterizada por un conjunto finito de obstrucciones mínimas.

En esta tesis estudiamos a las clases de cográficas que se definen de la siguiente manera. Sea $i$ un entero mayor o igual a uno, la clase $M_i$ es la clase de las cográficas cuyo conjunto de vértices acepta una partición en $i$ partes tal que cada parte induce una gráfica multipartita completa. Nos referimos a estas clases en conjunto como las clases $M_i$. Este problema no ha sido estudiado con anterioridad. Sin embargo, utilizamos la investigación realizada acerca de las cográficas polares como guía para el estudio de la clase $M_2$, que sirve como base de nuestra investigación.

Los resultados principales de nuestra tesis son los siguientes. Presentamos un algoritmo que, dada una clase de cográficas fija, representada a través de su conjunto de obstrucciones mínimas, es capaz de determinar si una cográfica pertenece a dicha clase en tiempo lineal con respecto al orden de la gráfica. Caracterizamos a la clase $M_2$ a través de su conjunto de obstrucciones mínimas. Presentamos un algoritmo de tiempo lineal para el reconocimiento de los elementos de la clase $M_2$. Presentamos un algoritmo certificador de tiempo lineal para la clase $M_2$. Tomando como base el estudio de las cográficas $(s,k)$-polares, estudiamos un conjunto de subclases de la clase $M_2$ a las que llamamos clases $(\alpha,\beta)$-$M_2$. Como resultado principal de este estudio, presentamos un algoritmo que, dados tres enteros $\alpha$, $\beta$ y $n$ mayores o iguales a uno, genera las obstrucciones mínimas de la clase $(\alpha,\beta)$-$M_2$ con hasta $n$ vértices. Presentamos el conjunto de obstrucciones mínimas de la clase $M_3$. Finalmente, presentamos dos familias de obstrucciones mínimas para cualquier clase $M_i$.

La tesis está organizada de la siguiente manera. En el Capítulo 2 se presentan de manera detallada y formal las definiciones necesarias para entender el problema que resolvemos con nuestro trabajo de tesis. Estas definiciones proporcionan una introducción a la Teoría de Gráficas y, dentro de esta área, a las cográficas, que son el tipo de gráficas con las que trabajamos en todo el documento. En el Capítulo 3 presentamos los resultados en la investigación de las cográficas polares que dan forma a nuestro estudio de la clase $M_2$. En el Capítulo 4, presentamos los resultados de nuestra investigación que comprenden un conjunto de lemas, teoremas, algoritmos y listas de obstrucciones mínimas para varias clases hereditarias de cográficas. La correctitud de cada uno de nuestros resultados se demuestra de manera formal. Finalmente, en el Capítulo 5, presentamos las conclusiones de nuestra investigación a la vez que proponemos varias formas de continuar con la investigación que iniciamos con este trabajo de tesis.

\chapter{Antecedentes}
En este capítulo proporcionamos las definiciones necesarias para presentar el problema principal de nuestro trabajo de tesis. Primero establecemos definiciones y notación de Teoría de Gráficas, área a la cual pertenece nuestra investigación. Posteriormente presentamos a las cográficas, la clase de gráficas con las que trabajamos a lo largo de todo el documento.


\section{Introducción a la Teoría de Gráficas}
    %A continuación se presentan los conceptos básicos de la Teoría de Gráficas. Todas las gráficas que se consideran en este trabajo son gráficas simples, es decir que no contienen aristas de un vértice a sí mismo o varias aristas diferentes que conecten al mismo par de vértices. En términos generales se siguen las definiciones de \cite{Bondy}. Sin embargo, algunas podrían diferir.

%Una \textit{\textbf{gráfica}} $G$ es una tercia ordenada $(V(G), E(G), \psi_G)$ formada por un conjunto no vacío $V(G)$ de \textit{vértices}, un conjunto $E(G)$, ajeno a $V(G)$, de \textit{aristas} y una \textit{función de incidencia} $\psi_G$ que asocia a cada arista de $G$ una pareja no ordenada de vértices (no necesariamente distintos)  de $G$. Si $e$ es una arista y tanto $u$ como $v$ son vértices tales que $\psi_G(e) = uv$, entonces decimos que $e$ une a $u$ y a $v$; llamamos a los vértices $u$ y $v$ los \textit{extremos} de $e$. Decimos que los extremos de una arista son  \textit{incidentes} con ésta y viceversa. Dos vértices que son incidentes con un mismo arista son \textit{adyacentes}.  Una arista cuyos extremos son idénticos es llamada un \textit{lazo}. 

%Una gráfica es \textit{finita} si tanto su conjunto de vértices como su conjunto de aristas son finitos. 

%Una gráfica es \textit{simple si no tiene lazos}

A continuación proporcionamos conceptos básicos del área de Teoría de Gráficas. Dado que nuestra investigación pertenece a dicha área, estos conceptos serán utilizados a lo largo de todo el documento. De manera general, las definiciones que se presentan fueron tomadas de Bollobás \cite{Bollobas}. Algunas de éstas pueden diferir de las proporcionadas en Bollobás \cite{Bollobas} de forma no sustancial.

\subsection{Conceptos básicos y notación}

Una \textbf{\emph{gráfica}} $G$ es una pareja ordenada de conjuntos ajenos $(V,E)$ tal que $E$ es subconjunto del conjunto de parejas no ordenadas de elementos de $V$. Decimos que $V$ es el conjunto de \textbf{\emph{vértices}} de $G$ y que $E$ es el conjunto de \textbf{\emph{aristas}} de $G$. 

Si $G$ es una gráfica, denotamos su conjunto de vértices como $V(G)$ y su conjunto de aristas como $E(G)$. Sin embargo, cuando $G$ es la única gráfica de la que estamos hablando, podemos referirnos a su conjunto de vértices simplemente como $V$ y a su conjunto de aristas como $E$. Si $x$ es un vértice de $G$, podemos escribir $x\in G$ en lugar de $x \in V(G)$. 

Dada una arista $(x,y)$ de una gráfica, decimos que $(x,y)$ \textbf{\emph{une}} a los vértices $x$ y $y$ y la denotamos como $xy$. Los vértices $x$ y $y$ son los \textbf{\emph{extremos}} de dicha arista. Si $xy \in E(G)$, entonces $x$ y $y$ son \textbf{\emph{adyacentes}} en $G$. Si el contexto es claro, simplemente decimos que $x$ y $y$ son adyacentes. Notemos que la arista $xy$ es exactamente la misma arista que $yx$.

Decimos que una gráfica $G$ es una \textbf{\textit{gráfica finita}} si tanto $V$ como $E$ son finitos. A lo largo de este documento, trabajaremos exclusivamente con gráficas finitas. Si $G$ es una gráfica finita, decimos que el \textbf{\emph{orden}} de $G$, denotado como $|V(G)|$, es el número de vértices de $G$. El \textbf{\emph{tamaño}} de $G$ es el número de aristas de $G$, denotado como $|E(G)|$.  

Las gráficas suelen ser representadas por medio de dibujos. En esta representación los vértices aparecen como pequeños círculos y las aristas como líneas que conectan a los vértices que son sus extremos. Podemos ver un ejemplo de esto en la Figura \ref{fig_ejemplo_graph}. Las etiquetas que aparecen a lado de cada vértice indican el nombre con el que se denota dicho vértice. Usualmente estas etiquetas no son incluidas. 

\begin{figure}[h]
\begin{center}
\begin{tikzpicture}
\begin{scope}[xshift=0cm,scale=1]
\node [style=vertex] (1) at (0,0) {};
\node [style=vertex] (2) at (1,0) {};
\node [style=vertex] (3) at (0,1) {};
\node [style=vertex] (4) at (1,1) {};
\node [style=vertex] (5) at (-1,-1) {};
\node [style=vertex] (6) at (2,-1) {};
\node [style=vertex] (7) at (-1,2) {};
\node [style=vertex] (8) at (2,2) {};
\node at (-0.5,0) {$e$};
\node at (1.5,0) {$f$};
\node at (-0.5,1) {$c$};
\node at (1.5,1) {$d$};
\node at (-1.5,-1) {$g$};
\node at (2.5,-1) {$h$};
\node at (-1.5,2) {$a$};
\node at (2.5,2) {$b$};
\foreach \i/\j in {1/4,2/3,5/1,6/2,7/3,8/4,5/6,5/7,8/6,8/7}
  \draw [style=edge] (\i) to (\j);
\end{scope}
\end{tikzpicture}
\end{center}
\caption{Ejemplo de la representación de una gráfica.}\label{fig_ejemplo_graph}
\end{figure}

Dadas dos gráficas $G=(V,E)$ y $G'=(V',E')$, decimos que $G'$ es \textbf{\emph{subgráfica}} de $G$ si $V' \subseteq V$ y $E' \subseteq E$. Denotamos esto como $G' \subseteq G$. La subgráfica de $G$ \emph{\textbf{inducida}} por $V'\in V$, a la que denotamos como $G[V']$, es la gráfica cuyo conjunto de vértices es $V'$ y cuyos aristas unen a dos vértices si y sólo si estos son adyacentes en $G$. 

En la Figura \ref{fig_ejemplo_subgraph} se muestran dos gráficas. Ambas gráficas son subgráficas de la gráfica de la Figura \ref{fig_ejemplo_graph}, a la que llamaremos $G$ para los fines de este ejemplo. Dado que la gráfica (a) no incluye las aristas $ab$ y $ac$, ésta no es una subgráfica inducida de $G$. Por otra parte, la gráfica (b) es la subgráfica de $G$ inducida por el conjunto de vértices $\{a,b,c,d,g,h\}$.


\begin{figure}[!htbp]
\centering
\begin{tikzpicture}

\begin{scope}[xshift=0cm,scale=1]
\node [style=vertex] (3) at (0,1) {};
\node [style=vertex] (4) at (1,1) {};
\node [style=vertex] (5) at (-1,-1) {};
\node [style=vertex] (6) at (2,-1) {};
\node [style=vertex] (7) at (-1,2) {};
\node [style=vertex] (8) at (2,2) {};
\node at (-0.5,1) {$c$};
\node at (1.5,1) {$d$};
\node at (-1.5,-1) {$g$};
\node at (2.5,-1) {$h$};
\node at (-1.5,2) {$a$};
\node at (2.5,2) {$b$};
\foreach \i/\j in {8/4,5/6,5/7,8/6}
  \draw [style=edge] (\i) to (\j);
\node [below of=5,xshift=1.5cm] {\parbox{0.3\linewidth}{\subcaption{Subgráfica}}};
\end{scope}

\begin{scope}[xshift=5cm,scale=1]
\node [style=vertex] (3) at (0,1) {};
\node [style=vertex] (4) at (1,1) {};
\node [style=vertex] (5) at (-1,-1) {};
\node [style=vertex] (6) at (2,-1) {};
\node [style=vertex] (7) at (-1,2) {};
\node [style=vertex] (8) at (2,2) {};
\node at (-0.5,1) {$c$};
\node at (1.5,1) {$d$};
\node at (-1.5,-1) {$g$};
\node at (2.5,-1) {$h$};
\node at (-1.5,2) {$a$};
\node at (2.5,2) {$b$};
\foreach \i/\j in {7/3,8/4,5/6,5/7,8/6,8/7}
  \draw [style=edge] (\i) to (\j);
\node [below of=5,xshift=1.5cm] {\parbox{0.3\linewidth}{\subcaption{Subgráfica inducida}}};
\end{scope}

\end{tikzpicture}
\caption{Ejemplo de una subgráfica y de una subgráfica inducida de la gráfica de la Figura \ref{fig_ejemplo_graph}.}
\label{fig_ejemplo_subgraph}
\end{figure}

Decimos que dos gráficas $G=(V,E)$ y $G'=(V',E')$ son \textbf{\emph{isomorfas}} si existe una biyección $\phi:V\rightarrow V'$ tal que $xy\in E$ si y sólo si $\phi(x)\phi(y) \in E' $. A lo largo de este documento no hacemos distinción entre gráficas isomorfas. Es decir que, si dos gráficas $G$ y $G'$ son isomorfas, las consideramos como la misma gráfica.

\subsection{Algunas gráficas distinguidas}

A continuación se presentan algunas familias de gráficas a cuyos elementos nos referimos por medio de un nombre específico. A lo largo del documento usaremos dicho nombre para referirnos a cualquier copia isomorfa de estas gráficas.  
 
El tamaño de una gráfica de orden $n$ es al menos $0$ y a lo más $\binom{n}{2}$. Una gráfica de orden $n$ y tamaño $\binom{n}{2}$ es llamada \textbf{\emph{gráfica completa}} de orden $n$ y se denota como $K_n$. En $K_n$ todo par de vértices son adyacentes. Decimos que la gráfica $K_1$ es una gráfica \textbf{\emph{trivial}}.

Un vértice que no se encuentra unido a ningún otro vértice es llamado un \textbf{\emph{vértice aislado}}. Un conjunto de vértices es un \textbf{\emph{conjunto independiente}} si no contiene elementos que sean adyacentes.

Un \textbf{\emph{camino}} es una gráfica $P$ tal que  $V(P) = {x_0,x_1,\dots,x_l}$ y $E(P) = {x_0x_1,x_1x_2,\dots,x_{l-1}x_l}$

El camino $P$ se denota usualmente como $x_0x_1\dots x_l$. Decimos que $P$ es un camino de $x_0$ a $x_l$. Denotamos como $P_l$ a un camino arbitrario de orden $l$.

Un \textbf{\emph{paseo}} $W$ en una gráfica $G$ es una secuencia alternante de vértices y aristas de $G$, digamos $x_0, e_1, x_1, e_2, \dots, e_l, x_l$ en donde $e_i=x_{i-1}x_i$, $0<i\leq l$. De acuerdo con la terminología anterior, $W$ es un $x_0-x_l$-paseo y es denotado como $x_0x_1\dots x_l$; la \textbf{\emph{longitud}} de $W$ es $l$. Este paseo es llamado un  \textbf{\emph{paseo simple}} si todas sus aristas son distintas. Notemos que un camino es un paseo en el que todos sus vértices son distintos. Un paseo simple cuyos vértices de inicio y fin coinciden es llamado un \textbf{\emph{circuito}}. Para ser más precisos, un circuito es un paseo simple cerrado sin vértices de inicio y fin distinguidos. Si un paseo $W = x_0x_1\dots x_l$ es tal que $l\geq 3$, $x_0=x_l$ y los vértices $x_i$, $0<i<l$ son distintos los unos de los otros y de $x_0$, decimos que $W$ es un \textbf{\emph{ciclo}}. 

Con frecuencia usamos la notación $P_l$ para referirnos a un camino arbitrario de longitud $l$ y la notación $C_l$ para referirnos a un ciclo de longitud $l$. En la Figura \ref{fig_tipos_graficas} podemos ver los ejemplos de una gráfica completa, un camino y un ciclo.

\begin{figure}[!htbp]
\centering

\begin{tikzpicture}
\begin{scope}[xshift=0cm,scale=1]
\node [style=vertex] (1) at (0,0) {};
\node [style=vertex] (2) at (2,0) {};
\node [style=vertex] (3) at (1,0.75) {};
\node [style=vertex] (4) at (1,2) {};
\foreach \i/\j in {1/2,1/3,1/4,2/3,2/4,3/4}
  \draw [style=edge] (\i) to (\j);
\end{scope}

\begin{scope}[xshift=3cm,scale=1]
\node [style=vertex] (1) at (0,0.5) {};
\node [style=vertex] (2) at (1,1.5) {};
\node [style=vertex] (3) at (2,0.5) {};
\node [style=vertex] (4) at (3,1.5) {};
\foreach \i/\j in {1/2,2/3,3/4}
  \draw [style=edge] (\i) to (\j);
\end{scope}

\begin{scope}[xshift=7cm,scale=1]
\node [style=vertex] (1) at (0,0) {};
\node [style=vertex] (2) at (2,0) {};
\node [style=vertex] (3) at (2,2) {};
\node [style=vertex] (4) at (0,2) {};
\foreach \i/\j in {1/2,2/3,3/4,4/1}
  \draw [style=edge] (\i) to (\j);
\end{scope}
\end{tikzpicture}
\caption{Las gráficas $K_4$, $P_4$ y $C_4$.}
\label{fig_tipos_graficas}
\end{figure}

Una gráfica es \textbf{\emph{conexa}} si para cada par de vértices distintos existe un camino de entre ellos. Una gráfica que no es conexa es llamada \textbf{\emph{inconexa}}. Dada una gráfica $G$, una subgráfica de $G$ conexa de tamaño máximo(es decir que se vuelve inconexa si se le agrega cualquier otro vértice) es llamada una \textbf{\emph{componente conexa}} de $G$.

Una gráfica $G$ es \textbf{\emph{bipartita}} con clases $V_1$ y $V_2$ si $V(G) = V_1 \cup V_2$, $V_1 \cap V_2 = \emptyset$ y cada arista en $E(G)$ une a un vértice de $V_1$ con un vértice de $V_2$. Decimos que $G$ tiene una \textbf{\emph{bipartición}} $(V_1,V_2)$. De manera similar, $G$ es \textbf{\emph{r-partita}} (\textbf{\emph{multipartita}} ) con clases $V_1, V_2, \dots, V_r$ si $V(G) = V_1 \cup V_2 \cup \dots \cup V_r$, $V_i \cap V_j = \emptyset$ para todos $1\leq i < j \leq r$ y ninguna arista une vértices de la misma clase. Una gráfica es \textbf{\emph{r-partita completa}} (o bien, \textbf{\emph{multipartita completa}}) si es multipartita y si todo par de vértices de clases distintas son adyacentes.

\subsection{Árboles}

A continuación se exponen los conceptos de árbol y de árbol arraigado (también conocido como árbol enraizado), dos tipos de gráficas que serán ampliamente utilizadas en el documento, en particular, en los algoritmos que presentamos. 

Una gráfica que no contiene ningún ciclo es un \emph{\textbf{bosque}}; un \emph{\textbf{árbol}} es un bosque conexo. Notemos que un bosque es un conjunto de árboles sin aristas entre ellos.

Si bien los textos sobre algoritmos mantienen una noción uniforme de lo que es un árbol arraigado, es difícil encontrar una definición formal. Por lo que a continuación se presenta una definición y notación propias.

Un \textbf{\emph{árbol arraigado}} es un árbol con un vértice distinguido al que llamamos \textbf{\emph{raíz}}. Un árbol arraigado se puede denotar como una pareja ordenada $(T,r)$, en donde $T$ es un árbol y $r \in V(T)$. Nos referiremos al árbol arraigado $(T,r)$ simplemente como $T$, indicando que $r$ es su raíz únicamente cuando sea necesario.

\begin{figure}[!htbp]
\centering
\begin{subfigure}{\textwidth}
\centering
\begin{tikzpicture}
\begin{scope}[xshift=8.5cm,scale=1]
\node [style=vertex] (1) at (1.5,4) {};
\node [style=vertex] (2) at (0.5,3) {};
\node [style=vertex] (3) at (1.5,3) {};
\node [style=vertex] (4) at (2.5,3) {};
\node [style=vertex] (5) at (0,2) {};
\node [style=vertex] (6) at (1,2) {};
\node [style=vertex] (7) at (1.75,2) {};
\node [style=vertex] (8) at (3.25,2) {};
\node [style=vertex] (12) at (2.75,1) {};
\node [style=vertex] (14) at (3.75,1) {};
\node [style=vertex] (15) at (3.5,0) {};
\node [style=vertex] (16) at (4,0) {};
\foreach \i/\j in {1/2,1/3,1/4,2/5,2/6,4/7,4/8,8/12,8/14,14/15,14/16}
  \draw [style=edge] (\i) to (\j);
\node [right of=1] {\parbox{0.1\linewidth}{$raíz$}};
\end{scope}
\end{tikzpicture}
\end{subfigure}
\caption{Ejemplo de un árbol arraigado. El vértice distinguido como raíz aparece como tal.}
\label{fig_ejemplo_arbol}
\end{figure}

En la Figura \ref{fig_ejemplo_arbol} podemos observar la representación de un árbol arraigado. A partir de ahora supondremos que el nodo que se dibuja con mayor altura en la representación es la raíz, sin la necesidad de incluir una etiqueta para identificarlo.
%Como podemos observar en la Figura \ref{fig_ejemplo_arbol}, un árbol arraigado se representa usualmente dibujando su nodo raíz en la parte superior. A partir de ahora, . 

Sea $T$ un árbol arraigado con raíz $r$, y sean $u$ y $v$ nodos de $T$. Denotamos al único camino desde $v$ hasta $r$ como $P_T(v)$, y establecemos un orden parcial en $V(T)$ con la relación $\geq$ tal que $u \geq v$ si y sólo si $u$ es un vértice en $P_T(v)$. Si $u$ y $v$ son vértices de T distintos tales que $u \geq v$, decimos que $u$ es \textbf{\emph{ancestro}} de $v$, y que $v$ es \textbf{\emph{descendiente}} de $u$. El \textbf{\emph{ancestro común más profundo}} de dos vértices de $T$ es el vértice que está tanto en $P_T(u)$ como en $P_T(v)$ y que se encuentra a mayor distancia de $r$. Si $v$ es un vértice de $T$, decimos que $v$ es una \textbf{\emph{hoja}} si no tiene descendientes, y decimos que es un \textbf{\emph{nodo interno}} en el caso contrario. Si $x$ y $y$ son nodos de un árbol tales que $x$ es ancestro de $y$ y estos son adyacentes, decimos que $x$ es el padre de $y$ y que $y$ es hijo de $x$. Podemos ver estos conceptos ilustrados en la Figura \ref{fig_ejemplo_arbol_2}.

\begin{figure}[!htbp]
\centering
\begin{subfigure}{\textwidth}
\centering
\begin{tikzpicture}
\begin{scope}[xshift=8.5cm,scale=1]
\node [style=vertex] (1) at (1.5,4) {};
\node [style=vertex] (2) at (0.5,3) {};
\node [style=vertex] (3) at (1.5,3) {};
\node [style=vertex] (4) at (2.5,3) {};
\node [style=vertex] (5) at (0,2) {};
\node [style=vertex] (6) at (1,2) {};
\node [style=vertex] (7) at (1.75,2) {};
\node [style=vertex] (8) at (3.25,2) {};
\node [style=vertex] (12) at (2.75,1) {};
\node [style=vertex] (14) at (3.75,1) {};
\node [style=vertex] (15) at (3.5,0) {};
\node [style=vertex] (16) at (4,0) {};
\foreach \i/\j in {1/2,1/3,1/4,2/5,2/6,4/7,4/8,8/12,8/14,14/15,14/16}
  \draw [style=edge] (\i) to (\j);
\node [right of=1] {\parbox{0.1\linewidth}{$a$}};
\node [right of=4] {\parbox{0.1\linewidth}{$b$}};
\node [left of=5, xshift=1cm] {\parbox{0.1\linewidth}{$c$}};
\node [left of=15, xshift=0.5cm] {$d$};
\end{scope}
\end{tikzpicture}
\end{subfigure}
\caption{El nodo $a$ es ancestro de los nodos $b,c $ y $d$. Los nodos $b,c $ y $d$ son descendientes de $a$. El ancestro común más profundo de los nodos $c$ y $d$ es el nodo $a$. El nodo $b$ es hijo del nodo $a$ y éste a su vez es padre del nodo $b$. Los nodos $c$ y $d$ son hojas del árbol. Los nodos $a$ y $b$ son nodos internos.}
\label{fig_ejemplo_arbol_2}
\end{figure}


\subsection{Operaciones en gráficas}

A continuación presentamos algunas operaciones que se pueden aplicar a las gráficas. Éstas son útiles para construir gráficas a partir de otras gráficas. También nos sirven para describir algunas gráficas y estructuras que se utilizan en el documento. 

Dada una gráfica $G$, el \textbf{\emph{complemento}} de $G$, denotado como $\overline{G}$ es la gráfica tal que $V(\overline{G}) = G$ y $E(\overline{G})$ es el conjunto de todas las parejas no ordenadas de elementos de $V$ menos las aristas de $E(G)$; así, dos vértices son adyacentes en $\overline{G}$ si y sólo si no son adyacentes en $G$.

 Dada una gráfica $G$ y uno de sus vértices $x$, denotamos como $G - x $ a la gráfica $G[V-\{x\}]$. Es decir, la subgráfica de $G$ inducida por el conjunto de vértices $V-\{x\}$. 

Decimos que la gráfica $G + H = (V(G)\cup V(H), E(G)\cup E(H))$ es la \textbf{\emph{unión ajena}} de $G$ y $H$. Por otra parte, obtenemos la \textbf{\emph{unión completa}} de $G$ y $H$, denotada como $G \oplus H$, a partir de $G + H$ al agregar todas las aristas entre los vértices de $G$ y los vértices de $H$. 

\subsection{Clases de gráficas hereditarias}

A continuación se presenta el concepto de clase hereditaria de gráficas y otras definiciones relacionadas. Todas las clases de gráficas que estudiamos en este trabajo son clases hereditarias.  Las definiciones fueron tomadas de Kitaev \cite{Kitaev}.

Una clase $X$ de gráficas que contiene una gráfica $G$ si y sólo si contiene también a todas las subgráicas inducidas de $G$ es llamada una \textbf{\emph{clase hereditaria}} de gráficas.

Dado un conjunto de gráficas $M$ (finito o infinito), denotamos como $Free(M)$ a la clase de gráficas que contiene todas las gráficas que no tienen a ninguna gráfica de $M$ como subgráfica inducida. Decimos que las gráficas en $M$ son \textbf{\emph{subgráficas inducidas prohibidas}} para la clase $Free(M)$, y que las gráficas en $Free(M)$ son libres de $M$. Diremos también que las gráficas en $Free(M)$ son libres de $G$ para cualquier gráfica $G$ en el conjunto $M$. Por convención, llamamos a los elementos de $M$ \textbf{\emph{obstrucciones}} de $Free(M)$.

Una clase $X$ de gráficas es hereditaria si y sólo si existe un conjunto $M$ tal que $X = Free(M)$.

Una gráfica es una \textbf{\emph{subgráfica inducida  prohibida mínima}}, también llamada una \textbf{\emph{obstrucción mínima}} de una clase hereditaria $X$ si $G$ no pertenece a $X$ pero toda subgráfica inducida de $G$ (con excepción de $G$ misma) pertenece a $X$. Denotamos como $Forb(X)$ al conjunto de todas las obstrucciones mínimas de la clase hereditaria $X$. 

Para cualquier clase hereditaria $X$, tenemos que $X = Free(Forb(X))$. Además, $Forb(X)$ es el único conjunto mínimo con esta propiedad. 



%%\section{Particiones matriciales}
%%    A continuación se presentan las definiciones que sirven de base al concepto de particiones matriciales que son una generalización de las coloraciones. Las definiciones básicas del tema de coloraciones se toman de \cite{Bondy}. Mientras que las definiciones de particiones matriciales se toman de \cite{Hell04}.

\subsection{Coloraciones}

Una \textbf{\emph{$k$-coloración}} $\mathcal{C}$ es una asignación de $k$ colores, $1,2,\dots,k$, a los vértices de $G$. La coloración $\mathcal{C}$ es correcta si no existen dos vértices adyacentes que tengan el mismo color. Así, una $k$-coloración correcta de una gráfica $G$ es una partición $(V_1, V_2, \dots, V_k)$ de $V$ en $k$ (posiblemente vacíos) conjuntos independientes. Decimos que $G$ es \textbf{\emph{$k$-coloreable}} si $G$ tiene una $k$-coloración correcta. Por simplicidad nos referiremos a una $k$-coloración correcta simpemente como una \textbf{\emph{$k$-coloración}}.

El número cromático, $\Chi(G)$, de una gráfica $G$ es el mínimo $k$ para el cual $G$ es $k$-coloreable. Si $\Chi(G) = k$, decimos que $G$ es \textbf{\emph{$k$-cromático}}.


\subsection{Particiones matriciales}

Sea $M$ una matriz fija de $m \times m$ con entradas $M_{i,j}\in \{0,1,*\}$. Una \textbf{\emph{$M$-partición}} de una gráfica $G$ es una partición de los vértices de $G$ en $m$ partes, indexadas por las filas (y columnas) de la matriz $M$, tal que para distintos vértices $x$ y $y$ de la gráfica $G$, posicionadas en partes $i$ y $j$ (posiblemente con $i = j$), respectivamente, tenemos lo siguiente:

\begin{itemize}
    \item Si $M(i,j) = 0$, entonces $xy$ no es una arista de $G$.
    \item Si $M(i,j) = 1$, entonces $xy$ es una arista de $G$.
    \item (Si $M(i,j) = *$, entonces $xy$ puede o no ser una arista de $G$).
\end{itemize}

Las particiones matriciales no solo generalizan las coloraciones y los homomorfismos, sino que también unifican muchos problemas de particiones en el estudio de las gráficas perfectas. 

El problemas central de esta investigación, el de encontrar una partición de una gráfica en dos partes cada una de las cuales induce una subgráfica multipartita completa, es una instancia del problema de encontrar una partición matricial en una gráfica. 

\section{Cográficas}
    A continuación presentamos el concepto de \emph{cográfica}. Las cográficas son la clase de gráficas que estudiamos en este trabajo de tesis. Éstas fueron introducidas independientemente por varios investigadores a principios de la década de 1970 y finalmente unificadas por Corneil en 1981\cite{Corneil}; de este artículo serán tomadas las definiciones básicas.

\subsection{Caracterización de las cográficas}

Una \emph{\textbf{cográfica}} se define recursivamente de la siguiente manera:

\begin{enumerate}[(i)]
    \item Una gráfica con un solo vértice es una cográfica.
    \item Si $G_1$ y $G_2$ son cográficas, también lo es $G_1 + G_2$.
    \item Si $G$ es una cográfica, también lo es su complemento $\overline{G}$.
\end{enumerate}

De igual forma, Corneil \cite{Corneil}, presenta la siguiente lista de equivalencias para las cográficas:

Sea $G$ una gráfica, las siguientes afirmaciones son equivalentes:

\begin{enumerate}[(1)]
    \item $G$ es una cográfica.
    \item Cualquier subgráfica no trivial de $G$ tiene al menos un par de \emph{gemelos}.
    \item Cualquier subgráfica de $G$ tiene la $CK$-propiedad.
    \item $G$ no contiene a $P_4$ como subgráfica inducida.
    \item El complemento de cualquier subgráfica inducida no trivial conexa de $G$ es inconexa.
    \item $G$ es una $HD$-gráfica.
    \item Toda subgráfica conexa de $G$ tiene diámetro menor o igual a 2.
    \item $G$ es la gráfica de comparabilidad de un multiárbol.
\end{enumerate}

\subsection{Coárboles}
 De la definición de las cográficas, podemos observar que éstas son todas las gráficas que se pueden obtener a partir de un solo nodo aplicando un número finito de operaciones de unión ajena y complemento. Esta serie de operaciones puede ser representada a través de un árbol único salvo isomorfismo conocido como \emph{coárbol}. En este documento utilizamos la definición de coárbol proporcionada por Corneil \cite{Corneil02}.

 Sean $G$ una gráfica y $T$ un árbol arraigado con raíz $r$ cuyos nodos están etiquetados y cuyas hojas representan cada una un vértice de $G$, decimos que $T$ es el \emph{\textbf{coárbol}} de $G$ si cumple con las siguientes condiciones:
 \begin{enumerate}
     \item Cada uno de sus nodos internos tiene la etiqueta 0 o la etiqueta 1.
     \item Los nodos etiquetados con 0 y los etiquetados con 1 son alternantes en cualquier camino desde $r$.
     \item Si $G$ es conexa, entonces $r$ tiene etiqueta 0. En el caso contrario, $r$ tiene etiqueta 1.
     \item dos vértices $x,y \in V(G)$ son adyacentes si y sólo si el camino desde la hoja que representa a $x$ hasta $r$ se encuentra con el camino desde la hoja que representa a $y$ hasta $r$ en un nodo con etiqueta 1.
 \end{enumerate}

 %los nodos internos de $T$ están etiquetados con 0 y 1 de manera tal que los nodos etiquetados con 0 y los etiquetados con 1 son alternantes en cualquier camino desde $r$. De igual manera, $r$ tendrá etiqueta 0 si $G$ es conexa y tendrá etiqueta 1 si $G$ es inconexa. Por último, dos vértices $x,y \in V(G)$ son adyacentes si y sólo si el camino desde la hoja que representa a $x$ hasta $r$ se encuentra con el camino desde la hoja que representa a $y$ hasta $r$ en un nodo (1).


\begin{figure}[ht!]
\begin{center}
\begin{tikzpicture}

\begin{scope}[xshift=0cm,scale=1]

\node [vertex] (1) at (0,1) {};
\node [vertex] (2) at (1,2) {};
\node [vertex] (3) at (3,2) {};
\node [vertex] (4) at (4,1) {};
\node [vertex] (5) at (3,0) {};
\node [vertex] (6) at (1,0) {};
\foreach \i/\j in {1/2,1/4,1/5,1/6,2/4,2/5,2/6,3/4,3/5,3/6}
\draw [edge] (\i) to (\j);

\node [left of=1, xshift=0.5cm] {$a$};
\node [above of=2, yshift=-0.5cm] {$b$};
\node [above of=3, yshift=-0.5cm] {$c$};
\node [right of=4, xshift=-0.5cm] {$d$};
\node [below of=5, yshift=0.5cm] {$e$};
\node [below of=6, yshift=0.5cm] {$f$};

%\node [below of=6,xshift=1cm] {\parbox{0.3\linewidth}{\subcaption{}}};

\end{scope}

\begin{scope}[xshift=7cm, yshift=1cm,scale=1]

\node [cotreenode] (1) at (1,1) {1};
\node [cotreenode] (2) at (0,0) {0};
\node [cotreenode] (3) at (2,0) {0};
\node [cotreenode] (4) at (-0.5,-1) {1};
\node [vertex] (5) at (0.5,-1) {};
\node [vertex] (6) at (1.5,-1) {};
\node [vertex] (7) at (2,-1) {};
\node [vertex] (8) at (2.5,-1) {};
\node [vertex] (9) at (-1,-2) {};
\node [vertex] (10) at (0,-2) {};

\foreach \i/\j in {1/2,1/3,2/4,2/5,3/6,3/7,3/8,4/9,4/10}
\draw [edge] (\i) to (\j);

\node [below of=9, yshift=0.5cm] {$a$};
\node [below of=10, yshift=0.5cm] {$b$};
\node [below of=5, yshift=0.5cm] {$c$};
\node [below of=6, yshift=0.5cm] {$d$};
\node [below of=7, yshift=0.5cm] {$e$};
\node [below of=8, yshift=0.5cm] {$f$};

%\node [below of=10,xshift=.5cm] {\parbox{0.3\linewidth}{\subcaption{}}};
\end{scope}
\end{tikzpicture}
\end{center}
\setlength{\abovecaptionskip}{-10pt}
\caption{Una cográfica y su coárbol.}\label{fig_ej_coarbol}
\end{figure}

Notemos que todos los nodos internos de un coárbol tienen al menos dos hijos. Además, si el árbol $T$ con raíz $r$ es el coárbol de una gráfica inconexa $G$, cada uno de los hijos de $r$ son las coárboles de las componentes conexas de $G$. Por otra parte, si $G$ es conexa,  los hijos de $r$ serán los coárboles de las componentes conexas de $\overline{G}$. De esta forma, todo nodo $x$ del coárbol $T$ es la raíz de un coárbol al que denotamos como $T_x$. La cográfica representada por $T_x$ es la subgráfica de $G$ inducida por los vértices representados por las hojas de $T_x$. A esta subgráfica la denotamos por $G[x]$.


%%\section{Teoría de la complejidad}
%%    \input{chapters/0304Complejidad.tex}

\chapter{Estado del arte}
En el presente capítulo hacemos una revisión de los artículos que dan forma a la investigación que realizamos. El problema de determinar si una cográfica acepta una partición tal que cada una de sus partes es una gráfica multipartita completa no ha sido estudiado con anterioridad. Sin embargo, la investigación realizada sobre las cográficas polares nos sirve como referencia acerca de cómo encontrar las obstrucciones mínimas de una clase hereditaria de cográficas en donde cada uno de sus elementos acepta una partición tal que cada parte cumple con una condición. Esta investigación también nos muestra que se pueden estudiar subclases de las clases anteriormente descritas al agregar restricciones a cada una de las partes. Los primeros tres artículos de los que hablamos abordan el tema de las cográficas polares. %%Tenemos que hablar del otro artículo.

\section{Cográficas polares}
    En este artículo de T.Ekim \cite{Ekim} se introduce la clase de las cográficas polares a la vez que se presenta su conjunto de obstrucciones mínimas (que existe dado que las cográficas polares son una clase hereditaria de gráficas). La metodología con la que se realiza la demostración de que el conjunto de gráficas que se proporciona (Figura \ref{obsts_cografics_polares}) es, en efecto, el conjunto de obstrucciones mínimas de las cográficas polares sirve como base de nuestra demostración de que el conjunto de gráficas que mostramos como conjunto de obstrucciones mínimas de la clase de cográficas que acepta una partición en dos gráficas multipartitas completas (a la que llamamos $M_2$) es correcto. En este artículo también se presentan las cográficas monopolares, una subclase de las cográficas polares, cuyo estudio sirve de inspiración para el estudio de algunas subclases de $M_2$.

Sea $G$ una cográfica, decimos que $G$ es una \emph{\textbf{cográfica polar}} si su conjunto de vértices $V$ acepta una partición $(A,B)$ tal que $A$ induce una gráfica multipartita completa y $B$ induce una unión ajena de clanes.
Decimos que $G$ es ($s,k$)-polar si existe una partición ($A,B$) de los vértices de $G$ en donde $A$ induce en $G$ una unión completa de a lo más $s$ conjuntos independientes (Es decir una gráfica $s$-partita) y $B$ induce en $G$ una unión ajena de a lo más $k$ clanes. Notemos que las cográficas polares son las gráficas ($\infty, \infty$)-polares.

\begin{theorem}
    Sea $G$ una cográfica, decimos que $G$ es una cográfica polar si y sólo si no contiene como subgráfica inducida a ninguna de las gráficas de la Figura \ref{obsts_cografics_polares}.
\end{theorem}

\begin{figure}[H]
\begin{center}
\begin{tikzpicture}

\begin{scope}[xshift=0cm,scale=1]

\node [style=vertex] (1) at (0.5,0) {};
\node [style=vertex] (2) at (0.2,0.5) {};
\node [style=vertex] (3) at (0.8,0.5) {};
\node [style=vertex] (4) at (0,1) {};
\node [style=vertex] (5) at (1,1) {};
\node [style=vertex] (6) at (0.5,1.5) {};
\node [style=vertex] (7) at (0,2) {};
\node [style=vertex] (8) at (1,2) {};
\foreach \i/\j in {1/2,1/3,4/5,4/6,4/7,5/6,5/8,6/7,6/8,7/8}
  \draw [style=edge] (\i) to (\j);
\node [below of=1] {\parbox{0.3\linewidth}{\subcaption*{$H_1$}}};

\end{scope}

\begin{scope}[xshift=2cm,scale=1]

\node [style=vertex] (1) at (0.75,0) {};
\node [style=vertex] (2) at (0.45,0.5) {};
\node [style=vertex] (3) at (1.05,0.5) {};
\node [style=vertex] (4) at (0.5,1) {};
\node [style=vertex] (5) at (1.5,1) {};
\node [style=vertex] (6) at (0,1.5) {};
\node [style=vertex] (7) at (1,1.5) {};
\node [style=vertex] (8) at (0.5,2) {};
\node [style=vertex] (9) at (1.5,2) {};

\foreach \i/\j in {1/2,1/3,4/6,4/7,5/7,5/9,6/7,6/8,7/9,7/8}
  \draw [style=edge] (\i) to (\j);
\node [below of=1] {\parbox{0.3\linewidth}{\subcaption*{$H_2$}}};

\end{scope}

\begin{scope}[xshift=4.5cm,scale=1]

\node [style=vertex] (1) at (0.75,0) {};
\node [style=vertex] (2) at (0.45,0.5) {};
\node [style=vertex] (3) at (1.05,0.5) {};
\node [style=vertex] (4) at (0,1) {};
\node [style=vertex] (5) at (1.5,1) {};
\node [style=vertex] (6) at (0.45,1.5) {};
\node [style=vertex] (7) at (1.05,1.5) {};
\node [style=vertex] (8) at (0,2) {};
\node [style=vertex] (9) at (1.5,2) {};

\foreach \i/\j in {1/2,1/3,4/5,4/6,4/7,4/8,5/6,5/7,5/9,6/7,6/8,7/9,8/9}
  \draw [style=edge] (\i) to (\j);
\node [below of=1] {\parbox{0.3\linewidth}{\subcaption*{$H_3$}}};

\end{scope}

\begin{scope}[xshift=7cm,scale=1]

\node [style=vertex] (1) at (1,0) {};
\node [style=vertex] (2) at (0.7,0.5) {};
\node [style=vertex] (3) at (1.3,0.5) {};
\node [style=vertex] (4) at (0,1) {};
\node [style=vertex] (5) at (1,1) {};
\node [style=vertex] (6) at (2,1) {};
\node [style=vertex] (7) at (0,2) {};
\node [style=vertex] (8) at (1,2) {};
\node [style=vertex] (9) at (2,2) {};
\foreach \i/\j in {1/2,1/3,4/5,4/7,4/8,5/6,5/7,5/8,5/9,6/8,6/9,7/8,8/9}
  \draw [style=edge] (\i) to (\j);
\node [below of=1] {\parbox{0.3\linewidth}{\subcaption*{$H_4$}}};
\end{scope}

\end{tikzpicture}
\end{center}
\caption{Obstrucciones mínimas para las gráficas polares.}
\label{obsts_cografics_polares}
\end{figure}

La demostración de que las gráficas de la Figura \ref{obsts_cografics_polares} forman el conjunto de obstrucciones mínimas de las cográficas polares se realiza describiendo a cada una de éstas en términos de la unión ajena y la unión completa de gráficas más pequeñas.

\begin{enumerate}[(1)]
    \item $H_1 = P_3 + ( \overline{K_2} \oplus P_3) = P_3+ (K_1 \oplus P_4)$
    \item $H_2 = P_3 + (K_1 \oplus (P_3 + K_2))$
    \item $H_3 = P_3 + ( \overline{P_3} \oplus \overline{P_3})$
    \item $H_4 = P_3 + (K_2 \oplus 2K_2)$
\end{enumerate}

Primero se demuestra que ninguna de estas gráficas es una cográfica polar y que, por lo tanto, cualquier gráfica que tenga a alguna de ellas como subgráfica inducida tampoco es una cográfica polar. Por último se muestra que si una cográfica $G$ no es polar, entonces debe de tener a alguna de estas gráficas como subgráfica inducida.

En este artículo también se presenta y se caracteriza la clase de las cográficas monopolares a través de su conjunto de obstrucciones mínimas. Éstas se muestran en la Figura \ref{obsts_cografics_monopolares}

%Sea $G$ una cográfica, decimos que $G$ es una cográfica monopolar si $G$ es una gráfica ($s,k$)-polar con $s\leq 1$ o $k \leq 1$.


\begin{figure}[H]
\begin{center}
\begin{subfigure}{\textwidth}
\begin{tikzpicture}

\begin{scope}[xshift=0cm,scale=1]
\node [style=vertex] (2) at (0.2,0.5) {};
\node [style=vertex] (3) at (0.8,0.5) {};
\node [style=vertex] (4) at (0,1) {};
\node [style=vertex] (5) at (1,1) {};
\node [style=vertex] (6) at (0.5,1.5) {};
\node [style=vertex] (7) at (0,2) {};
\node [style=vertex] (8) at (1,2) {};
\foreach \i/\j in {4/5,4/6,4/7,5/6,5/8,6/7,6/8,7/8}
  \draw [style=edge] (\i) to (\j);
\node at (0.5,-0.25) {\parbox{0.3\linewidth}{\subcaption*{$G_1$}}};
\end{scope}

\begin{scope}[xshift=2cm,scale=1]
\node [style=vertex] (2) at (0.5,0.5) {};
\node [style=vertex] (3) at (1,0.5) {};
\node [style=vertex] (4) at (0,1) {};
\node [style=vertex] (5) at (1,1) {};
\node [style=vertex] (6) at (0.5,1.5) {};
\node [style=vertex] (7) at (0,2) {};
\node [style=vertex] (8) at (1,2) {};
\foreach \i/\j in {2/6,4/5,4/6,4/7,5/6,5/8,6/7,6/8,7/8}
  \draw [style=edge] (\i) to (\j);
\node at (0.5,-0.25) {\parbox{0.3\linewidth}{\subcaption*{$G_2$}}};
\end{scope}

\begin{scope}[xshift=4cm, yshift=0.5cm,scale=1]
\node [style=vertex] (1) at (1,0) {};
\node [style=vertex] (2) at (0.5,0.5) {};
\node [style=vertex] (3) at (1.5,0.5) {};
\node [style=vertex] (4) at (0,1) {};
\node [style=vertex] (5) at (1,1) {};
\node [style=vertex] (6) at (0.5,1.5) {};
\node [style=vertex] (7) at (1.5,1.5) {};
\foreach \i/\j in {2/4,2/5,3/5,3/7,4/5,4/6,5/6,5/7}
  \draw [style=edge] (\i) to (\j);
\node at (0.75,-0.75) {\parbox{0.3\linewidth}{\subcaption*{$G_3$}}};
\end{scope}

\begin{scope}[xshift=6.5cm, yshift=0.5cm,scale=1]
\node [style=vertex] (1) at (0.75,0) {};
\node [style=vertex] (2) at (0,0.5) {};
\node [style=vertex] (3) at (0.75,0.5) {};
\node [style=vertex] (4) at (1.5,0.5) {};
\node [style=vertex] (5) at (0,1.5) {};
\node [style=vertex] (6) at (0.75,1.5) {};
\node [style=vertex] (7) at (1.5,1.5) {};
\foreach \i/\j in {2/3,2/5,2/6,3/4,3/5,3/6,3/7,4/6,4/7,5/6,6/7}
  \draw [style=edge] (\i) to (\j);
\node at (0.75,-0.75) {\parbox{0.3\linewidth}{\subcaption*{$G_4$}}};
\end{scope}

\begin{scope}[xshift=9cm, yshift=0.5cm,scale=1]
\node [style=vertex] (1) at (0.75,0) {};
\node [style=vertex] (2) at (0.25,0.5) {};
\node [style=vertex] (3) at (1.25,0.5) {};
\node [style=vertex] (4) at (0,1) {};
\node [style=vertex] (5) at (0.75,1) {};
\node [style=vertex] (6) at (1.5,1) {};
\node [style=vertex] (7) at (0.75,1.5) {};
\foreach \i/\j in {2/4,2/5,2/6,3/4,3/5,3/6,4/5,4/7,5/6,6/7}
  \draw [style=edge] (\i) to (\j);
\node at (0.75,-0.75) {\parbox{0.3\linewidth}{\subcaption*{$G_5$}}};
\end{scope}

\end{tikzpicture}
\end{subfigure}

\begin{subfigure}{\textwidth}
\begin{center}
\begin{tikzpicture}

\begin{scope}[xshift=0cm, yshift=0.5cm,scale=1]
\node [style=vertex] (1) at (0.75,0) {};
\node [style=vertex] (2) at (0.25,0.5) {};
\node [style=vertex] (3) at (1.25,0.5) {};
\node [style=vertex] (4) at (0,1) {};
\node [style=vertex] (5) at (0.75,1) {};
\node [style=vertex] (6) at (1.5,1) {};
\node [style=vertex] (7) at (0.75,1.5) {};
\foreach \i/\j in {2/3,2/4,2/5,2/6,3/4,3/5,3/6,4/5,4/7,5/6,6/7}
  \draw [style=edge] (\i) to (\j);
\node at (0.75,-0.75) {\parbox{0.3\linewidth}{\subcaption*{$G_6$}}};
\end{scope}

\begin{scope}[xshift=2.5cm, yshift=0.5cm,scale=1]
\node [style=vertex] (1) at (0.75,0) {};
\node [style=vertex] (2) at (0.25,0.5) {};
\node [style=vertex] (3) at (1.25,0.5) {};
\node [style=vertex] (4) at (0,1) {};
\node [style=vertex] (5) at (0.75,1) {};
\node [style=vertex] (6) at (1.5,1) {};
\node [style=vertex] (7) at (0.75,1.5) {};
\foreach \i/\j in {2/3,2/4,2/5,2/6,3/4,3/5,3/6,4/5,4/7,5/6,5/7,6/7}
  \draw [style=edge] (\i) to (\j);
\node at (0.75,-0.75) {\parbox{0.3\linewidth}{\subcaption*{$G_7$}}};
\end{scope}

\begin{scope}[xshift=5cm, yshift=0.5cm,scale=1]
\node [style=vertex] (1) at (0.75,0) {};
\node [style=vertex] (2) at (0,0.5) {};
\node [style=vertex] (3) at (1.5,0.5) {};
\node [style=vertex] (4) at (0.25,1) {};
\node [style=vertex] (5) at (1.25,1) {};
\node [style=vertex] (6) at (0,1.5) {};
\node [style=vertex] (7) at (1.5,1.5) {};
\foreach \i/\j in {2/3,2/4,2/5,2/6,3/4,3/5,3/7,4/5,4/6,5/7,6/7}
  \draw [style=edge] (\i) to (\j);
\node at (0.75,-0.75) {\parbox{0.3\linewidth}{\subcaption*{$G_8$}}};
\end{scope}

\begin{scope}[xshift=7.5cm, yshift=0.5cm,scale=1]
\node [style=vertex] (1) at (0.75,0) {};
\node [style=vertex] (2) at (0,0.5) {};
\node [style=vertex] (3) at (1.5,0.5) {};
\node [style=vertex] (4) at (0.25,1) {};
\node [style=vertex] (5) at (1.25,1) {};
\node [style=vertex] (6) at (0,1.5) {};
\node [style=vertex] (7) at (1.5,1.5) {};
\foreach \i/\j in {2/3,2/4,2/5,2/6,3/4,3/5,3/7,4/5,4/6,4/7,5/6,5/7,6/7}
  \draw [style=edge] (\i) to (\j);
\node at (0.75,-0.75) {\parbox{0.3\linewidth}{\subcaption*{$G_9$}}};
\end{scope}

\end{tikzpicture}
\end{center}
\end{subfigure}

\end{center}
\caption{Obstrucciones mínimas para las gráficas monopolares.}
\label{obsts_cografics_monopolares}
\end{figure}


Como podemos observar, varias de las obstrucciones mínimas de las gráficas monopolares se asemejan a obstrucciones mínimas de las gráficas polares. Por ejemplo, las gráficas $G_1$, $G_3$, $G_4$ y $G_8$ son subgráficas de $H_1$, $H_2$, $H_3$ y $H_4$ respectivamente. Notemos que en los cuatro casos se obtiene una obstrucción mínima para las gráficas monopolares $G$ a partir de una obstrucción mínima para las gráficas polares $H$ al restar vértices de la componente conexa de $H$ que forma un $P_3$. En los últimos tres casos, el $P_3$ es reemplazado por un vértice aislado. Esto nos lleva a pensar en que se puede encontrar una relación entre el conjunto de obstrucciones mínimas de una clase y los conjuntos de obstrucciones mínimas de sus subclases.


\section{Obstrucciones mínimas para cográficas ($s$, 1)-polares}
    En \cite{Fernando}, Contreras-Mendoza y Hern\'andez-Cruz
exhiben el conjunto de obstrucciones mínimas de las
cográficas $(\infty, 1)$-polares. Este conjunto es
utilizado para describir cómo se puede obtener el conjunto
de obstrucciones mínimas de cualquiera de las clases de
cográficas $(s,1)$-polares dado un entero $s \ge 2$. En
nuestra investigación encontramos un resultado similar,
el Lema \ref{lema_1infM2}, donde describimos a las obstrucciones
m\'inimas para las gráficas que aceptan una partición en
un conjunto independiente y una gráfica multipartita completa.

A continuaci\'on reproducimos el resultado antes mencionado.

\begin{theorem}[\cite{Fernando}]
\label{thm:s,1-ess}
  Sea $G$ una cográfica. Entonces $G$ es $(\infty,1)$-polar
  si y sólo si no contiene alguna de las gráficas de la Figura
  \ref{obsts_cografics_esenciales_1spolares} como subgráfica
  inducida. Este conjunto es llamado el conjunto de obstrucciones
  esenciales.
\end{theorem}

\begin{figure}[ht!]
\begin{center}
\begin{tikzpicture}

\begin{scope}[xshift=0cm,scale=1]
\node [style=vertex] (1) at (0,0) {};
\node [style=vertex] (2) at (1,0) {};
\node [style=vertex] (3) at (0,1) {};
\node [style=vertex] (4) at (1,1) {};
\node [style=vertex] (5) at (0.5,2) {};
\foreach \i/\j in {1/2,3/4}
  \draw [style=edge] (\i) to (\j);
\node at (0.5,-0.75) {\parbox{0.3\linewidth}{\subcaption*{$G_1$}}};
\end{scope}

\begin{scope}[xshift=2cm,scale=1]
\node [style=vertex] (1) at (0,0) {};
\node [style=vertex] (2) at (1,0) {};
\node [style=vertex] (3) at (0,1) {};
\node [style=vertex] (4) at (1,1) {};
\node [style=vertex] (5) at (0,2) {};
\node [style=vertex] (6) at (1,2) {};
\foreach \i/\j in {1/2,1/3,2/4,3/4}
  \draw [style=edge] (\i) to (\j);
\node at (0.5,-0.75) {\parbox{0.3\linewidth}{\subcaption*{$G_2$}}};
\end{scope}

\begin{scope}[xshift=4cm,scale=1]
\node [style=vertex] (1) at (0,0) {};
\node [style=vertex] (2) at (1,0) {};
\node [style=vertex] (3) at (0,1) {};
\node [style=vertex] (4) at (1,1) {};
\node [style=vertex] (5) at (0,2) {};
\node [style=vertex] (6) at (1,2) {};
\foreach \i/\j in {1/3,2/4,3/5,4/6}
  \draw [style=edge] (\i) to (\j);
\node at (0.5,-0.75) {\parbox{0.3\linewidth}{\subcaption*{$G_3$}}};
\end{scope}

\begin{scope}[xshift=6cm,scale=1]
\node [style=vertex] (1) at (0.75,0) {};
\node [style=vertex] (2) at (0,0.75) {};
\node [style=vertex] (3) at (0.75,0.75) {};
\node [style=vertex] (4) at (1.5,0.75) {};
\node [style=vertex] (5) at (0.75,1.5) {};
\node [style=vertex] (6) at (0.75,2) {};
\foreach \i/\j in {1/2,1/3,1/4,2/3,2/5,3/4,4/5}
  \draw [style=edge] (\i) to (\j);
\node at (0.75,-0.75) {\parbox{0.3\linewidth}{\subcaption*{$G_4$}}};
\end{scope}

\end{tikzpicture}
\end{center}
\caption{Obstrucciones mínimas para las gráficas polares.}
\label{obsts_cografics_esenciales_1spolares}
\end{figure}

Haciendo uso de estas obstrucciones esenciales,
los autores caracterizan, para cualquier entero
$s$, con $s \ge 2$, a las obstrucciones m\'inimas
para la clase de cogr\'aficas $(s,1)$-polares.
Notemos que, adem\'as de las
obstrucciones esenciales, existen cuatro familias
espor\'adicas de obstrucciones m\'inimas, y se
plantea una regla recursiva para generar
obstrucciones m\'inimas para las cogr\'aficas
$(s,1)$-polares utilizando obstrucciones
m\'inimas no esenciales para las cogr\'aficas
$(t,1)$-polares, con $t < s$.

\begin{theorem}
\label{thm:s,1}
  Sea $G$ una cográfica y $s \ge 2$ un entero. Entonces
  $G$ es una obstrucción mínima de las cográficas
  $(s,1)$-polares si y sólo si es una de las siguientes
  gráficas:

  \begin{itemize}
    \item Una de las cuatro obstrucciones esenciales.

    \item $2K_{s+1}$.

    \item $K_2 + (\overline{K_2}\oplus K_s)$.

    \item $K_1 + (C_4 \oplus K_{s-1})$.

    \item $\overline{(s+1)K_2}$.

    \item El complemento de $G$ es inconexo con componentes
      $G_1, \dots, G_t$ tales que $t \leq s$, y cada $G_i$
      es el complemento de una obstrucción mínima no esencial
      de la clase de cográficas $(s_i, 1)$-polares con
      $\sum^{t}_{i=1}s_i = s-t+1$.
  \end{itemize}
\end{theorem}

%Si $s$ y $k$ son enteros positivos arbitrarios, determinar
%las obstrucciones m\'inimas para las cogr\'aficas
%$(s,k)$-polares parece ser un problema dif\'icil de
%resolver.

El resultado presentado en el Teorema
\ref{thm:s,1} es un ejemplo de c\'omo, al restringir
un problema a un subproblema bien definido, es posible
encontrar soluciones parciales al problema general.   En
particular, se obtuvo una caracterizaci\'on por obstrucciones
m\'inimas para una subclase infinita de las cogr\'aficas
$(s,k)$-polares.

M\'as a\'un, una interpretaci\'on de los Teoremas \ref{thm:s,1-ess}
y \ref{thm:s,1} es que resulta posible pensar a las gr\'aficas
$(\infty,1)$-polares como el l\'imite cuando $s$ tiende al
infinito de las gr\'aficas $(s,1)$-polares. Podemos observar que
las gráficas que son obstrucciones mínimas de las gráficas
$(s,1)$-polares para cualquier entero $s \ge 2$ también son
obstrucciones mínimas de las gráficas $(\infty,1)$-polares. En
nuestra investigación encontramos un resultado similar para las
gráficas $(\alpha,\beta)$-$M_2$\footnote{La clase
$(\alpha,\beta)$-$M_2$ es la clase constituida por todas las
gráficas que aceptan una partición en dos gráficas multipartitas
completas, una formada por a lo más $\alpha$ conjuntos estables
y la otra formada por a lo más $\beta$ conjuntos estables.}, ya
que las obstrucciones mínimas de la clase $(1,\infty)$-$M_2$ son
obstrucciones mínimas de la clase $(1,m)$-$M_2$, para cualquier
valor de $m$ con $m \ge 2$.  Con base en la idea anterior,
a partir de las obstrucciones mínimas de las clases
$(2,m)$-$M_2$, con $m \in \{ 3, \dots, 7 \}$ que generamos
computacionalmente, encontramos algunas obstrucciones mínimas
de la clase $(2,\infty)$-$M_2$. Aplicando este mismo proceso,
podemos encontrar obstrucciones mínimas para la clase $\alpha,
\infty$-$M_2$ dado un entero $\alpha \ge 3$.


\section{Obstrucciones mínimas para cográficas 2-polares}
    %%Por definir: k-polares
En \cite{Hell03}, Hell, Hern\'andez-Cruz y Linhares-Sales
exhiben el conjunto de obstrucciones mínimas de la clase
de las cográficas 2-polares. Para construir este conjunto
primero se presentan resultados preliminares sobre la
estructura de las obstrucciones mínimas de las cográficas
$(k,k)$-polares (llamadas simplemente cográficas $k$-polares)
para cualquier entero positivo $k$. Posteriormente se presenta
el complemento parcial, una operación que conserva la
$2$-polaridad, para construir las 24 obstrucciones m\'inimas
inconexas de las cogr\'aficas $2$-polares a partir de un
conjunto de cuatro obtrucciones m\'inimas.

El siguiente lema describe la estructura de las
obstrucciones m\'inimas para las cogr\'aficas
$k$-polares con el m\'aximo n\'umero posible de
componentes conexas.

\begin{lemma}
\label{lema_2polares_01}
Sean $l$ y $k$ enteros tales que $1 \le l \le k+1$. Salvo isomorfismo, hay exactamente una obstrucción mínima para las cográficas $k$-polares con $k+2$ componentes en total y $l$ componentes triviales. Esta obstrucción mínima es isomorfa a
$$lk_1+(k-l+1)k_2+k_{l,l}$$
\end{lemma}

Aplicando el Lema \ref{lema_2polares_01}, podemos encontrar
tres obstrucciones mínimas para las cográficas $2$-polares.
Éstas se muestran en la Figura \ref{obsts_2polares_01}.

\begin{figure}[ht!]
\begin{center}
\begin{tikzpicture}

\begin{scope}[xshift=0cm,scale=1]
\node [vertex] (1) at (0,0) {};
\node [vertex] (2) at (1,0) {};
\node [vertex] (3) at (0,0.5) {};
\node [vertex] (4) at (1,0.5) {};
\node [vertex] (5) at (0,1) {};
\node [vertex] (6) at (1,1) {};
\node [vertex] (7) at (0.5,1.5) {};
\foreach \i/\j in {1/2,3/4,5/6}
  \draw [edge] (\i) to (\j);
\node at(0.5,-1) {\parbox{0.3\linewidth}{\subcaption*{$F_{1}$}}};
\end{scope}

\begin{scope}[xshift=2.5cm,scale=1]
\node [vertex] (1) at (0,0) {};
\node [vertex] (2) at (1,0) {};
\node [vertex] (3) at (0,0.5) {};
\node [vertex] (4) at (1,0.5) {};
\node [vertex] (5) at (0,1) {};
\node [vertex] (6) at (1,1) {};
\node [vertex] (7) at (0.25,1.5) {};
\node [vertex] (8) at (0.75,1.5) {};
\foreach \i/\j in {1/2,1/4,2/3,3/4,5/6}
  \draw [edge] (\i) to (\j);
\node at(0.5,-1) {\parbox{0.3\linewidth}{\subcaption*{$F_{13}$}}};
\end{scope}

\begin{scope}[xshift=5cm,scale=1]
\node [vertex] (1) at (0,0) {};
\node [vertex] (2) at (1,0) {};
\node [vertex] (3) at (0,0.5) {};
\node [vertex] (4) at (1,0.5) {};
\node [vertex] (5) at (0,1) {};
\node [vertex] (6) at (1,1) {};
\node [vertex] (7) at (0,1.5) {};
\node [vertex] (8) at (0.5,1.5) {};
\node [vertex] (9) at (1,1.5) {};
\foreach \i/\j in {1/2,1/4,2/3,3/4,3/6,4/5,5/6}
  \draw [edge] (\i) to (\j);
\node at(0.5,-1) {\parbox{0.3\linewidth}{\subcaption*{$F_{21}$}}};
\end{scope}

\end{tikzpicture}
\end{center}
\caption{Obstrucciones mínimas para las cográficas 2-polaes obtenidas con el Lema \ref{lema_2polares_01}.}
\label{obsts_2polares_01}
\end{figure}

Ahora introducimos una operaci\'on que preserva la
$2$-polaridad y la propiedad de ser cogr\'afica, por lo
que resulta bastante \'util en el estudio de las
obstrucciones m\'inimas para las cogr\'aficas $2$-polares.
Sea $H$ una gráfica, un \textbf{\emph{complemento parcial}}
de $H$ es una gráfica obtenida de $H$ al dividir a sus
componentes conexas en dos gráficas $H'$ y $H''$, y tomando
de forma separada el complemento de cada una.

Tomando todos los posibles complementos parciales de las
gráficas $F_1$, $F_{13}$ y $F_{21}$ (Figura
\ref{obsts_2polares_01}), encontramos tres familias de
obstrucciones mínimas; \'estas se muestran en las Figuras
\ref{obsts_2polares_02}, \ref{obsts_2polares_03} y
\ref{obsts_2polares_04} respectivamente. Podemos encontrar
una cuarta familia de obstrucciones mínimas tomar todos los
posibles complementos parciales de la gráfica $F_7$ (Figura
\ref{obsts_2polares_05}) que también es una obstrucción mínima
de las gráficas 2-polares. La gráfica $F_7$ se puede construir
de forma natural agregando un $K_2$ a una de las obstrucciones
m\'inimas para $(2,1)$-polaridad en cogr\'aficas.

\begin{figure}[ht!]
\begin{subfigure}{\textwidth}
\begin{center}
\begin{tikzpicture}

\begin{scope}[xshift=0cm,scale=1]
\node [vertex] (1) at (0,0) {};
\node [vertex] (2) at (1,0) {};
\node [vertex] (3) at (0,0.5) {};
\node [vertex] (4) at (1,0.5) {};
\node [vertex] (5) at (0,1) {};
\node [vertex] (6) at (1,1) {};
\node [vertex] (7) at (0.5,1.5) {};
\foreach \i/\j in {1/2,3/4,5/6}
  \draw [edge] (\i) to (\j);
\node at(0.5,-1) {\parbox{0.3\linewidth}{\subcaption*{$F_{1}$}}};
\end{scope}

\begin{scope}[xshift=2.5cm,scale=1]
\node [vertex] (1) at (0,0) {};
\node [vertex] (2) at (0.5,0) {};
\node [vertex] (3) at (1,0) {};
\node [vertex] (4) at (0.5,0.5) {};
\node [vertex] (5) at (0,1) {};
\node [vertex] (6) at (1,1) {};
\node [vertex] (7) at (0.5,1.5) {};
\foreach \i/\j in {1/2,2/3,4/5,4/6,5/7,6/7}
  \draw [edge] (\i) to (\j);
\node at(0.5,-1) {\parbox{0.3\linewidth}{\subcaption*{$F_{2}$}}};
\end{scope}

\begin{scope}[xshift=5cm,scale=1]
\node [vertex] (1) at (0.5,0) {};
\node [vertex] (2) at (0,0.5) {};
\node [vertex] (3) at (0.5,0.5) {};
\node [vertex] (4) at (1,0.5) {};
\node [vertex] (5) at (0.5,1) {};
\node [vertex] (6) at (0,1.5) {};
\node [vertex] (7) at (1,1.5) {};
\foreach \i/\j in {2/5,2/6,3/5,4/5,4/7,5/6,5/7}
  \draw [edge] (\i) to (\j);
\node at(0.5,-1) {\parbox{0.3\linewidth}{\subcaption*{$F_{3}$}}};
\end{scope}

\begin{scope}[xshift=7.5cm,scale=1]
\node [vertex] (1) at (0,0) {};
\node [vertex] (2) at (1,0) {};
\node [vertex] (3) at (0.5,0.5) {};
\node [vertex] (4) at (0,1) {};
\node [vertex] (5) at (0.5,1) {};
\node [vertex] (6) at (1,1) {};
\node [vertex] (7) at (0.5,1.5) {};
\foreach \i/\j in {3/4,3/5,3/6,4/5,4/7,5/6,5/7,6/7}
  \draw [edge] (\i) to (\j);
\node at(0.5,-1) {\parbox{0.3\linewidth}{\subcaption*{$F_{4}$}}};
\end{scope}

\begin{scope}[xshift=10.25cm,scale=1]
\node [vertex] (1) at (0,0) {};
\node [vertex] (2) at (1,0) {};
\node [vertex] (3) at (-0.25,0.5) {};
\node [vertex] (4) at (1.25,0.5) {};
\node [vertex] (5) at (0,1) {};
\node [vertex] (6) at (1,1) {};
\node [vertex] (7) at (0.5,1.5) {};
\foreach \i/\j in {1/3,1/4,1/5,1/6,2/3,2/4,2/5,2/6,3/5,3/6,4/5,4/6}
  \draw [edge] (\i) to (\j);
\node at(0.5,-1) {\parbox{0.3\linewidth}{\subcaption*{$F_{5}$}}};
\end{scope}

\end{tikzpicture}
\end{center}
\end{subfigure}

\caption{Obstrucciones mínimas de las cográficas $2$-polares con 7 vértices.}
\label{obsts_2polares_02}
\end{figure}


\begin{figure}[ht!]
\begin{subfigure}{\textwidth}
\begin{center}
\begin{tikzpicture}

\begin{scope}[xshift=0cm,scale=1]
\node [vertex] (1) at (0,0) {};
\node [vertex] (2) at (1,0) {};
\node [vertex] (3) at (0,0.5) {};
\node [vertex] (4) at (1,0.5) {};
\node [vertex] (5) at (0,1) {};
\node [vertex] (6) at (1,1) {};
\node [vertex] (7) at (0.25,1.5) {};
\node [vertex] (8) at (0.75,1.5) {};
\foreach \i/\j in {1/2,1/4,2/3,3/4,5/6}
  \draw [edge] (\i) to (\j);
\node at(0.5,-1) {\parbox{0.3\linewidth}{\subcaption*{$F_{13}$}}};
\end{scope}

\begin{scope}[xshift=2.5cm,scale=1]
\node [vertex] (1) at (0,0) {};
\node [vertex] (2) at (1,0) {};
\node [vertex] (3) at (0,0.5) {};
\node [vertex] (4) at (1,0.5) {};
\node [vertex] (5) at (0,1) {};
\node [vertex] (6) at (1,1) {};
\node [vertex] (7) at (0,1.5) {};
\node [vertex] (8) at (1,1.5) {};
\foreach \i/\j in {1/2,1/3,1/4,2/4,3/4,5/6,7/8}
  \draw [edge] (\i) to (\j);
\node at(0.5,-1) {\parbox{0.3\linewidth}{\subcaption*{$F_{14}$}}};
\end{scope}

\begin{scope}[xshift=5cm,scale=1]
\node [vertex] (1) at (0,0) {};
\node [vertex] (2) at (1,0) {};
\node [vertex] (3) at (0,0.5) {};
\node [vertex] (4) at (1,0.5) {};
\node [vertex] (5) at (0,1) {};
\node [vertex] (6) at (1,1) {};
\node [vertex] (7) at (0,1.5) {};
\node [vertex] (8) at (1,1.5) {};
\foreach \i/\j in {1/2,1/3,1/4,2/3,2/4,3/5,3/6,4/5,4/6,5/6,7/8}
  \draw [edge] (\i) to (\j);
\node at(0.5,-1) {\parbox{0.3\linewidth}{\subcaption*{$F_{15}$}}};
\end{scope}

\begin{scope}[xshift=7.5cm,scale=1]
\node [vertex] (1) at (0,0) {};
\node [vertex] (2) at (0,0.5) {};
\node [vertex] (3) at (0.5,0) {};
\node [vertex] (4) at (1,0.5) {};
\node [vertex] (5) at (0.5,1) {};
\node [vertex] (6) at (0,1.5) {};
\node [vertex] (7) at (1,1.5) {};
\node [vertex] (8) at (1,0) {};
\foreach \i/\j in {2/5,2/6,4/5,4/7,5/6,5/7,1/3,3/8}
  \draw [edge] (\i) to (\j);
\node at(0.5,-1) {\parbox{0.3\linewidth}{\subcaption*{$F_{16}$}}};
\end{scope}

\begin{scope}[xshift=10cm,scale=1]
\node [vertex] (1) at (0,0) {};
\node [vertex] (2) at (1,0) {};
\node [vertex] (3) at (0,0.5) {};
\node [vertex] (4) at (1,0.5) {};
\node [vertex] (5) at (0,1) {};
\node [vertex] (6) at (1,1) {};
\node [vertex] (7) at (0,1.5) {};
\node [vertex] (8) at (1,1.5) {};
\foreach \i/\j in {1/2,1/3,1/4,2/3,2/4,3/4,3/5,3/6,4/5,4/6,5/6} \draw [edge] (\i) to (\j);
\node at(0.5,-1) {\parbox{0.3\linewidth}{\subcaption*{$F_{17}$}}};
\end{scope}

\end{tikzpicture}
\end{center}
\end{subfigure}

\begin{subfigure}{\textwidth}
\begin{center}
\begin{tikzpicture}

\begin{scope}[xshift=0cm,scale=1]
\node [vertex] (1) at (0.75,0) {};
\node [vertex] (2) at (0.25,0.5) {};
\node [vertex] (3) at (1.25,0.5) {};
\node [vertex] (4) at (0,1) {};
\node [vertex] (5) at (0.75,1) {};
\node [vertex] (6) at (1.5,1) {};
\node [vertex] (7) at (0.25,1.5) {};
\node [vertex] (8) at (1.25,1.5) {};
\foreach \i/\j in {2/3,2/4,2/5,2/6,3/4,3/5,3/6,4/5,4/7,4/8,5/6,5/7,5/8,6/7,6/8,7/8}
  \draw [edge] (\i) to (\j);
\node at(0.75,-1) {\parbox{0.3\linewidth}{\subcaption*{$F_{18}$}}};
\end{scope}

\begin{scope}[xshift=3cm,scale=1]
\node [vertex] (1) at (0,0) {};
\node [vertex] (2) at (1,0) {};
\node [vertex] (3) at (0,0.5) {};
\node [vertex] (4) at (1,0.5) {};
\node [vertex] (5) at (0,1) {};
\node [vertex] (6) at (1,1) {};
\node [vertex] (7) at (0,1.5) {};
\node [vertex] (8) at (1,1.5) {};
\foreach \i/\j in {1/2,1/3,1/4,2/3,2/4,3/5,3/6,4/5,4/6}
  \draw [edge] (\i) to (\j);
\node at(0.5,-1) {\parbox{0.3\linewidth}{\subcaption*{$F_{19}$}}};
\end{scope}

\begin{scope}[xshift=5.5cm,scale=1]
\node [vertex] (1) at (0,0) {};
\node [vertex] (2) at (0.75,0) {};
\node [vertex] (3) at (0,0.75) {};
\node [vertex] (4) at (0.75,0.75) {};
\node [vertex] (5) at (1.5,0.75) {};
\node [vertex] (6) at (0.5,1) {};
\node [vertex] (7) at (1,1) {};
\node [vertex] (8) at (0.75,1.5) {};
\foreach \i/\j in {2/3,2/4,2/5,3/4,3/8,4/5,4/6,4/7,4/8,5/8}
  \draw [edge] (\i) to (\j);
\node at(0.75,-1) {\parbox{0.3\linewidth}{\subcaption*{$F_{20}$}}};
\end{scope}

\end{tikzpicture}
\end{center}
\end{subfigure}

\caption{Familia A de obstrucciones mínimas de las cográficas $2$-polares con 8 vértices.}
\label{obsts_2polares_03}
\end{figure}

\begin{figure}[ht!]
\begin{subfigure}{\textwidth}
\begin{center}
\begin{tikzpicture}

\begin{scope}[xshift=0cm,scale=1]
\node [vertex] (1) at (0,0) {};
\node [vertex] (2) at (1,0) {};
\node [vertex] (3) at (0,0.5) {};
\node [vertex] (4) at (1,0.5) {};
\node [vertex] (5) at (0,1) {};
\node [vertex] (6) at (1,1) {};
\node [vertex] (7) at (0,1.5) {};
\node [vertex] (8) at (0.5,1.5) {};
\node [vertex] (9) at (1,1.5) {};
\foreach \i/\j in {1/2,1/4,2/3,3/4,3/6,4/5,5/6}
  \draw [edge] (\i) to (\j);
\node at(0.5,-1) {\parbox{0.3\linewidth}{\subcaption*{$F_{21}$}}};
\end{scope}

\begin{scope}[xshift=2.5cm,scale=1]
\node [vertex] (1) at (0,0) {};
\node [vertex] (2) at (0.5,0.25) {};
\node [vertex] (3) at (0,0.5) {};
\node [vertex] (4) at (1,0.5) {};
\node [vertex] (5) at (0.5,0.75) {};
\node [vertex] (6) at (1,1) {};
\node [vertex] (7) at (0,1) {};
\node [vertex] (8) at (0.5,1.25) {};
\node [vertex] (9) at (0,1.5) {};
\foreach \i/\j in {1/2,1/3,2/3,4/5,4/6,5/6,7/8,7/9,8/9}
  \draw [edge] (\i) to (\j);
\node at(0.5,-1) {\parbox{0.3\linewidth}{\subcaption*{$F_{22}$}}};
\end{scope}

\begin{scope}[xshift=5cm,scale=1]
\node [vertex] (1) at (0.5,0) {};
\node [vertex] (2) at (1.5,0) {};
\node [vertex] (3) at (0.5,0.5) {};
\node [vertex] (4) at (1.5,0.5) {};
\node [vertex] (5) at (0,1) {};
\node [vertex] (6) at (1,1) {};
\node [vertex] (7) at (2,1) {};
\node [vertex] (8) at (0.5,1.5) {};
\node [vertex] (9) at (1.5,1.5) {};
\foreach \i/\j in {1/2,3/5,3/6,3/8,4/6,4/7,4/9,5/6,5/8,6/7,6/8,6/9,7/9}
  \draw [edge] (\i) to (\j);
\node at(0.75,-1) {\parbox{0.3\linewidth}{\subcaption*{$F_{23}$}}};
\end{scope}

\begin{scope}[xshift=8.5cm,scale=1]
\node [vertex] (1) at (0.75,0) {};
\node [vertex] (2) at (0,0.5) {};
\node [vertex] (3) at (0.75,0.5) {};
\node [vertex] (4) at (1.5,0.5) {};
\node [vertex] (5) at (0.25,1) {};
\node [vertex] (6) at (1.25,1) {};
\node [vertex] (7) at (0,1.5) {};
\node [vertex] (8) at (0.75,1.5) {};
\node [vertex] (9) at (1.5,1.5) {};
\foreach \i/\j in {2/3,2/5,2/6,3/4,3/5,3/6,4/5,4/6,5/6,5/7,5/8,5/9,6/7,6/8,6/9,7/8,8/9}
  \draw [edge] (\i) to (\j);
\node at(0.75,-1) {\parbox{0.3\linewidth}{\subcaption*{$F_{24}$}}};
\end{scope}

\end{tikzpicture}
\end{center}
\end{subfigure}

\caption{Obstrucciones mínimas de las cográficas $2$-polares con 9 vértices.}
\label{obsts_2polares_04}
\end{figure}


\begin{figure}[ht!]
\begin{subfigure}{\textwidth}
\begin{center}
\begin{tikzpicture}

\begin{scope}[xshift=0cm,scale=1]
\node [vertex] (1) at (0.25,0) {};
\node [vertex] (2) at (0.75,0) {};
\node [vertex] (3) at (0,0.75) {};
\node [vertex] (4) at (0.5,0.75) {};
\node [vertex] (5) at (1,0.75) {};
\node [vertex] (6) at (0,1.5) {};
\node [vertex] (7) at (0.5,1.5) {};
\node [vertex] (8) at (1,1.5) {};
\foreach \i/\j in {1/2,3/4,4/5,6/7,7/8}
  \draw [edge] (\i) to (\j);
\node at(0.5,-1) {\parbox{0.3\linewidth}{\subcaption*{$F_{6}$}}};
\end{scope}

\begin{scope}[xshift=2.5cm,scale=1]
\node [vertex] (1) at (0,0) {};
\node [vertex] (8) at (0.5,0) {};
\node [vertex] (2) at (1,0) {};
\node [vertex] (3) at (0.5,0.5) {};
\node [vertex] (4) at (0,1) {};
\node [vertex] (5) at (0.5,1) {};
\node [vertex] (6) at (1,1) {};
\node [vertex] (7) at (0.5,1.5) {};
\foreach \i/\j in {3/4,3/5,3/6,4/5,4/7,5/6,6/7,8/2}
  \draw [edge] (\i) to (\j);
\node at(0.5,-1) {\parbox{0.3\linewidth}{\subcaption*{$F_{7}$}}};
\end{scope}

\begin{scope}[xshift=5cm,scale=1]
\node [vertex] (1) at (0,0) {};
\node [vertex] (2) at (1,0) {};
\node [vertex] (3) at (0,0.5) {};
\node [vertex] (4) at (1,0.5) {};
\node [vertex] (5) at (0,1) {};
\node [vertex] (6) at (1,1) {};
\node [vertex] (7) at (0,1.5) {};
\node [vertex] (8) at (1,1.5) {};
\foreach \i/\j in {1/2,1/3,1/4,2/3,2/4,3/4,3/6,4/5,5/6} \draw [edge] (\i) to (\j);
\node at(0.5,-1) {\parbox{0.3\linewidth}{\subcaption*{$F_{8}$}}};
\end{scope}

\begin{scope}[xshift=7.5cm,scale=1]
\node [vertex] (1) at (0.5,0) {};
\node [vertex] (2) at (0,0.5) {};
\node [vertex] (3) at (1,0.5) {};
\node [vertex] (4) at (0,1) {};
\node [vertex] (5) at (1,1) {};
\node [vertex] (6) at (0,1.5) {};
\node [vertex] (7) at (0.5,1.5) {};
\node [vertex] (8) at (1,1.5) {};
\foreach \i/\j in {2/3,2/4,2/5,3/4,3/5,4/6,4/7,4/8,5/6,5/7,5/8,6/7,7/8} \draw [edge] (\i) to (\j);
\node at(0.5,-1) {\parbox{0.3\linewidth}{\subcaption*{$F_{9}$}}};
\end{scope}

\begin{scope}[xshift=10cm,scale=1]
\node [vertex] (1) at (0.5,0) {};
\node [vertex] (2) at (1,0) {};
\node [vertex] (3) at (0.5,0.5) {};
\node [vertex] (4) at (1.5,0.5) {};
\node [vertex] (5) at (0,1) {};
\node [vertex] (6) at (1,1) {};
\node [vertex] (7) at (0.5,1.5) {};
\node [vertex] (8) at (1.5,1.5) {};
\foreach \i/\j in {1/2,3/5,3/6,4/6,4/8,5/6,5/7,6/7,6/8} \draw [edge] (\i) to (\j);
\node at(0.5,-1) {\parbox{0.3\linewidth}{\subcaption*{$F_{10}$}}};
\end{scope}

\end{tikzpicture}
\end{center}
\end{subfigure}

\begin{subfigure}{\textwidth}
\begin{center}
\begin{tikzpicture}

\begin{scope}[xshift=5cm,scale=1]
\node [vertex] (1) at (1,0) {};
\node [vertex] (3) at (0.5,0.5) {};
\node [vertex] (4) at (1.5,0.5) {};
\node [vertex] (5) at (0,1) {};
\node [vertex] (6) at (1,1) {};
\node [vertex] (7) at (2,1) {};
\node [vertex] (8) at (0.5,1.5) {};
\node [vertex] (9) at (1.5,1.5) {};
\foreach \i/\j in {3/5,3/6,4/6,4/7,5/6,5/8,6/7,6/8,6/9,7/9}
  \draw [edge] (\i) to (\j);
\node at(0.75,-1) {\parbox{0.3\linewidth}{\subcaption*{$F_{11}$}}};
\end{scope}

\begin{scope}[xshift=8.5cm,scale=1]
\node [vertex] (1) at (0,0) {};
\node [vertex] (2) at (1.25,0) {};
\node [vertex] (3) at (2,0.25) {};
\node [vertex] (4) at (0,0.75) {};
\node [vertex] (5) at (0.75,0.75) {};
\node [vertex] (6) at (2,1.25) {};
\node [vertex] (7) at (0,1.5) {};
\node [vertex] (8) at (1.25,1.5) {};
\foreach \i/\j in {2/3,2/4,2/5,2/6,3/5,3/8,3/6,4/5,4/8,5/7,5/8,5/6,8/6}
  \draw [edge] (\i) to (\j);
\node at(0.75,-1) {\parbox{0.3\linewidth}{\subcaption*{$F_{12}$}}};
\end{scope}

\end{tikzpicture}
\end{center}
\end{subfigure}

\caption{Familia B de obstrucciones mínimas de las cográficas $2$-polares con 8 vértices.}
\label{obsts_2polares_05}
\end{figure}

En este artículo podemos observar un ejemplo en el que, dada
una clase hereditaria de cográficas $C$ y una operación
$o$ tales que $C$ es cerrada bajo $o$, es posible encontrar
familias de obstrucciones mínimas para $C$ a partir de
obstrucciones mínimas de $C$ ya conocidas aplic\'andoles
la operación $o$. Las clases de cográficas que estudiamos en
nuestra investigación son cerradas bajo la unión completa y,
de manera parecida a lo que sucede en \cite{Hell03},
encontramos reglas para generar obstrucciones mínimas para
una clase hereditaria $D$ al aplicar la unión completa a
obstrucciones mínimas de subclases de $D$.

% Quizá valdría la pena ser más específicos en la última
% oración de este párrafo.


\section{Un algoritmo lineal para el reconocimiento de cográficas}
    En \cite{Corneil02}, Corneil, Perl y Stewart presentan un algoritmo
que, dada una gráfica, determina si ésta es una cográfica y, si lo
es, devuelve su coárbol. Éste es un algoritmo incremental, se
construye un coárbol agregando los vértices de la gráfica recibida
como entrada uno a uno. Para agregar un nuevo vértice $x$, de la
gráfica de entrada $G$, al coárbol $T$ que se va construyendo, el
algoritmo ejecuta primero una subrutina que se encarga de marcar los
nodos de $T$ empezando por sus hojas. Una hoja de $T$ se marca sólo
si ésta representa un vértice de $G$ adyacente a $x$. La subrutina
continúa marcando nodos de $G$ subiendo hacia la raíz. En nuestra
investigación presentamos un algoritmo similar, el Algoritmo
\ref{alg_subcoarbol}, que ayuda a determinar si una cográfica
$H$ es subcográfica de otra cográfica $I$ marcando los vértices de
un coárbol binario\footnote{Un coárbol binario es un coárbol en el
que cada nodo tiene necesariamente dos hijos y pueden haber nodos
adyacentes con la misma etiqueta.} de $I$ partiendo desde las hojas
y subiendo hacia la raíz.

A continuación se reproducen los resultados de \cite{ Corneil02},
empezando por el algoritmo de marcado antes mencionado.

\begin{algorithm}[!htbp]
\caption{Marcar}
\label{alg_mark_corneil}
\DontPrintSemicolon % Some LaTeX compilers require you to use \dontprintsemicolon instead
\KwIn{$T$, un coárbol con raíz $R$ cuyas hojas son vértices de una gráfica $G$; $x$, el vértice de $G$ que se busca agregar a $T$}
\KwOut{Se marcan y se desmarcan algunos nodos de $T$ }

$D\gets \{\}$ \tcp*[h]{El conjunto de los nodos que han sido marcados y desmarcados} \; 

Marcar todas las hojas de $T$ que sean adyacentes a $x$\;

\ForEach{\emph{nodo marcado $u$ de $T$ tal que todos sus hijos están en el conjunto $D$}}{
    Desmarcar a $u$\;
    Agregar $u$ a $D$\;

    \If{$u \neq R$}{
        \emph{Marcar al padre de $u$}\;
    }
}

\end{algorithm}

Sean $T$ un coárbol, $x$ un vértice y $M$ el conjunto de nodos
de $T$ que se encuentran marcados al terminar la ejecución del
Algoritmo \ref{alg_mark_corneil} al recibir como entrada a $T$
y a $x$. El Teorema \ref{teo_teo1_corneil} puede ser utilizado
para determinar si la gráfica que se obtiene al agregar $x$ a
la gráfica representada por $T$ es una cográfica.
Las siguientes definiciones son necesarias para presentar dicho
teorema. Sea $\alpha$ un nodo de $M$ de profundidad máxima en $T$
y sea $\beta$ un nodo en $M-\{a\}$ de profundidad máxima en $T$.
Decimos que un nodo $\gamma$ de $T$ con etiqueta 1 está
correctamente marcado si y sólo si todos sus hijos, excepto uno,
fueron marcados y desmarcados. Un camino legítimamente alternante
en un coárbol marcado es un camino alternante de nodos correctamente
marcados y nodos sin marcar, con etiqueta 0, cuyos extremos son nodos
con etiqueta 1.


\begin{theorem}
    \label{teo_teo1_corneil}
    Si $G$ es una cográfica con árbol $T$, entonces $G+x$ es una cográfica si y sólo si:
    \begin{itemize}
        \item $M$ es vacío o
        \item se cumplen las siguientes dos condiciones
        \begin{itemize}
            \item $M-\{\alpha\}$ consiste exactamente de los nodos
              con etiqueta 1 de un camino legítimamente alternante
              que termina en $R$.

            \item $\alpha$ es un nodo con etiqueta 0 cuyo padre es
              $\beta$ o $\alpha$ es un nodo con etiqueta 1 cuyo
              abuelo, si existe, es $\beta$.
        \end{itemize}
    \end{itemize}
\end{theorem}

El Algoritmo \ref{alg_principal_corneil}, que constituye el resultado
principal del artículo, utiliza el Algoritmo \ref{alg_mark_corneil} y
el Teorema \ref{teo_teo1_corneil} para determinar si una gráfica es
una cográfica, y construir su coárbol en caso de que lo sea.

\begin{algorithm}[!htbp]
\caption{AlgoritmoDeReconocimiento}
\label{alg_principal_corneil}
\DontPrintSemicolon % Some LaTeX compilers require you to use \dontprintsemicolon instead
\KwIn{$G$, una gráfica cuyo conunto de vértices es $V=\{v_1, v_2, \dots, v_n\}$}
\KwOut{El coárbol de $G$ si $G$ es una cográfica, $null$ en el caso contrario}

Crear el coárbol $T$ cuya raíz $R$ tiene etiqueta 1 si $v_1$ y $v_2$ son adyacentes o etiqueta 0 en el caso contrario\;

\ForEach{x \emph{\textbf{en}} $V(G)-\{v_1,v_2\}$}{
    Marcar($T$, $x$)\;
    \If{Todos los nodos de $T$ fueron marcados y desmarcados}{
        \If{$R$ tiene etiqueta 1}{
            Agregar a $x$ como hijo de $R$\;
        }
        \Else{
            $R\gets$ un nuevo nodo con etiqueta 1 cuyos hijos sean $x$ y $R$\;
        }
    }
    \ElseIf{Ninguno de los nodos de $T$ fue marcado}{
        \If{$R$ tiene etiqueta 0}{
            Agregar a $x$ como hijo de $R$\;
        }
        \Else{
            $R\gets$ un nuevo nodo con etiqueta 0 cuyos hijos sean $x$ y $R$\;
        }
    }
    \ElseIf{La gráfica representada por $T$ agregando $x$ es una cográfica}{
        $u\gets$ el nodo marcado más profundo de $T$\;
        Encontrar en qué nodo del coárbol con raíz $u$ se debe insertar $x$\;
    }
    \Else{
        \Return $null$\;
    }
	Desmarcar todos los nodos de $T$\;
}

\end{algorithm}

%Cierre
El Algoritmo \ref{alg_mark_corneil} consolida un ejemplo
de un algoritmo que funciona marcando los nodos de un coárbol
empezando por las hojas y dirigiéndose hacia la raíz\footnote{En
ingl\'es un {\em bottom-up} algorithm.}. Por otro lado, el Algoritmo
\ref{alg_mark_corneil} es un ejemplo de un algoritmo que, al ser
ejecutado, no sólo resuelve un problema de decisión, sino que
proporciona una prueba (o {\em certificado}) de que su respuesta
es correcta. En este caso, la prueba es el coárbol de la
gráfica recibida como entrada. Sin embargo, cuando la gr\'afica
que se recibe como entrada no es una cogr\'afica, el Algoritmo
\ref{alg_mark_corneil} s\'olo devuelve {\em null}. Si desconfi\'aramos
de una implementaci\'on de este algoritmo, y quisi\'eramos comprobar
que la salida es correcta, podr\'iamos utilizar el co\'arbol en un
caso, pero en el otro, no tendr\'iamos informaci\'on adicional.

En nuestra investigación, el Algoritmo \ref{alg_cert_m2} es un
algoritmo certificador, es decir, devuelve certificados para
verificar, de manera eficiente, que la salida del algoritmo es
correcta, en cualquier caso. Nuestro algoritmo no sólo es capaz de
determinar si una cográfica $G$ pertenece a la clase $M_2$, sino que
devuelve una coloración de las hojas de su coárbol $T$ tal que si
$G \in M_2$, las hojas de $T$ tienen uno de dos colores y las hojas
del mismo color inducen una gráfica multipartita completa en $G$
(un {\em s\'i-certificado}). En el caso contrario, algunas de las
hojas de $T$ tendrán un color distintivo que indica que esos
vértices inducen una obstrucción mínima de $M_2$ en $G$ (un
{\em no-certificado}).


\label{cap3}
\chapter{Resultados}\label{cap4}
En este capítulo presentamos el producto de nuestra investigación. La primera
sección del capítulo tiene como resultado principal un algoritmo capaz de
determinar si una cográfica pertenece a una clase hereditaria de gráficas fija
en tiempo lineal. A partir de la segunda sección abordamos el problema
principal de nuestra tesis, determinar si una cográfica acepta una partición en
un número dado de gráficas multipartitas completas. Comenzamos realizando un
estudio detallado de la clase de cográficas que aceptan una partición en dos
gráficas multipartitas completas, a la que llamamos $M_2$. En la segunda
sección del capítulo caracterizamos a la clase $M_2$ a través de su conjunto de
obstrucciones mínimas, proporcionamos un algoritmo para reconocer a sus
elementos (utilizando el resultado principal de la primera sección) y
presentamos un algoritmo certificador que no sólo es capaz de reconocer si una
cográfica $G$ pertenece a  $M_2$, sino que encuentra una partición de $G$ en
dos gráficas multipartitas completas o una obstrucción mínima de la clase como
subgráfica inducida de $G$. En la tercera sección estudiamos a las clases
$(\alpha, \beta)-M_2$, subclases de $M_2$ cuyos elementos aceptan una partición
en dos gráficas multipartitas completas de tamaños restringidos. El resultado
principal de esta sección es un algoritmo para encontrar obstrucciones mínimas
para cualquier clase $(\alpha, \beta)-M_2$. La cuarta y última sección del
capítulo da un paso en la generalización de los resultados de las secciones
anteriores. En éste se proporcionan algunas familias de obstrucciones mínimas
para caracterizar a la clase de cográficas que aceptan una partición en $i$
gráficas multipartitas completas dado un entero $i \geq 2$.

\section{Términos y algoritmos generales}
    En esta sección se presenta un conjunto de conceptos y algoritmos
    \'utiles para cogr\'aficas en general. El resultado
    principal de la sección es un algoritmo de tiempo lineal capaz de
    determinar si una cográfica representada por su coárbol pertenece a una
    clase hereditaria fija de cográficas caracterizada por su conjunto de
    obstrucciones mínimas.

    \subsection{Coárbol binario}
        Tomando como base el concepto de coárbol, podemos imaginar otra estructura de
tipo árbol para la representación de las cográficas en la que cada nodo tenga a
lo más un número $k$ de hijos. Esta limitante resulta útil para formular
algoritmos rápidos en cográficas. El menor valor que puede tomar $k$ es 2,
lo que resulta en una representaci\'on de cualquier cogr\'afica
mediante un co\'arbol binario.   Esta representaci\'on es la
que utilizaremos principalmente en el presente cap\'itulo.

Sean $G=(V,E)$ una cográfica, $C = \{c_1, c_2, \dots, c_n$\} el conjunto de
las componentes conexas de $G$, $D = \{d_1, d_2, \dots, d_m\}$ el conjunto
de las componentes conexas de $\overline{G}$, $(C_1, C_2)$ una partición en
dos partes de $C$ y $(D_1, D_2)$ una partición en dos partes de $D$. Decimos
que el árbol binario arraigado etiquetado, $(T,r)$, es un
\textbf{\emph{coárbol binario}} de $G$ si se puede construir de la siguiente
manera: Si $G$ consta de un sólo vértice, entonces $T$ sólo contiene a $r$,
que es igual al único vértice de $G$.
De lo contrario, si $G$ es conexa, entonces $r$ tiene la etiqueta $1$, uno
de los hijos de $r$ es el coárbol binario de $G-D_1$ y el otro es el coárbol
binario de $G-D_2$. Y finalmente, si $G$ es inconexa, entonces $r$ tiene la
etiqueta $0$, uno de sus hijos es el coárbol binario de $G-C_1$ y el otro el
coárbol binario de $G-C_2$.

Claramente, una cográfica puede ser representada por más de un coárbol binario
diferente como se muestra en la Figura \ref{fig_coar_bin01}. Sin embargo, la
propiedad de que dos vértices son adyacentes si y sólo si su ancestro común más
profundo tiene la etiqueta 1 se mantiene.

\begin{figure}[ht!]
\begin{center}
\begin{tikzpicture}

\begin{scope}[xshift=0cm,scale=1]

\node [vertex] (1) at (0,0) {};
\node [vertex] (2) at (1,0) {};
\node [vertex] (3) at (0,1) {};
\node [vertex] (4) at (1,1) {};
\foreach \i/\j in {1/2,1/3,1/4,2/3,2/4,3/4}
  \draw [edge] (\i) to (\j);
\node [below of=1,xshift=.5cm] {\parbox{0.3\linewidth}{\subcaption{}}};

\end{scope}

\begin{scope}[xshift=3.5cm,scale=1]

\node [cotreenode] (1) at (1,1) {1};
\node [cotreenode] (2) at (0,0) {1};
\node [cotreenode] (3) at (2,0) {1};
\node [vertex] (4) at (-0.5,-1) {};
\node [vertex] (5) at (0.5,-1) {};
\node [vertex] (6) at (1.5,-1) {};
\node [vertex] (7) at (2.5,-1) {};
\foreach \i/\j in {1/2,1/3,2/4,2/5,2/4,3/6,3/7}
  \draw [edge] (\i) to (\j);
\node [below of=5,xshift=.5cm] {\parbox{0.3\linewidth}{\subcaption{}}};

\end{scope}

\begin{scope}[xshift=7.5cm,scale=1]

\node [cotreenode] (1) at (1,1) {1};
\node [vertex] (2) at (0,0) {};
\node [cotreenode] (3) at (2,0) {1};
\node [vertex] (6) at (1,-1) {};
\node [cotreenode] (7) at (3,-1) {1};
\node [vertex] (8) at (2,-2) {};
\node [vertex] (9) at (4,-2) {};

\foreach \i/\j in {1/2,1/3,3/6,3/7,7/8,7/9}
  \draw [edge] (\i) to (\j);
\node [below of=8] {\parbox{0.3\linewidth}{\subcaption{}}};

\end{scope}

\end{tikzpicture}
\end{center}
\setlength{\abovecaptionskip}{-10pt}
\caption{(b) y (c) son dos coárboles binarios diferentes que representan a la cográfica (a).}\label{fig_coar_bin01}
\end{figure}

    \subsection{Algoritmo para generar un coárbol binario}
        Podemos obtener un coárbol binario a partir de un coárbol con el Algoritmo \ref{alg_coa_bin01}. En este algoritmo un nodo interno con al menos tres hijos, $r$, de un coárbol, se procesa creando un nuevo coárbol binario de la siguiente forma: La raíz del coárbol binario tiene como primer hijo al coárbol binario resultante de procesar al primer hijo de $r$ y como segundo hijo un nodo con la misma etiqueta de $r$ que a su vez tiene como primer hijo al árbol binario resultante de procesar al segundo hijo de $r$ y como segundo hijo un nuevo nodo con la misma etiqueta y así sucesivamente. Cuando sólo quedan los últimos dos hijos de $r$, estos se procesan y los árboles binarios resultantes son los hijos del último nodo creado. El árbol binario resultante es un árbol cargado a la derecha. La Figura \ref{fig_alg_coa_bin01} muestra una ejecución ilustrativa del algoritmo.

\begin{algorithm}[h]
\caption{CrearArbolBinario}
\label{alg_coa_bin01}
\DontPrintSemicolon % Some LaTeX compilers require you to use \dontprintsemicolon instead
\KwIn{$r$ la raíz del coárbol}
\KwOut{$r'$ la raíz del coárbol binario}

$r' \gets \text{nuevo nodo de árbol binario}$\;

\If{$r\ \emph{es un nodo interno} $}{
    $r'.etiqueta = r.etiqueta$\;
    $s \gets r'$\;
    $i \gets 0$\;
    \While{$i < r.children.size - 2$}{
        $s.primerHijo \gets \text{CrearArbolBinario}(r.hijos[i])$\;
        $s.segundoHijo \gets \text{nuevo nodo de árbol binario}$\;
        $s \gets s.segundoHijo$\;
        $s.etiqueta \gets r.etiqueta$\;
        $i \gets i+1$\;
    }
    $s.primerHijo \gets \text{CrearArbolBinario}(r.hijos[i])$\;
    $s.segundoHijo \gets \text{CrearArbolBinario}(r.hijos[i+1])$\;
}
\Return $r'$\;

\end{algorithm}

\begin{figure}[h!]
\centering

\begin{subfigure}{0.7\textwidth}
\begin{tikzpicture}
\begin{scope}[xshift=0cm,scale=1]
\node [style=cotreenode, fill=lightgray] (1) at (2,1) {1};
\node [style=vertex] (2) at (0.5,0) {};
\node [style=vertex] (3) at (1.5,0) {};
\node [style=vertex] (4) at (2.5,0) {};
\node [style=cotreenode] (5) at (3.5,0) {0};
\node [style=vertex] (6) at (3,-1) {};
\node [style=vertex] (7) at (4,-1) {};
\foreach \i/\j in {1/2,1/3,1/4,1/5,5/6,5/7}
  \draw [style=edge] (\i) to (\j);
\end{scope}
\begin{scope}[xshift=6.5cm,scale=1]
\node [style=cotreenode] (1) at (1,1) {1};
\node [style=vertex] (2) at (0.5,0) {};
\node [style=cotreenode] (3) at (1.5,0) {1};
\foreach \i/\j in {1/2,1/3}
  \draw [style=edge] (\i) to (\j);
\end{scope}
\end{tikzpicture}
\end{subfigure}

\par\bigskip

\begin{subfigure}{0.7\textwidth}
\begin{tikzpicture}
\begin{scope}[xshift=0cm,scale=1]
\node [style=cotreenode, fill=lightgray] (1) at (2,1) {1};
\node [style=vertex] (2) at (0.5,0) {};
\node [style=vertex] (3) at (1.5,0) {};
\node [style=vertex] (4) at (2.5,0) {};
\node [style=cotreenode] (5) at (3.5,0) {0};
\node [style=vertex] (6) at (3,-1) {};
\node [style=vertex] (7) at (4,-1) {};
\foreach \i/\j in {1/2,1/3,1/4,1/5,5/6,5/7}
  \draw [style=edge] (\i) to (\j);
\end{scope}
\begin{scope}[xshift=6.5cm,scale=1]
\node [style=cotreenode] (1) at (1,1) {1};
\node [style=vertex] (2) at (0.5,0) {};
\node [style=cotreenode] (3) at (1.5,0) {1};
\node [style=vertex] (4) at (1,-1) {};
\node [style=cotreenode] (5) at (2,-1) {1};
\foreach \i/\j in {1/2,1/3,3/4,3/5}
  \draw [style=edge] (\i) to (\j);
\end{scope}
\end{tikzpicture}
\end{subfigure}

\par\bigskip

\begin{subfigure}{0.7\textwidth}
\begin{tikzpicture}
\begin{scope}[xshift=0cm,scale=1]
\node [style=cotreenode, fill=lightgray] (1) at (2,1) {1};
\node [style=vertex] (2) at (0.5,0) {};
\node [style=vertex] (3) at (1.5,0) {};
\node [style=vertex] (4) at (2.5,0) {};
\node [style=cotreenode] (5) at (3.5,0) {0};
\node [style=vertex] (6) at (3,-1) {};
\node [style=vertex] (7) at (4,-1) {};
\foreach \i/\j in {1/2,1/3,1/4,1/5,5/6,5/7}
  \draw [style=edge] (\i) to (\j);
\end{scope}
\begin{scope}[xshift=6.5cm,scale=1]
\node [style=cotreenode] (1) at (1,1) {1};
\node [style=vertex] (2) at (0.5,0) {};
\node [style=cotreenode] (3) at (1.5,0) {1};
\node [style=vertex] (4) at (1,-1) {};
\node [style=cotreenode] (5) at (2,-1) {1};
\node [style=vertex] (6) at (1.5,-2) {};
\node [style=cotreenode] (7) at (2.5,-2) {0};
\foreach \i/\j in {1/2,1/3,3/4,3/5,5/6,5/7}
  \draw [style=edge] (\i) to (\j);
\end{scope}
\end{tikzpicture}
\end{subfigure}

\par\bigskip

\begin{subfigure}{0.7\textwidth}
\begin{tikzpicture}
\begin{scope}[xshift=0cm,scale=1]
\node [style=cotreenode] (1) at (2,1) {1};
\node [style=vertex] (2) at (0.5,0) {};
\node [style=vertex] (3) at (1.5,0) {};
\node [style=vertex] (4) at (2.5,0) {};
\node [style=cotreenode, fill=lightgray] (5) at (3.5,0) {0};
\node [style=vertex] (6) at (3,-1) {};
\node [style=vertex] (7) at (4,-1) {};
\foreach \i/\j in {1/2,1/3,1/4,1/5,5/6,5/7}
  \draw [style=edge] (\i) to (\j);
\end{scope}
\begin{scope}[xshift=6.5cm,scale=1]
\node [style=cotreenode] (1) at (1,1) {1};
\node [style=vertex] (2) at (0.5,0) {};
\node [style=cotreenode] (3) at (1.5,0) {1};
\node [style=vertex] (4) at (1,-1) {};
\node [style=cotreenode] (5) at (2,-1) {1};
\node [style=vertex] (6) at (1.5,-2) {};
\node [style=cotreenode] (7) at (2.5,-2) {0};
\node [style=vertex] (8) at (2,-3) {};
\node [style=vertex] (9) at (3,-3) {};
\foreach \i/\j in {1/2,1/3,3/4,3/5,5/6,5/7,7/8,7/9}
  \draw [style=edge] (\i) to (\j);
\end{scope}
\end{tikzpicture}
\end{subfigure}


\caption{Ejemplo de la ejecución del Algoritmo \ref{alg_coa_bin01}. A la izquierda se muestra el coárbol original, mienrtras se marca con gris el nodo que se está procesando. A la derecha aparece el coárbol binario que se va construyendo.}\label{fig_alg_coa_bin01}


\end{figure}


En términos de las particiones de las componentes conexas de la gráfica, el algoritmo realiza lo siguiente. Si la etiqueta de $r$ es $0$, entonces el coárbol con raíz en $r$ representa una cográfica inconexa y se elige la partición de sus vértices en la que la primera parte es una componente conexa y la segunda parte es el resto. Sucede lo mismo si la etiqueta de $r$ es uno, pero como la cográfica representada es conexa, en su lugar se toman una componente conexa del complemento de la cográfica representada en la primera parte y el resto en la segunda.

Dado que el Algoritmo \ref{alg_coa_bin01} recorre a lo más una vez cada nodo de $r$, su tiempo de ejecución es $O(n)$ en donde $n$ es el número total de nodos del árbol con raíz $r$.

    \subsection{Algoritmo para generar todos los coárboles binarios de una gráfica}
        Podemos obtener todos los coárboles binarios correspondientes a un coárbol
haciendo uso del Algoritmo \ref{alg_coa_bin02}. Este algoritmo recibe como
entrada la raíz del coárbol, $r$, y devuelve un conjunto de nodos, $S$, cada uno
de cuyos elementos es la raíz de un coárbol binario. Los nodos internos son
procesados creando un nuevo coárbol binario para cada posible partición del
conjunto de hijos de dicho nodo. Al procesar las hojas, simplemente se crea un
nuevo nodo que será una hoja en los árboles binarios. 

\begin{algorithm}[ht!]
\caption{CrearÁrbolesBinarios}
\label{alg_coa_bin02}
\DontPrintSemicolon % Some LaTeX compilers require you to use \dontprintsemicolon instead
\KwIn{$r$ la raíz de un coárbol $T$}
\KwOut{$S = \{r'_1, r'_2, \dots, r'_n\}$ con $r_i$ la raíz de un coárbol binario de $T$ para todo $1\geq i \geq n$}

$S \gets \emptyset$\;
\If{$r$ es una hoja}{
    Agregar un nuevo nodo de coárbol binario a $S$\;    
}
\Else{
    \ForEach{partición en dos partes $(A,B)$ del conjunto de hijos de $r$ tal que ni $A$ ni $B$ son vacíos}{
        \If{$A$ tiene sólo un elemento $a$}{
            $L \gets \text{CrearÁrbolesBinarios(\emph{a})}$\;
        }
        \Else{
            $a'\gets$ nuevo nodo de coárbol con la etiqueta de $r$\;
            Agregar todos los elementos de $A$ como hijos de $a'$\;
            $L \gets \text{CrearÁrbolesBinarios(\emph{a'})}$\;
        }
        \If{$B$ tiene sólo un elemento $b$}{
            $R \gets \text{CrearÁrbolesBinarios(\emph{b})}$\;
        }
        \Else{
            $b'\gets$ nuevo nodo de coárbol con la etiqueta de $r$\;
            Agregar todos los elementos de $B$ como hijos de $b'$\;
            $R \gets \text{CrearÁrbolesBinarios(\emph{b'})}$\;
        }
        \ForEach{$l\in L$ \textbf{y cada} $r \in R$}{
            $s\gets $ nuevo nodo de coárbol binario con la etiqueta de $r$\;
            $s.izquierda \gets l$\;
            $s.derecha \gets r$\;
            Agregar $s$ a $S$\;
        }
    }
}
\Return $S$\;
    
\end{algorithm}

Dado que el número de coárboles binarios que representan a una gráfica crece de forma exponencial, el tiempo en el que se pueden generar dichos coárboles binarios crece al menos de forma exponencial. El Algoritmo \ref{alg_coa_bin02} no es óptimo, ya que todos los nodos con excepción de la raíz y las hojas se procesan múltiples veces. Esto no repercute en el resultado principal de la sección, el Algoritmo \ref{alg_esta_en_clase}, que sirve para identificar a los elementos de una clase hereditaria de cográficas $C$, ya que, aunque se requiere encontrar todos los coárboles binarios de cada una de las obstrucciones mínimas de $C$, se contempla que este cómputo se realice antes de la ejecución del algoritmo.
    \subsection{Subcoárbol}
        A continuación presentamos el concepto de subcoárbol binario
que será utilizado para determinar si una cográfica $H$ es subgráfica
inducida de una cográfica $G$ en el Algoritmo \ref{alg_subgraph}.

Sean $T$ y $U$ dos coárboles binarios y $u_1$, $u_2$ y $u_3$ nodos de $U$, decimos
que $U$ es un \emph{\textbf{subcoárbol binario}} de $T$ si existe una función
inyectiva $f:V(U)\rightarrow V(T)$ tal que, si $u_1$ es una hoja, entonces $f(u_1)$
es una hoja también; si no, entonces $u_1.etiqueta = f(u_1).etiqueta$ y, si $u_3$ es
el ancestro común más profundo de $u_1$ y $u_2$, entonces $f(u_3)$ es el ancestro
común más profundo de $f(u_1)$ y $f(u_2)$. Llamamos a $f$ la \textbf{\emph{función de
coasignación}} de $U$ a $T$.

El concepto de subcoárbol binario es diferente del de subárbol dado que, si $T$ y $U$
son coárboles con $U$ subcoárbol binario de $T$, entonces tenemos que los nodos de
$U$ se pueden encontrar dispersos entre los nodos de $T$ a diferencia de lo que se
tendría si $U$ fuera subárbol de $T$. Esto se puede apreciar en la Figura
\ref{fig_subcoarbol01}.

\begin{figure}[h!]
\begin{center}
\begin{tikzpicture}

\begin{scope}[xshift=0cm,scale=1]
\node [style=cotreenode] (1) at (1,1) {0};
\node [style=cotreenode] (2) at (-0.5,0) {1};
\node [style=cotreenode] (3) at (2.5,0) {1};
\node [style=cotreenode] (4) at (-1.25,-1) {0};
\node [style=cotreenode] (5) at (0.25,-1) {0};
\node [style=cotreenode] (6) at (1.75,-1) {0};
\node [style=cotreenode] (7) at (3.25,-1) {0};
\node [style=vertex] (8) at (-1.5,-2) {};
\node [style=vertex] (9) at (-1,-2) {};
\node [style=vertex] (10) at (0,-2) {};
\node [style=vertex] (11) at (0.5,-2) {};
\node [style=vertex] (12) at (1.5,-2) {};
\node [style=vertex] (13) at (2,-2) {};
\node [style=vertex] (14) at (3,-2) {};
\node [style=vertex] (15) at (3.5,-2) {};

\node (16) at (0.25,1) {$f(a)$};
\node (17) at (-1.6,-2.4) {$f(b)$};
\node (18) at (3.25,0) {$f(c)$};
\node (19) at (1.4,-2.4) {$f(d)$};
\node (20) at (3.6,-2.4) {$f(e)$};

\foreach \i/\j in {1/2,1/3,2/4,2/5,3/6,3/7,4/8,4/9,5/10,5/11,6/12,6/13,7/14,7/15}
  \draw [style=edge] (\i) to (\j);
\node [below of=19,xshift=-0.25cm] {\parbox{0.3\linewidth}{\subcaption{}}};
\end{scope}

\begin{scope}[xshift=6cm,scale=1]
\node [style=cotreenode] (1) at (1,1) {0};
\node [style=vertex] (2) at (0,0) {};
\node [style=cotreenode] (3) at (2,0) {1};
\node [style=vertex] (4) at (1.5,-1) {};
\node [style=vertex] (5) at (2.5,-1) {};

\node (6) at (0.5,1) {$a$};
\node (7) at (-0.3,0) {$b$};
\node (8) at (2.5,0) {$c$};
\node (9) at (1.5,-1.3) {$d$};
\node (10) at (2.5,-1.3) {$e$};

\foreach \i/\j in {1/2,1/3,3/4,3/5}
  \draw [style=edge] (\i) to (\j);
\node [below of=9,xshift=-0.25cm] {\parbox{0.3\linewidth}{\subcaption{}}};
\end{scope}

\end{tikzpicture}
\end{center}
\setlength{\abovecaptionskip}{-10pt}
\caption{El coárbol (b) es subcoárbol binario del coárbol (a). Las etiquetas en los nodos de ambos coárboles binarios indican la asignación de los nodos de (b) a los nodos de (a).}\label{fig_subcoarbol01}
\end{figure}

\begin{lemma}
Sean $G$ y $H$ cográficas, $T_G$ un coárbol binario de $G$ y $T_H$ un coárbol binario de $H$. Si $T_H$ es subcoárbol binario de $T_G$, entonces $H$ es una subgráfica inducida de $G$.
\end{lemma}

\begin{proof}
Sean $h_1$ y $h_2$ hojas diferentes de $T_H$ y $h_3$ el ancestro común más profundo
de $h_1$ y $h_2$. Como $T_H$ es subcoárbol binario de $T_G$, entonces existe una
función de coasignación, $f$, de $T_H$ a $T_G$. Luego, tenemos que $f(h_1)$ y
$f(h_2)$ son hojas de $T_G$ y que las etiquetas de $h_3$ y $f(h_3)$ coinciden. Así,
$h_1$ y $h_2$ son adyacentes en $H$ si y sólo si $f(h_1)$ y $f(h_2)$ son adyacentes
en $G$. Luego, $G[f[V(H)]]$ es una subgráfica indicida de $G$ que es isomorfa a $H$.
Así, $H$ es una subgráfica inducida de $G$.
\end{proof}


\begin{lemma}
Sean $G$ y $H$ cográficas y $B_G$ un coárbol binario de $G$. Si $H$ es una subgráfica inducida de $G$, entonces existe un coárbol binario $B_H$ de $H$ tal que $B_H$ es subcoárbol binario de $B_G$.
\end{lemma}

\begin{proof}
Como $H$ es una subgráfica inducida de $G$, $V(H)$ es un subconjunto de las hojas de $B_G$. Consideremos el siguiente proceso recursivo para crear el árbol $B_H$.

\begin{algorithm}[H]
\DontPrintSemicolon
\KwIn{$V(H)$}
\KwOut{$r$, la raíz de $B_H$}
    \If{$V(H)$ tiene un solo elemento}{
        $r\gets v$, el único elemento de $V(H)$\;
    }
    \Else{
        $v\gets$ el nodo más profundo de $B_G$ que es ancestro de cada elemento de $V(H)$\;
        $r\gets$ nuevo nodo de coárbol binario con la misma etiqueta que $v$\;
        Asignar al hijo izquierdo de $r$ el resultado de procesar recursivamente los elementos de $V(H)$ que estén en la rama izquierda de $v$\;
        Asignar al hijo derecho de $r$ el resultado de procesar recursivamente los elementos de $V(H)$ que estén en la rama derecha de $v$\;
    }
    \Return $r$,
\end{algorithm}

Notemos que cada nodo de $B_H$ se construye tomando como base a un nodo de $B_G$. Esto ocurre en la línea 2 si dicho nodo es una hoja o en la línea 4 si es un nodo interno. Sea $x$ un nodo de $B_H$, denotamos por $x'$ al nodo de $B_G$ que se toma como base para construir a $x$.
% Aquí hay un abuso del lenguaje que puede confundir al lector, cuando
% usas $x$ para referirte al nodo y al coárbol con raíz $x$.

Sea $x$ un nodo de $B_H$, notemos lo siguiente. Si $x$ es una hoja de $B_H$, entonces $x$ es un vértice de $H$, por lo que $x = x'$, y si $x$ es un nodo interno, entonces la etiqueta de $x$ es la misma que la etiqueta de $x'$. Además, el hijo derecho de $x$ es ancestro de los mismos vértices de $H$ que el hijo derecho de $x'$. Análogamente para el hijo izquierdo. De esto se sigue que para cualesquiera dos vértices de $H$, su ancestro común más profundo en $B_H$ tiene la misma etiqueta que su ancestro común más profundo en $B_G$.

Veamos que $B_H$ es un coárbol binario de $H$. Notemos que las hojas de $B_H$ son todos los vértices de $H$. Dado que $H$ es una subgráfica inducida de $G$, cualesquiera dos vértices de $H$ son adyacentes si y sólo si su ancestro común más profundo en $B_G$ tiene etiqueta 1. Luego, dos vértices de $H$ son adyacentes si y sólo si su ancestro común más profundo en $B_H$ tiene etiqueta 1. Así, $B_H$ es un subcoárbol binario de $H$.

Sea $f$ un subconjunto de $V(B_H) \times V(B_G)$ tal que $(x,x')\in f$ para cualquier nodo $x$ de $V(B_H)$, veamos que $f$ es una función de coasignación de $B_H$ a $B_G$. Como cada pareja en $f$ tiene como primer elemento a un nodo único de $B_H$ y como segundo elemento un nodo único de $B_G$, $f$ es una función inyectiva. Sean $x_1$, $x_2$ y $x_3$ nodos de $B_H$. Es claro que si $x$ es una hoja, entonces $f(x)=x'$ es una hoja. Y si $x$ es un nodo interno, entonces $f(x)=x'$ tiene la misma etiqueta que $x$. Si $x_3$ es el ancestro común más profundo de $x_1$ y $x_2$, notemos lo siguiente:
\begin{itemize}
    \item el conjunto de vértices de $H$ que son descendientes de $x_1$ es igual al conjunto de vértices de $H$ que son descendientes de $f(x_1)$.
    \item el conjunto de vértices de $H$ que son descendientes de $x_2$ es igual al conjunto de vértices de $H$ que son descendientes de $f(x_2)$.
    \item el hijo derecho de $x_3$ es ancestro de los mismos vértices de $H$ que el hijo derecho de $f(x_3)$.
\end{itemize}
Supongamos sin pérdida de generalidad que $x_1$ es descendiente del hijo izquierdo de $x_3$ y que $x_2$ es descendiente de su hijo derecho. Luego,  $f(x_1)$ es descendiente del hijo izquierdo de $f(x_3)$ y $f(x_2)$ es descendiente de su hijo derecho. Así, $f(x_3)$ es el ancestro común más profundo de $f(x_1)$ y $f(x_2)$. Luego, $f$ es una función de coasignación de $B_H$ a $B_G$.

\end{proof}

    \subsection{Algoritmo para encontrar obstrucciones mínimas} \label{sec_AlgoSub}
        La presente sección aborda el problema de determinar si una cográfica $G$ tiene
a otra cográfica $H$ como subgráfica inducida haciendo uso de los conceptos de
coárbol binario y subcoárbol. Se proporciona un algoritmo (Algoritmo
\ref{alg_subgraph}) para resolver este problema tal que, si se fija el tamaño de
$H$, su tiempo de ejecución crece de forma lineal con respecto al tamaño de $G$.
Este algoritmo es útil para identificar a las gráficas pertenecientes a una clase 
caracterizada a través de su conjunto de obstrucciones mínimas de forma rápida.

\subsubsection{Algoritmo para determinar si un coárbol binario es subcoárbol binario de otro}

\begin{definition}
    Sean $T$ y $U$ coárboles (binarios) y $u$ un nodo de $U$, decimos que
    $f:V(U)\rightarrow\{marcado, no\_marcado\}$ es una \textbf{\emph{función de
    verificación}} de $T$ para $U$ si $f(u) = marcado$ si y sólo si el coárbol
    (binario) con raíz $u$ es subcoárbol (binario) de $T$. Si $f(u) = marcado$,
    decimos que $f$ \textbf{\emph{marca}} a $u$.
\end{definition}

El Algoritmo \ref{alg_subcoarbol} recibe como entradas dos coárboles binarios,
$G$ y $H$ representados por sus raíces $g$ y $h$ respectivamente, y devuelve una
función de verificación, $f_g$, de $G$ para $H$.

Este algoritmo funciona creando la función de verificación de cada subárbol de $G$ para $H$, empezando aquellos cuya raíz es más profunda. De esta manera, si la función de verificación de $G$ para $H$ evaluada en $h$ es $marcado$, entonces  $H$ es subcoárbol de $G$.

\begin{algorithm}[ht!]
\caption{Función\_de\_coasignación}
\label{alg_subcoarbol}
\DontPrintSemicolon % Some LaTeX compilers require you to use \dontprintsemicolon instead
\KwIn{$g$ y $h$, las raíces de dos coárboles binarios para las gráficas $G$ y $H$ respectivamente}
\KwOut{$func$, la función de verificación de $G$ para $H$}

 $func \gets \text{nueva función de coasignación tal que} func(x)=no\_marcado \text{ para todo } x\in V(H)$\;

 \If{$g$ es una hoja}{
    $func \text{ marca a todas las hojas de } H$\;
 }
 \Else{
    $v_{izq} \gets \text{Función\_de\_coasignación}(g.izquierda, h)$\;
    $v_{der} \gets \text{Función\_de\_coasignación}(g.derecha, h)$\;

    \ForEach{nodo \textbf{\emph{de}} H}{
        \If{$v_{izq}(nodo) = marcado \emph{ \textbf{o} } v_{der}(nodo) = marcado$}{
            $func(nodo) \gets marcado$\;
        }
        \ElseIf{nodo.etiqueta = g.etiqueta \emph{\textbf{y}} $v_{izq}$ \emph{marca a uno de los hijos de} nodo \emph{y} $v_{der}$ \emph{al otro}}{
            $func(nodo) \gets marcado$\;
        }
    }

 }

$\Return func$

\end{algorithm}

\begin{theorem}
    La ejecución del Algoritmo \ref{alg_subcoarbol}, \emph{Función\_de\_coasignación($g$, $h$)} regresa una función, $func$, tal que $func$ es una función de verificación de $\acute{a}rbol(g)$ para $\acute{a}rbol(h)$.
\end{theorem}

\begin{proof}

    Sea $n$ un nodo de $\acute{a}rbol(h)$. Para probar que $func$ es una función
    de verificación de $\acute{a}rbol(g)$ para $\acute{a}rbol(h)$, tenemos que
    probar que $func(n) = marcado$ si y sólo si $\acute{a}rbol(n)$ es subcoárbol
    de $\acute{a}rbol(g)$.

    \textbf{Necesidad}: En esta parte de la demostración, se supone que el algoritmo ha sido ejecutado y que $func$ marca a $n$. Procedamos por inducción sobre la altura de $g$.

    \emph{Caso base:} Si $g$ tiene altura 0, entonces $g$ es una hoja, por lo
    que $func$ marca únicamente a las hojas de $\acute{a}rbol(h)$. Como $func$
    marca a $n$, entonces $n$ es una hoja. Luego, la función $f=\{(n,g)\}$ es
    una función de coasignación de $\acute{a}rbol(n)$ a $\acute{a}rbol(g)$, por
    lo que $\acute{a}rbol(n)$ es subcoárbol de $\acute{a}rbol(n)$.

    \emph{Paso inductivo:} Si $g$ tiene altura $k > 0$. Supongamos como
    hipotesis inductiva (H.I.) que, para todo nodo de un coárbol binario, $g'$,
    de altura $k' < k$ se cumple que, si $func' = $
    Función\_de\_coasignación$(g',h)$ marca a un nodo $n'$ de
    $\acute{a}rbol(h)$, entonces $\acute{a}rbol(n')$ es subcoárbol de
    $\acute{a}rbol(g')$. Como $g$ no es una hoja, el algoritmo debió de entrar
    al bloque de instrucciones de las líneas 5 a 11. En las líneas 5 y 6 se
    crean dos funciones que cumplen con la H.I., ya que $g.izquierda$ y
    $g.derecha$ tienen ambas una altura menor a $k$. Como $func$ marca a $n$,
    entonces $n$ debe de cumplir la condición de la línea 8 o la condición de la
    línea 10. Si se cumple la condición de la línea 8, entonces $v_{izq}(n) =
    marcado$ o $v_{der}(n) = marcado$, por lo que $\acute{a}rbol(n)$ es
    subcoárbol de $\acute{a}rbol(g.izquierda)$ o de $\acute{a}rbol(g.derecha)$,
    y por lo tanto es subcoárbol de $\acute{a}rbol(g)$. De lo contrario, se
    cumple la condición de la línea 10, entonces $v_{izq}$ marca a $n.izquierda$
    o a $n.derecha$ y $v_{der}$ marca al otro. Supongamos sin pérdida de
    generalidad que $v_{izq}$ marca a $n.izquierda$ y $v_{der}$ marca a
    $n.derecha$. Sean $f_i:V(\acute{a}rbol(n.izquierda))\rightarrow
    V(\acute{a}rbol(g.izquierda))$ la función de coasignación de
    $\acute{a}rbol(n.izquierda)$ a $\acute{a}rbol(g.izquierda)$ y
    $f_d:V(\acute{a}rbol(n.derecha))\rightarrow V(\acute{a}rbol(g.derecha))$ la
    función de coasignación de $\acute{a}rbol(n.derecha)$ a
    $\acute{a}rbol(g.derecha)$, mostremos que la función $f = f_i \cup f_d \cup
    \{(n,g)\}$ es una función de coasignación de $\acute{a}rbol(n)$ a
    $\acute{a}rbol(g)$. Como los dominios de $f_i$ y $f_d$ son ajenos y ninguno
    contiene a $n$, entonces $f$ es una función. Como los rangos de $f_i$ y
    $f_d$ son ajenos, ninguno contiene a $g$ y tanto $f_i$ como $f_d$ son
    inyectivas, entonces $f$ es inyectiva. Por otra parte, por la condición de
    la línea 10, sabemos que $n.etiqueta = g.etiqueta$. También sabemos que, sea
    $x \in V(\acute{a}rbol(n.izquierda))$, si $x$ es una hoja, entonces $f(x) =
    f_i(x)$ es una hoja y si no, entonces $x.etiqueta = f_i(x).etiqueta =
    f(x).etiqueta$. Análogamente para un $y \in V(\acute{a}rbol(n.derecha))$ y
    $f_d$. Finalmente, si $n$ es el ancestro común más profundo de dos nodos
    $z_1$ y $z_2$, entonces $z_1$ es descendiente de $n.derecha$ y $z_2$ es
    descendiente de $n.izquierda$ o viceversa. Supongamos lo primero sin pérdida
    de generalidad. Luego, por la condición de la línea 10, $v_{izq}$ marca a
    uno y $v_{der}$ marca al otro. Supongamos sin pérdida de generalidad que
    $v_{izq}$ marca a $z_1$ y $v_{der}$ marca a $z_2$. Entonces, $f(z_1) =
    f_i(z_1) \in V(\acute{a}rbol(g.izquierda))$ y $f(z_2) = f_i(z_2) \in
    V(\acute{a}rbol(g.derecha))$, por lo que el ancestro común más profundo de
    $f(z_1)$ y $f(z_2)$ es $g = f(n)$. Así, $f$ es una función de coasignación
    de $\acute{a}rbol(n)$ a $\acute{a}rbol(g)$ y $\acute{a}rbol(n)$ es
    subcoárbol de $\acute{a}rbol(g)$.

     \textbf{Suficiencia}: En esta parte de la demostración se supone que
     $\acute{a}rbol(n)$ es subcoárbol de $\acute{a}rbol(g)$ y se sigue la
     ejecución del algoritmo para mostrar que, al final de la misma, $func$
     marcará a $n$. Sea $f$ la función de cosignación de $\acute{a}rbol(n)$ a
     $\acute{a}rbol(g)$, procedamos por inducción sobre la altura de $g$.

    \emph{Caso base:} Si la altura de $g$ es 0, entonces $g$ es una hoja, por lo
    que se cumple con la condición de la línea 2 y se ejecuta la línea 3,
    haciendo que $func$ marque todas las hojas de $H$. Como $\acute{a}rbol(n)$
    es subcoárbol de $\acute{a}rbol(g)$ y $\acute{a}rbol(g)$ sólo tiene un nodo,
    entonces $n$ debe de ser una hoja. Luego, $func$ marca a $n$.

    \emph{Paso inductivo:} Si $g$ tiene altura $k > 0$. Supongamos como H.I. que
    todo coárbol, $g'$, con altura $k' < k$ cumple con que, siendo $n'$ un nodo
    de $\acute{a}rbol(h)$, si $\acute{a}rbol(n')$ es subcoárbol de
    $\acute{a}rbol(g')$, entonces $func'=$Función\_de\_coasignación $(g',h)$
    marca a $n'$. Como $g$ no es una hoja, el algoritmo ejecuta las líneas 5 y 6
    y posteriormente el bloque de las líneas 8 a 11 para cada nodo de $H$. Si
    $n$ es marcada por $v_{izq}$ o $v_{der}$, entonces se ejecuta la línea 9 y
    $func$ marca a $n$. En el caso contrario, probemos que se cumple la
    condición de la línea 10. Mostremos primero que $f(n) = g$ procediendo por
    contradicción. Supongamos que $f(n) = x$ para algún $x\in
    V(\acute{a}rbol(g))-\{g\}$. Como $x$ es descendiente de $g$, tiene altura
    menor a $k$. También sabemos que $f$ es una función de coasignación de
    $\acute{a}rbol(n)$ a $\acute{a}rbol(x)$, por lo que, por H.I., $n$ debería
    de ser marcado ya sea por $v_{izq}$ o por $v_{der}$, lo que es una
    contradicción. Luego, $f(n) = g$, y por lo tanto $n$ no es una hoja y
    $f(n).etiqueta = g.etiqueta$. Mostremos ahora que tanto $n.izquierda$ como
    $n.derecha$ son marcados cada uno ya sea por $v_{izq}$ o por $v_{der}$.
    Sabemos que $f(n.izquierda)$ y $f(n.derecha)$ son descendientes de $r$. Como
    $f\mid_{V(\acute{a}rbol(n.izquierda))}$ es una función de coasignación de
    $\acute{a}rbol(n.izquierda)$ a $\acute{a}rbol(f(n.izquierda))$ y $f(n.izquierda)
    \neq g$ ya que $f$ es inyectiva, entonces $\acute{a}rbol(n.izquierda)$ es
    subcoárbol de algún descendiente de $g$, al que llamaremos $y$. Como $y$
    tiene altura menor a $k$, su función de verificación correspondiente marca a
    $n.izquierda$ (por H.I.), y por la condición de la línea 8, sus ancestros
    también lo marcan. Luego $v_{izq}$ o $v_{der}$ marcan a $n.izquierda$.
    Análogamente para $n.derecha$. Así, tanto $n.izquierda$ como $n.derecha$
    están marcados cada uno ya sea en $v_{izq}$ o en $v_{der}$. Mostremos, por
    último, que uno es marcado por $v_{izq}$ y el otro es marcado por $v_{der}$.
    Como el ancestro común más profundo de $n.izquierda$ y $n.derecha$ es $n$, y
    $f(n)=g$, entonces el ancestro común más profundo de $f(n.izquierda)$ y
    $f(n.derecha)$ debe de ser $g$. Luego, $f(n.izquierda)$ está en una rama de
    $g$ y $f(n.derecha)$ está en la otra. Supongamos sin pérdida de generalidad
    que $f(n.izquierda)$ está en la rama izquierda de $g$ y $f(n.derecha)$ está
    en la rama derecha. Como $f\mid_{V(\acute{a}rbol(n.izquierda))}$ es una
    función de coasignación de $\acute{a}rbol(n.izquierda)$ a
    $\acute{a}rbol(g.izquierda)$ y por H.I., entonces $v_{izq}$ marca a
    $n.izquierda$. De forma análoga, $v_{der}$ marca a $n.derecha$. Concluyendo,
    como $n.etiqueta = r.etiqueta$ y tanto $n.izquierda$ como $n.derecha$ son
    marcados uno por $v_{izq}$ y el otro por $v_{der}$, se cumple la condición
    de la línea 10 y $func$ marca a $n$. Así, al final de la ejecución del
    algoritmo, $n$ estará marcado.

\end{proof}

Dado que, para cada nodo de $G$, se crea una función de verificación cuyo
dominio es el conjunto de los nodos de $H$, el tiempo de ejecución del algoritmo
crece de la forma $O(\mid V(G) \mid \mid V(H) \mid)$. 

\subsubsection{Determinar si una cográfica es subcográfica de otra}

Haciendo uso del Algoritmo \ref{alg_subcoarbol}, se puede idear otro algoritmo
para determinar si una cográfica, $H$ es subgráfica de otra cográfica, $G$, al
buscar todas las formas del coárbol binario de $H$ en un solo coárbol binario de
$G$.

\begin{algorithm}[ht!]
\caption{Es\_subgráfica}
\label{alg_subgraph}
\DontPrintSemicolon % Some LaTeX compilers require you to use \dontprintsemicolon instead
\KwIn{$g$ y $h$, las raíces de dos coárboles, $G$ y $H$ respectivamente.}
\KwOut{$verdadero$ si la cográfica representada por $H$ es subgráfica de la cográfica representada por $G$. $falso$ en el caso contrario.}

$g\_bin \gets \text{CrearÁrbolBinario}(g)$\;
$h\_bins \gets$ las raíces de todos los coárboles binarios correspondientes a $H$\;

\ForEach{bin \textbf{\emph{en}} h\_bins}{
    $f = \text{Función\_de\_coasignación}(g\_bin,bin)$\;
    \If{f(bin) = marcado}{
        $\Return\ verdadero$\;
    }
}

$\Return\ falso$\;

\end{algorithm}

Como la línea 1 Algoritmo \ref{alg_subgraph} se ejecuta en tiempo $O(\mid V(G)
\mid)$, la complejidad temporal de éste depende del número de coárboles binarios
correspondientes a $H$ (que crece con mayor rapidez). Sin embargo, si se fija
$H$, la complejidad temporal de éste es simplemente  $O(\mid V(G) \mid)$. Fijar
$H$ resultará útil cuando se esté resolviendo un problema específico como el de
encontrar una obstrucción mínima en una gráfica. La aplicación de este algoritmo
en el presente trabajo de tesis es desarrollar algoritmos que nos permitan
identificar a los elementos de una clase hereditaria de cográficas en tiempo
lineal. Esto se muestra en el Algoritmo \ref{alg_esta_en_clase}. 

\begin{algorithm}[ht!]
\caption{Pertenece_a_la_clase}
\label{alg_esta_en_clase}
\DontPrintSemicolon % Some LaTeX compilers require you to use \dontprintsemicolon instead
\KwIn{$g$, la raíz del coárbol de una cográfica $G$.}
\KwOut{$verdadero$ si $G$ pertenece a la clase hereditaria de cográficas $C$}

$g\_bin \gets \text{CrearÁrbolBinario}(g)$\;
$C\_bins \gets$ las raíces de todos los coárboles binarios de todas las obstrucciones mínimas de la clase $C$\;

\ForEach{$bin$ \textbf{\emph{en}} $C\_bins$}{
    $f = \text{Función\_de\_coasignación}(g\_bin,bin)$\;
    \If{$f(bin) = marcado$}{
        $\Return\ verdadero$\;
    }
}

$\Return\ falso$\;

\end{algorithm}

El Algoritmo \ref{alg_esta_en_clase} es una variación del Algoritmo
\ref{alg_subgraph} en el que se fija una clase $C$ cuyos elementos se desea
identificar. Al computar el conjunto de coárboles binarios de cada una de las
obstrucciones mínimas de $C$ antes de la ejecución del algoritmo
\ref{alg_esta_en_clase}, la línea 2 del algoritmo se puede ejecutar en tiempo
constante. Gracias a esto obtenemos un algoritmo que determina si una gráfica
$G$ pertenece a la clase $C$ en tiempo $O(\mid V(G) \mid)$.



\section{La clase $M_2$}

    En el presente capítulo se desarrolla el tema principal de la tesis, del cual se desprenden el resto de los resultados.

    \begin{definition}
        La \textbf{\emph{clase $M_2$}} es la clase de cográficas cuyo conjunto de vértices acepta una partición en dos partes tal que cada parte induce una gráfica multipartita completa. %Es decir que $M_2 = \{G \mid G $ es una cográfica y existe una partición de $ V = (A,B) $ tal que $ G(A) $ y $ G(B) $  son gráficas multipartitas completas$\}$.
    \end{definition}

    Claramente, $M_2$ es una clase hereditaria, pues si tomamos una gráfica de esta clase y sustraemos uno de sus vértices, de cualquiera de sus dos partes, dicha parte seguirá siendo una gráfica multipartita completa. Al ser una clase hereditaria, $M_2$ puede ser caracterizada por un conjunto de obstrucciones mínimas. También se puede decidir si una cográfica pertenece a ésta en tiempo lineal \cite{unknown}.

    \subsection{Obstrucciones mínimas}
        \begin{theorem} \label{teo_obsts_m2}

    Para una cográfica $G$, las siguientes afirmaciones son equivalentes.
    \begin{enumerate}[(a)]
        \item $G \in M_2$.
        \item $G$ no contiene a ninguna de las gráficas de las Figuras \ref{obsts_O_M3} como subgráficas inducidas. $H, I$ ni a $J$ como subgráficas inducidas.
    \end{enumerate}

\end{theorem}

\begin{figure}[ht!]
\begin{center}
\begin{tikzpicture}

\begin{scope}[xshift=0cm,scale=1]

\node [style=vertex] (1) at (0,0) {};
\node [style=vertex] (2) at (1,0) {};
\node [style=vertex] (3) at (0,0.5) {};
\node [style=vertex] (4) at (1,0.5) {};
\node [style=vertex] (5) at (0.5,1.25) {};
\node [style=vertex] (6) at (0.5,2) {};
\foreach \i/\j in {1/2,3/4,3/5,4/5}
  \draw [style=edge] (\i) to (\j);
\node [below of=1,xshift=.5cm]
{\parbox{0.3\linewidth}{\subcaption*{$H$}}};

\end{scope}

\begin{scope}[xshift=3cm,scale=1]

\node [style=vertex] (1) at (0,0) {};
\node [style=vertex] (2) at (1,0) {};
\node [style=vertex] (3) at (0,0.5) {};
\node [style=vertex] (4) at (1,0.5) {};
\node [style=vertex] (5) at (0.5,1.25) {};
\node [style=vertex] (6) at (0.5,2) {};
\foreach \i/\j in {1/2,3/4,3/5,4/5,5/6}
  \draw [style=edge] (\i) to (\j);
\node [below of=1,xshift=.5cm]  {\parbox{0.3\linewidth}{\subcaption*{$I$}}};

\end{scope}

\begin{scope}[xshift=6cm,scale=1]

\node [style=vertex] (1) at (0,0) {};
\node [style=vertex] (2) at (0.5,0.5) {};
\node [style=vertex] (3) at (1.5,0.5) {};
\node [style=vertex] (4) at (0.5,1.5) {};
\node [style=vertex] (5) at (1.5,1.5) {};
\node [style=vertex] (6) at (0,2) {};
\node [style=vertex] (7) at (2,1) {};

\foreach \i/\j in {1/2,1/3,1/6,2/3,2/4,2/5,3/4,3/5,4/5,4/6,5/6}
  \draw [style=edge] (\i) to (\j);
\node [below of=1,xshift=1cm] {\parbox{0.3\linewidth}{\subcaption*{$J$}}};

\end{scope}
\end{tikzpicture}
\end{center}
\setlength{\abovecaptionskip}{-15pt}
\caption{Obstrucciones mínimas para la clase $M_2$.}
\label{obsts_O_M3}
\end{figure}

\begin{proof}

    Notemos que las subgráficas $H, I$ y $J$ pueden ser descritas de la siguiente manera:

    \begin{enumerate}[(1)]
        \item $I = K_1 + K_2 + K_3$.
        \item $H = Paw + K_2$.
        \item $J = (\overline{P_3} \oplus \overline{P_3}) + K_1$.
    \end{enumerate}

    \textbf{\emph{Necesidad}}: Dado que $M_2$ es una clase hereditaria, todas las subgráficas inducidas en $G$ deben de estar en $M_2$. Procedamos por contrapositiva mostrando que ni $H$ ni $I$ ni $J$ están en $M_2$. Si ambos vértices del $K_2$ en $H$ se encuentran en la misma parte, entonces no puede haber ningún vértice adicional en dicha parte, o ésta contendría un $\overline{P_3}$. Como los vértices restantes inducen una gráfica que no es multipartita completa, esto no puede suceder. Así, un vértice del $K_2$ debe estar en una parte y el otro en la otra. Como son vértices independientes, cada parte debe de ser un conjunto independiente. Como la gráfica inducida por los vértices restantes contiene un $K_3$, no es bipartita y no se puede dividir en dos conjuntos independientes. Así, $H$ no pertenece a $M_2$. Análogamente, se muestra que $I$ no está en $M_2$. Por otra parte, como $J$ tiene un vértice aislado, el resto de sus vértices (mismos que forman un $\overline{P_3} \oplus \overline{P_3}$) deben de poder dividirse en dos partes de manera tal que una induzca un conjunto independiente y la otra una gráfica multipartita completa. Siempre que tomamos uno de los vértices de uno de los dos $\overline{P_3}$ para agregarlo al conjunto independiente, ninguno los vértices del otro $\overline{P_3}$ puede ser agregado al conjunto independiente, pues es adyacente al vértice que agregamos primero. Así, la subgráfica inducida $\overline{P_3} \oplus \overline{P_3}$ no acepta una partición en un conjunto independiente y una gráfica multipartita completa. Luego, $J$ no está en $M_2$.

    \textbf{\emph{Suficiencia}}: Consideramos los siguientes casos.

    \emph{Caso 1:} $G$ tiene al menos dos componentes conexas no triviales.

    Consideremos la partición de $V = (A,B)$ tal que $A$ contiene únicamente una componente no trivial y $B$ el resto. Como $G[A]$ y $G[B]$ contienen ambas componentes no triviales, las dos poseen un $K_2$. Luego, ni $G[B]$ ni $G[A]$ contienen un $Paw$, o $G$ tendría a $I$ como subgráfica inducida. Dado que $G[A]$ y $G[B]$ son cográficas, son también gráficas perfectas, y al ninguna tener un $Paw$ como subgráfica inducida, cada una es bipartita o multipartita completa. Si ambas son multipartitas, entonces $G \in M_2$.

    Si ambas son bipartitas, entonces $G$ es bipartita también y acepta una partición en dos conjuntos independientes, cada uno de los cuales es una gráfica multipartita completa, por lo que $G \in M_2$.

    Si $G[A]$ es bipartita y $G[B]$ es multipartita completa, como $G[A]$ es una cográfica conexa, entonces es una gráfica multipartita completa y $G \in M_2$.

    Finalmente, si $G[A]$ es multipartita completa y $G[B]$ es bipartita. Si $G[B]$ tiene una sola componente, $G[B]$ es bipartita completa y $G \in M_2$. Si $G[B]$ tiene más de una componente, debe tener a $\overline{P_3}$ como subgráfica inducida. Luego, $G[A]$ debe ser libre de $K_3$ o $G$ tendría a $H$ como subráfica inducida. Así, $G[A]$ es bipartita. Como ambas son bipartitas, $G \in M_2$.

    \emph{Caso 2:} $G$ tiene exactamente una componente conexa no trivial y al menos una trivial.

    Como $G$ contiene al menos una componente trivial, la única partición que puede aceptar en dos gráficas multipartitas completas es una partición en un conjunto independiente y una gráfica multipartita completa. Luego, la componente no trivial de $G$, a la que llamaremos $G'$, debe de aceptar una partición en un conjunto independiente y una gráfica multipartita completa.

    Si $G'$ es bipartita, entonces acepta una partición en dos conjuntos independientes, y por lo tanto $G \in M_2$. Si $G'$ es una gráfica multipartita completa, $G \in M_2$. Si $G$ no es una gráfica bipartita ni multipartita completa, dado que es una cográfica, y por lo tanto una gráfica perfecta, contiene un $Paw$. Sea $y$ la raíz del coárbol de $G'$ y sea $z$, descendiente de $y$, el nodo más profundo que tiene un $Paw$ como subgráfica inducida, probemos por inducción sobre la distancia desde $y$ hasta $z$, $d$, que $G'[y]$ acepta una partición en un conjunto independiente y una gráfica multipartita completa.

    \textbf{Caso base}: $d = 0$. O bien, $y = z$.

    Notemos que $z$ tiene etiqueta 1, pues $Paw$ es una gráfica conexa. Dado que $z$ tiene etiqueta 1, todos sus hijos inducen gráficas multipartitas completas menos uno, $w$, que tiene etiqueta 0. Mostremos por contradicción que todos los hijos de $w$ inducen gráficas multipartitas completas. Supongamos que alguno de los hijos de $w$ contiene un $\overline{P_3}$. Como el nodo más profundo que contiene un $\overline{P_3}$ debe tener etiqueta 0, $w$ tiene un hijo de etiqueta 1 que tiene al menos 2 hijos, uno de los cuales contiene al $\overline{P_3}$ y el otro que tiene al menos un $K_1$. Luego, dicho hijo contiene un $Paw$, lo que es una contradicción. Si $w$ tiene un sólo hijo que no es un vértice, el resto de sus hijos forman un conjunto independiente, $C$. Si eliminamos este conjunto independiente, el único hijo de $w$ induce una gráfica multipartita completa, luego $G'[w] - C$ es una gráfica multipartita completa, entonces $G'[z] - C$ es una unión completa de de gráficas multipartitas completas y por lo tanto una gráfica multipartita completa. De esto se sigue que $G[z]$ acepta una partición en un conjunto independiente, $C$, y una gráfica multipartita completa. Si $w$ tiene al menos dos hijos no triviales, notemos que ninguno de ellos puede contener a $K_3$, o de lo contrario $w$ contendría a $K_2+K_3$ y $G$ no sería libre de $I$. Luego, todos los hijos de $w$ han de inducir gráficas bipartitas, es decir que $w$ induce también una gráfica bipartita. En otras palabras, $G'[w]$ acepta una partición en dos conjuntos independientes. Si sustraemos uno de estos conjuntos independientes, denoatdo como $D$, entonces $G'[w]-D$ es un conjunto independiente. Luego $G'[z]-D$ es la unión completa de gráficas multipartitas completas y un conjunto independiente, así $G'[z]-D$ es una gráfica multipartita completa. Luego, $G'[z]$ acepta una partición en un conjunto independiente, $D$ y una gráfica multipartita completa, $G'[z] - D$.

    Como en todos los casos $z$ acepta una partición en un conjunto independiente y una gráfica multipartita completa y $y = z$, entonces $y$ acepta la misma partición.

    \textbf{Paso inductivo}: $d \geq 2$.

    Notemos que $d$ siempre será par, ya que tanto $y$ como $z$ son nodos con etiqueta 1. Sea $k$ un entero tal que $k \geq 2$. Supongamos, como hipótesis inductiva, que si $G''$ es una cográfica conexa libre de $H, I$ y $J$ tal que la distancia, $d'$, entre la raíz, $y'$ de su coárbol y el nodo más profundo que contiene un $Paw$ es igual a $k-2$, entonces $G''$ acepta una partición en un conjunto independiente y una gráfica multipartita completa.

    Dado que $G'$ es libre de $J$, todos los hijos de $y$ menos uno inducen gráficas multipartitas completas. Dicho hijo, $v$, tiene etiqueta 0 y al menos uno de sus hijos debe de contener un $Paw$. Denotemos a dicho hijo como $u$. El resto de los hijos de $v$ deben de ser vértices, o de lo contrario, $G'[v]$ contendría a $K_2 + K_3$ como subgráfica inducida y por lo tanto $G$ contendría a $I$. Denotemos a este conjunto de vértices como $E$. Luego, $G'[u]$  es una cográfica que cumple con las condiciones de la hipótesis inductiva, por lo que acepta una partición en un conjunto independiente, $D$ y una gráfica multipartita completa. Si eliminamos de $u$ los vértices de $D$, entonces teneos que $G'[u] - D$ es una gráfica multipartita completa. Luego, si eliminamos también los vértices de $E$ de $G'[v]$, tenemos que $G'[v]-D-E$ es una gráfica multipartita completa. Luego, $G'-D-E$ es una unión completa de gráficas multipartitas completas por lo que también es una gráfica multipartita completa. Notemos que, dado que $v$ tiene etiqueta 0, no existen aristas entre los vértices en $D$ y los vértices en $E$, es decir que $D \cup E$ es un conjunto independiente. Así, $G'$ acepta una partición en un conjunto independiente, $D \cup E$ y una gráfica multipartita completa, $G' - D - E$.


    Como $G'$ acepta una partición en un conjunto independiente y una gráfica multipartita completa, entonces $G \in M_2$.


    \emph{Caso 3:} $G$ es un conjunto independiente con al menos dos vértices.

    Dado que $G$ es una gráfica multipartita completa, se sigue inmediatamente que está en $M_2$.

    \emph{Caso 4:} $G$ es conexa.

    Dado que toda cográfica inconexa libre de $H$, $I$ y $J$ acepta una partición en dos gráficas multipartitas completas, una cográfica conexa es o un vértice aislado o una unión de cográficas inconexas y la clase $M_2$ es cerrada bajo la unión completa, $G \in M_2$.


\end{proof}


    \subsection{Reconocimiento de la clase $M_2$}
        Haciendo uso  del Algoritmo \ref{alg_esta_en_clase} se puede determinar si una cográfica pertenece o no a la clase $M_2$. Como se especificó en la Sección \ref{sec_AlgoSub}, el tiempo de este algoritmo crece de forma lineal de acuerdo con el tamaño de la gráfica de entrada si encontramos primero todos los coárboles binarios de las obstrucciones de la clase. Como conocemos las obstrucciones mínimas de la clase $M_2$, que son finitas, se puede buscar cada una en tiempo lineal y por lo tanto se puede reconocer si una cográfica pertenece a la clase $M_2$ en tiempo lineal. El Algoritmo \ref{alg_decision}, que es una instancia del Algoritmo \ref{alg_esta_en_clase} corresponde a este proceso. Los árboles binarios de cada una de las obstrucciones de la clase $M_2$ se muestran en la Figura \ref{fig_obsts_bin}.

\begin{figure}[ht!]
\centering

\begin{subfigure}{0.85\textwidth}
\begin{tikzpicture}

\begin{scope}[xshift=0cm,scale=1]
\node [style=cotreenode] (1) at (1,1) {0};
\node [style=cotreenode] (2) at (0,0) {0};
\node [style=cotreenode] (3) at (2,0) {1};
\node [style=vertex] (4) at (-0.5,-1) {};
\node [style=cotreenode] (5) at (0.5,-1) {1};
\node [style=vertex] (6) at (1.5,-1) {};
\node [style=cotreenode] (7) at (2.5,-1) {1};
\node [style=vertex] (8) at (0.25,-2) {};
\node [style=vertex] (9) at (0.75,-2) {};
\node [style=vertex] (10) at (2.25,-2) {};
\node [style=vertex] (11) at (2.75,-2) {};
\foreach \i/\j in {1/2,1/3,2/4,2/5,3/6,3/7,5/8,5/9,7/10,7/11}
  \draw [style=edge] (\i) to (\j);
\node [below of=9,xshift=0.25cm] {\parbox{0.3\linewidth}{\subcaption*{$H_1$}}};
\end{scope}

\begin{scope}[xshift=4.5cm,scale=1]
\node [style=cotreenode] (1) at (1,1) {0};
\node [style=cotreenode] (2) at (0,0) {0};
\node [style=cotreenode] (3) at (2,0) {1};
\node [style=vertex] (4) at (-0.5,-1) {};
\node [style=cotreenode] (5) at (0.5,-1) {1};
\node [style=vertex] (6) at (1.5,-1) {};
\node [style=vertex] (7) at (2.5,-1) {};
\node [style=vertex] (8) at (0.125,-2) {};
\node [style=cotreenode] (9) at (0.875,-2) {1};
\node [style=vertex] (10) at (0.625,-3) {};
\node [style=vertex] (11) at (1.125,-3) {};
\foreach \i/\j in {1/2,1/3,2/4,2/5,3/6,3/7,5/8,5/9,9/10,9/11}
  \draw [style=edge] (\i) to (\j);
\node [below of=11] {\parbox{0.3\linewidth}{\subcaption*{$H_2$}}};
\end{scope}

\begin{scope}[xshift=9cm,scale=1]
\node [style=cotreenode] (1) at (1,1) {0};
\node [style=cotreenode] (2) at (0,0) {0};
\node [style=vertex] (3) at (2,0) {};
\node [style=cotreenode] (4) at (-0.5,-1) {1};
\node [style=cotreenode] (5) at (0.5,-1) {1};
\node [style=vertex] (8) at (0.125,-2) {};
\node [style=cotreenode] (9) at (0.875,-2) {1};
\node [style=vertex] (10) at (0.625,-3) {};
\node [style=vertex] (11) at (1.125,-3) {};
\node [style=vertex] (12) at (-0.75,-2) {};
\node [style=vertex] (13) at (-0.25,-2) {};
\foreach \i/\j in {1/2,1/3,2/4,2/5,5/8,5/9,9/10,9/11,4/12,4/13}
  \draw [style=edge] (\i) to (\j);
\node [below of=11] {\parbox{0.3\linewidth}{\subcaption*{$H_3$}}};
\end{scope}


\end{tikzpicture}
\end{subfigure}


\begin{subfigure}{0.6\textwidth}
\begin{tikzpicture}

\begin{scope}[xshift=0cm,scale=1]
\node [style=cotreenode] (1) at (1,1) {0};
\node [style=cotreenode] (2) at (0,0) {1};
\node [style=cotreenode] (3) at (2,0) {1};
\node [style=vertex] (4) at (-0.5,-1) {};
\node [style=vertex] (5) at (0.5,-1) {};
\node [style=vertex] (6) at (1.5,-1) {};
\node [style=cotreenode] (7) at (2.5,-1) {0};
\node [style=vertex] (8) at (2.125,-2) {};
\node [style=cotreenode] (9) at (2.875,-2) {1};
\node [style=vertex] (10) at (2.625,-3) {};
\node [style=vertex] (11) at (3.125,-3) {};

\foreach \i/\j in {1/2,1/3,2/4,2/5,3/6,3/7,7/8,7/9,9/10,9/11}
  \draw [style=edge] (\i) to (\j);
\node [below of=10,xshift=-1.625cm] {\parbox{0.3\linewidth}{\subcaption*{$I_1$}}};
\end{scope}

\begin{scope}[xshift=5cm,scale=1]
\node [style=cotreenode] (1) at (1,1) {0};
\node [style=vertex] (2) at (0,0) {};
\node [style=cotreenode] (3) at (2,0) {1};
\node [style=cotreenode] (4) at (1.25,-1) {0};
\node [style=cotreenode] (5) at (2.75,-1) {0};
\node [style=cotreenode] (6) at (0.825,-2) {1};
\node [style=vertex] (7) at (1.625,-2) {};
\node [style=vertex] (8) at (2.375,-2) {};
\node [style=cotreenode] (9) at (3.125,-2) {1};
\node [style=vertex] (10) at (0.575,-3) {};
\node [style=vertex] (11) at (1.075,-3) {};
\node [style=vertex] (12) at (2.875,-3) {};
\node [style=vertex] (13) at (3.375,-3) {};


\foreach \i/\j in {1/2,1/3,3/4,3/5,4/6,4/7,5/8,5/9,6/10,6/11,9/12,9/13}
  \draw [style=edge] (\i) to (\j);
\node [below of=12,xshift=-1.625cm] {\parbox{0.3\linewidth}{\subcaption*{$J_1$}}};
\end{scope}
\end{tikzpicture}
\end{subfigure}
\setlength{\abovecaptionskip}{5pt}
\caption{$H_1, H_2$ y $H_3$ son los coárboles binarios correspondientes a la obstrucción $H$. El coárbol binario $I_1$ es el único que corresponde a la obstrucción $I$. El coárbol binario $J_1$ es el único que corresponde a la obstrucción $J$.}\label{fig_obsts_bin}


\end{figure}


\begin{algorithm}[ht!]
\caption{Pertenece\_a\_M2}
\label{alg_decision}
\DontPrintSemicolon % Some LaTeX compilers require you to use \dontprintsemicolon instead
\KwIn{$g$, la raíz del coárbol de una cográfica $G$.}
\KwOut{$verdadero$ si $G$ pertenece a la clase hereditaria de cográficas $C$}

$g\_bin \gets \text{CrearÁrbolBinario}(g)$\;
$C\_bins \gets \{H_1, H_2, H_3, I_1, J_1\}$\;

\ForEach{$bin$ \textbf{\emph{en}} $C\_bins$}{
    $f = \text{Función\_de\_coasignación}(g\_bin,bin)$\;
    \If{$f(bin) = marcado$}{
        $\Return\ verdadero$\;
    }
}

$\Return\ falso$\;

\end{algorithm}




    \subsection{Algoritmo certificador}
        Si bien, el Algoritmo \ref{alg_decision} es capaz de identificar a las cográficas que pertenece a la clase $M_2$, no es posible determinar a partir de éste cuáles son las dos partes en las que se puede dividir una gráfica de la clase. La presente sección muestra un algoritmo (Algoritmo \ref{alg_cert_m2}) que, dada una cográfica representada por su coárbol, devuelve una coloración de las hojas de este último. Si la gráfica pertenece a $M_2$, las hojas tendrán dos colores, $verde$ y $azul$, cada uno de los cuales corresponde a una parte de la partición en dos gráficas multipartitas completas. En el caso contrario, las hojas correspondientes a los vértices que forman una obstrucción mínima tendrán un color que indique de qué obstrucción mínima se trata ($amarillo$ para $H$, $anaranjado$ para $I$ y $rojo$ para $J$). El Algoritmo \ref{alg_cert_m2} hace uso de los Algoritmos \ref{alg_cert_caso1} y \ref{alg_cert_caso2}, que funcionan de la misma manera para casos específicos del problema. La correctitud de estos algoritmo se sigue de la demostración del Teorema \ref{teo_obsts_m2}

\subsubsection{Algoritmo para reconocer gráficas bipartitas completas conexas}

El Algoritmo \ref{alg_bpc} es un algoritmo que resulta útil para los algoritmos subsecuentes. Éste recibe la raíz de un coárbol, $g$, y devuelve $verdadero$ si la gráfica representada por dicho coárbol es una gráfica bipartita completa conexa, coloreando los vértices de una parte de color $azul$ y los de la otra parte de $verde$. En el caso contrario, se colorean con $amarillo$  tres hojas cuyo ancestro común más profundo sea un nodo con etiqueta 1. Es decir que se colorean los vértices que inducen un $K_3$ en la gráfica. El bloque de la línea 9 a la 28 se ejecuta sólo si $g$ tiene exactamente dos hijos. En las líneas 10 a 18 se busca un $K_3$ en el primer hijo de $g$ y en las líneas 19 a 27 se busca en el segundo hijo. La Figura \ref{fig_bipartita} muestra el resultado de la ejecución de este algoritmo para algunos coárboles.

\begin{algorithm}[!htbp]
\caption{Es\_bipartita_completa}
\label{alg_bpc}

\DontPrintSemicolon % Some LaTeX compilers require you to use \dontprintsemicolon instead
\KwIn{$g$, la raíz de un coárbol, $G$}
\KwOut{Verdadero si la gráfica representada por $G$ es bipartita completa. Falso en el caso contrario. Las hojas de $árbol(g)$ se colorean.}

\If{g \text{es una hoja}}{
    $g.color \gets azul$\;
    $\Return\ verdadero$\;
}
\ElseIf{$g.etiqueta = 0$}{
    $\Return\ falso$\;
}
\ElseIf{$g.hijos.tamaño > 2$}{
    Marcar con amarillo: una hoja en cada uno de tres hijos diferentes de $g$\;
    $\Return\ falso$\;
}
\Else(\tcp*[h]{Hay exactamente dos hijos}){
    \If{g.hijos\emph{[0]} \text{es una hoja}}{
        $g.hijos[0].color \gets azul$\;
    }
    \Else{
        \For{gchild \textbf{\emph{en}} g.hijos\emph{[0]}.hijos}{
            \If{gchild \text{es una hoja}}{
                $gchild.color \gets azul$\;
            }
            \Else{
                Marcar con amarillo: dos hojas que tengan como ancestro común más profundo a $gchild$ y una hoja descendiente de $g.hijos[1]$\;
                $\Return\ falso$\;
            }
        }
    }
    \If{g.hijos\emph{[1]} \text{es una hoja}}{
        $g.hijos[1].color \gets verde$\;
    }
    \Else{
        \For{gchild \textbf{\emph{en}} g.hijos\emph{[1]}.hijos}{
            \If{gchild \text{es una hoja}}{
                $gchild.color \gets verde$\;
            }
            \Else{
                Marcar con amarillo: dos hojas que tengan como ancestro común más profundo a $gchild$ y una hoja descendiente de $g.hijos[0]$\;
                $\Return\ falso$\;
            }
        }
    }
}

$\Return\ verdadero$\;

\end{algorithm}

\begin{figure}[!htbp]
\begin{center}
\begin{tikzpicture}

\begin{scope}[xshift=0cm,scale=1]
\node [style=cotreenode] (1) at (1,3) {1};
\node [style=vertex, fill=blue] (2) at (0.5,2) {};
\node [style=vertex, fill=green] (3) at (1.5,2) {};
\foreach \i/\j in {1/2,1/3}
  \draw [style=edge] (\i) to (\j);
\end{scope}

\begin{scope}[xshift=2cm,scale=1]
\node [style=cotreenode] (1) at (1,3) {1};
\node [style=vertex, fill=blue] (2) at (0.5,2) {};
\node [style=cotreenode] (3) at (1.5,2) {0};
\node [style=vertex, fill=green] (4) at (1,1) {};
\node [style=vertex, fill=green] (5) at (1.5,1) {};
\node [style=vertex, fill=green] (6) at (2,1) {};
\foreach \i/\j in {1/2,1/3,3/4,3/5,3/6}
  \draw [style=edge] (\i) to (\j);
\end{scope}

\begin{scope}[xshift=4.75cm,scale=1]
\node [style=cotreenode] (1) at (1,3) {1};
\node [style=cotreenode] (2) at (0.5,2) {0};
\node [style=cotreenode] (3) at (1.5,2) {0};
\node [style=vertex, fill=blue] (4) at (0.25,1) {};
\node [style=vertex, fill=blue] (5) at (0.5,1) {};
\node [style=vertex, fill=blue] (6) at (0.75,1) {};
\node [style=vertex, fill=green] (7) at (1.25,1) {};
\node [style=vertex, fill=green] (8) at (1.5,1) {};
\node [style=vertex, fill=green] (9) at (1.75,1) {};

\foreach \i/\j in {1/2,1/3,2/4,2/5,2/6,3/7,3/8,3/9}
  \draw [style=edge] (\i) to (\j);
\end{scope}

\begin{scope}[xshift=7cm,scale=1]
\node [style=cotreenode] (1) at (1,3) {1};
\node [style=vertex, fill=yellow] (2) at (0.5,2) {};
\node [style=vertex, fill=yellow] (3) at (1,2) {};
\node [style=vertex, fill=yellow] (4) at (1.5,2) {};
\foreach \i/\j in {1/2,1/3,1/4}
  \draw [style=edge] (\i) to (\j);
\end{scope}

\begin{scope}[xshift=9.25cm,scale=1]
\node [style=cotreenode] (1) at (1,3) {1};
\node [style=cotreenode] (2) at (0.5,2) {0};
\node [style=cotreenode] (3) at (1.5,2) {0};
\node [style=vertex, fill=yellow] (4) at (0.25,1) {};
\node [style=vertex, fill=blue] (6) at (0.75,1) {};
\node [style=vertex, fill=green] (7) at (1.25,1) {};
\node [style=cotreenode] (9) at (1.75,1) {1};
\node [style=vertex, fill=yellow] (10) at (1.5,0) {};
\node [style=vertex, fill=yellow] (11) at (2,0) {};

\foreach \i/\j in {1/2,1/3,2/4,2/6,3/7,3/9,9/10,9/11}
  \draw [style=edge] (\i) to (\j);
\end{scope}

\end{tikzpicture}
\end{center}
\caption{Ejemplos del resultado de la ejecución del Algoritmo \ref{alg_bpc}.}
\label{fig_bipartita}
\end{figure}



\subsubsection{Caso 1}

El algoritmo \ref{alg_cert_caso1} corresponde al $Caso\ 1$ de la demostración del Teorema \ref{teo_obsts_m2}. Éste recibe como entrada la raíz de un coárbol que representa una cográfica inconexa que tiene al menos dos componentes conexas no triviales. En el bloque de las líneas 1 a 12 se aborda el caso en el que la gráfica tiene exactamente dos componentes conexas y se busca un $Paw$ que pueda formar la obstrucción $I$. En el bloque de las líneas 13 a 17 se aborda el caso en el que hay al menos 3 componente conexas y se busca un $K_3$ en cada componente para formar la obstrucción $H$. Si no se encuentra ninguna de las obstrucciones mínimas, se devuelve $verdadero$ y cada una de las hojas del coárbol tendrán color $azul$ o $verde$. Las Figuras \ref{fig_certificador_caso1_01} y \ref{fig_certificador_caso1_02} muestran la ejecución del algoritmo para gráficas sin ninguna de las obstrucciones. La Figura \ref{fig_certificador_caso1_03} muestra el resultado de la ejecución para tres gráficas, cada una de las cuales contiene una obstrucción.

\begin{algorithm}[!htbp]
\small
\caption{M2\_Caso\_1}
\label{alg_cert_caso1}

\DontPrintSemicolon % Some LaTeX compilers require you to use \dontprintsemicolon instead
\KwIn{$g$, la raíz de un coárbol con etiqueta 0 y al menos dos hijos que no son hojas}
\KwOut{Verdadero si $G$ pertenece a la clase $M_2$. Falso en el caso contrario. Las hojas de $árbol(g)$ se colorean.}

\If{g.hijos.tamaño = 2}{
    \For{gchild \textbf{\emph{en}} g.hijos\emph{[0]}}{
        \If{gchild \text{es una hoja}}{
            $gchild.color \gets azul$\;
        }
        \Else{
            \For{ggchild \textbf{\emph{en}} gchild.hijos}{
                \If{ggchild \text{es una hoja}}{
                    $gchild.color \gets azul$\;
                }
                \Else(\tcp*[h]{Se marca la obstrucción $I$}){
                    Marcar con anaranjado: una hoja en $ggchild.hijos[0]$, una hoja en $ggchild.hijos[1]$, una hoja en un hermano de $ggchild$, una hoja en un hermano de $gchild$ y dos hojas cuyo ancestro común más profundo sea el hermano de $g.hijos[0]$\;

                    $\Return\ falso$\;
                }
            }
        }
    }
    Repetir el procedimiento de las líneas 2 a 11 para $g.hijos[1]$, pero marcando con color $verde$ en vez de $azul$\;
}
\Else{
    \For{child \textbf{\emph{en}} g.hijos}{
        \If{\emph{Es\_bipartita\_completa(}$child$\emph{)} = falso}{
             Marcar con amarillo: dos hojas cuyo ancestro común más profundo sea un hermano de $child$ que no sea una hoja y una hoja en un hermano diferente\;
                        $\Return\ falso$\;
        }
    }
}


$\Return\ verdadero$\;

\end{algorithm}


\begin{figure}[!htbp]
\centering

\begin{subfigure}{\textwidth}
\centering
\begin{tikzpicture}
\begin{scope}[xshift=0cm,scale=1]
\node [style=cotreenode, fill=lightgray] (1) at (1.5,4) {0};
\node [style=cotreenode] (2) at (0.5,3) {1};
\node [style=cotreenode] (3) at (2.5,3) {1};
\node [style=vertex] (4) at (0,2) {};
\node [style=vertex] (6) at (1,2) {};
\node [style=vertex] (7) at (2,2) {};
\node [style=cotreenode] (8) at (3,2) {0};
\node [style=vertex] (9) at (2.5,1) {};
\node [style=vertex] (10) at (3,1) {};
\node [style=vertex] (11) at (3.5,1) {};
\foreach \i/\j in {1/2,1/3,2/4,2/6,3/7,3/8,8/9,8/10,8/11}
  \draw [style=edge] (\i) to (\j);
\end{scope}
\begin{scope}[xshift=4cm,scale=1]
\node [style=cotreenode, fill=lightgray] (1) at (1.5,4) {0};
\node [style=cotreenode, fill=lightgray] (2) at (0.5,3) {1};
\node [style=cotreenode] (3) at (2.5,3) {1};
\node [style=vertex] (4) at (0,2) {};
\node [style=vertex] (6) at (1,2) {};
\node [style=vertex] (7) at (2,2) {};
\node [style=cotreenode] (8) at (3,2) {0};
\node [style=vertex] (9) at (2.5,1) {};
\node [style=vertex] (10) at (3,1) {};
\node [style=vertex] (11) at (3.5,1) {};
\foreach \i/\j in {1/2,1/3,2/4,2/6,3/7,3/8,8/9,8/10,8/11}
  \draw [style=edge] (\i) to (\j);
\end{scope}
\begin{scope}[xshift=8cm,scale=1]
\node [style=cotreenode, fill=lightgray] (1) at (1.5,4) {0};
\node [style=cotreenode, fill=lightgray] (2) at (0.5,3) {1};
\node [style=cotreenode] (3) at (2.5,3) {1};
\node [style=vertex, fill=lightgray] (4) at (0,2) {};
\node [style=vertex] (6) at (1,2) {};
\node [style=vertex] (7) at (2,2) {};
\node [style=cotreenode] (8) at (3,2) {0};
\node [style=vertex] (9) at (2.5,1) {};
\node [style=vertex] (10) at (3,1) {};
\node [style=vertex] (11) at (3.5,1) {};
\foreach \i/\j in {1/2,1/3,2/4,2/6,3/7,3/8,8/9,8/10,8/11}
  \draw [style=edge] (\i) to (\j);
\end{scope}
\end{tikzpicture}
\end{subfigure}

\begin{subfigure}{\textwidth}
\centering
\begin{tikzpicture}
\begin{scope}[xshift=0cm,scale=1]
\node [style=cotreenode, fill=lightgray] (1) at (1.5,4) {0};
\node [style=cotreenode, fill=lightgray] (2) at (0.5,3) {1};
\node [style=cotreenode] (3) at (2.5,3) {1};
\node [style=vertex, fill=blue] (4) at (0,2) {};
\node [style=vertex, fill=lightgray] (6) at (1,2) {};
\node [style=vertex] (7) at (2,2) {};
\node [style=cotreenode] (8) at (3,2) {0};
\node [style=vertex] (9) at (2.5,1) {};
\node [style=vertex] (10) at (3,1) {};
\node [style=vertex] (11) at (3.5,1) {};
\foreach \i/\j in {1/2,1/3,2/4,2/6,3/7,3/8,8/9,8/10,8/11}
  \draw [style=edge] (\i) to (\j);
\end{scope}
\begin{scope}[xshift=4cm,scale=1]
\node [style=cotreenode, fill=lightgray] (1) at (1.5,4) {0};
\node [style=cotreenode] (2) at (0.5,3) {1};
\node [style=cotreenode, fill=lightgray] (3) at (2.5,3) {1};
\node [style=vertex, fill=blue] (4) at (0,2) {};
\node [style=vertex, fill=blue] (6) at (1,2) {};
\node [style=vertex] (7) at (2,2) {};
\node [style=cotreenode] (8) at (3,2) {0};
\node [style=vertex] (9) at (2.5,1) {};
\node [style=vertex] (10) at (3,1) {};
\node [style=vertex] (11) at (3.5,1) {};
\foreach \i/\j in {1/2,1/3,2/4,2/6,3/7,3/8,8/9,8/10,8/11}
  \draw [style=edge] (\i) to (\j);
\end{scope}
\begin{scope}[xshift=8cm,scale=1]
\node [style=cotreenode, fill=lightgray] (1) at (1.5,4) {0};
\node [style=cotreenode] (2) at (0.5,3) {1};
\node [style=cotreenode, fill=lightgray] (3) at (2.5,3) {1};
\node [style=vertex, fill=blue] (4) at (0,2) {};
\node [style=vertex, fill=blue] (6) at (1,2) {};
\node [style=vertex, fill=lightgray] (7) at (2,2) {};
\node [style=cotreenode] (8) at (3,2) {0};
\node [style=vertex] (9) at (2.5,1) {};
\node [style=vertex] (10) at (3,1) {};
\node [style=vertex] (11) at (3.5,1) {};
\foreach \i/\j in {1/2,1/3,2/4,2/6,3/7,3/8,8/9,8/10,8/11}
  \draw [style=edge] (\i) to (\j);
\end{scope}
\end{tikzpicture}
\end{subfigure}

\begin{subfigure}{\textwidth}
\centering
\begin{tikzpicture}
\begin{scope}[xshift=0cm,scale=1]
\node [style=cotreenode, fill=lightgray] (1) at (1.5,4) {0};
\node [style=cotreenode] (2) at (0.5,3) {1};
\node [style=cotreenode, fill=lightgray] (3) at (2.5,3) {1};
\node [style=vertex, fill=blue] (4) at (0,2) {};
\node [style=vertex, fill=blue] (6) at (1,2) {};
\node [style=vertex, fill=green] (7) at (2,2) {};
\node [style=cotreenode, fill=lightgray] (8) at (3,2) {0};
\node [style=vertex] (9) at (2.5,1) {};
\node [style=vertex] (10) at (3,1) {};
\node [style=vertex] (11) at (3.5,1) {};
\foreach \i/\j in {1/2,1/3,2/4,2/6,3/7,3/8,8/9,8/10,8/11}
  \draw [style=edge] (\i) to (\j);
\end{scope}
\begin{scope}[xshift=4cm,scale=1]
\node [style=cotreenode, fill=lightgray] (1) at (1.5,4) {0};
\node [style=cotreenode] (2) at (0.5,3) {1};
\node [style=cotreenode, fill=lightgray] (3) at (2.5,3) {1};
\node [style=vertex, fill=blue] (4) at (0,2) {};
\node [style=vertex, fill=blue] (6) at (1,2) {};
\node [style=vertex, fill=green] (7) at (2,2) {};
\node [style=cotreenode, fill=lightgray] (8) at (3,2) {0};
\node [style=vertex, fill=lightgray] (9) at (2.5,1) {};
\node [style=vertex] (10) at (3,1) {};
\node [style=vertex] (11) at (3.5,1) {};
\foreach \i/\j in {1/2,1/3,2/4,2/6,3/7,3/8,8/9,8/10,8/11}
  \draw [style=edge] (\i) to (\j);
\end{scope}
\begin{scope}[xshift=8cm,scale=1]
\node [style=cotreenode, fill=lightgray] (1) at (1.5,4) {0};
\node [style=cotreenode] (2) at (0.5,3) {1};
\node [style=cotreenode, fill=lightgray] (3) at (2.5,3) {1};
\node [style=vertex, fill=blue] (4) at (0,2) {};
\node [style=vertex, fill=blue] (6) at (1,2) {};
\node [style=vertex, fill=green] (7) at (2,2) {};
\node [style=cotreenode, fill=lightgray] (8) at (3,2) {0};
\node [style=vertex, fill=green] (9) at (2.5,1) {};
\node [style=vertex, fill=lightgray] (10) at (3,1) {};
\node [style=vertex] (11) at (3.5,1) {};
\foreach \i/\j in {1/2,1/3,2/4,2/6,3/7,3/8,8/9,8/10,8/11}
  \draw [style=edge] (\i) to (\j);
\end{scope}
\end{tikzpicture}
\end{subfigure}

\begin{subfigure}{\textwidth}
\centering
\begin{tikzpicture}
\begin{scope}[xshift=0cm,scale=1]
\node [style=cotreenode, fill=lightgray] (1) at (1.5,4) {0};
\node [style=cotreenode] (2) at (0.5,3) {1};
\node [style=cotreenode, fill=lightgray] (3) at (2.5,3) {1};
\node [style=vertex, fill=blue] (4) at (0,2) {};
\node [style=vertex, fill=blue] (6) at (1,2) {};
\node [style=vertex, fill=green] (7) at (2,2) {};
\node [style=cotreenode, fill=lightgray] (8) at (3,2) {0};
\node [style=vertex, fill=green] (9) at (2.5,1) {};
\node [style=vertex, fill=green] (10) at (3,1) {};
\node [style=vertex, fill=lightgray] (11) at (3.5,1) {};
\foreach \i/\j in {1/2,1/3,2/4,2/6,3/7,3/8,8/9,8/10,8/11}
  \draw [style=edge] (\i) to (\j);
\end{scope}
\begin{scope}[xshift=4cm,scale=1]
\node [style=cotreenode, fill=lightgray] (1) at (1.5,4) {0};
\node [style=cotreenode] (2) at (0.5,3) {1};
\node [style=cotreenode, fill=lightgray] (3) at (2.5,3) {1};
\node [style=vertex, fill=blue] (4) at (0,2) {};
\node [style=vertex, fill=blue] (6) at (1,2) {};
\node [style=vertex, fill=green] (7) at (2,2) {};
\node [style=cotreenode, fill=lightgray] (8) at (3,2) {0};
\node [style=vertex, fill=green] (9) at (2.5,1) {};
\node [style=vertex, fill=green] (10) at (3,1) {};
\node [style=vertex, fill=green] (11) at (3.5,1) {};
\foreach \i/\j in {1/2,1/3,2/4,2/6,3/7,3/8,8/9,8/10,8/11}
  \draw [style=edge] (\i) to (\j);
\end{scope}
\end{tikzpicture}
\end{subfigure}

\caption{Ejemplo de la ejecución del Algoritmo \ref{alg_cert_caso1}. Se muestran en color gris los nodos del árbol que están siendo procesados. Los colores de las hojas corresponden a los colores que asigna el algoritmo.}
\label{fig_certificador_caso1_01}
\end{figure}

\begin{figure}[!htbp]
\centering
\begin{subfigure}{\textwidth}
\centering
\begin{tikzpicture}
\begin{scope}[xshift=0cm,scale=1]
\node [style=cotreenode, fill=lightgray] (1) at (2,4) {0};
\node [style=cotreenode] (2) at (0.5,3) {1};
\node [style=vertex] (3) at (2,3) {};
\node [style=cotreenode] (4) at (3.5,3) {1};
\node [style=vertex] (5) at (0,2) {};
\node [style=vertex] (6) at (1,2) {};
\node [style=cotreenode] (7) at (2.75,2) {0};
\node [style=cotreenode] (8) at (4.25,2) {0};
\node [style=vertex] (9) at (2.25,1) {};
\node [style=vertex] (10) at (2.75,1) {};
\node [style=vertex] (11) at (3.25,1) {};
\node [style=vertex] (12) at (3.75,1) {};
\node [style=vertex] (13) at (4.25,1) {};
\node [style=vertex] (14) at (4.75,1) {};
\foreach \i/\j in {1/2,1/3,1/4,2/5,2/6,4/7,4/8,7/9,7/10,7/11,8/12,8/13,8/14}
  \draw [style=edge] (\i) to (\j);
\end{scope}
\begin{scope}[xshift=6cm,scale=1]
\node [style=cotreenode, fill=lightgray] (1) at (2,4) {0};
\node [style=cotreenode, fill=lightgray] (2) at (0.5,3) {1};
\node [style=vertex] (3) at (2,3) {};
\node [style=cotreenode] (4) at (3.5,3) {1};
\node [style=vertex] (5) at (0,2) {};
\node [style=vertex] (6) at (1,2) {};
\node [style=cotreenode] (7) at (2.75,2) {0};
\node [style=cotreenode] (8) at (4.25,2) {0};
\node [style=vertex] (9) at (2.25,1) {};
\node [style=vertex] (10) at (2.75,1) {};
\node [style=vertex] (11) at (3.25,1) {};
\node [style=vertex] (12) at (3.75,1) {};
\node [style=vertex] (13) at (4.25,1) {};
\node [style=vertex] (14) at (4.75,1) {};
\foreach \i/\j in {1/2,1/3,1/4,2/5,2/6,4/7,4/8,7/9,7/10,7/11,8/12,8/13,8/14}
  \draw [style=edge] (\i) to (\j);
\end{scope}
\end{tikzpicture}
\end{subfigure}

\begin{subfigure}{\textwidth}
\centering
\begin{tikzpicture}
\begin{scope}[xshift=0cm,scale=1]
\node [style=cotreenode, fill=lightgray] (1) at (2,4) {0};
\node [style=cotreenode] (2) at (0.5,3) {1};
\node [style=vertex, fill=lightgray] (3) at (2,3) {};
\node [style=cotreenode] (4) at (3.5,3) {1};
\node [style=vertex, fill=blue] (5) at (0,2) {};
\node [style=vertex, fill=green] (6) at (1,2) {};
\node [style=cotreenode] (7) at (2.75,2) {0};
\node [style=cotreenode] (8) at (4.25,2) {0};
\node [style=vertex] (9) at (2.25,1) {};
\node [style=vertex] (10) at (2.75,1) {};
\node [style=vertex] (11) at (3.25,1) {};
\node [style=vertex] (12) at (3.75,1) {};
\node [style=vertex] (13) at (4.25,1) {};
\node [style=vertex] (14) at (4.75,1) {};
\foreach \i/\j in {1/2,1/3,1/4,2/5,2/6,4/7,4/8,7/9,7/10,7/11,8/12,8/13,8/14}
  \draw [style=edge] (\i) to (\j);
\end{scope}
\begin{scope}[xshift=6cm,scale=1]
\node [style=cotreenode, fill=lightgray] (1) at (2,4) {0};
\node [style=cotreenode] (2) at (0.5,3) {1};
\node [style=vertex, fill=blue] (3) at (2,3) {};
\node [style=cotreenode, fill=lightgray] (4) at (3.5,3) {1};
\node [style=vertex, fill=blue] (5) at (0,2) {};
\node [style=vertex, fill=green] (6) at (1,2) {};
\node [style=cotreenode] (7) at (2.75,2) {0};
\node [style=cotreenode] (8) at (4.25,2) {0};
\node [style=vertex] (9) at (2.25,1) {};
\node [style=vertex] (10) at (2.75,1) {};
\node [style=vertex] (11) at (3.25,1) {};
\node [style=vertex] (12) at (3.75,1) {};
\node [style=vertex] (13) at (4.25,1) {};
\node [style=vertex] (14) at (4.75,1) {};
\foreach \i/\j in {1/2,1/3,1/4,2/5,2/6,4/7,4/8,7/9,7/10,7/11,8/12,8/13,8/14}
  \draw [style=edge] (\i) to (\j);
\end{scope}
\end{tikzpicture}
\end{subfigure}

\begin{subfigure}{\textwidth}
\centering
\begin{tikzpicture}
\begin{scope}[xshift=0cm,scale=1]
\node [style=cotreenode, fill=lightgray] (1) at (2,4) {0};
\node [style=cotreenode] (2) at (0.5,3) {1};
\node [style=vertex, fill=blue] (3) at (2,3) {};
\node [style=cotreenode, fill=lightgray] (4) at (3.5,3) {1};
\node [style=vertex, fill=blue] (5) at (0,2) {};
\node [style=vertex, fill=green] (6) at (1,2) {};
\node [style=cotreenode] (7) at (2.75,2) {0};
\node [style=cotreenode] (8) at (4.25,2) {0};
\node [style=vertex, fill=blue] (9) at (2.25,1) {};
\node [style=vertex, fill=blue] (10) at (2.75,1) {};
\node [style=vertex, fill=blue] (11) at (3.25,1) {};
\node [style=vertex, fill=green] (12) at (3.75,1) {};
\node [style=vertex, fill=green] (13) at (4.25,1) {};
\node [style=vertex, fill=green] (14) at (4.75,1) {};
\foreach \i/\j in {1/2,1/3,1/4,2/5,2/6,4/7,4/8,7/9,7/10,7/11,8/12,8/13,8/14}
  \draw [style=edge] (\i) to (\j);
\end{scope}
\end{tikzpicture}
\end{subfigure}

\caption{Ejemplo de la ejecución del Algoritmo \ref{alg_cert_caso1}. Se muestran en color gris los nodos del árbol que están siendo procesados. Los colores de las hojas corresponden a los colores que asigna el algoritmo.}
\label{fig_certificador_caso1_02}
\end{figure}

\begin{figure}[!htbp]
\centering
\begin{subfigure}{\textwidth}
\centering
\begin{tikzpicture}
\begin{scope}[xshift=0cm,scale=1]
\node [style=cotreenode] (1) at (1.5,4) {0};
\node [style=cotreenode] (2) at (0.5,3) {1};
\node [style=cotreenode] (4) at (2.5,3) {1};
\node [style=vertex, fill=orange] (5) at (0,2) {};
\node [style=vertex, fill=blue] (6) at (1,2) {};
\node [style=vertex, fill=orange] (7) at (1.75,2) {};
\node [style=cotreenode] (8) at (3.25,2) {0};
\node [style=vertex, fill=orange] (12) at (2.75,1) {};
\node [style=cotreenode] (14) at (3.75,1) {1};
\node [style=vertex, fill=orange] (15) at (3.5,0) {};
\node [style=vertex, fill=orange] (16) at (4,0) {};
\foreach \i/\j in {1/2,1/4,2/5,2/6,4/7,4/8,8/12,8/14,14/15,14/16}
  \draw [style=edge] (\i) to (\j);
\end{scope}
\begin{scope}[xshift=4.5cm,scale=1]
\node [style=cotreenode] (1) at (1.5,4) {0};
\node [style=cotreenode] (2) at (0.5,3) {1};
\node [style=vertex, fill=yellow] (3) at (1.5,3) {};
\node [style=cotreenode] (4) at (2.5,3) {1};
\node [style=vertex, fill=yellow] (5) at (0,2) {};
\node [style=vertex, fill=yellow] (6) at (1,2) {};
\node [style=vertex, fill=yellow] (7) at (2,2) {};
\node [style=vertex, fill=yellow] (8) at (2.5,2) {};
\node [style=vertex, fill=yellow] (9) at (3,2) {};
\foreach \i/\j in {1/2,1/3,1/4,2/5,2/6,4/7,4/8,4/9}
  \draw [style=edge] (\i) to (\j);
\end{scope}
\begin{scope}[xshift=8.5cm,scale=1]
\node [style=cotreenode] (1) at (1.5,4) {0};
\node [style=cotreenode] (2) at (0.5,3) {1};
\node [style=vertex, fill=yellow] (3) at (1.5,3) {};
\node [style=cotreenode] (4) at (2.5,3) {1};
\node [style=vertex, fill=yellow] (5) at (0,2) {};
\node [style=vertex, fill=yellow] (6) at (1,2) {};
\node [style=vertex, fill=yellow] (7) at (1.75,2) {};
\node [style=cotreenode] (8) at (3.25,2) {0};
\node [style=vertex, fill=green] (12) at (2.75,1) {};
\node [style=cotreenode] (14) at (3.75,1) {1};
\node [style=vertex, fill=yellow] (15) at (3.5,0) {};
\node [style=vertex, fill=yellow] (16) at (4,0) {};
\foreach \i/\j in {1/2,1/3,1/4,2/5,2/6,4/7,4/8,8/12,8/14,14/15,14/16}
  \draw [style=edge] (\i) to (\j);
\end{scope}
\end{tikzpicture}
\end{subfigure}


\caption{Ejemplos del resultado de la ejecución del Algoritmo \ref{alg_cert_caso1} en los que se encuentra una obstrucción.}
\label{fig_certificador_caso1_03}
\end{figure}

\subsubsection{Caso 2}

El algoritmo \ref{alg_cert_caso2} corresponde al $Caso\ 2$ de la demostración del Teorema \ref{teo_obsts_m2}. Éste recibe como entrada la raíz, $g$, de un coárbol que representa una cográfica inconexa que tiene exactamente una componente conexa no trivial y al menos una trivial. En el bloque de las líneas 6 a 28 se procesa el hijo de $g$ que no es una hoja. En las líneas 7 a 19 se procesan los nietos de $g$ y se registra si alguno tiene un hijo que no sea una hoja (es decir que dicho nieto de $g$ corresponde a una gráfica no multipartita completa) en la variable $aux\_gchild$. La cantidad de hijos diferentes de una hoja de éste se registra en $ggchildren\_no\_hojas$. Si hay más de un nieto que tenga hijos que no son hojas, se marca la obstrucción $J$ (Línea 18). Una vez procesados los nietos de $g$, se decide cómo será procesado el nieto de $g$ que no corresponde a una gráfica multipartita completa. Si tal hijo no existe, la partición ya se habrá hecho (Líneas 20 y 21), esto corresponde a una parte del caso base del $Caso\_2$ de la demostración del Teorema \ref{teo_obsts_m2}. Si dicho nieto tiene un solo hijo que no es una hoja, se procesa recursivamente (Líneas 22 y 23), esto corresponde al paso inductivo del $Caso\_2$ de la demostración ya mencionada. Y finalmente, si tiene más de un hijo que no es una hoja, se busca que todos estos hijos sean bipartitas, esta es la otra parte del caso base del $Caso\_2$ de la demostración. La Figura \ref{fig_certificador_caso2_01} muestra un ejemplo de la ejecución del algoritmo para un coárbol cuya cográfica no contiene a ninguna de las obstrucciones mínimas de $M_2$. La Figura \ref{fig_certificador_caso2_02} muestra el resultado de la ejecución para coárboles que contienen una obstrucción.




\begin{algorithm}[!htbp]
\SetInd{1pt}{10pt}
\footnotesize
\caption{M2\_Caso\_2}
\label{alg_cert_caso2}

\DontPrintSemicolon % Some LaTeX compilers require you to use \dontprintsemicolon instead
\KwIn{$g$, la raíz de un coárbol con etiqueta 0 que tiene exactamente un hijo que no es una hoja y al menos uno que es una hoja}
\KwOut{Verdadero si $G$ peretenece a la clase $M_2$. Falso en el caso contrario. Las hojas de $G$ se colorean.}


    $aux\_gchild \gets null$\;
    $ggchildren\_no\_hojas \gets 0$\;

    \For{child \textbf{\emph{en}} $g.hijos$}{
        \If{child \text{es una hoja}}{
            $child.color \gets azul$\;
        }
        \Else(\tcp*[h]{Sólo se ejecuta una vez}){
            \For{gchild \textbf{\emph{en}} $child.hijos$}{
                \If{gchild \text{es una hoja}}{
                    $gchild.color \gets verde$\;
                }
                \Else{
                    \For{ggchild \textbf{\emph{en}} $gchild.hijos$}{
                        \If{ggchild \text{es una hoja}}{
                            $ggchild.color \gets verde$\;
                        }
                        \ElseIf{$aux\_gchild = null \emph{\textbf{ o }} aux\_gchild = gchild$}{
                            $aux\_gchild \gets gchild$\;
                            $ggchildren\_no\_hojas \gets ggchildren\_no\_hojas + 1$\;
                        }
                        \Else{
                            Marcar con rojo: Un hijo de $g$ que sea una hoja, dos hojas cuyo ancestro común más profundo sea $ggchild$, una hoja en un hermano de $ggchild$, dos hojas cuyo ancestro común más profundo sea un hijo de $aux\_gchild$ que no es una hoja y una hoja en un hijo de $aux\_gchild$ diferente del anterior\;
                            $\Return\ falso$\;
                        }
                    }
                }
            }

            \If{ggchildren\_no\_hojas = 0}{
                $\Return\ verdadero$\;
            }
            \ElseIf{ggchildren\_no\_hojas = 1}{
                $\Return$ M2\_Caso\_2($aux\_gchild$)\;
            }
            \Else{
                \For{ggchild \textbf{\emph{en}} aux\_gchild}{
                    \If{\emph{Es\_bipartita_completa(}$ggchild$\emph{)} = falso}{
                        Marcar con amarillo: dos hojas cuyo ancestro común más profundo sea un hermano de $ggchild$ que no sea una hoja y un hijo de $g$ que sea una hoja\;
                        $\Return\ falso$\;
                    }
                }
            }

        }
    }

    $\Return\ verdadero$\;


\end{algorithm}

\begin{figure}[!htbp]
\centering

\begin{subfigure}{\textwidth}
\centering
\begin{tikzpicture}
\begin{scope}[xshift=0cm,scale=1]
\node [style=cotreenode, fill=lightgray] (1) at (3.5,5) {0};
\node [style=vertex] (2) at (0.5,4) {};
\node [style=vertex] (3) at (1.5,4) {};
\node [style=vertex] (4) at (2.5,4) {};
\node [style=cotreenode] (5) at (3.5,4) {1};
\node [style=vertex] (6) at (0.5,3) {};
\node [style=cotreenode] (7) at (2,3) {0};
\node [style=cotreenode] (8) at (3.5,3) {0};
\node [style=vertex] (9) at (1.75,2) {};
\node [style=vertex] (10) at (2.25,2) {};
\node [style=vertex] (11) at (2.75,2) {};
\node [style=cotreenode] (12) at (3.5,2) {1};
\node [style=cotreenode] (13) at (4.5,2) {1};
\node [style=vertex] (14) at (3.25,1) {};
\node [style=vertex] (15) at (3.75,1) {};
\node [style=vertex] (16) at (4.25,1) {};
\node [style=vertex] (17) at (4.75,1) {};
\foreach \i/\j in {1/2,1/3,1/4,1/5,5/6,5/7,5/8,7/9,7/10,8/11,8/12,8/13,12/14,12/15,13/16,13/17}
  \draw [style=edge] (\i) to (\j);
\end{scope}
\begin{scope}[xshift=5cm,scale=1]
\node [style=cotreenode, fill=lightgray] (1) at (3.5,5) {0};
\node [style=vertex, fill=blue] (2) at (0.5,4) {};
\node [style=vertex, fill=blue] (3) at (1.5,4) {};
\node [style=vertex, fill=blue] (4) at (2.5,4) {};
\node [style=cotreenode, fill=lightgray] (5) at (3.5,4) {1};
\node [style=vertex] (6) at (0.5,3) {};
\node [style=cotreenode] (7) at (2,3) {0};
\node [style=cotreenode] (8) at (3.5,3) {0};
\node [style=vertex] (9) at (1.75,2) {};
\node [style=vertex] (10) at (2.25,2) {};
\node [style=vertex] (11) at (2.75,2) {};
\node [style=cotreenode] (12) at (3.5,2) {1};
\node [style=cotreenode] (13) at (4.5,2) {1};
\node [style=vertex] (14) at (3.25,1) {};
\node [style=vertex] (15) at (3.75,1) {};
\node [style=vertex] (16) at (4.25,1) {};
\node [style=vertex] (17) at (4.75,1) {};
\foreach \i/\j in {1/2,1/3,1/4,1/5,5/6,5/7,5/8,7/9,7/10,8/11,8/12,8/13,12/14,12/15,13/16,13/17}
  \draw [style=edge] (\i) to (\j);
\end{scope}
\begin{scope}[xshift=10cm,scale=1]
\node [style=cotreenode, fill=lightgray] (1) at (3.5,5) {0};
\node [style=vertex, fill=blue] (2) at (0.5,4) {};
\node [style=vertex, fill=blue] (3) at (1.5,4) {};
\node [style=vertex, fill=blue] (4) at (2.5,4) {};
\node [style=cotreenode, fill=lightgray] (5) at (3.5,4) {1};
\node [style=vertex, fill=green] (6) at (0.5,3) {};
\node [style=cotreenode, fill=lightgray] (7) at (2,3) {0};
\node [style=cotreenode] (8) at (3.5,3) {0};
\node [style=vertex] (9) at (1.75,2) {};
\node [style=vertex] (10) at (2.25,2) {};
\node [style=vertex] (11) at (2.75,2) {};
\node [style=cotreenode] (12) at (3.5,2) {1};
\node [style=cotreenode] (13) at (4.5,2) {1};
\node [style=vertex] (14) at (3.25,1) {};
\node [style=vertex] (15) at (3.75,1) {};
\node [style=vertex] (16) at (4.25,1) {};
\node [style=vertex] (17) at (4.75,1) {};
\foreach \i/\j in {1/2,1/3,1/4,1/5,5/6,5/7,5/8,7/9,7/10,8/11,8/12,8/13,12/14,12/15,13/16,13/17}
  \draw [style=edge] (\i) to (\j);
\end{scope}
\end{tikzpicture}
\end{subfigure}
\begin{subfigure}{\textwidth}
\centering
\begin{tikzpicture}
\begin{scope}[xshift=0cm,scale=1]
\node [style=cotreenode, fill=lightgray] (1) at (3.5,5) {0};
\node [style=vertex, fill=blue] (2) at (0.5,4) {};
\node [style=vertex, fill=blue] (3) at (1.5,4) {};
\node [style=vertex, fill=blue] (4) at (2.5,4) {};
\node [style=cotreenode, fill=lightgray] (5) at (3.5,4) {1};
\node [style=vertex, fill=green] (6) at (0.5,3) {};
\node [style=cotreenode] (7) at (2,3) {0};
\node [style=cotreenode, fill=lightgray] (8) at (3.5,3) {0};
\node [style=vertex, fill=green] (9) at (1.75,2) {};
\node [style=vertex, fill=green] (10) at (2.25,2) {};
\node [style=vertex] (11) at (2.75,2) {};
\node [style=cotreenode] (12) at (3.5,2) {1};
\node [style=cotreenode] (13) at (4.5,2) {1};
\node [style=vertex] (14) at (3.25,1) {};
\node [style=vertex] (15) at (3.75,1) {};
\node [style=vertex] (16) at (4.25,1) {};
\node [style=vertex] (17) at (4.75,1) {};
\foreach \i/\j in {1/2,1/3,1/4,1/5,5/6,5/7,5/8,7/9,7/10,8/11,8/12,8/13,12/14,12/15,13/16,13/17}
  \draw [style=edge] (\i) to (\j);
\end{scope}
\begin{scope}[xshift=5cm,scale=1]
\node [style=cotreenode, fill=lightgray] (1) at (3.5,5) {0};
\node [style=vertex, fill=blue] (2) at (0.5,4) {};
\node [style=vertex, fill=blue] (3) at (1.5,4) {};
\node [style=vertex, fill=blue] (4) at (2.5,4) {};
\node [style=cotreenode, fill=lightgray] (5) at (3.5,4) {1};
\node [style=vertex, fill=green] (6) at (0.5,3) {};
\node [style=cotreenode] (7) at (2,3) {0};
\node [style=cotreenode, fill=lightgray] (8) at (3.5,3) {0};
\node [style=vertex, fill=green] (9) at (1.75,2) {};
\node [style=vertex, fill=green] (10) at (2.25,2) {};
\node [style=vertex, fill=green] (11) at (2.75,2) {};
\node [style=cotreenode, fill=lightgray] (12) at (3.5,2) {1};
\node [style=cotreenode] (13) at (4.5,2) {1};
\node [style=vertex] (14) at (3.25,1) {};
\node [style=vertex] (15) at (3.75,1) {};
\node [style=vertex] (16) at (4.25,1) {};
\node [style=vertex] (17) at (4.75,1) {};
\foreach \i/\j in {1/2,1/3,1/4,1/5,5/6,5/7,5/8,7/9,7/10,8/11,8/12,8/13,12/14,12/15,13/16,13/17}
  \draw [style=edge] (\i) to (\j);
\end{scope}
\begin{scope}[xshift=10cm,scale=1]
\node [style=cotreenode, fill=lightgray] (1) at (3.5,5) {0};
\node [style=vertex, fill=blue] (2) at (0.5,4) {};
\node [style=vertex, fill=blue] (3) at (1.5,4) {};
\node [style=vertex, fill=blue] (4) at (2.5,4) {};
\node [style=cotreenode, fill=lightgray] (5) at (3.5,4) {1};
\node [style=vertex, fill=green] (6) at (0.5,3) {};
\node [style=cotreenode] (7) at (2,3) {0};
\node [style=cotreenode, fill=lightgray] (8) at (3.5,3) {0};
\node [style=vertex, fill=green] (9) at (1.75,2) {};
\node [style=vertex, fill=green] (10) at (2.25,2) {};
\node [style=vertex, fill=green] (11) at (2.75,2) {};
\node [style=cotreenode] (12) at (3.5,2) {1};
\node [style=cotreenode, fill=lightgray] (13) at (4.5,2) {1};
\node [style=vertex] (14) at (3.25,1) {};
\node [style=vertex] (15) at (3.75,1) {};
\node [style=vertex] (16) at (4.25,1) {};
\node [style=vertex] (17) at (4.75,1) {};
\foreach \i/\j in {1/2,1/3,1/4,1/5,5/6,5/7,5/8,7/9,7/10,8/11,8/12,8/13,12/14,12/15,13/16,13/17}
  \draw [style=edge] (\i) to (\j);
\end{scope}
\end{tikzpicture}
\end{subfigure}

\begin{subfigure}{\textwidth}
\centering
\begin{tikzpicture}
\begin{scope}[xshift=0cm,scale=1]
\node [style=cotreenode, fill=lightgray] (1) at (3.5,5) {0};
\node [style=vertex, fill=blue] (2) at (0.5,4) {};
\node [style=vertex, fill=blue] (3) at (1.5,4) {};
\node [style=vertex, fill=blue] (4) at (2.5,4) {};
\node [style=cotreenode, fill=lightgray] (5) at (3.5,4) {1};
\node [style=vertex, fill=green] (6) at (0.5,3) {};
\node [style=cotreenode] (7) at (2,3) {0};
\node [style=cotreenode, fill=lightgray] (8) at (3.5,3) {0};
\node [style=vertex, fill=green] (9) at (1.75,2) {};
\node [style=vertex, fill=green] (10) at (2.25,2) {};
\node [style=vertex, fill=green] (11) at (2.75,2) {};
\node [style=cotreenode, fill=lightgray] (12) at (3.5,2) {1};
\node [style=cotreenode] (13) at (4.5,2) {1};
\node [style=vertex] (14) at (3.25,1) {};
\node [style=vertex] (15) at (3.75,1) {};
\node [style=vertex] (16) at (4.25,1) {};
\node [style=vertex] (17) at (4.75,1) {};
\foreach \i/\j in {1/2,1/3,1/4,1/5,5/6,5/7,5/8,7/9,7/10,8/11,8/12,8/13,12/14,12/15,13/16,13/17}
  \draw [style=edge] (\i) to (\j);
\end{scope}
\begin{scope}[xshift=5cm,scale=1]
\node [style=cotreenode, fill=lightgray] (1) at (3.5,5) {0};
\node [style=vertex, fill=blue] (2) at (0.5,4) {};
\node [style=vertex, fill=blue] (3) at (1.5,4) {};
\node [style=vertex, fill=blue] (4) at (2.5,4) {};
\node [style=cotreenode, fill=lightgray] (5) at (3.5,4) {1};
\node [style=vertex, fill=green] (6) at (0.5,3) {};
\node [style=cotreenode] (7) at (2,3) {0};
\node [style=cotreenode, fill=lightgray] (8) at (3.5,3) {0};
\node [style=vertex, fill=green] (9) at (1.75,2) {};
\node [style=vertex, fill=green] (10) at (2.25,2) {};
\node [style=vertex, fill=green] (11) at (2.75,2) {};
\node [style=cotreenode] (12) at (3.5,2) {1};
\node [style=cotreenode, fill=lightgray] (13) at (4.5,2) {1};
\node [style=vertex, fill=blue] (14) at (3.25,1) {};
\node [style=vertex, fill=green] (15) at (3.75,1) {};
\node [style=vertex] (16) at (4.25,1) {};
\node [style=vertex] (17) at (4.75,1) {};
\foreach \i/\j in {1/2,1/3,1/4,1/5,5/6,5/7,5/8,7/9,7/10,8/11,8/12,8/13,12/14,12/15,13/16,13/17}
  \draw [style=edge] (\i) to (\j);
\end{scope}
\begin{scope}[xshift=10cm,scale=1]
\node [style=cotreenode, fill=lightgray] (1) at (3.5,5) {0};
\node [style=vertex, fill=blue] (2) at (0.5,4) {};
\node [style=vertex, fill=blue] (3) at (1.5,4) {};
\node [style=vertex, fill=blue] (4) at (2.5,4) {};
\node [style=cotreenode, fill=lightgray] (5) at (3.5,4) {1};
\node [style=vertex, fill=green] (6) at (0.5,3) {};
\node [style=cotreenode] (7) at (2,3) {0};
\node [style=cotreenode, fill=lightgray] (8) at (3.5,3) {0};
\node [style=vertex, fill=green] (9) at (1.75,2) {};
\node [style=vertex, fill=green] (10) at (2.25,2) {};
\node [style=vertex, fill=green] (11) at (2.75,2) {};
\node [style=cotreenode] (12) at (3.5,2) {1};
\node [style=cotreenode, fill=lightgray] (13) at (4.5,2) {1};
\node [style=vertex, fill=blue] (14) at (3.25,1) {};
\node [style=vertex, fill=green] (15) at (3.75,1) {};
\node [style=vertex, fill=blue] (16) at (4.25,1) {};
\node [style=vertex, fill=green] (17) at (4.75,1) {};
\foreach \i/\j in {1/2,1/3,1/4,1/5,5/6,5/7,5/8,7/9,7/10,8/11,8/12,8/13,12/14,12/15,13/16,13/17}
  \draw [style=edge] (\i) to (\j);
\end{scope}
\end{tikzpicture}
\end{subfigure}
\caption{Ejemplo de la ejecución del Algoritmo \ref{alg_cert_caso2}. Se muestran en color gris los nodos del árbol que están siendo procesados. El procesamiento de las hojas hermanas se realiza en una sola imagen. Los colores de las hojas corresponden a los colores que asigna el algoritmo.}
\label{fig_certificador_caso2_01}
\end{figure}


\begin{figure}[!htbp]
\centering

\begin{subfigure}{\textwidth}
\centering
\begin{tikzpicture}
\begin{scope}[xshift=0cm,scale=1]
\node [style=cotreenode] (1) at (3.5,5) {0};
\node [style=vertex, fill=yellow] (2) at (0.5,4) {};
\node [style=vertex, fill=blue] (3) at (1.5,4) {};
\node [style=vertex, fill=blue] (4) at (2.5,4) {};
\node [style=cotreenode] (5) at (3.5,4) {1};
\node [style=vertex, fill=green] (6) at (0.5,3) {};
\node [style=cotreenode] (7) at (2,3) {0};
\node [style=cotreenode] (8) at (3.5,3) {0};
\node [style=vertex, fill=green] (9) at (1.75,2) {};
\node [style=vertex, fill=green] (10) at (2.25,2) {};
\node [style=vertex, fill=green] (11) at (2.75,2) {};
\node [style=cotreenode] (12) at (3.5,2) {1};
\node [style=cotreenode] (13) at (4.5,2) {1};
\node [style=vertex, fill=yellow] (14) at (3.25,1) {};
\node [style=vertex, fill=yellow] (15) at (3.75,1) {};
\node [style=vertex, fill=yellow] (16) at (4.25,1) {};
\node [style=vertex, fill=yellow] (17) at (4.75,1) {};
\node [style=vertex, fill=yellow] (18) at (4.5,1) {};
\foreach \i/\j in {1/2,1/3,1/4,1/5,5/6,5/7,5/8,7/9,7/10,8/11,8/12,8/13,12/14,12/15,13/16,13/17,13/18}
  \draw [style=edge] (\i) to (\j);
\end{scope}
\begin{scope}[xshift=5cm,scale=1]
\node [style=cotreenode] (1) at (3.5,5) {0};
\node [style=vertex, fill=red] (2) at (0.5,4) {};
\node [style=vertex, fill=blue] (3) at (1.5,4) {};
\node [style=vertex, fill=blue] (4) at (2.5,4) {};
\node [style=cotreenode] (5) at (3.5,4) {1};
\node [style=vertex, fill=green] (6) at (0.5,3) {};
\node [style=vertex, fill=green] (7) at (1.5,3) {};
\node [style=cotreenode] (8) at (2.5,3) {0};
\node [style=cotreenode] (9) at (4.5,3) {0};
\node [style=cotreenode] (10) at (2,2) {1};
\node [style=vertex, fill=red] (11) at (3,2) {};
\node [style=cotreenode] (12) at (4,2) {1};
\node [style=vertex, fill=red] (13) at (5,2) {};
\node [style=vertex, fill=red] (14) at (1.5,1) {};
\node [style=vertex, fill=red] (15) at (2.5,1) {};
\node [style=vertex, fill=red] (16) at (3.5,1) {};
\node [style=vertex, fill=red] (17) at (4.5,1) {};
\foreach \i/\j in {1/2,1/3,1/4,1/5,5/6,5/7,5/8,5/9,8/10,8/11,9/12,9/13,10/14,10/15,12/16,12/17}
  \draw [style=edge] (\i) to (\j);
\end{scope}
\end{tikzpicture}
\end{subfigure}
\caption{Ejemplo del resultado de la ejecución del Algoritmo \ref{alg_cert_caso2} para coárboles que incluyen una obstrucción.}
\label{fig_certificador_caso2_02}
\end{figure}


\subsubsection{Algoritmo certificador}

Finalmente, el Algoritmo \ref{alg_cert_m2} utiliza los algoritmos anteriores para colorear las hojas del coárbol recibido como entrada, $g$. En el caso de que la gráfica sea conexa (líneas 4 a 8), simplemente se llama el algoritmo para cada una de los hijos de $g$. Esto no significa que sea un algoritmo recursivo, dado que, para las gráficas inconexas y las hojas, el algoritmo no vuelve a ser llamado. En el caso de que la gráfica sea inconexa, se ejecuta el bloque de las líneas 10 a 21. En las líneas 10 a 15 se cuenta el número de componentes conexas de la gráfica representada (es decir que se cuentan los hijos de $g$ que no son hojas). Y por último se toma la decisión de qué caso debe llamarse.


\begin{algorithm}[!htbp]
\caption{M2\_Certificador}
\label{alg_cert_m2}

\DontPrintSemicolon % Some LaTeX compilers require you to use \dontprintsemicolon instead
\KwIn{$g$, la raíz de un coárbol, $G$}
\KwOut{Verdadero si la gráfica representada por $G$ pertenece a la clase $M_2$. Falso en el caso contrario. Las hojas de $G$ se colorean.}

    \If{$g$ \text{es una hoja}}{
        $g.color \gets azul$\;
        $\Return\ verdadero$\;
    }
    \ElseIf{$g.etiqueta = 1$}{
        \For{child \textbf{\emph{en}} $g.hijos$}{
            \If{\emph{M2\_Certificador(}child\emph{)} = falso}{
                $\Return\ falso$\;
            }
            $\Return\ verdadero$\;
        }
    }
    \Else{
        $componentes\_no\_triviales \gets 0$\;
        \For{child \textbf{\emph{en}} $g.hijos$}{
            \If{$child$ \text{es una hoja}}{
                $child.color \gets azul$\;
            }
            \Else{
                $componentes\_no\_triviales \gets componentes\_no\_triviales + 1$\;
            }
        }
        \If{componentes\_no\_triviales = 0}{
            $\Return\ verdadero$\;
        }
        \ElseIf{componentes\_no\_triviales = 1}{
            $\Return$ M2\_Caso\_2($g$)\;
        }
        \Else{
            $\Return$ M2\_Caso\_1($g$)\;
        }
    }


$\Return\ verdadero$\;

\end{algorithm}


\section{Subclases de $M_2$}

    \subsection{Clases $(\alpha, \beta)-M_2$}

    \subsection{Conjunto de parejas mínimas}

    \subsection{Reconocimiento de las clases $(\alpha, \beta)-M_2$}

    \subsection{Algoritmo para generar obstrucciones mínimas}

\section{Particiones en más de dos partes}
    \subsection{Las Clases $M_i$}

    \subsection{Obstrucciones mínimas de la clase $M_3$}
        \begin{theorem} \label{teo_obsts_m2}

    Para una cográfica $G$, las siguientes afirmaciones son equivalentes.
    \begin{enumerate}[(a)] 
        \item $G \in M_3$.
        \item $G$ no contiene a ninguna de las gráficas de las Figuras \ref{obsts_O_M3} como subgráfica inducida.
    \end{enumerate}

\end{theorem}

\begin{figure}[h]





\begin{subfigure}{\textwidth}
\begin{center}
\begin{tikzpicture}
\begin{scope}[xshift=0cm,scale=1]
%K4
\node [style=vertex] (1) at (0,1) {};
\node [style=vertex] (2) at (1,1) {};
\node [style=vertex] (3) at (0.5,1.3) {};
\node [style=vertex] (4) at (0.5,1.75) {};
%K3
\node [style=vertex] (5) at (0,2.25) {};
\node [style=vertex] (6) at (1,2.25) {};
\node [style=vertex] (7) at (0.5,3) {};
%K2
\node [style=vertex] (8) at (0,3.5) {};
\node [style=vertex] (9) at (1,3.5) {};
%K1
\node [style=vertex] (10) at (0.5,4) {};

\foreach \i/\j in {1/2,1/3,1/4,2/3,2/4,3/4,5/6,5/7,6/7,8/9} \draw [style=edge] (\i) to (\j);
\node at (0.5,0) {\parbox{0.3\linewidth}{\subcaption*{$O_{3,1}$}}};
\end{scope}

\begin{scope}[xshift=2.5cm,scale=1]
%K4
\node [style=vertex] (1) at (0,1) {};
\node [style=vertex] (2) at (1,1) {};
\node [style=vertex] (3) at (0.5,1.3) {};
\node [style=vertex] (4) at (0.5,1.75) {};
%K3
\node [style=vertex] (5) at (0,2.25) {};
\node [style=vertex] (6) at (1,2.25) {};
\node [style=vertex] (7) at (0.5,3) {};
%K2
\node [style=vertex] (8) at (0,4) {};
\node [style=vertex] (9) at (1,4) {};
%K1
\node [style=vertex] (10) at (0.5,3.5) {};

\foreach \i/\j in {1/2,1/3,1/4,2/3,2/4,3/4,5/6,5/7,6/7,8/9} \draw [style=edge] (\i) to (\j);
\foreach \i/\j in {7/10} \draw [style=edge] (\i) to (\j);
\node at (0.5,0) {\parbox{0.3\linewidth}{\subcaption*{$O_{3,2}$}}};
\end{scope}

\begin{scope}[xshift=5cm,scale=1]
%K4
\node [style=vertex] (1) at (0,1) {};
\node [style=vertex] (2) at (1,1) {};
\node [style=vertex] (3) at (0.5,1.3) {};
\node [style=vertex] (4) at (0.5,1.75) {};
%K3
\node [style=vertex] (5) at (0,2.75) {};
\node [style=vertex] (6) at (1,2.75) {};
\node [style=vertex] (7) at (0.5,3.5) {};
%K2
\node [style=vertex] (8) at (0,4) {};
\node [style=vertex] (9) at (1,4) {};
%K1
\node [style=vertex] (10) at (0.5,2.25) {};

\foreach \i/\j in {1/2,1/3,1/4,2/3,2/4,3/4,5/6,5/7,6/7,8/9} \draw [style=edge] (\i) to (\j);
\foreach \i/\j in {4/10} \draw [style=edge] (\i) to (\j);
\node at (0.5,0) {\parbox{0.3\linewidth}{\subcaption*{$O_{3,3}$}}};
\end{scope}

\begin{scope}[xshift=7.5cm,scale=1]
%K4
\node [style=vertex] (1) at (0,1) {};
\node [style=vertex] (2) at (1,1) {};
\node [style=vertex] (3) at (0.5,1.3) {};
\node [style=vertex] (4) at (0.5,1.75) {};
%K3
\node [style=vertex] (5) at (0,2.75) {};
\node [style=vertex] (6) at (1,2.75) {};
\node [style=vertex] (7) at (0.5,3.5) {};
%K2
\node [style=vertex] (8) at (0,4) {};
\node [style=vertex] (9) at (1,4) {};
%K1
\node [style=vertex] (10) at (0.5,2.25) {};

\foreach \i/\j in {1/2,1/3,1/4,2/3,2/4,3/4,5/6,5/7,6/7,8/9} \draw [style=edge] (\i) to (\j);
\foreach \i/\j in {4/10,1/10} \draw [style=edge] (\i) to (\j);
\node at (0.5,0) {\parbox{0.3\linewidth}{\subcaption*{$O_{3,4}$}}};
\end{scope}

\begin{scope}[xshift=10cm,scale=1]
%K4
\node [style=vertex] (1) at (0,1) {};
\node [style=vertex] (2) at (1,1) {};
\node [style=vertex] (3) at (0.5,1.3) {};
\node [style=vertex] (4) at (0.5,1.75) {};
%K3
\node [style=vertex] (5) at (0,2.75) {};
\node [style=vertex] (6) at (1,2.75) {};
\node [style=vertex] (7) at (0.5,3.5) {};
%K2
\node [style=vertex] (8) at (0,2.25) {};
\node [style=vertex] (9) at (1,2.25) {};
%K1
\node [style=vertex] (10) at (0.5,4) {};

\foreach \i/\j in {1/2,1/3,1/4,2/3,2/4,3/4,5/6,5/7,6/7,8/9} \draw [style=edge] (\i) to (\j);
\foreach \i/\j in {4/8,4/9} \draw [style=edge] (\i) to (\j);
\node at (0.5,0) {\parbox{0.3\linewidth}{\subcaption*{$O_{3,5}$}}};
\end{scope}
\end{tikzpicture}
\end{center}
\end{subfigure}

\begin{subfigure}{\textwidth}
\begin{center}
\begin{tikzpicture}

\begin{scope}[xshift=0cm,scale=1]
%K4
\node [style=vertex] (1) at (0,1) {};
\node [style=vertex] (2) at (1,1) {};
\node [style=vertex] (3) at (0.5,1.3) {};
\node [style=vertex] (4) at (0.5,1.75) {};
%K3
\node [style=vertex] (5) at (0,3.25) {};
\node [style=vertex] (6) at (1,3.25) {};
\node [style=vertex] (7) at (0.5,4) {};
%K2
\node [style=vertex] (8) at (0.5,2.5) {};
\node [style=vertex] (9) at (1,2.5) {};
%K1
\node [style=vertex] (10) at (0,2.5) {};

\foreach \i/\j in {1/2,1/3,1/4,2/3,2/4,3/4,5/6,5/7,6/7,8/9} \draw [style=edge] (\i) to (\j);
\foreach \i/\j in {4/8,4/9,4/10} \draw [style=edge] (\i) to (\j);
\node at (0.5,0) {\parbox{0.3\linewidth}{\subcaption*{$O_{3,6}$}}};
\end{scope}

\begin{scope}[xshift=2.5cm,scale=1]
%K4
\node [style=vertex] (1) at (0,1) {};
\node [style=vertex] (2) at (1,1) {};
\node [style=vertex] (3) at (0.5,1.3) {};
\node [style=vertex] (4) at (0.5,1.75) {};
%K3
\node [style=vertex] (5) at (0,3.25) {};
\node [style=vertex] (6) at (1,3.25) {};
\node [style=vertex] (7) at (0.5,4) {};
%K2
\node [style=vertex] (8) at (0.5,2.5) {};
\node [style=vertex] (9) at (1,2.5) {};
%K1
\node [style=vertex] (10) at (0,2.5) {};

\foreach \i/\j in {1/2,1/3,1/4,2/3,2/4,3/4,5/6,5/7,6/7,8/9} \draw [style=edge] (\i) to (\j);
\foreach \i/\j in {4/8,4/9,4/10,1/10} \draw [style=edge] (\i) to (\j);
\node at (0.5,0) {\parbox{0.3\linewidth}{\subcaption*{$O_{3,7}$}}};
\end{scope}

\begin{scope}[xshift=5cm,scale=1]
%K4
\node [style=vertex] (1) at (0,1) {};
\node [style=vertex] (2) at (1,1) {};
\node [style=vertex] (3) at (0.5,1.3) {};
\node [style=vertex] (4) at (0.5,1.75) {};
%K3
\node [style=vertex] (5) at (0,2.75) {};
\node [style=vertex] (6) at (1,2.75) {};
\node [style=vertex] (7) at (0.5,3.5) {};
%K2
\node [style=vertex] (8) at (0,2.25) {};
\node [style=vertex] (9) at (1,2.25) {};
%K1
\node [style=vertex] (10) at (0.5,4) {};

\foreach \i/\j in {1/2,1/3,1/4,2/3,2/4,3/4,5/6,5/7,6/7,8/9} \draw [style=edge] (\i) to (\j);
\foreach \i/\j in {4/8,4/9,7/10} \draw [style=edge] (\i) to (\j);
\node at (0.5,0) {\parbox{0.3\linewidth}{\subcaption*{$O_{3,5}$}}};
\end{scope}

\end{tikzpicture}
\end{center}
\end{subfigure}

\setlength{\abovecaptionskip}{-15pt}
\caption{Obstrucciones mínimas para la clase $M_2$.}
\label{obsts_O_M3}
\end{figure}

\begin{proof}

\end{proof}


    \subsection{Familia $O$ de obstrucciones}

    \subsection{Familia $P$ de obstrucciones}

\chapter{Conclusiones y trabajo futuro}
En esta tesis estudiamos a la clase de cográficas que aceptan una partición en un número $i$ de gráficas multipartitas completas, para distintos valores de $i$. Denotamos a estas clases por $M_i$. Nuestra investigación tiene como punto de partida la clase $M_2$, a la cual estudiamos a profundidad tomando como referencia la investigación realizada sobre las cográficas polares.

A partir del estudio de la clase $M_2$, encontramos un algoritmo (Algoritmo \ref{alg_esta_en_clase}) que es capaz de determinar si una cográfica $G$, representada a través de su coárbol, es elemento de una clase hereditaria de cográficas fija en tiempo lineal con respecto al orden de $G$. Este algoritmo funciona utilizando una variación de los coárboles, los coárboles binarios y el concepto de subcoárbol binario,
que introducimos en este trabajo.

Se puede caracterizar a la clase $M_2$ a través de su conjunto de obstrucciones mínimas que se muestra en la Figura \ref{obsts_M2}. De igual manera, se presentan dos algoritmos para determinar si una cográfica $G$, representada a través de su coárbol, pertenece a la clase $M_2$. El primero de \'estos (Algoritmo \ref{alg_decision}) es una instancia del algoritmo mencionado en el párrafo anterior. Éste funciona generando un coárbol binario $B$ de $G$ y determinando si alguno de los coárboles binarios de las obstrucciones mínimas de $M_2$ es subárbol binario de $B$. El segundo algoritmo (Algoritmo \ref{alg_cert_m2}) es un algoritmo certificador. Es decir que no sólo determina si $G$ pertenece o no a la clase $M_2$, sino que devuelve un sí-certificado o un no-certificado. En el caso de que $G$, pertenezca a la clase $M_2$, el algoritmo colorea las hojas del coárbol de $G$ de dos colores que conforman una $M_2$-partición de $G$. En el caso contrario, el algoritmo colorea algunas hojas del coárbol de $G$ que representan vértices de $G$ que forman una obstrucción mínima de la clase. Este algoritmo se realizó siguiendo la demostración del Teorema \ref{teo_obsts_m2}, en el que se presentan las obstrucciones mínimas de la clase $M_2$. La ejecución de ambos algoritmos toma una cantidad de tiempo que crece de forma lineal con respecto al orden de $G$. Sin embargo, una vez implementados, el algoritmo certificador tuvo un desempeño mucho mejor. Se comprobó con éxito que los resultados de ambos algoritmos son coherentes.

Siguiendo la investigación realizada sobre las cográficas polares, estudiamos las clases $(\alpha, \beta)$-$M_2$, que son las subclases de $M_2$ que contienen a una gráfica si ésta acepta una partición en dos gráficas multipartitas completas tales que una de ellas está formada por a lo más $\alpha$ conjuntos independientes y la otra está formada por a lo más $\beta$ conjuntos independientes con $\alpha$ y $\beta$ enteros mayores o iguales a cero tales que $\alpha \le \beta$. Presentamos un algoritmo (Algoritmo \ref{alg_obstrucciones_alfabeta}) que, dados dos enteros $\alpha$ y $\beta$, genera todas las obstrucciones mínimas de la clase $(\alpha, \beta)$-$M_2$ hasta cierto orden. Este algoritmo utiliza el algoritmo certificador de la clase $M_2$. Utilizando este algoritmo, generamos las obstrucciones mínimas de hasta 15 vértices de varias clases $(\alpha, \beta)$-$M_2$. Si fijamos $\alpha$, podemos observar que las obstrucciones generadas para clases consecutivas siguen un comportamiento predecible que podría ser descrito definiendo familias de obstrucciones como lo hicimos con la clase $(1,\infty)$-$M_2$.

El siguiente paso en nuestra investigación fue caracterizar a la clase $M_3$ a través de su conjunto de obstrucciones mínimas (Teorema \ref{teo_obsts_m3}). Esto nos ayudó a encontrar dos familias de obstrucciones mínimas para cualquier clase $M_i$. Llamamos a estas familias la familia $O$ y la familia $P$ de obstrucciones. Todas las obstrucciones mínimas de las clases $M_1$ y $M_2$ pertenecen a las familias $O$ y $P$, mientras que 14 de 21 obstrucciones mínimas de la clase $M_3$ pertenecen a estas dos familias. Las otras 7 fueron agrupadas en otras 3 familias que podrían generalizarse.

\section{Trabajo a futuro}

En nuestra tesis encontramos varios temas con los que se puede continuar el estudio de las clases $M_i$.

Uno de los resultados principales de nuestra tesis, el Algoritmo \ref{alg_esta_en_clase}, funciona determinando si cada uno de los coárboles binarios de un conjunto es subcoárbol binario de otro coárbol binario. El número de coárboles binarios de una gráfica (en este caso la obstrucción mínima de alguna clase) crece de forma exponencial, por lo que, aunque el algoritmo funcione en tiempo lineal, éste es poco eficiente. A partir de esto, nos preguntamos si una definición más general de sucoárbol pueda dar lugar a un algoritmo más eficiente.

El Algoritmo \ref{alg_obstrucciones_alfabeta} recibe como entrada tres enteros $\alpha$, $\beta$ y $n$ y genera todas las cográficas de hasta $n$ vértices que son obstrucciones mínimas de la clase $(\alpha,\beta)$-$M_2$. Los resultados que generamos con este algoritmo son obstrucciones mínimas de hasta 15 vértices. Si bien, creemos que los conjuntos de obstrucciones mínimas generados para clases que no tienen obstrucciones mínimas ni con 13 ni con 14 vértices son exhaustivos, no podemos estar seguros de ello ya que no contamos con una cota superior para el número de vértices de una obstrucción mínima lo suficientemente pequeña. Así, un resultado que tendría gran impacto en la investigación de las clases $M_i$ sería encontrar dicha cota superior.

El el Apéndice \ref{apéndiceListaAlfaBeta} se listan las obstrucciones mínimas de varias clases $(\alpha,\beta)$-$M_2$. Si fijamos $\alpha$, podemos observar un comportamiento bastante predecible para valores de $\beta$ consecutivos. Tomando como ejemplo el análisis que realizamos de las clases $(1,\beta)$-$M_2$, nos preguntamos si podemos encontrar resultados parecidos para valores de $\alpha$ diferentes de 1.

Así como encontrar el conjunto de obstrucciones mínimas de la clase $M_3$ nos llevó a identificar familias de obstrucciones mínimas para cualquier clase $M_i$, seguir el estudio de la clase $M_3$ como lo hicimos con la clase $M_2$ podría llevarnos a resultados más generales. Así, quedan pendientes las siguientes tareas:
\begin{itemize}
    \item Encontrar un algoritmo para identificar a los elementos de $M_3$.
    \item Encontrar un algoritmo certificador para la clase $M_3$.
    \item El estudio de las clases $(\alpha,\beta,\gamma)$-$M_3$.
\end{itemize}

Otra forma de darle continuidad a los resultados presentados en esta tesis es encontrar un algoritmo para computar a los elementos de la familia $O_n$ de obstrucciones para un entero $n$.

Finalmente, un resultado importante sería identificar otras familias de obstrucciones para cualquier clase $M_i$.


\chapter*{Apéndices}
%\appendix

\bibliographystyle{unsrt}
\bibliography{biblio.bib}

\end{document}
