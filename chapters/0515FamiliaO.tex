En esta subsección presentamos a la familia $O$ de obstrucciones mínimas para cualquier clase $M_i$.

\subsubsection{$O$-componentes}

Sean $G$ una cográfica y $l$ un entero mayor o igual a cero, decimos que $G$ es una \textbf{\emph{$O$-componente}} si su conjunto de vértices acepta una partición $(A,B_1,B_2,\dots,B_l)$ tal que, para cualesquiera dos enteros $1\le i,j \le l$, se cumplen las siguientes condiciones :

\begin{itemize}
    \item $G[A]$ es una gráfica completa con $m_0$ vértices.
    \item $G[B_i]$ es una gráfica completa con $m_i$ vértices en donde $1\le m_i<m_0-1$.
    \item Si $i\neq j$, entonces $m_i\neq m_j$.
    \item Cada uno de los vértices de $B_i$ es adyacente a los mismos $n_i$ vértices de $A$, con $0<n_i<m_0-m_i$.
    \item Si $i\neq j$, entonces ningún vértice de $B_i$ es adyacente a algún vértice de $B_j$.
\end{itemize}

Decimos que $(A,B_1,B_2,\dots,B_l)$ es una $O$-partición de $G$ y que $G$ incluye a la gráfica completa $K_m$ si $m$ es igual a $m_k$ para algún $0\le k\le l$. Decimos que $G[A]$ es el cuerpo de $G$ y que, para cualquier $0\le k\le l$, $G[B_k]$ es una extremidad de $G$.

Sea $G$ una $O$-componente y $(A,B_1,B_2,\dots,B_l)$ una $O$-partición de $G$, notemos que el cuerpo de $G$ es necesariamente el clan más grande en $G$.

En la Figura \ref{fig_ejemplos_O} podemos ver algunos ejemplos de gráficas que son $O$-componentes y otros ejemplos de gráficas que no son $O$-componentes. Las gráficas $G_1$, $G_2$, $G_3$ y $G_4$ son $O$-componentes. La gráfica $G_1$ acepta la $O$-partición $(\{a,b,c\},\{d\})$. La gráfica $G_2$ acepta la $O$-partición $(\{a,b,c,d\},\{e\})$. La gráfica $G_3$ acepta la $O$-partición $(\{a,b,c,d\},\{e\})$. La gráfica $G_4$ acepta la $O$-partición $(\{a,b,c,d\},\{e\},\{f,g\})$.

Por el contrario, las gráficas $G_5$, $G_6$, $G_7$ y $G_8$ no son $O$-componentes. La gráfica $G_5$ tiene dos extremidades del mismo orden. El cuerpo de la gráfica $G_6$ es un $K_3$, pero ésta tiene una extremidad de orden mayor a  1. La gráfica $G_7$ tiene una extremidad que se conecta a más de 2 vértices del cuerpo de $G$. La gráfica $G_8$ cumple con las características listadas, pero los vértices $e$, $a$, $d$ y $f$ forman un $P_4$, por lo que $G_8$ no es una cográfica, y por lo tanto, tampoco es una $O$-componente. Esto nos lleva a realizar la siguiente observación.

Sean $G$ una $O$-componente, $l$ un  entero mayor o igual a 0, $i,j$ enteros diferentes entre sí tales que $1\le i,j \le l$ , $(A,B_2,\dots,B_l)$ una $O$-partición de $G$, $x$ un vértice de $B_i$ y $y$ un vértice de $B_j$, notemos que no existen vértices diferentes $v,w\in A$ tales que $v$ es adyacente a los vértices en $B_i$, pero no a los vértices en $B_j$ y $w$ es adyacente a los vértices en $B_j$, pero no a los vértices en $B_i$, ya que, de ser así, $x$, $v$, $w$ y $y$ formarían un $P_4$. De esto se sigue que si $|B_i| < |B_j|$, los vértices en $B_i$ serán adyacentes a cada uno de los vértices de $A$ a los que sean adyacentes los vértices de $B_j$.

\begin{figure}[ht!]
\begin{subfigure}{\textwidth}
\begin{center}
\begin{tikzpicture}

\begin{scope}[xshift=0cm,scale=1]

\node [style=vertex] (1) at (0,0.5) {};
\node [style=vertex] (2) at (1.5,0.5) {};
\node [style=vertex] (3) at (0.75,1.75) {};
\node [style=vertex] (4) at (0.75,2.5) {};

\node at (-0.25,0.5) {$a$};
\node at (1.75,0.5) {$b$};
\node at (1,1.75) {$c$};
\node at (1,2.5) {$d$};

\foreach \i/\j in {1/2,1/3,2/3,3/4} \draw [style=edge] (\i) to (\j);
\node at (0.75,0) {\parbox{0.3\linewidth}{\subcaption*{$G_1$}}};
\end{scope}

\begin{scope}[xshift=3cm,scale=1]

\node [style=vertex] (1) at (0,0.5) {};
\node [style=vertex] (2) at (1.5,0.5) {};
\node [style=vertex] (3) at (0.75,1) {};
\node [style=vertex] (4) at (0.75,1.75) {};
\node [style=vertex] (5) at (0.75,2.5) {};

\node at (-0.25,0.5) {$a$};
\node at (1.75,0.5) {$b$};
\node at (0.75,0.75) {$c$};
\node at (1,1.75) {$d$};
\node at (1,2.5) {$e$};

\foreach \i/\j in {1/2,1/3,1/4,2/3,2/4,3/4,4/5} \draw [style=edge] (\i) to (\j);
\node at (0.75,0) {\parbox{0.3\linewidth}{\subcaption*{$G_2$}}};
\end{scope}

\begin{scope}[xshift=6cm,scale=1]

\node [style=vertex] (1) at (0,0.5) {};
\node [style=vertex] (2) at (1.5,0.5) {};
\node [style=vertex] (3) at (0.75,1) {};
\node [style=vertex] (4) at (0.75,1.75) {};
\node [style=vertex] (5) at (0.75,2.5) {};

\node at (-0.25,0.5) {$a$};
\node at (1.75,0.5) {$b$};
\node at (0.75,0.75) {$c$};
\node at (1,1.75) {$d$};
\node at (1,2.5) {$e$};

\foreach \i/\j in {1/2,1/3,1/4,1/5,2/3,2/4,3/4,4/5} \draw [style=edge] (\i) to (\j);
\node at (0.75,0) {\parbox{0.3\linewidth}{\subcaption*{$G_2$}}};
\end{scope}

\begin{scope}[xshift=9cm,scale=1]

\node [style=vertex] (1) at (0,0.5) {};
\node [style=vertex] (2) at (1.5,0.5) {};
\node [style=vertex] (3) at (0.75,1) {};
\node [style=vertex] (4) at (0.75,1.75) {};
\node [style=vertex] (5) at (0,2.5) {};
\node [style=vertex] (6) at (0.75,2.5) {};
\node [style=vertex] (7) at (1.5,2.5) {};

\node at (-0.25,0.5) {$a$};
\node at (1.75,0.5) {$b$};
\node at (0.75,0.75) {$c$};
\node at (1,1.75) {$d$};
\node at (-0.25,2.5) {$e$};
\node at (0.5,2.5) {$f$};
\node at (1.75,2.5) {$g$};

\foreach \i/\j in {1/2,1/3,1/4,2/3,2/4,3/4,4/5,4/6,4/7,6/7} \draw [style=edge] (\i) to (\j);
\node at (0.75,0) {\parbox{0.3\linewidth}{\subcaption*{$G_4$}}};
\end{scope}

\end{tikzpicture}
\end{center}
\end{subfigure}

\begin{subfigure}{\textwidth}
\begin{center}
\begin{tikzpicture}

\begin{scope}[xshift=0cm,scale=1]

\node [style=vertex] (1) at (0,0.5) {};
\node [style=vertex] (2) at (1.5,0.5) {};
\node [style=vertex] (3) at (0.75,1.75) {};
\node [style=vertex] (4) at (0,2.5) {};
\node [style=vertex] (5) at (1.5,2.5) {};

\node at (-0.25,0.5) {$a$};
\node at (1.75,0.5) {$b$};
\node at (1,1.75) {$c$};
\node at (-0.25,2.5) {$d$};
\node at (1.75,2.5) {$e$};

\foreach \i/\j in {1/2,1/3,2/3,3/4,3/5} \draw [style=edge] (\i) to (\j);
\node at (0.75,0) {\parbox{0.3\linewidth}{\subcaption*{$G_5$}}};
\end{scope}

\begin{scope}[xshift=3cm,scale=1]

\node [style=vertex] (1) at (0,0.5) {};
\node [style=vertex] (2) at (1.5,0.5) {};
\node [style=vertex] (3) at (0.75,1.75) {};
\node [style=vertex] (4) at (0,2.5) {};
\node [style=vertex] (5) at (1.5,2.5) {};

\node at (-0.25,0.5) {$a$};
\node at (1.75,0.5) {$b$};
\node at (1,1.75) {$c$};
\node at (-0.25,2.5) {$d$};
\node at (1.75,2.5) {$e$};

\foreach \i/\j in {1/2,1/3,2/3,3/4,3/5,4/5} \draw [style=edge] (\i) to (\j);
\node at (0.75,0) {\parbox{0.3\linewidth}{\subcaption*{$G_6$}}};
\end{scope}

\begin{scope}[xshift=6cm,scale=1]

\node [style=vertex] (1) at (0,0.5) {};
\node [style=vertex] (2) at (1.5,0.5) {};
\node [style=vertex] (3) at (0.75,1) {};
\node [style=vertex] (4) at (0.75,1.75) {};
\node [style=vertex] (5) at (0.75,2.5) {};

\node at (-0.25,0.5) {$a$};
\node at (1.75,0.5) {$b$};
\node at (0.75,0.75) {$c$};
\node at (1.25,1.75) {$d$};
\node at (1,2.5) {$e$};

\foreach \i/\j in {1/2,1/3,1/4,1/5,2/3,2/4,2/5,3/4,4/5} \draw [style=edge] (\i) to (\j);
\node at (0.75,0) {\parbox{0.3\linewidth}{\subcaption*{$G_7$}}};
\end{scope}

\begin{scope}[xshift=9cm,scale=1]

\node [style=vertex] (1) at (0,0.5) {};
\node [style=vertex] (2) at (1.5,0.5) {};
\node [style=vertex] (3) at (0.75,1) {};
\node [style=vertex] (4) at (0.75,1.75) {};
\node [style=vertex] (5) at (0,2.5) {};
\node [style=vertex] (6) at (0.75,2.5) {};
\node [style=vertex] (7) at (1.5,2.5) {};

\node at (-0.25,0.5) {$a$};
\node at (1.75,0.5) {$b$};
\node at (0.75,0.75) {$c$};
\node at (1,1.75) {$d$};
\node at (-0.25,2.5) {$e$};
\node at (0.5,2.5) {$f$};
\node at (1.75,2.5) {$g$};

\foreach \i/\j in {1/2,1/3,1/4,1/5,2/3,2/4,3/4,4/6,4/7,6/7} \draw [style=edge] (\i) to (\j);
\node at (0.75,0) {\parbox{0.3\linewidth}{\subcaption*{$G_8$}}};
\end{scope}

\end{tikzpicture}
\end{center}
\end{subfigure}

%\setlength{\abovecaptionskip}{-15pt}
\caption{Ejemplos de $O$-componentes ($G_1$,$G_2$,$G_3$ y $G_4$) y de gráficas que no son $O$-componentes ($G_5$,$G_6$,$G_7$ y $G_8$).}
\label{fig_ejemplos_O}
\end{figure}

\begin{lemma}
Sea $G$ una $O$-componente, $G$ es una gráfica completa o $G$ no es una gráfica multipartita completa.
\end{lemma}

\begin{proof}
Sean $l$ un entero mayor o igual a cero y $(A,B_1,B_2,\dots,B_l)$ una $O$-partición de $G$. Si $l=0$, entonces $G$ es igual a $G[A]$, y por lo tanto es una gráfica completa. En el caso contrario, veamos que, dado un entero $1\le i\le l$, $A$ tiene al menos dos vértices que no son adyacentes a ningún $B_i$. Sea $v\in B_i$, sabemos por la definición de $O$-componente que $v$ es adyacente a máximo $|A|-|B_i|-1$ vértices de $A$. Como $i$ es mayor o igual a 1, entonces $v$ es adyacente a lo más a $|A|-2$ vértices de $A$. Luego, existen dos vértices $w,x\in A$ tales que no son adyacentes a $v$, pero que sí son adyacentes el uno al otro. Así, $v$, $w$ y $x$ forman un $\overline{P_3}$. De esto se sigue que $G$ no es una gráfica multipartita completa.
\end{proof}

\subsubsection{$O$-obstrucciones}

Sean $G$ una cográfica y $n,m$ enteros mayores o iguales a uno, decimos que $G$ es una \emph{\textbf{$O_n$-obstrucción}} si $G$ es la unión ajena de algunas $O$-componentes $G_1,G_2,\dots G_m$ tales que, para cualesquiera enteros $1\le i \le m$ y $1\le j \le n+1$, existe una única $O$-componente $G_i$ que contiene a $K_j$.

Sea $n$ un entero mayor o igual a uno, la \emph{\textbf{familia $O_n$ de obstrucciones}}, denotada simplemente por $O_n$, es el conjunto de todas las $O_n$-obstrucciones.

\begin{theorem}
Sean $n$ un entero mayor o igual a uno y $G$ una cográfica. Si $G$ es una $O_n$ obstrucción, entonces $G$ es una obstruccción mínima de la clase $M_n$.
\end{theorem}

\begin{proof}
Mostremos primero que $G$ no está en la clase $M_n$. Procedamos por inducción sobre $n$.

Si $n = 1$, entonces tenemos que la única $O_1$-obstrucción que hay es $K_1+K_2$. Luego, $G=K_1+K_2=\overline{P_3}$. Como $\overline{P_3}$ es la única obstrucción de las gráficas multipartitas completas, se sigue que $G\notin M_1$.

Supongamos como hipótesis inductiva que si una cográfica $G'$ es una $O_{n-1}$-obstrucción, entonces $G'$ es una obstrucción mínima de la clase $M_{n-1}$, y consideremos a $G$ una $O_n$-obstrucci\'on.

Sea $P = (A_1, A_2, \dots, A_n)$ una partición de los vértices de $G$, veamos que $P$ no es una $M_n$-partición. Recordemos que, como $G$ es una $O_n$-obstrucción, ésta es la unión ajena de hasta $n+1$ $O$-componentes. Y que exactamente una de estas $O$-componentes $H$ contiene a $K_{n+1}$. Como $G$ tiene a $K_{n+1}$ como subgráfica inducida, $G$ no es $n$-coloreable, y por lo tanto al menos dos vértices de dicho $K_{n+1}$ están en la misma parte de $P$. Supongamos sin pérdida de generalidad que $A_1$ contiene al menos dos vértices del $K_{n+1}$ de $G$.

Como $H$ es una $O$-componente, ésta acepta una $O$-partición $Q=(C,B_1,B_2,\dots,B_i)$ para algún entero $0\le i \le n-2$. Notemos que $C$ es el conjunto de los vértices del $K_{n+1}$ de $H$. Denotemos por $m_j$ la cardinalidad de $B_j$ para todo $1\le j\le i$. Supongamos sin pérdida de generalidad que $m_k < m_{k+1}$ para todo $1\le k < i$. Recordemos que $A_1$ contiene al menos dos vértices que están en $C$ y que hay al menos dos vértices de $C$ que no son adyacentes a vértice alguno de $B_l$ para todo $1\le l \le i$.

Si $A_1$ contiene sólo vértices de $C$, los vértices de $H-A_1$ inducen en $G$ la unión ajena de un conjunto de gráficas que tienen como subgráficas inducidas $O$-componentes que, en conjunto, contienen a las gráficas completas $G[B_1], G[B_2],\dots,G[B_i]$. Los vértices de $H-A_1$ y las $O$-componentes de $G$ diferentes de $H$ deben de repartirse entre las $n-1$ partes de $P$ diferentes de $A_1$. Notemos que $G-A_1$ tiene como subgráfica inducida a un conjunto de $O$-componentes que contienen todas las gráficas completas desde $K_1$ hasta $K_{n}$. Es decir que $G-A_1$ tiene como subgráfica inducida una $O_{n-1}$-obstrucción que, por hipótesis inductiva, no está en la clase $M_{n-1}$. Luego, $G$ no está en la clase $M_n$. En la Figura \ref{fig_dem_O_01} se muestra un ejemplo de esto.

\begin{figure}[ht!]

\begin{subfigure}{\textwidth}
\begin{center}
\begin{tikzpicture}
\begin{scope}[xshift=0cm,scale=1]
%K6
\node [style=vertex, fill=red] (1) at (1,0.5) {};
\node [style=vertex, fill=red] (2) at (2,0.5) {};
\node [style=vertex, fill=red] (3) at (0,1) {};
\node [style=vertex, fill=red] (4) at (3,1) {};
\node [style=vertex, fill=red] (5) at (1,1.5) {};
\node [style=vertex, fill=red] (6) at (2,1.5) {};

%K1
\node [style=vertex] (7) at (0,3) {};

%K2
\node [style=vertex] (8) at (1,3) {};
\node [style=vertex] (9) at (2,3) {};

%K3
\node [style=vertex] (10) at (2.5,2.5) {};
\node [style=vertex] (11) at (3.5,2) {};
\node [style=vertex] (12) at (3.25,2.5) {};

%K4
\node [style=vertex] (13) at (4,0.625) {};
\node [style=vertex] (14) at (4.75,0.625) {};
\node [style=vertex] (15) at (4,1.375) {};
\node [style=vertex] (16) at (4.75,1.375) {};

\foreach \i/\j in {1/2,1/3,1/4,1/5,1/6,2/3,2/4,2/5,2/6,3/4,3/5,3/6,4/5,4/6,5/6} \draw [style=edge] (\i) to (\j);
\foreach \i/\j in {7/3,7/4,7/5,7/6} \draw [style=edge] (\i) to (\j);
\foreach \i/\j in {8/4,8/5,8/6,8/9,9/4,9/5,9/6} \draw [style=edge] (\i) to (\j);
\foreach \i/\j in {10/11,10/12,11/12,10/4,10/6,11/4,11/6,12/4,12/6} \draw [style=edge] (\i) to (\j);
\foreach \i/\j in {13/14,13/15,13/16,14/15,14/16,15/16,13/4,14/4,15/4,16/4} \draw [style=edge] (\i) to (\j);

\node at (2.375,0) {\parbox{0.3\linewidth}{\subcaption*{$G$}}};
\end{scope}

\end{tikzpicture}
\end{center}
\end{subfigure}

%\setlength{\abovecaptionskip}{-15pt}
\caption{La gráfica $G$ es una $O$-componente que contiene a $K_1$, $K_2$, $K_3$, $K_4$ y $K_6$. En rojo se marcan los vértices que se agregan a $A_1$. Se puede apreciar que el resto de los vértices inducen la gráfica $K_1+K_2+K_3+K_4$.}
\label{fig_dem_O_01}
\end{figure}


Abordemos el caso en el que $A_1$ contiene, además de dos vértices de $C$, a lo más un vértice de al menos una parte de $Q$ diferente de $C$. Sean $j,k$ enteros tales que $A_1$ contiene un vértice de $B_j$ y un vértice de $B_k$. Si $A_1$ contiene un vértice $v$ de $C$ tal que $v$ es adyacente a los vértices de $B_j$ pero no es adyacente a los vértices de $B_k$, entonces el subconjunto de $V(A_1)$ que contiene a $v$, a un vértice de $B_j$ y a un vértice de $B_k$ induce un $\overline{P_3}$, con lo que $G[A_1]$ no es una gráfica multipartita completa y $P$ no es una $O$-partición.

En el caso contrario, un elemento de $C$ está en $A_1$ si y sólo si es adyacente a todos los elementos de $A_1-C$ o, en su defecto, no es adyacente a elemento alguno de $A_1-C$. Si $A_1$ contiene dos elementos de $C$ que no son adyacentes a elemento alguno de $A_1-C$, entonces $G[A_1]$ no es una gráfica multipartita completa y por lo tanto $P$ no es una $M_n$-partición.

Si todos los elementos de $C$ que están en $A_1$, excepto a lo más uno, son adyacentes a todos los vértices de $A_1-C$, veamos que $H-A_1$ tiene como subgráfica inducida a la unión ajena de algunas $O$-componentes que, en conjunto, contienen a $K_{m_l}$ para cualquier entero $1\le l\le i$ . Sea $m$ el máximo entero tal que $A_1$ tiene un vértice de $B_m$. Notemos que para cualquier entero $m < l \le i$, se cumple que $B_{l}\subset V(H)-A_1$, ya que no hay vértices de $B_l$ en $A_1$. Por otra parte, para cualquier entero $1 < l \le m$, se cumple que $B_{l}-A_1$ tiene como subgráfica inducida a la gráfica completa con $m_{l-1}$ vértices, ya que $m_l>m_{l-1}$ y a lo más uno de los vértices de $B_l$ está en $A_1$. Como ya hemos verificado que todas las gráficas completas con $m_1$, $m_2$, \dots, $m_i$ vértices menos la que tiene $|B_m|$ vértices son subgráficas inducidas de $H-A_1$ y no tienen vértices en común, sólo nos falta verificar que la gráfica completa con $|B_m|$ vértices también está en $H-A_1$, que no tiene vértices en común con las otras gráficas completas mencionadas, y que éstas forman $O$-obstrucciones.

Dado que todos los vértices de $C$ que están en $A_1$, excepto a lo más uno, son adyacentes a todos los vértices de $A_1-C$, y $m$ es el mayor entero tal que $A_1$ tiene un vértice de $B_m$, se cumple que $A_1$ contiene a lo más $n-|B_m|+1$ vértices de $C$. Luego, $C-A_1$ debe de tener al menos $|B_m|$ vértices, por lo que $H-A_1$ contiene a la gráfica completa con $|B_m|$ vértices como subgráfica inducida. Como $C$ es ajeno a $B_l$ para cualquier entero $1\le l\le i$, entonces esta gráfica completa con $|B_m|$ vértices no tiene vértices en común con ninguna de las otras gráficas completas encontradas. Ahora veamos que estas gráficas completas forman un conjunto ajeno de $O$-obstrucciones.

Sean $C'$ un subconjunto de $C$ tal que contiene $|B_m|$ elementos de $C$ que no son adyacentes a elemento alguno de $B_m$, $B_l'$ un subconjunto de $B_{l+1}$ con $|B_l|$ elementos para cada entero $1\le l < m$, veamos que $Q'=(C',B_1', B_2', \dots, B_{m-1}')$ es una $O$-partición. Sabemos que $G[C']$ es una gráfica completa con $m$ vértices. De igual manera, sabemos que, para cualquier entero $1\le l < m$, $G[B_l']$ es una gráfica completa con $m_l$ vértices con $1\le m_l < m$. De igual manera, sabemos que ninguna de las partes de $Q$ tiene la misma cardinalidad y que vértices de diferentes partes diferentes de $C'$ no son adyacentes. Sólo falta por mostrar que, para cualquier entero $1\le m_l < m$, cada uno de los vértices de $B_l'$ es adyacente a los mismos $a_l$ vértices de $C'$ con $0< a_l < m - m_l$. Como $H$ es una $O$-componente, sabemos que cada uno de los vértices de $B_l'$ es adyacente a los mismos $a_l$ vértices de $C'$. Como cada vértice de $C$ en $A_1$ es adyacente a cada uno de los vértices de $B_{l+1}$, entonces se mantiene la propiedad de que $0< a_l < m - m_l$. Luego, $Q'$ es una $O$-partición. Notemos que para cualquier entero $m<l\le 1$ se cumple que $B_l$ no tiene vértices adyacentes a vértice alguno de $C'$. Así, $H-A_1$ es la unión ajena de una $O$-componente y varias gráficas completas que, en conjunto, contienen a todas las gráficas completas con $B_1$, $B_2$, $\dots$ y $B_i$ vértices. En la Figura \ref{fig_dem_O_02} se muestra un ejemplo de esto.

Finalmente, $G-A_1$ tiene como subgráfica inducida a la unión de algunas $O$-componentes que, en conjunto, contienen a todas las gráficas completas desde 1 hasta $n$ vértices. Luego $G-A_1$ tiene como subgráfica inducida una $O_{n-1}$-onstrucción, que, por hipótesis inductiva, no está en la clase $M_{n-1}$. Así, $P$ no es una $M_n$-partición.

\begin{figure}[ht!]

\begin{subfigure}{\textwidth}
\begin{center}
\begin{tikzpicture}
\begin{scope}[xshift=0cm,scale=1]
%K6
\node [style=vertex] (1) at (1,0.5) {};
\node [style=vertex, fill=red] (2) at (2,0.5) {};
\node [style=vertex] (3) at (0,1) {};
\node [style=vertex, fill=red] (4) at (3,1) {};
\node [style=vertex] (5) at (1,1.5) {};
\node [style=vertex, fill=red] (6) at (2,1.5) {};

%K1
\node [style=vertex, fill=red] (7) at (0,3) {};

%K2
\node [style=vertex] (8) at (1,3) {};
\node [style=vertex] (9) at (2,3) {};

%K3
\node [style=vertex, fill=red] (10) at (2.5,2.5) {};
\node [style=vertex] (11) at (3.5,2) {};
\node [style=vertex] (12) at (3.25,2.5) {};

%K4
\node [style=vertex] (13) at (4,0.625) {};
\node [style=vertex] (14) at (4.75,0.625) {};
\node [style=vertex] (15) at (4,1.375) {};
\node [style=vertex] (16) at (4.75,1.375) {};

\foreach \i/\j in {1/2,1/3,1/4,1/5,1/6,2/3,2/4,2/5,2/6,3/4,3/5,3/6,4/5,4/6,5/6} \draw [style=edge] (\i) to (\j);
\foreach \i/\j in {7/3,7/4,7/5,7/6} \draw [style=edge] (\i) to (\j);
\foreach \i/\j in {8/4,8/5,8/6,8/9,9/4,9/5,9/6} \draw [style=edge] (\i) to (\j);
\foreach \i/\j in {10/11,10/12,11/12,10/4,10/6,11/4,11/6,12/4,12/6} \draw [style=edge] (\i) to (\j);
\foreach \i/\j in {13/14,13/15,13/16,14/15,14/16,15/16,13/4,14/4,15/4,16/4} \draw [style=edge] (\i) to (\j);

\node at (2.375,0) {\parbox{0.3\linewidth}{\subcaption*{$G$}}};
\end{scope}

\begin{scope}[xshift=6.5cm,scale=1]
%K6
\node [style=vertex, fill=blue] (1) at (0,0.5) {};
\node [style=vertex, fill=blue] (3) at (1.5,0.5) {};
\node [style=vertex, fill=blue] (5) at (0.75,1.75) {};

%K2
\node [style=vertex, fill=blue] (8) at (0,3) {};
\node [style=vertex] (9) at (1.5,3) {};

%K3
\node [style=vertex, fill=blue] (11) at (2,3) {};
\node [style=vertex, fill=blue] (12) at (3.5,3) {};

%K4
\node [style=vertex, fill=blue] (13) at (2,0.5) {};
\node [style=vertex, fill=blue] (14) at (3.5,0.5) {};
\node [style=vertex, fill=blue] (15) at (2.75,1) {};
\node [style=vertex, fill=blue] (16) at (2.75,1.75) {};

\foreach \i/\j in {1/3,1/5,3/5} \draw [style=edge] (\i) to (\j);

\foreach \i/\j in {8/5,8/9,9/5} \draw [style=edge] (\i) to (\j);
\foreach \i/\j in {11/12} \draw [style=edge] (\i) to (\j);
\foreach \i/\j in {13/14,13/15,13/16,14/15,14/16,15/16} \draw [style=edge] (\i) to (\j);

\node at (1.75,0) {\parbox{0.3\linewidth}{\subcaption*{$G-A_1$}}};
\end{scope}

\end{tikzpicture}
\end{center}
\end{subfigure}

%\setlength{\abovecaptionskip}{-15pt}
\caption{La gráfica $G$ es una $O$-componente que contiene a $K_1$, $K_2$, $K_3$, $K_4$ y a $K_6$. En rojo se marcan los vértices que se agregan a $A_1$. A la derecha se muestra $G-A_1$. En azul se muestran los vértices de una $O$-obstrucción que contiene a $K_1$, $K_2$, $K_3$ y a $K_4$.}
\label{fig_dem_O_02}
\end{figure}

Abordemos el caso en el que $A_1$ contiene, además de dos vértices de $C$, al menos dos vértices de $B_j$ para algún $1\le j \le i$. Si $A_1$ contiene un vértice de $C$ que no es adyacente a los vértices de $B_j$ o algún vértice que no es elemento ni de $C$ ni de $B_j$, entonces $G[A_1]$ no es una grafica multipartita completa. Y por lo tanto, $P$ no es una $M_n$-partición. Si $A_1$ contiene sólo vértices de $B_j$ y vértices de $C$ adyacentes a los vértices de $B_j$, como los vértices del $B_j$ son adyacentes a máximo $n-m_j$ vértices de $C$, entonces al menos $m_j+1$ vértices de $C$ no son adyacentes a los vértices de $B_j$. Luego, al menos $m_j$ vértices de $C$ no están en $A_1$. Así, los vértices de $H$ que no están en $A_1$ inducen una gráfica que tiene como subgráfica inducida la unión ajena de algunas $O$-componentes que, en conjunto, contienen a las gráficas completas con $m_1, m_2,\dots$ y $m_{i-1}$ vértices. Luego, $G-A_1$ tiene como subgráfica inducida una $O_{n-1}$-obstrucción que, por hipótesis inductiva, no está en la clase $M_{n-1}$. Luego, $G$ no está en la clase $M_n$. En la Figura \ref{fig_dem_O_03} se muestra un ejemplo de esto.

\begin{figure}[ht!]

\begin{subfigure}{\textwidth}
\begin{center}
\begin{tikzpicture}
\begin{scope}[xshift=0cm,scale=1]
%K6
\node [style=vertex] (1) at (1,0.5) {};
\node [style=vertex] (2) at (2,0.5) {};
\node [style=vertex] (3) at (0,1) {};
\node [style=vertex, fill=red] (4) at (3,1) {};
\node [style=vertex] (5) at (1,1.5) {};
\node [style=vertex,fill=red] (6) at (2,1.5) {};

%K1
\node [style=vertex] (7) at (0,3) {};

%K2
\node [style=vertex] (8) at (1,3) {};
\node [style=vertex] (9) at (2,3) {};

%K3
\node [style=vertex, fill=red] (10) at (2.5,2.5) {};
\node [style=vertex, fill=red] (11) at (3.5,2) {};
\node [style=vertex, fill=red] (12) at (3.25,2.5) {};

%K4
\node [style=vertex] (13) at (4,0.625) {};
\node [style=vertex] (14) at (4.75,0.625) {};
\node [style=vertex] (15) at (4,1.375) {};
\node [style=vertex] (16) at (4.75,1.375) {};

\foreach \i/\j in {1/2,1/3,1/4,1/5,1/6,2/3,2/4,2/5,2/6,3/4,3/5,3/6,4/5,4/6,5/6} \draw [style=edge] (\i) to (\j);
\foreach \i/\j in {7/3,7/4,7/5,7/6} \draw [style=edge] (\i) to (\j);
\foreach \i/\j in {8/4,8/5,8/6,8/9,9/4,9/5,9/6} \draw [style=edge] (\i) to (\j);
\foreach \i/\j in {10/11,10/12,11/12,10/4,10/6,11/4,11/6,12/4,12/6} \draw [style=edge] (\i) to (\j);
\foreach \i/\j in {13/14,13/15,13/16,14/15,14/16,15/16,13/4,14/4,15/4,16/4} \draw [style=edge] (\i) to (\j);

\node at (2.375,0) {\parbox{0.3\linewidth}{\subcaption*{$G$}}};
\end{scope}

\begin{scope}[xshift=6.5cm,scale=1]
%K6
\node [style=vertex, fill=blue] (1) at (1.5,0.5) {};
\node [style=vertex, fill=blue] (3) at (0,0.5) {};
\node [style=vertex] (5) at (0.75,1.75) {};
\node [style=vertex, fill=blue] (6) at (0.75,1) {};

%K1
\node [style=vertex, fill=blue] (7) at (0,3) {};

%K2
\node [style=vertex, fill=blue] (8) at (0.75,3) {};
\node [style=vertex, fill=blue] (9) at (1.75,3) {};

%K4
\node [style=vertex, fill=blue] (13) at (3.5,0.5) {};
\node [style=vertex, fill=blue] (14) at (2,0.5) {};
\node [style=vertex, fill=blue] (15) at (2.75,1.75) {};
\node [style=vertex, fill=blue] (16) at (2.75,1) {};

\foreach \i/\j in {1/3,1/5,1/6,3/5,3/6,5/6} \draw [style=edge] (\i) to (\j);
\foreach \i/\j in {7/3,7/5} \draw [style=edge] (\i) to (\j);
\foreach \i/\j in {8/5,8/9,9/5} \draw [style=edge] (\i) to (\j);
\foreach \i/\j in {13/14,13/15,13/16,14/15,14/16,15/16} \draw [style=edge] (\i) to (\j);

\node at (1.75,0) {\parbox{0.3\linewidth}{\subcaption*{$G-A_1$}}};
\end{scope}

\end{tikzpicture}
\end{center}
\end{subfigure}

%\setlength{\abovecaptionskip}{-15pt}
\caption{La gráfica $G$ es una $O$-componente que contiene a $K_1$, $K_2$, $K_3$, $K_4$ y a $K_6$. En rojo se marcan los vértices que se agregan a $A_1$. A la derecha se muestra $G-A_1$. En azul se muestran los vértices de una $O$-obstrucción que contiene a $K_1$, $K_2$, $K_3$ y a $K_4$.}
\label{fig_dem_O_03}
\end{figure}

Para mostrar que $G$ es una obstrucción mínima, basta con notar que en cada uno de los casos anteriores, si se sustrae un vértice de $G$, entonces la gráfica resultante está en $M_n$.

\end{proof}

\subsubsection{$O$-obstrucciones conocidas}

A lo largo de este documento, podemos identificar varias $O$-obstrucciones. Notemos que $\overline{P_3}$ es una $O$-obstrucción. En la Figura \ref{obsts_M2}, las gráficas $H$ e $I$ son $O$-obstrucciones. Todas las gráficas de la Figura \ref{obsts_M3_O} son $O$-obstrucciones. En el apéndice \ref{apéndice_O} podemos observar a la familia $O_4$ de obstrucciones.
