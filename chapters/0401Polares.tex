En \cite{Ekim}, Ekim, de Werra y Mahadeev introducen
la clase de las cográficas polares y presentan una
caracterizaci\'on de la misma mediante un conjunto de
obstrucciones mínimas (que existe dado que las
cográficas polares son una clase hereditaria de gráficas).
La metodología que los autores utilizan para demostrar su
resultado principal, i.e., que el conjunto de gráficas que
se propone (Figura \ref{obsts_cografics_polares}) es, en
efecto, el conjunto de obstrucciones mínimas para las
cográficas polares, fue una gu\'ia importante para realizar
la demostración de que nuestra propuesta de conjunto de
obstrucciones mínimas de la clase $M_2$\footnote{Una
cogr\'afica est\'a en la clase $M_2$ si acepta una
partición en dos gráficas multipartitas completas.} es
correcta.   En este artículo también se consideran las
cográficas monopolares, una subclase de las cográficas
polares, cuyo estudio sirve como inspiración para el
estudio de algunas subclases de $M_2$.

Dada una cogr\'afica $G$, decimos que $G$ es una
\emph{\textbf{cográfica polar}} si su conjunto de vértices
$V$ acepta una partición $(A,B)$ tal que $A$ induce una
gráfica multipartita completa y $B$ induce una unión ajena
de clanes. Decimos que $G$ es $(s,k)$-polar si existe una
partición $(A,B)$ de los vértices de $G$ en donde $A$ induce
en $G$ una unión completa de a lo más $s$ conjuntos
independientes (es decir una gráfica $s$-partita) y $B$
induce en $G$ una unión ajena de a lo más $k$ clanes. Notemos
que las cográficas polares son las gráficas ($\infty,
\infty$)-polares.

\begin{theorem}
  Si $G$ es una cográfica, entonces $G$ es polar si y sólo
  si no contiene como subgráfica inducida a ninguna de las
  gráficas de la Figura \ref{obsts_cografics_polares}.
\end{theorem}

\begin{figure}[ht!]
\begin{center}
\begin{tikzpicture}

\begin{scope}[xshift=0cm,scale=1]

\node [style=vertex] (1) at (0.5,0) {};
\node [style=vertex] (2) at (0.2,0.5) {};
\node [style=vertex] (3) at (0.8,0.5) {};
\node [style=vertex] (4) at (0,1) {};
\node [style=vertex] (5) at (1,1) {};
\node [style=vertex] (6) at (0.5,1.5) {};
\node [style=vertex] (7) at (0,2) {};
\node [style=vertex] (8) at (1,2) {};
\foreach \i/\j in {1/2,1/3,4/5,4/6,4/7,5/6,5/8,6/7,6/8,7/8}
  \draw [style=edge] (\i) to (\j);
\node [below of=1] {\parbox{0.3\linewidth}{\subcaption*{$H_1$}}};

\end{scope}

\begin{scope}[xshift=2cm,scale=1]

\node [style=vertex] (1) at (0.75,0) {};
\node [style=vertex] (2) at (0.45,0.5) {};
\node [style=vertex] (3) at (1.05,0.5) {};
\node [style=vertex] (4) at (0.5,1) {};
\node [style=vertex] (5) at (1.5,1) {};
\node [style=vertex] (6) at (0,1.5) {};
\node [style=vertex] (7) at (1,1.5) {};
\node [style=vertex] (8) at (0.5,2) {};
\node [style=vertex] (9) at (1.5,2) {};

\foreach \i/\j in {1/2,1/3,4/6,4/7,5/7,5/9,6/7,6/8,7/9,7/8}
  \draw [style=edge] (\i) to (\j);
\node [below of=1] {\parbox{0.3\linewidth}{\subcaption*{$H_2$}}};

\end{scope}

\begin{scope}[xshift=4.5cm,scale=1]

\node [style=vertex] (1) at (0.75,0) {};
\node [style=vertex] (2) at (0.45,0.5) {};
\node [style=vertex] (3) at (1.05,0.5) {};
\node [style=vertex] (4) at (0,1) {};
\node [style=vertex] (5) at (1.5,1) {};
\node [style=vertex] (6) at (0.45,1.5) {};
\node [style=vertex] (7) at (1.05,1.5) {};
\node [style=vertex] (8) at (0,2) {};
\node [style=vertex] (9) at (1.5,2) {};

\foreach \i/\j in {1/2,1/3,4/5,4/6,4/7,4/8,5/6,5/7,5/9,6/7,6/8,7/9,8/9}
  \draw [style=edge] (\i) to (\j);
\node [below of=1] {\parbox{0.3\linewidth}{\subcaption*{$H_3$}}};

\end{scope}

\begin{scope}[xshift=7cm,scale=1]

\node [style=vertex] (1) at (1,0) {};
\node [style=vertex] (2) at (0.7,0.5) {};
\node [style=vertex] (3) at (1.3,0.5) {};
\node [style=vertex] (4) at (0,1) {};
\node [style=vertex] (5) at (1,1) {};
\node [style=vertex] (6) at (2,1) {};
\node [style=vertex] (7) at (0,2) {};
\node [style=vertex] (8) at (1,2) {};
\node [style=vertex] (9) at (2,2) {};
\foreach \i/\j in {1/2,1/3,4/5,4/7,4/8,5/6,5/7,5/8,5/9,6/8,6/9,7/8,8/9}
  \draw [style=edge] (\i) to (\j);
\node [below of=1] {\parbox{0.3\linewidth}{\subcaption*{$H_4$}}};
\end{scope}

\end{tikzpicture}
\end{center}
\caption{Obstrucciones mínimas para las gráficas polares.}
\label{obsts_cografics_polares}
\end{figure}

La demostración de que las gráficas de la Figura
\ref{obsts_cografics_polares} conforman el conjunto de
obstrucciones mínimas de las cográficas polares se
realiza describiendo a cada una de éstas en términos
de la unión ajena y la unión completa de gráficas más
pequeñas.

\begin{enumerate}[(1)]
    \item $H_1 = P_3 + ( \overline{K_2} \oplus P_3) = P_3+ (K_1 \oplus P_4)$
    \item $H_2 = P_3 + (K_1 \oplus (P_3 + K_2))$
    \item $H_3 = P_3 + ( \overline{P_3} \oplus \overline{P_3})$
    \item $H_4 = P_3 + (K_2 \oplus 2K_2)$
\end{enumerate}

Primero se demuestra que ninguna de estas gráficas es
una cográfica polar y que, por lo tanto, cualquier
gráfica que tenga a alguna de ellas como subgráfica
inducida tampoco es una cográfica polar. Por último se
muestra que si una cográfica $G$ no es polar, entonces
debe de tener a alguna de estas gráficas como subgráfica
inducida.

En este artículo también se presenta y se caracteriza,
a través de su conjunto de obstrucciones mínimas, a la
clase de las cográficas monopolares. El conjunto de
obstrucciones m\'inimas se muestra en la Figura
\ref{obsts_cografics_monopolares}

%Sea $G$ una cográfica, decimos que $G$ es una cográfica monopolar si $G$ es una gráfica $(s,k)$-polar con $s\leq 1$ o $k \leq 1$.


\begin{figure}[ht!]
\begin{center}
\begin{subfigure}{\textwidth}
\begin{tikzpicture}

\begin{scope}[xshift=0cm,scale=1]
\node [style=vertex] (2) at (0.2,0.5) {};
\node [style=vertex] (3) at (0.8,0.5) {};
\node [style=vertex] (4) at (0,1) {};
\node [style=vertex] (5) at (1,1) {};
\node [style=vertex] (6) at (0.5,1.5) {};
\node [style=vertex] (7) at (0,2) {};
\node [style=vertex] (8) at (1,2) {};
\foreach \i/\j in {4/5,4/6,4/7,5/6,5/8,6/7,6/8,7/8}
  \draw [style=edge] (\i) to (\j);
\node at (0.5,-0.25) {\parbox{0.3\linewidth}{\subcaption*{$G_1$}}};
\end{scope}

\begin{scope}[xshift=2cm,scale=1]
\node [style=vertex] (2) at (0.5,0.5) {};
\node [style=vertex] (3) at (1,0.5) {};
\node [style=vertex] (4) at (0,1) {};
\node [style=vertex] (5) at (1,1) {};
\node [style=vertex] (6) at (0.5,1.5) {};
\node [style=vertex] (7) at (0,2) {};
\node [style=vertex] (8) at (1,2) {};
\foreach \i/\j in {2/6,4/5,4/6,4/7,5/6,5/8,6/7,6/8,7/8}
  \draw [style=edge] (\i) to (\j);
\node at (0.5,-0.25) {\parbox{0.3\linewidth}{\subcaption*{$G_2$}}};
\end{scope}

\begin{scope}[xshift=4cm, yshift=0.5cm,scale=1]
\node [style=vertex] (1) at (1,0) {};
\node [style=vertex] (2) at (0.5,0.5) {};
\node [style=vertex] (3) at (1.5,0.5) {};
\node [style=vertex] (4) at (0,1) {};
\node [style=vertex] (5) at (1,1) {};
\node [style=vertex] (6) at (0.5,1.5) {};
\node [style=vertex] (7) at (1.5,1.5) {};
\foreach \i/\j in {2/4,2/5,3/5,3/7,4/5,4/6,5/6,5/7}
  \draw [style=edge] (\i) to (\j);
\node at (0.75,-0.75) {\parbox{0.3\linewidth}{\subcaption*{$G_3$}}};
\end{scope}

\begin{scope}[xshift=6.5cm, yshift=0.5cm,scale=1]
\node [style=vertex] (1) at (0.75,0) {};
\node [style=vertex] (2) at (0,0.5) {};
\node [style=vertex] (3) at (0.75,0.5) {};
\node [style=vertex] (4) at (1.5,0.5) {};
\node [style=vertex] (5) at (0,1.5) {};
\node [style=vertex] (6) at (0.75,1.5) {};
\node [style=vertex] (7) at (1.5,1.5) {};
\foreach \i/\j in {2/3,2/5,2/6,3/4,3/5,3/6,3/7,4/6,4/7,5/6,6/7}
  \draw [style=edge] (\i) to (\j);
\node at (0.75,-0.75) {\parbox{0.3\linewidth}{\subcaption*{$G_4$}}};
\end{scope}

\begin{scope}[xshift=9cm, yshift=0.5cm,scale=1]
\node [style=vertex] (1) at (0.75,0) {};
\node [style=vertex] (2) at (0.25,0.5) {};
\node [style=vertex] (3) at (1.25,0.5) {};
\node [style=vertex] (4) at (0,1) {};
\node [style=vertex] (5) at (0.75,1) {};
\node [style=vertex] (6) at (1.5,1) {};
\node [style=vertex] (7) at (0.75,1.5) {};
\foreach \i/\j in {2/4,2/5,2/6,3/4,3/5,3/6,4/5,4/7,5/6,6/7}
  \draw [style=edge] (\i) to (\j);
\node at (0.75,-0.75) {\parbox{0.3\linewidth}{\subcaption*{$G_5$}}};
\end{scope}

\end{tikzpicture}
\end{subfigure}

\begin{subfigure}{\textwidth}
\begin{center}
\begin{tikzpicture}

\begin{scope}[xshift=0cm, yshift=0.5cm,scale=1]
\node [style=vertex] (1) at (0.75,0) {};
\node [style=vertex] (2) at (0.25,0.5) {};
\node [style=vertex] (3) at (1.25,0.5) {};
\node [style=vertex] (4) at (0,1) {};
\node [style=vertex] (5) at (0.75,1) {};
\node [style=vertex] (6) at (1.5,1) {};
\node [style=vertex] (7) at (0.75,1.5) {};
\foreach \i/\j in {2/3,2/4,2/5,2/6,3/4,3/5,3/6,4/5,4/7,5/6,6/7}
  \draw [style=edge] (\i) to (\j);
\node at (0.75,-0.75) {\parbox{0.3\linewidth}{\subcaption*{$G_6$}}};
\end{scope}

\begin{scope}[xshift=2.5cm, yshift=0.5cm,scale=1]
\node [style=vertex] (1) at (0.75,0) {};
\node [style=vertex] (2) at (0.25,0.5) {};
\node [style=vertex] (3) at (1.25,0.5) {};
\node [style=vertex] (4) at (0,1) {};
\node [style=vertex] (5) at (0.75,1) {};
\node [style=vertex] (6) at (1.5,1) {};
\node [style=vertex] (7) at (0.75,1.5) {};
\foreach \i/\j in {2/3,2/4,2/5,2/6,3/4,3/5,3/6,4/5,4/7,5/6,5/7,6/7}
  \draw [style=edge] (\i) to (\j);
\node at (0.75,-0.75) {\parbox{0.3\linewidth}{\subcaption*{$G_7$}}};
\end{scope}

\begin{scope}[xshift=5cm, yshift=0.5cm,scale=1]
\node [style=vertex] (1) at (0.75,0) {};
\node [style=vertex] (2) at (0,0.5) {};
\node [style=vertex] (3) at (1.5,0.5) {};
\node [style=vertex] (4) at (0.25,1) {};
\node [style=vertex] (5) at (1.25,1) {};
\node [style=vertex] (6) at (0,1.5) {};
\node [style=vertex] (7) at (1.5,1.5) {};
\foreach \i/\j in {2/3,2/4,2/5,2/6,3/4,3/5,3/7,4/5,4/6,5/7,6/7}
  \draw [style=edge] (\i) to (\j);
\node at (0.75,-0.75) {\parbox{0.3\linewidth}{\subcaption*{$G_8$}}};
\end{scope}

\begin{scope}[xshift=7.5cm, yshift=0.5cm,scale=1]
\node [style=vertex] (1) at (0.75,0) {};
\node [style=vertex] (2) at (0,0.5) {};
\node [style=vertex] (3) at (1.5,0.5) {};
\node [style=vertex] (4) at (0.25,1) {};
\node [style=vertex] (5) at (1.25,1) {};
\node [style=vertex] (6) at (0,1.5) {};
\node [style=vertex] (7) at (1.5,1.5) {};
\foreach \i/\j in {2/3,2/4,2/5,2/6,3/4,3/5,3/7,4/5,4/6,4/7,5/6,5/7,6/7}
  \draw [style=edge] (\i) to (\j);
\node at (0.75,-0.75) {\parbox{0.3\linewidth}{\subcaption*{$G_9$}}};
\end{scope}

\end{tikzpicture}
\end{center}
\end{subfigure}

\end{center}
\caption{Obstrucciones mínimas para las gráficas monopolares.}
\label{obsts_cografics_monopolares}
\end{figure}


Como podemos observar, varias de las obstrucciones mínimas
de las gráficas monopolares se asemejan a obstrucciones
mínimas de las gráficas polares. Por ejemplo, las gráficas
$G_1$, $G_3$, $G_4$ y $G_8$ son subgráficas de $H_1$, $H_2$,
$H_3$ y $H_4$ respectivamente. Notemos que en los cuatro
casos se obtiene una obstrucción mínima $G$ para las gráficas
monopolares, a partir de una obstrucción mínima $H$ para las
gráficas polares al eliminar vértices de la componente conexa
de $H$ que forma un $P_3$. En los últimos tres casos, el
$P_3$ es reemplazado por un vértice aislado. Esto nos lleva
a pensar en que se puede encontrar una relación entre el
conjunto de obstrucciones mínimas de una clase y los conjuntos
de obstrucciones mínimas de sus subclases.
