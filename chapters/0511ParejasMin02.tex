A continuación presentamos una serie de lemas para encontrar el conjunto de parejas mínimas de una gráfica. 

\begin{lemma}\label{lema_parejas_01}
Sea $G\in M_2$ un conjunto independiente inconexo, entonces $\mu(G)=\{(0,1)\}$.
\end{lemma}

\begin{proof}
Sea $(A,B)$ una $M_2$-partición de $G$. Si ni $A$ ni $B$ son vacíos, entonces $(A,B)$ es de tamaño $(1,1)$. Si uno de los dos es vacío, entonces $(A,B)$ es de tamaño $(0,1)$. Luego, $S(G)=\{(0,1),(1,1)\}$. Como $(1,1)$ domina a $(0,1)$ y $(0,1)$ no domina a ningún elemento de $S(G)$, entonces $\mu(G)=\{(0,1)\}$.
\end{proof}

\begin{lemma}\label{lema_parejas_02}
Sea $G\in M_2$ una gráfica inconexa con al menos una componente trivial y exactamente una componente no trivial que es una gráfica multipartita completa formada por $n$ conjuntos independientes, entonces $\mu(G)=\{(1,n-1)\}$.
\end{lemma}

\begin{proof}
Notemos que, como $G$ contiene vértices aislados, toda $M_2$-partición que $G$ acepta es una partición en un conjunto independiente y una gráfica multipartita completa. Sea $(A,B)$ una $M_2$-partición de $G$ y $G[A]$ un conjunto independiente, abordemos los siguientes dos casos que son exhaustivos.

\emph{Caso 1}: $n = 2$. 

Notemos que la componente no trivial de $G$ es bipartita y por lo tanto $G$ también lo es. Si tanto $G[A]$ como $G[B]$ son conjuntos independientes, entonces $(A,B)$ es de tamaño $(1,1)$. En el caso contrario $B$ contiene al menos un par de vértices que son adyacentes. Luego, $A$ contiene a todos los vértices aislados de $G$ y $B$ contiene sólo vértices de la componente no trivial de $G$. Como ésta es bipartita, entonces $(A,B)$ es de tamaño $(1,2)$. Así, $S(G)=\{(1,1),(1,2)\}$. Luego, $\mu(G)=\{(1,1)\} = \{(1,n-1)\}$.

\emph{Caso 2}: $n > 2$.

Como $G$ no es bipartita, entonces $B$ debe contener al menos un par de vértices adyacentes. Luego, $A$ debe de contener a todos los vértices aislados de $G$, y $B$ sólo puede contener vértices de la componente no trivial de $G$. Sea $(B_1, B_2, \dots, B_n)$ una partición de la componente no trivial de $G$ tal que $G[B_i]$ es un conjunto independiente para todo $0\le i \le n$. Si $A$ contiene todos los vértices de $B_j$ para algún $0\le j \le n$, entonces $(A,B)$ es de tamaño $(1,n-1)$. En el caso contrario, $(A,B)$ es de tamaño $(1,n)$. Así, $S(G)=\{(1,n-1),(1,n)\}$. Luego, $\mu(G)=\{(1,n-1)\}$.

\end{proof}

\begin{lemma}\label{lema_parejas_03}
Sea $G\in M_2$ una gráfica inconexa con al menos una componente trivial y exactamente una componente no trivial que no es una gráfica multipartita completa, $T$ el coárbol de $G$ y $r$ la raíz de $T$, entonces $\mu(G)=\{(1,B(r))\}$, en donde $B(x)$ se define de la siguiente manera para cualquier nodo $x$ de $T$ tal que $x$ tiene etiqueta 0 y al menos un hijo que no es una hoja.
\begin{enumerate}
    \item Si $x$ es el nodo más profundo con etiqueta 0 que tiene al menos un hijo que no es una hoja.
    \begin{enumerate}
        \item Si $x$ tiene al menos dos hijos que no son hojas, entonces $B(x) = 1$.
        \item Si $x$ tiene un solo hijo $y$ que no es una hoja tal que $G[y]$ es una gráfica multipartita completa formada por $n$ conjuntos independientes, entonces $B(x) = n-1$. 
    \end{enumerate}
    \item En el caso contrario. Sea $y$ el único hijo de $x$ que no es una hoja, $n$ el número de hijos de $y$ y $x'$ el único hijo de $y$ tal que $G[x']$ no es una gráfica multipartita completa, entonces $B(x) = n-1 + B(x')$.
\end{enumerate}
\end{lemma}

\begin{proof}

Podemos verificar que estos casos son exhaustivos a través del \emph{Caso 2} de la demostración del Teorema \ref{teo_obsts_m2}. 

Sea $(A,B)$ una $M_2$-partición de $G$, notemos que $(A,B)$ es de tamaño $(1,\beta)$ para algún entero $\beta \geq 2$. Luego $(1,\beta)\in \mu(G)$ si y sólo si $\beta$ es mínimo. Es decir que $\mu(G)$ tendrá un solo elemento, y éste será el tamaño de una $M_2$-partición de $G$ en un conjunto independiente y una gráfica multipartita completa con el menor número posible de partes. 

Sea $x$ un nodo de $T$. Mostremos que si $x$ tiene etiqueta 0 y al menos un hijo que no es una hoja, entonces una $M_2$-partición de $G[x]$ en un conjunto independiente y una gráfica multipartita completa con el menor número posible de partes será de tamaño $(1,B(x))$. Sea $z$ el nodo más profundo de $T$ con etiqueta 0 y al menos un hijo que no es una hoja, y $d$ la distancia desde $x$ hasta $z$, probemos por inducción sobre $d$.

\emph{Caso base}: $d = 0$, es decir que $x = z$.

Por el \emph{Caso 2} de la demostración del Teorema \ref{teo_obsts_m2}, sabemos que $x$ tiene al menos dos hijos que no son hojas o $x$ tiene un solo hijo $y$ tal que $y$ no es una hoja y que $G[y]$ es una gráfica multipartita completa formada por $n$ conjuntos independientes. 

Si $x$ tiene al menos dos hijos que no son hojas, cada uno de estos hijos es la raíz del coárbol de una gráfica bipartita. Luego, $G[x]$ es bipartita. Así, $G[x]$ acepta una $M_2$-partición $(A,B)$ de tamaño $(1,1)$. Como $G[x]$ no es un conjunto independiente, entonces $(A,B)$ es una $M_2$-partición de $G[x]$ en un conjunto independiente y una gráfica multipartita completa con el menor número posible de partes. Finalmente, notemos que $(A,B)$ es de tamaño $(1,B(x))$ dado que se cumple la primera condición \emph{1.a)} del lema.

Si $x$ tiene un solo hijo $y$ que no es una hoja tal que $G[y]$ es una gráfica multipartita completa formada por $n$ conjuntos independientes, por el Lema \ref{lema_parejas_02}, sabemos que $\mu(G[x]) = (1,n-1)$. Es claro que no existe una $M_2$-partición de $G[x]$ en un conjunto independiente y una gráfica multipartita completa con menos de $n-1$ partes. Así, dado que se cumple la condición \emph{1.b)} del lema, una $M_2$-partición de $G[x]$ en un conjunto independiente y una gráfica multipartita completa con el menor número posible de partes será de tamaño $(1,n-1)=(1,B(x))$. 
 
\emph{Paso inductivo}: $d > 0$.

Sea $x'$ un nodo de $T$ con etiqueta 0 y al menos un hijo que no es una hoja. Supongamos como hipótesis inductiva (H.I.) que, si la distancia desde $x'$ hasta $z$ es menor a $d$, entonces una $M_2$-partición de $G[x']$ en un conjunto independiente y una gráfica multipartita completa con el menor número posible de partes es de tamaño $(1,B(x'))$.

Por el \emph{Caso 2} de la demostración del Teorema \ref{teo_obsts_m2}, sabemos que $x$ tiene un sólo hijo $y$ que no es una hoja mientras que el resto son hojas. Asimismo, sabemos que $y$ tiene un solo hijo $x'$ tal que $G[x']$ no es una gráfica multipartita completa. Es decir que $x'$, que tiene etiqueta 0, tiene al menos un hijo que no es una hoja. Por H.I., una $M_2$-partición de $G[x']$ en un conjunto independiente y una gráfica multipartita completa con el menor número posible de partes es de tamaño $(1,B(x'))$. 

Sean $(A',B')$ una partición de $G[x']$ de tamaño $(1,B(x'))$ tal que $G[A']$ es un conjunto independiente y $(A,B)$ una partición de $G[x]$ tal que $G[A]$ es un conjunto independiente y $G[B]$ una gráfica multipartita completa. Como $G$ no es bipartita, entonces $A$ debe contener todos los vértices de $G$ que son hijos de $x$ (es decir, todos los hijos de $x$ que son hojas), mientras que $B$ sólo puede contener vértices de $G[y]$. Para que $G[B]$ sea una una gráfica multipartita completa, ésta debe de ser la unión completa de un conjunto de gráficas multipartitas completas. Como todos los hijos de $y$ representan gráficas multipartitas completas menos $x'$, entonces $A$ debe de contener un subconjunto $S$ de los vértices de $G[x']$ tal que $G[x']-S$ sea una gráfica multipartita completa. 

Sea $\beta'$ un entero tal que $G[x']-S$ es una gráfica multipartita completa
formada por $\beta'$ conjuntos independientes y $n$ el número de hijos de $y$,
calculemos el tamaño $(\alpha,\beta)$ de $(A,B)$ en función de $\beta'$ y $n$.
Sabemos que $\alpha=1$ y que $\beta$ es el número de conjuntos independientes que
forman a la gráfica multipartita completa $G[x']-S$. Como $G[x']-S$ es la unión
completa de $n$ gráficas multipartitas completas, entonces $\beta$ es la suma del
número de conjuntos independientes que conforman a cada una de estas gráficas
multipartitas completas. Todos los hijos de $y$ que no son $x'$ son hojas o
representan gráficas multipartitas completas inconexas. Es decir, conjuntos independientes. En cualquier caso, cada uno de los $n-1$ hijos de $y$ representa un conjunto independiente. Como $G[x']-S$ es la unión completa de $\beta'$ conjuntos independientes, entonces $\beta = n-1+\beta'$.

Finalmente, encontremos el menor $\beta'$ posible. Si $S=A'$, entonces $\beta'=B(x')$. Como $B(x')$ es mínimo por H.I., entonces el mínimo valor que puede tomar $\beta'$ es $B(x')$. Luego, el valor mínimo de $\beta$ es $n-1+B(x')$ . Así, una $M_2$-partición de $G[x]$ en un conjunto independiente y una gráfica multipartita completa con el menor número posible de partes es de tamaño $(1,n-1+B(x'))$. Como $x$ cumple la condición 2 del lema, este tamaño es igual a $(1,B(x))$. 

\end{proof}

\begin{lemma}\label{lema_parejas_04}
Sea $G\in M_2$ una gráfica inconexa no bipartita con dos componentes conexas no triviales. Si las componentes de $G$ son una la unión ajena de $\alpha$ conjuntos independientes y la otra la unión ajena de $\beta$ conjuntos independientes con $\alpha \le \beta$, entonces $\mu(G)=\{(\alpha,\beta)\}$.
\end{lemma}

\begin{proof}
Sea $(A,B)$ una $M_2$-partición de $G$. Como $G$ no es bipartita, entonces $A$ debe de contener todos los vértices de una de las componentes de $G$ y $B$ debe contener todos los vértices de la otra componente. Como ésta es la única posible $M_2$-partición de $G$ y es de tamaño $(\alpha, \beta)$, entonces $\mu(G)=\{(\alpha,\beta)\}$.
\end{proof}

\begin{lemma}\label{lema_parejas_05}
Sea $G\in M_2$ una gráfica inconexa bipartita con al menos dos componentes conexas no triviales, entonces $\mu(G)=\{(1,1)\}$.
\end{lemma}

\begin{proof}
Como $G$ es bipartita, entonces $G$ acepta una $M_2$-partición de tamaño $(1,1)$. Como $G$ tiene un $\overline{P_3}$, entonces $G$ no es una gráfica multipartita completa. Luego, $G$ no acepta ninguna $M_2$-partición de tamaño $(0,\beta)$ para cualquier entero $\beta \geq 1$. Así, cualquier $M_2$-partición de $G$ será de tamaño $(\alpha,\beta')$ para algunos $\alpha \geq 1$ y $\beta \geq 1$. Luego, $\mu(G)=\{(1,1)\}$.
\end{proof}

\begin{lemma}\label{lema_parejas_06}
Sea $G$ una gráfica isomorfa a $K_1$, entonces $\mu(G)=\{0,1\}$.
\end{lemma}

\begin{proof}
Es claro que la única $M_2$-partición posible de $G$ es de tamaño $(0,1)$. Luego, $\mu(G)=\{0,1\}$. 
\end{proof}

\begin{lemma}\label{lema_parejas_07}
Sean $G_1,G_2\in M_2$, $G=G_1 \oplus G_2$ y $S'(G)$ el conjunto de todas las parejas $(\alpha, \beta)$ tales que para algunos $(\alpha_1,\beta_1)\in \mu(G_1)$ y $(\alpha_2,\beta_2)\in \mu(G_2)$ cumplen alguna de las siguientes condiciones
\begin{itemize}
    \item $\alpha = \alpha_1+\alpha_2$ y $\beta = \beta_1 + \beta_2$ o
    \item $\alpha = \emph{min}(\alpha_1+\beta_2, \alpha_2+\beta_1)$ y $\beta = \emph{max}(\alpha_1+\beta_2, \alpha_2+\beta_1)$. 
\end{itemize}
Una pareja de enteros $P$ está en $\mu(G)$ si y sólo si $P\in S'(G)$ y $P$ no domina a ningún otro elemento de $S'(G)$.
\end{lemma}

\begin{proof}
Notemos que $(A,B)$ es una $M_2$-partición de $G$ si y sólo si existen $M_2$-particiones $(A_1,B_1)$ y $(A_2,B_2)$ de $G_1$ y $G_2$ respectivamente tales que $(A,B)=(A_1\cup A_2, B_1 \cup B_2)$. 

Veamos que $S'(G)\subset S(G)$. Sea $(x,y)$ un elemento de $S'(G)$. Sabemos que existen $(\alpha_1,\beta_1)\in \mu(G_1)$ y $(\alpha_2,\beta_2)\in \mu(G_2)$ tales que se cumple alguna de las condiciones mencionadas en el lema. Como $(\alpha_1,\beta_1)\in \mu(G_1)$, entonces existe una $M_2$-partición $(A_1,B_1)$ de $G_1$ de tamaño $(\alpha_1, \beta_1)$. Análogamente, existe una $M_2$-partición $(A_2,B_2)$ de $G_2$ de tamaño $(\alpha_2, \beta_2)$. Luego, $(A_1\cup A_2, B_1 \cup B_2)$ es una $M_2$ partición de $G$ de tamaño $(x,y)$ o $(A_1\cup B_2, B_1 \cup A_2)$ es una $M_2$ partición de $G$ de tamaño $(x,y)$. En cualquiera de los casos, $(x,y)\in S(G)$. Así, $S'(G)\subset S(G)$.

Ahora mostremos la doble contención. Sea $(x,y)$ un elemento de $S'(G)$ tal que $(x,y)$ no domina a ningún otro elemento de $S'(G)$. Como $S'(G)\subset S(G)$, entonces $(x,y)\in S(G)$. Luego, existe una $M_2$-partición $(A,B)$ de $G$ de tamaño $(x,y)$. Mostremos que para cualquier $M_2$-partición $(A',B')$ de $G$ de tamaño $(\alpha',\beta')\in S(G)$, se cumple que $(x,y)$ no domina a $(\alpha',\beta')$. Si $(\alpha',\beta')\in S'(G)$, entonces $(x,y)$ no domina a $(\alpha',\beta')$. En el caso contrario, para cualesquiera $M_2$-particiones $P_1=(A_1,B_1)$ y $P_2=(A_2,B_2)$ de $G_1$ y $G_1$ respectivamente de tamaños $(\alpha_1,\beta_1)\in \mu(G_1)$ y $(\alpha_2,\beta_2)\in \mu(G_2)$ respectivamente, se cumplen las siguientes condiciones:
\begin{itemize}
    \item $\alpha' \neq \alpha_1+\alpha_2$ o $\beta' \neq \beta_1 + \beta_2$ y
    \item $\alpha' \neq \emph{min}(\alpha_1+\beta_2, \alpha_2+\beta_1)$ o  $\beta' \neq \emph{max}(\alpha_1+\beta_2, \alpha_2+\beta_1)$. 
\end{itemize}
Luego, ni $A'$ ni $B'$ son el resultado de la unión de una parte de $P_1$ y una parte de $P_2$. Sea $(A'',B'')$ una $M_2$-partición de $G$ tal que $A''=A_1\cup A_2$ y $B''=B_1\cup B_2$ y $(\alpha'', \beta'')$ el tamaño de $(A'',B'')$, notemos que $(\alpha'', \beta'')\in S'(G)$ y que necesariamente $(\alpha', \beta')$ domina a $(\alpha'', \beta'')$. Luego, como $(\alpha'', \beta'')\in S'(G)$, entonces $(x,y)$ no domina a $(\alpha'', \beta'')$. Así, como la dominación es una relación transitiva, $(x,y)$ no domina a $(\alpha', \beta')$.

Recíprocamente. Sea $(x,y)\in\mu(G)$, entonces existe una $M_2$-partición $(A,B)$ de $G$ de tamaño $(x,y)$. Luego, existen $M_2$-particiones $(A_1,B_1)$ de tamaño $(\alpha_1, \beta_1)$ y $(A_2,B_2)$ de tamaño $(\alpha_2, \beta_2)$ de $G_1$ y $G_2$ respectivamente tales que $(A,B)=(A_1\cup A_2, B_1 \cup B_2)$. Como todos los vértices de $A_1$ son adyacentes a todos los vértices de $A_2$ y todos los vértices de $B_1$ son adyacentes a todos los vértices de $B_2$, se cumple que $\alpha = \alpha_1+\alpha_2$ y $\beta = \beta_1 + \beta_2$, o bien, se cumple que $\alpha = \emph{min}(\alpha_1+\beta_2, \alpha_2+\beta_1)$ y $\beta = \emph{max}(\alpha_1+\beta_2, \alpha_2+\beta_1)$. 

Veamos que $(\alpha_1, \beta_1)\in\mu(G_1)$. Procedamos por contradicción. Supongamos que $(\alpha_1, \beta_1)\notin\mu(G_1)$. Como $(A_1,B_1)$ existe, entonces $(\alpha_1, \beta_1)\in S(G_1)$. Luego debe de existir otra pareja $(\alpha_1', \beta_1')\in\S(G_1)$ tal que $(\alpha_1, \beta_1)$ domina a $(\alpha_1', \beta_1')$. De esto se sigue que existe una $M_2$-partición $(A_1',B_1')$ de $G_1$ de tamaño $(\alpha_1', \beta_1')$. Como $(A_1'\cup A_2, B_1'\cup B_2)$ es una $M_2$-partición de $G$ cuyo tamaño es dominado por $(x,y)$, entonces $(x,y)\notin\mu(G)$. Lo cual es una contradicción. Así, $(\alpha_1, \beta_1)\in\mu(G_1)$. Análogamente, $(\alpha_2, \beta_2)\in\mu(G_2)$. Así, $(x,y)\in S'(G)$. 

Como $S'(G)\subset S(G)$ y $(x,y)\in\mu(G)$, entonces $(x,y)$ no domina a ningún elemento de $S(G)$, y por lo tanto tampoco domina a ningún elemento de $S'(G)$.

\end{proof}
