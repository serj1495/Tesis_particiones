La presente sección aborda el problema de determinar si una cográfica $G$ tiene
a otra cográfica $H$ como subgráfica inducida haciendo uso de los conceptos de
coárbol binario y subcoárbol. Se proporciona un algoritmo (Algoritmo
\ref{alg_subgraph}) para resolver este problema tal que, si se fija el tamaño de
$H$, su tiempo de ejecución crece de forma lineal con respecto al tamaño de $G$.
Este algoritmo es útil para identificar a las gráficas pertenecientes a una clase
caracterizada a través de su conjunto de obstrucciones mínimas de forma rápida.

\subsubsection{Algoritmo para determinar si un coárbol binario es subcoárbol binario de otro}

% Nunca haces referencia a las definiciones por numero, es innecesario
% usar el ambiente de definicion.
Sean $T$ y $U$ coárboles binarios y $u$ un nodo de $U$, decimos que
$f \colon V(U)\rightarrow\{marcado, no\_marcado\}$ es una
\textbf{\emph{función de verificación}} de $T$ para $U$ si $f(u) = marcado$
si y sólo si el coárbol binario con raíz $u$ es subcoárbol binario de
$T$. Si $f(u) = marcado$, decimos que $f$ \textbf{\emph{marca}} a $u$.

El Algoritmo \ref{alg_subcoarbol} recibe como entradas dos coárboles binarios,
$G$ y $H$ representados por sus raíces $g$ y $h$ respectivamente, y devuelve
una función de verificación, $f_g$, de $G$ para $H$. Este algoritmo crea la
función de verificación de cada subárbol de $G$ para $H$, empezando por
aquellos cuya raíz es más profunda. De esta manera, si la función de
verificación de $G$ para $H$ evaluada en $h$ es $marcado$, entonces $H$ es
subcoárbol de $G$.

\begin{algorithm}[ht!]
\caption{Función\_de\_coasignación}
\label{alg_subcoarbol}
\DontPrintSemicolon % Some LaTeX compilers require you to use \dontprintsemicolon instead
\KwIn{$g$ y $h$, las raíces de dos coárboles binarios para las gráficas $G$ y $H$ respectivamente}
\KwOut{$func$, la función de verificación de $G$ para $H$}

 $func \gets \text{nueva función de coasignación tal que} func(x)=no\_marcado \text{ para todo } x\in V(H)$\;

 \If{$g$ es una hoja}{
    $func \text{ marca a todas las hojas de } H$\;
 }
 \Else{
    $v_{izq} \gets \text{Función\_de\_coasignación}(g.izquierda, h)$\;
    $v_{der} \gets \text{Función\_de\_coasignación}(g.derecha, h)$\;

    \ForEach{nodo \textbf{\emph{de}} H}{
        \If{$v_{izq}(nodo) = marcado \emph{ \textbf{o} } v_{der}(nodo) = marcado$}{
            $func(nodo) \gets marcado$\;
        }
        \ElseIf{nodo.etiqueta = g.etiqueta \emph{\textbf{y}} $v_{izq}$ \emph{marca a uno de los hijos de} nodo \emph{y} $v_{der}$ \emph{al otro}}{
            $func(nodo) \gets marcado$\;
        }
    }

 }

$\Return func$

\end{algorithm}

\begin{theorem}
    La ejecución del Algoritmo \ref{alg_subcoarbol},
    \emph{Función\_de\_coasignación($g$, $h$)} regresa una función, $func$,
    tal que $func$ es una función de verificación de $\acute{a}rbol(g)$ para
    $\acute{a}rbol(h)$.
\end{theorem}

\begin{proof}

    Sea $n$ un nodo de $\acute{a}rbol(h)$. Para probar que $func$ es una
    función de verificación de $\acute{a}rbol(g)$ para $\acute{a}rbol(h)$,
    tenemos que probar que $func(n) = marcado$ si y sólo si
    $\acute{a}rbol(n)$ es subcoárbol de $\acute{a}rbol(g)$.

    Supongamos primero que el algoritmo ha sido ejecutado y que $func$ marca
    a $n$. Procedamos por inducción sobre la altura de $g$.

    \emph{Caso base:} Si $g$ tiene altura 0, entonces $g$ es una hoja, por lo
    que $func$ marca únicamente a las hojas de $\acute{a}rbol(h)$. Como $func$
    marca a $n$, entonces $n$ es una hoja. Luego, la función $f=\{(n,g)\}$ es
    una función de coasignación de $\acute{a}rbol(n)$ a $\acute{a}rbol(g)$,
    por lo que $\acute{a}rbol(n)$ es subcoárbol de $\acute{a}rbol(n)$.

    \emph{Paso inductivo:} Si $g$ tiene altura $k > 0$. Supongamos como
    hipotesis inductiva que, para todo nodo de un coárbol binario, $g'$,
    de altura $k' < k$ se cumple que, si $func' = $
    Función\_de\_coasignación$(g',h)$ marca a un nodo $n'$ de
    $\acute{a}rbol(h)$, entonces $\acute{a}rbol(n')$ es subcoárbol de
    $\acute{a}rbol(g')$.

    Como $g$ no es una hoja, el algoritmo debió de entrar al bloque de
    instrucciones de las líneas 5 a 11. En las líneas 5 y 6 se crean dos
    funciones que cumplen con la hip\'otesis inductiva, ya que $g.izquierda$
    y $g.derecha$ tienen ambas una altura menor a $k$. Como $func$ marca a
    $n$, entonces $n$ debe de cumplir la condición de la línea 8 o la
    condición de la línea 10. Si se cumple la condición de la línea 8,
    entonces $v_{izq}(n) = marcado$ o $v_{der}(n) = marcado$, por lo que
    $\acute{a}rbol(n)$ es subcoárbol de $\acute{a}rbol(g.izquierda)$ o de
    $\acute{a}rbol(g.derecha)$, y por lo tanto es subcoárbol de
    $\acute{a}rbol(g)$. De lo contrario, se cumple la condición de la línea
    10, entonces $v_{izq}$ marca a $n.izquierda$ o a $n.derecha$ y $v_{der}$
    marca al otro.

    Supongamos sin pérdida de generalidad que $v_{izq}$ marca a
    $n.izquierda$ y $v_{der}$ marca a $n.derecha$. Sean $f_i \colon
    V(\acute{a}rbol(n.izquierda)) \rightarrow V(\acute{a}rbol(g.izquierda))$
    la función de coasignación de $\acute{a}rbol(n.izquierda)$ a
    $\acute{a}rbol(g.izquierda)$ y $f_d \colon V(\acute{a}rbol(n.derecha))
    \rightarrow V(\acute{a}rbol(g.derecha))$ la función de coasignación de
    $\acute{a}rbol(n.derecha)$ a $\acute{a}rbol(g.derecha)$; mostremos que la
    función $f = f_i \cup f_d \cup \{(n,g)\}$ es una función de coasignación
    de $\acute{a}rbol(n)$ a $\acute{a}rbol(g)$.

    Como los dominios de $f_i$ y $f_d$ son ajenos y ninguno contiene a $n$,
    entonces $f$ es una función. Como los rangos de $f_i$ y $f_d$ son ajenos,
    ninguno contiene a $g$, y como tanto $f_i$ como $f_d$ son inyectivas,
    entonces $f$ es inyectiva. Por otra parte, por la condición de la línea
    10, sabemos que $n.etiqueta = g.etiqueta$. También sabemos que, sea $x
    \in V(\acute{a}rbol (n.izquierda))$, si $x$ es una hoja, entonces $f(x) =
    f_i(x)$ es una hoja y si no, entonces $x.etiqueta = f_i(x).etiqueta =
    f(x).etiqueta$. Análogamente para un $y \in V(\acute{a}rbol(n.derecha))$ y
    $f_d$. Finalmente, si $n$ es el ancestro común más profundo de dos nodos
    $z_1$ y $z_2$, entonces $z_1$ es descendiente de $n.derecha$ y $z_2$ es
    descendiente de $n.izquierda$ o viceversa. Supongamos lo primero sin
    pérdida de generalidad. Luego, por la condición de la línea 10, $v_{izq}$
    marca a uno y $v_{der}$ marca al otro. Supongamos sin pérdida de
    generalidad que $v_{izq}$ marca a $z_1$ y $v_{der}$ marca a $z_2$.
    Entonces, $f(z_1) = f_i(z_1) \in V(\acute{a}rbol(g.izquierda))$ y $f(z_2)
    = f_i(z_2) \in V(\acute{a}rbol(g.derecha))$, por lo que el ancestro común
    más profundo de $f(z_1)$ y $f(z_2)$ es $g = f(n)$. Así, $f$ es una
    función de coasignación de $\acute{a}rbol(n)$ a $\acute{a}rbol(g)$ y
    $\acute{a}rbol(n)$ es subcoárbol de $\acute{a}rbol(g)$.

    Rec\'iprocamente, supongamos $\acute{a}rbol(n)$ es subcoárbol de
    $\acute{a}rbol(g)$. Seguiremos la ejecución del algoritmo para mostrar
    que, al final de la misma, $func$ marcará a $n$. Sea $f$ la función de
    cosignación de $\acute{a}rbol(n)$ a $\acute{a}rbol(g)$, procedamos por
    inducción sobre la altura de $g$.

    \emph{Caso base:} Si la altura de $g$ es 0, entonces $g$ es una hoja,
    por lo que se cumple con la condición de la línea 2 y se ejecuta la
    línea 3, haciendo que $func$ marque todas las hojas de $H$. Como
    $\acute{a}rbol(n)$ es subcoárbol de $\acute{a}rbol(g)$ y
    $\acute{a}rbol(g)$ sólo tiene un nodo, entonces $n$ debe de ser una
    hoja. Luego, $func$ marca a $n$.

    \emph{Paso inductivo:} Si $g$ tiene altura $k > 0$. Supongamos como
    hip\'otesis inductiva que todo coárbol, $g'$, con altura $k' < k$ cumple
    que, siendo $n'$ un nodo de $\acute{a}rbol(h)$, si $\acute{a}rbol(n')$ es
    subcoárbol de $\acute{a}rbol(g')$, entonces $func' =
    $Función\_de\_coasignación $(g',h)$ marca a $n'$.

    Como $g$ no es una hoja, el algoritmo ejecuta las líneas 5 y 6, y
    posteriormente el bloque de las líneas 8 a 11 para cada nodo de $H$. Si
    $n$ es marcada por $v_{izq}$ o $v_{der}$, entonces se ejecuta la línea 9 y
    $func$ marca a $n$. En el caso contrario, probemos que se cumple la
    condición de la línea 10.

    Mostremos primero que $f(n) = g$ procediendo por contradicción.
    Supongamos que $f(n) = x$ para algún $x \in V(\acute{a}rbol(g))-\{g\}$.
    Como $x$ es descendiente de $g$, tiene altura menor a $k$. También
    sabemos que $f$ es una función de coasignación de $\acute{a}rbol(n)$ a
    $\acute{a}rbol(x)$, por lo que, por hip\'otesis inductiva, $n$ debería de
    ser marcado ya sea por $v_{izq}$ o por $v_{der}$, lo que es una
    contradicción. Luego, $f(n) = g$, y por lo tanto $n$ no es una hoja y
    $f(n).etiqueta = g.etiqueta$.

    Mostremos ahora que tanto $n.izquierda$ como $n.derecha$ son marcados
    cada uno ya sea por $v_{izq}$ o por $v_{der}$.  Sabemos que
    $f(n.izquierda)$ y $f(n.derecha)$ son descendientes de $r$. Como
    $f \mid_{V(\acute{a}rbol(n.izquierda))}$ es una función de coasignación
    de $\acute{a}rbol(n.izquierda)$ a $\acute{a}rbol(f(n.izquierda))$ y
    $f(n.izquierda) \neq g$ ya que $f$ es inyectiva, entonces
    $\acute{a}rbol(n.izquierda)$ es subcoárbol de algún descendiente de $g$,
    al que llamaremos $y$. Como $y$ tiene altura menor a $k$, su función de
    verificación correspondiente marca a $n.izquierda$ (por hip\'otesis
    inductiva), y por la condición de la línea 8, sus ancestros también lo
    marcan. Luego $v_{izq}$ o $v_{der}$ marcan a $n.izquierda$. Análogamente
    para $n.derecha$. Así, tanto $n.izquierda$ como $n.derecha$ están
    marcados cada uno ya sea en $v_{izq}$ o en $v_{der}$.

    Mostremos, por último, que uno es marcado por $v_{izq}$ y el otro es
    marcado por $v_{der}$. Como el ancestro común más profundo de
    $n.izquierda$ y $n.derecha$ es $n$, y $f(n)=g$, entonces el ancestro
    común más profundo de $f(n.izquierda)$ y $f(n.derecha)$ debe de ser $g$.
    Luego, $f(n.izquierda)$ está en una rama de $g$ y $f(n.derecha)$ está en
    la otra. Supongamos sin pérdida de generalidad que $f(n.izquierda)$ está
    en la rama izquierda de $g$ y $f(n.derecha)$ está en la rama derecha.
    Como $f\mid_{V(\acute{a}rbol(n.izquierda))}$ es una función de
    coasignación de $\acute{a}rbol(n.izquierda)$ a $\acute{a}rbol
    (g.izquierda)$ y por hip\'otesis inductiva, entonces $v_{izq}$ marca a
    $n.izquierda$. De forma análoga, $v_{der}$ marca a $n.derecha$.
    Concluyendo, como $n.etiqueta = r.etiqueta$ y tanto $n.izquierda$ como
    $n.derecha$ son marcados uno por $v_{izq}$ y el otro por $v_{der}$, se
    cumple la condición de la línea 10 y $func$ marca a $n$. Así, al final de
    la ejecución del algoritmo, $n$ estará marcado.

\end{proof}

Dado que, para cada nodo de $G$ se crea una función de verificación cuyo
dominio es el conjunto de los nodos de $H$, el tiempo de ejecución del
algoritmo tiene crecimiento $O(|V(G)| |V(H)|)$.

\subsubsection{Determinar si una cográfica es subcográfica de otra}

Usando el Algoritmo \ref{alg_subcoarbol} como subrutina, se puede idear
un algoritmo para determinar si una cográfica $H$ es subgráfica de otra
cográfica $G$, al buscar todas las posibles formas del coárbol binario de
$H$ en un coárbol binario fijo de $G$.

\begin{algorithm}[ht!]
\caption{Es\_subgráfica}
\label{alg_subgraph}
\DontPrintSemicolon % Some LaTeX compilers require you to use \dontprintsemicolon instead
\KwIn{$g$ y $h$, las raíces de dos coárboles, $G$ y $H$ respectivamente.}
\KwOut{$verdadero$ si la cográfica representada por $H$ es subgráfica de la cográfica representada por $G$. $falso$ en el caso contrario.}

$g\_bin \gets \text{CrearÁrbolBinario}(g)$\;
$h\_bins \gets$ las raíces de todos los coárboles binarios correspondientes a $H$\;

\ForEach{bin \textbf{\emph{en}} h\_bins}{
    $f = \text{Función\_de\_coasignación}(g\_bin,bin)$\;
    \If{f(bin) = marcado}{
        $\Return\ verdadero$\;
    }
}

$\Return\ falso$\;

\end{algorithm}

Como la línea 1 Algoritmo \ref{alg_subgraph} se ejecuta en tiempo
$O(|V(G)|)$, la complejidad temporal de éste depende del número de coárboles
binarios correspondientes a $H$ (que crece con mayor rapidez). Sin embargo,
si se fija $H$, la complejidad temporal de éste es simplemente  $O(|V(G)|)$.
Fijar $H$ resultará útil cuando se esté resolviendo un problema específico
como el de encontrar una obstrucción mínima en una gráfica. La aplicación de
este algoritmo en el presente trabajo de tesis es desarrollar algoritmos que
nos permitan identificar a los elementos de una clase hereditaria de
cográficas en tiempo lineal. Esto se muestra en el Algoritmo
\ref{alg_esta_en_clase}.

\begin{algorithm}[ht!]
\caption{Pertenece_a_la_clase}
\label{alg_esta_en_clase}
\DontPrintSemicolon % Some LaTeX compilers require you to use \dontprintsemicolon instead
\KwIn{$g$, la raíz del coárbol de una cográfica $G$.}
\KwOut{$verdadero$ si $G$ pertenece a la clase hereditaria de cográficas $C$}

$g\_bin \gets \text{CrearÁrbolBinario}(g)$\;
$C\_bins \gets$ las raíces de todos los coárboles binarios de todas las obstrucciones mínimas de la clase $C$\;

\ForEach{$bin$ \textbf{\emph{en}} $C\_bins$}{
    $f = \text{Función\_de\_coasignación}(g\_bin,bin)$\;
    \If{$f(bin) = marcado$}{
        $\Return\ verdadero$\;
    }
}

$\Return\ falso$\;

\end{algorithm}

El Algoritmo \ref{alg_esta_en_clase} es una variación del Algoritmo
\ref{alg_subgraph} en el que se fija una clase $C$ cuyos elementos se desea
identificar. Al computar el conjunto de coárboles binarios de cada una de las
obstrucciones mínimas de $C$ antes de la ejecución del algoritmo
\ref{alg_esta_en_clase}, la línea 2 del algoritmo se puede ejecutar en tiempo
constante. Gracias a esto obtenemos un algoritmo que determina si una gráfica
$G$ pertenece a la clase $C$ en tiempo $O(|V(G)|)$.
