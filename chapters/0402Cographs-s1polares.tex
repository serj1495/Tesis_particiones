En \cite{Fernando}, Contreras-Mendoza y Hern\'andez-Cruz
exhiben el conjunto de obstrucciones mínimas de las
cográficas $(\infty, 1)$-polares. Este conjunto es
utilizado para describir cómo se puede obtener el conjunto
de obstrucciones mínimas de cualquiera de las clases de
cográficas $(s,1)$-polares dado un entero $s \ge 2$. En
nuestra investigación encontramos un resultado similar,
el Lema \ref{lema_1infM2}, donde describimos a las obstrucciones
m\'inimas para las gráficas que aceptan una partición en
un conjunto independiente y una gráfica multipartita completa.

A continuaci\'on reproducimos el resultado antes mencionado.

\begin{theorem}[\cite{Fernando}]
\label{thm:s,1-ess}
  Sea $G$ una cográfica. Entonces $G$ es $(\infty,1)$-polar
  si y sólo si no contiene alguna de las gráficas de la Figura
  \ref{obsts_cografics_esenciales_1spolares} como subgráfica
  inducida. Este conjunto es llamado el conjunto de obstrucciones
  esenciales.
\end{theorem}

\begin{figure}[ht!]
\begin{center}
\begin{tikzpicture}

\begin{scope}[xshift=0cm,scale=1]
\node [style=vertex] (1) at (0,0) {};
\node [style=vertex] (2) at (1,0) {};
\node [style=vertex] (3) at (0,1) {};
\node [style=vertex] (4) at (1,1) {};
\node [style=vertex] (5) at (0.5,2) {};
\foreach \i/\j in {1/2,3/4}
  \draw [style=edge] (\i) to (\j);
\node at (0.5,-0.75) {\parbox{0.3\linewidth}{\subcaption*{$G_1$}}};
\end{scope}

\begin{scope}[xshift=2cm,scale=1]
\node [style=vertex] (1) at (0,0) {};
\node [style=vertex] (2) at (1,0) {};
\node [style=vertex] (3) at (0,1) {};
\node [style=vertex] (4) at (1,1) {};
\node [style=vertex] (5) at (0,2) {};
\node [style=vertex] (6) at (1,2) {};
\foreach \i/\j in {1/2,1/3,2/4,3/4}
  \draw [style=edge] (\i) to (\j);
\node at (0.5,-0.75) {\parbox{0.3\linewidth}{\subcaption*{$G_2$}}};
\end{scope}

\begin{scope}[xshift=4cm,scale=1]
\node [style=vertex] (1) at (0,0) {};
\node [style=vertex] (2) at (1,0) {};
\node [style=vertex] (3) at (0,1) {};
\node [style=vertex] (4) at (1,1) {};
\node [style=vertex] (5) at (0,2) {};
\node [style=vertex] (6) at (1,2) {};
\foreach \i/\j in {1/3,2/4,3/5,4/6}
  \draw [style=edge] (\i) to (\j);
\node at (0.5,-0.75) {\parbox{0.3\linewidth}{\subcaption*{$G_3$}}};
\end{scope}

\begin{scope}[xshift=6cm,scale=1]
\node [style=vertex] (1) at (0.75,0) {};
\node [style=vertex] (2) at (0,0.75) {};
\node [style=vertex] (3) at (0.75,0.75) {};
\node [style=vertex] (4) at (1.5,0.75) {};
\node [style=vertex] (5) at (0.75,1.5) {};
\node [style=vertex] (6) at (0.75,2) {};
\foreach \i/\j in {1/2,1/3,1/4,2/3,2/5,3/4,4/5}
  \draw [style=edge] (\i) to (\j);
\node at (0.75,-0.75) {\parbox{0.3\linewidth}{\subcaption*{$G_4$}}};
\end{scope}

\end{tikzpicture}
\end{center}
\caption{Obstrucciones mínimas para las gráficas polares.}
\label{obsts_cografics_esenciales_1spolares}
\end{figure}

Haciendo uso de estas obstrucciones esenciales,
los autores caracterizan, para cualquier entero
$s$, con $s \ge 2$, a las obstrucciones m\'inimas
para la clase de cogr\'aficas $(s,1)$-polares.
Notemos que, adem\'as de las
obstrucciones esenciales, existen cuatro familias
espor\'adicas de obstrucciones m\'inimas, y se
plantea una regla recursiva para generar
obstrucciones m\'inimas para las cogr\'aficas
$(s,1)$-polares utilizando obstrucciones
m\'inimas no esenciales para las cogr\'aficas
$(t,1)$-polares, con $t < s$.

\begin{theorem}
\label{thm:s,1}
  Sea $G$ una cográfica y $s \ge 2$ un entero. Entonces
  $G$ es una obstrucción mínima de las cográficas
  $(s,1)$-polares si y sólo si es una de las siguientes
  gráficas:

  \begin{itemize}
    \item Una de las cuatro obstrucciones esenciales.

    \item $2K_{s+1}$.

    \item $K_2 + (\overline{K_2}\oplus K_s)$.

    \item $K_1 + (C_4 \oplus K_{s-1})$.

    \item $\overline{(s+1)K_2}$.

    \item El complemento de $G$ es inconexo con componentes
      $G_1, \dots, G_t$ tales que $t \leq s$, y cada $G_i$
      es el complemento de una obstrucción mínima no esencial
      de la clase de cográficas $(s_i, 1)$-polares con
      $\sum^{t}_{i=1}s_i = s-t+1$.
  \end{itemize}
\end{theorem}

%Si $s$ y $k$ son enteros positivos arbitrarios, determinar
%las obstrucciones m\'inimas para las cogr\'aficas
%$(s,k)$-polares parece ser un problema dif\'icil de
%resolver.

El resultado presentado en el Teorema
\ref{thm:s,1} es un ejemplo de c\'omo, al restringir
un problema a un subproblema bien definido, es posible
encontrar soluciones parciales al problema general.   En
particular, se obtuvo una caracterizaci\'on por obstrucciones
m\'inimas para una subclase infinita de las cogr\'aficas
$(s,k)$-polares.

M\'as a\'un, una interpretaci\'on de los Teoremas \ref{thm:s,1-ess}
y \ref{thm:s,1} es que resulta posible pensar a las gr\'aficas
$(\infty,1)$-polares como el l\'imite cuando $s$ tiende al
infinito de las gr\'aficas $(s,1)$-polares. Podemos observar que
las gráficas que son obstrucciones mínimas de las gráficas
$(s,1)$-polares para cualquier entero $s \ge 2$ también son
obstrucciones mínimas de las gráficas $(\infty,1)$-polares. En
nuestra investigación encontramos un resultado similar para las
gráficas $(\alpha,\beta)$-$M_2$\footnote{La clase
$(\alpha,\beta)$-$M_2$ es la clase constituida por todas las
gráficas que aceptan una partición en dos gráficas multipartitas
completas, una formada por a lo más $\alpha$ conjuntos estables
y la otra formada por a lo más $\beta$ conjuntos estables.}, ya
que las obstrucciones mínimas de la clase $(1,\infty)$-$M_2$ son
obstrucciones mínimas de la clase $(1,m)$-$M_2$, para cualquier
valor de $m$ con $m \ge 2$.  Con base en la idea anterior,
a partir de las obstrucciones mínimas de las clases
$(2,m)$-$M_2$, con $m \in \{ 3, \dots, 7 \}$ que generamos
computacionalmente, encontramos algunas obstrucciones mínimas
de la clase $(2,\infty)$-$M_2$. Aplicando este mismo proceso,
podemos encontrar obstrucciones mínimas para la clase $\alpha,
\infty$-$M_2$ dado un entero $\alpha \ge 3$.
