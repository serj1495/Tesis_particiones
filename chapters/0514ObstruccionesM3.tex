\begin{theorem} \label{teo_obsts_m2}

    Para una cográfica $G$, las siguientes afirmaciones son equivalentes.
    \begin{enumerate}[(a)] 
        \item $G \in M_3$.
        \item $G$ no contiene a ninguna de las gráficas de las Figuras \ref{obsts_O_M3} como subgráfica inducida.
    \end{enumerate}

\end{theorem}

\begin{figure}[h]





\begin{subfigure}{\textwidth}
\begin{center}
\begin{tikzpicture}
\begin{scope}[xshift=0cm,scale=1]
%K4
\node [style=vertex] (1) at (0,1) {};
\node [style=vertex] (2) at (1,1) {};
\node [style=vertex] (3) at (0.5,1.3) {};
\node [style=vertex] (4) at (0.5,1.75) {};
%K3
\node [style=vertex] (5) at (0,2.25) {};
\node [style=vertex] (6) at (1,2.25) {};
\node [style=vertex] (7) at (0.5,3) {};
%K2
\node [style=vertex] (8) at (0,3.5) {};
\node [style=vertex] (9) at (1,3.5) {};
%K1
\node [style=vertex] (10) at (0.5,4) {};

\foreach \i/\j in {1/2,1/3,1/4,2/3,2/4,3/4,5/6,5/7,6/7,8/9} \draw [style=edge] (\i) to (\j);
\node at (0.5,0) {\parbox{0.3\linewidth}{\subcaption*{$O_{3,1}$}}};
\end{scope}

\begin{scope}[xshift=2.5cm,scale=1]
%K4
\node [style=vertex] (1) at (0,1) {};
\node [style=vertex] (2) at (1,1) {};
\node [style=vertex] (3) at (0.5,1.3) {};
\node [style=vertex] (4) at (0.5,1.75) {};
%K3
\node [style=vertex] (5) at (0,2.25) {};
\node [style=vertex] (6) at (1,2.25) {};
\node [style=vertex] (7) at (0.5,3) {};
%K2
\node [style=vertex] (8) at (0,4) {};
\node [style=vertex] (9) at (1,4) {};
%K1
\node [style=vertex] (10) at (0.5,3.5) {};

\foreach \i/\j in {1/2,1/3,1/4,2/3,2/4,3/4,5/6,5/7,6/7,8/9} \draw [style=edge] (\i) to (\j);
\foreach \i/\j in {7/10} \draw [style=edge] (\i) to (\j);
\node at (0.5,0) {\parbox{0.3\linewidth}{\subcaption*{$O_{3,2}$}}};
\end{scope}

\begin{scope}[xshift=5cm,scale=1]
%K4
\node [style=vertex] (1) at (0,1) {};
\node [style=vertex] (2) at (1,1) {};
\node [style=vertex] (3) at (0.5,1.3) {};
\node [style=vertex] (4) at (0.5,1.75) {};
%K3
\node [style=vertex] (5) at (0,2.75) {};
\node [style=vertex] (6) at (1,2.75) {};
\node [style=vertex] (7) at (0.5,3.5) {};
%K2
\node [style=vertex] (8) at (0,4) {};
\node [style=vertex] (9) at (1,4) {};
%K1
\node [style=vertex] (10) at (0.5,2.25) {};

\foreach \i/\j in {1/2,1/3,1/4,2/3,2/4,3/4,5/6,5/7,6/7,8/9} \draw [style=edge] (\i) to (\j);
\foreach \i/\j in {4/10} \draw [style=edge] (\i) to (\j);
\node at (0.5,0) {\parbox{0.3\linewidth}{\subcaption*{$O_{3,3}$}}};
\end{scope}

\begin{scope}[xshift=7.5cm,scale=1]
%K4
\node [style=vertex] (1) at (0,1) {};
\node [style=vertex] (2) at (1,1) {};
\node [style=vertex] (3) at (0.5,1.3) {};
\node [style=vertex] (4) at (0.5,1.75) {};
%K3
\node [style=vertex] (5) at (0,2.75) {};
\node [style=vertex] (6) at (1,2.75) {};
\node [style=vertex] (7) at (0.5,3.5) {};
%K2
\node [style=vertex] (8) at (0,4) {};
\node [style=vertex] (9) at (1,4) {};
%K1
\node [style=vertex] (10) at (0.5,2.25) {};

\foreach \i/\j in {1/2,1/3,1/4,2/3,2/4,3/4,5/6,5/7,6/7,8/9} \draw [style=edge] (\i) to (\j);
\foreach \i/\j in {4/10,1/10} \draw [style=edge] (\i) to (\j);
\node at (0.5,0) {\parbox{0.3\linewidth}{\subcaption*{$O_{3,4}$}}};
\end{scope}

\begin{scope}[xshift=10cm,scale=1]
%K4
\node [style=vertex] (1) at (0,1) {};
\node [style=vertex] (2) at (1,1) {};
\node [style=vertex] (3) at (0.5,1.3) {};
\node [style=vertex] (4) at (0.5,1.75) {};
%K3
\node [style=vertex] (5) at (0,2.75) {};
\node [style=vertex] (6) at (1,2.75) {};
\node [style=vertex] (7) at (0.5,3.5) {};
%K2
\node [style=vertex] (8) at (0,2.25) {};
\node [style=vertex] (9) at (1,2.25) {};
%K1
\node [style=vertex] (10) at (0.5,4) {};

\foreach \i/\j in {1/2,1/3,1/4,2/3,2/4,3/4,5/6,5/7,6/7,8/9} \draw [style=edge] (\i) to (\j);
\foreach \i/\j in {4/8,4/9} \draw [style=edge] (\i) to (\j);
\node at (0.5,0) {\parbox{0.3\linewidth}{\subcaption*{$O_{3,5}$}}};
\end{scope}
\end{tikzpicture}
\end{center}
\end{subfigure}

\begin{subfigure}{\textwidth}
\begin{center}
\begin{tikzpicture}

\begin{scope}[xshift=0cm,scale=1]
%K4
\node [style=vertex] (1) at (0,1) {};
\node [style=vertex] (2) at (1,1) {};
\node [style=vertex] (3) at (0.5,1.3) {};
\node [style=vertex] (4) at (0.5,1.75) {};
%K3
\node [style=vertex] (5) at (0,3.25) {};
\node [style=vertex] (6) at (1,3.25) {};
\node [style=vertex] (7) at (0.5,4) {};
%K2
\node [style=vertex] (8) at (0.5,2.5) {};
\node [style=vertex] (9) at (1,2.5) {};
%K1
\node [style=vertex] (10) at (0,2.5) {};

\foreach \i/\j in {1/2,1/3,1/4,2/3,2/4,3/4,5/6,5/7,6/7,8/9} \draw [style=edge] (\i) to (\j);
\foreach \i/\j in {4/8,4/9,4/10} \draw [style=edge] (\i) to (\j);
\node at (0.5,0) {\parbox{0.3\linewidth}{\subcaption*{$O_{3,6}$}}};
\end{scope}

\begin{scope}[xshift=2.5cm,scale=1]
%K4
\node [style=vertex] (1) at (0,1) {};
\node [style=vertex] (2) at (1,1) {};
\node [style=vertex] (3) at (0.5,1.3) {};
\node [style=vertex] (4) at (0.5,1.75) {};
%K3
\node [style=vertex] (5) at (0,3.25) {};
\node [style=vertex] (6) at (1,3.25) {};
\node [style=vertex] (7) at (0.5,4) {};
%K2
\node [style=vertex] (8) at (0.5,2.5) {};
\node [style=vertex] (9) at (1,2.5) {};
%K1
\node [style=vertex] (10) at (0,2.5) {};

\foreach \i/\j in {1/2,1/3,1/4,2/3,2/4,3/4,5/6,5/7,6/7,8/9} \draw [style=edge] (\i) to (\j);
\foreach \i/\j in {4/8,4/9,4/10,1/10} \draw [style=edge] (\i) to (\j);
\node at (0.5,0) {\parbox{0.3\linewidth}{\subcaption*{$O_{3,7}$}}};
\end{scope}

\begin{scope}[xshift=5cm,scale=1]
%K4
\node [style=vertex] (1) at (0,1) {};
\node [style=vertex] (2) at (1,1) {};
\node [style=vertex] (3) at (0.5,1.3) {};
\node [style=vertex] (4) at (0.5,1.75) {};
%K3
\node [style=vertex] (5) at (0,2.75) {};
\node [style=vertex] (6) at (1,2.75) {};
\node [style=vertex] (7) at (0.5,3.5) {};
%K2
\node [style=vertex] (8) at (0,2.25) {};
\node [style=vertex] (9) at (1,2.25) {};
%K1
\node [style=vertex] (10) at (0.5,4) {};

\foreach \i/\j in {1/2,1/3,1/4,2/3,2/4,3/4,5/6,5/7,6/7,8/9} \draw [style=edge] (\i) to (\j);
\foreach \i/\j in {4/8,4/9,7/10} \draw [style=edge] (\i) to (\j);
\node at (0.5,0) {\parbox{0.3\linewidth}{\subcaption*{$O_{3,5}$}}};
\end{scope}

\end{tikzpicture}
\end{center}
\end{subfigure}

\setlength{\abovecaptionskip}{-15pt}
\caption{Obstrucciones mínimas para la clase $M_2$.}
\label{obsts_O_M3}
\end{figure}

\begin{proof}

\end{proof}
