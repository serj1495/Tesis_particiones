La Teoría de Gráficas es la rama de las Matemáticas Discretas encargada del estudio de los objetos matemáticos conocidos como gráficas. Una gráfica $G$ es una pareja ordenada de conjuntos ajenos $(V,E)$ tal que $E$ es un conjunto de parejas no ordenadas de elementos de $V$. Llamamos a $V$ el conjunto de vértices de $G$ y a $E$ el conjunto de aristas de $G$. Si dos elementos de $V$ forman una pareja que está en el conjunto $E$, decimos que estos son adyacentes. Con frecuencia, las gráficas son representadas con un dibujo en el que los vértices aparecen como puntos o pequeños círculos y las aristas como líneas que unen a los vértices adyacentes. En esta tesis trabajamos con un tipo particular de gráficas conocidas como cográficas, que pueden ser caracterizadas de múltiples maneras. Una de ellas es que son las gráficas que no tienen a la trayectoria de $4$ v\'ertices, $P_4$, como subgráfica inducida.

Un problema clásico en la Teoría de Gráficas es la coloración de gráficas que consiste en determinar si los vértices de una gráfica se pueden etiquetar con un número determinado de etiquetas diferentes, también llamadas colores, de forma tal que, si dos vértices son adyacentes, estos tienen etiquetas diferentes. Una generalización de las coloraciones de gráficas son las particiones matriciales, que consisten en determinar si los vértices de una gráfica se pueden etiquetar con un número determinado de etiquetas diferentes de manera tal que los vértices con la misma etiqueta cumplan con una propiedad de homogeneidad. En esta tesis abordamos un problema de particiones matriciales. Estudiamos a las clases de cográficas que se definen de la siguiente manera. Dado $i$, un entero mayor o igual a uno, la clase $M_i$ es la clase de las cográficas cuyo conjunto de vértices acepta una partición en $i$ partes tal que cada parte induce una gráfica multipartita completa. La clase $M_1$, que es la clase de las cográficas multipartitas completas, ha sido ampliamente estudiada. Sin embargo, las clases $M_i$ para valores de $i$ mayores a uno no han sido estudiadas con anterioridad. Nuestra investigación tiene como base el estudio de la clase $M_2$, para el cuál tomamos como guía la investigación realizada sobre las cográficas polares. Caracterizamos a la clase $M_2$ a través de su conjunto de obstrucciones mínimas, presentamos un algoritmo para reconocer a sus elementos y un algoritmo certificador y estudiamos a un conjunto de subclases de esta clase a las que llamamos clases $(\alpha,\beta)$-$M_2$. De igual manera, caracterizamos a la clase $M_3$ a través de su conjunto de obstrucciones mínimas y proporcionamos dos familias de obstrucciones mínimas para cualquier clase $M_i$.
