En esta subsección mostramos las obstrucciones mínimas de algunas clases $(\alpha, \beta)$-$M_2$ encontradas con el Algoritmo \ref{alg_obstrucciones_alfabeta}. Las obstrucciones se buscaron en coárboles de hasta 15 hojas.

\subsubsection{Clases $(1,\beta)$-$M_2$}

Las Figuras \ref{obsts_1_1_M2},\ref{obsts_1_2_M2},\ref{obsts_1_3_M2},\ref{obsts_1_4_M2} y \ref{obsts_1_5_M2} muestran los conjuntos de obstrucciones mínimas encontradas con el Algoritmo \ref{alg_obstrucciones_alfabeta} para las clases $(1,1)$-$M_2$, $(1,2)$-$M_2$, $(1,3)$-$M_2$, $(1,4)$-$M_2$ y $(1,5)$-$M_2$ respectivamente.

\begin{figure}[ht!]

\begin{subfigure}{\textwidth}
\begin{center}
\begin{tikzpicture}
\begin{scope}[xshift=0cm,scale=1]

\node [style=vertex] (1) at (0,0.5) {};
\node [style=vertex] (2) at (1,0.5) {};
\node [style=vertex] (3) at (0.5,1.25) {};

\foreach \i/\j in {1/2,1/3,2/3} \draw [style=edge] (\i) to (\j);
\node at (0.5,0) {\parbox{0.3\linewidth}{\subcaption*{$o_{(1,1),1}$}}};
\end{scope}

\end{tikzpicture}
\end{center}
\end{subfigure}

%\setlength{\abovecaptionskip}{-15pt}
\caption{Obstrucción mínima de la clase $(1,1)$-$M_2$.}
\label{obsts_1_1_M2}
\end{figure}
\begin{figure}[ht!]

\begin{subfigure}{\textwidth}
\begin{center}
\begin{tikzpicture}

\begin{scope}[xshift=0cm,scale=1]

\node [style=vertex] (1) at (0,0.5) {};
\node [style=vertex] (2) at (1,0.5) {};
\node [style=vertex] (3) at (0,1) {};
\node [style=vertex] (4) at (1,1) {};
\node [style=vertex] (5) at (0.5,1.75) {};

\foreach \i/\j in {1/2,3/4,3/5,4/5} \draw [style=edge] (\i) to (\j);
\node at (0.5,0) {\parbox{0.3\linewidth}{\subcaption*{$o_{(1,2),1}$}}};
\end{scope}

\begin{scope}[xshift=2.5cm,scale=1]

\node [style=vertex] (1) at (0,0.5) {};
\node [style=vertex] (2) at (1,0.5) {};
\node [style=vertex] (3) at (0.5,0.85) {};
\node [style=vertex] (4) at (0.5,1.5) {};

\foreach \i/\j in {1/2,1/3,1/4,2/3,2/4,3/4} \draw [style=edge] (\i) to (\j);
\node at (0.5,0) {\parbox{0.3\linewidth}{\subcaption*{$o_{(1,2),2}$}}};
\end{scope}

\end{tikzpicture}
\end{center}
\end{subfigure}

%\setlength{\abovecaptionskip}{-15pt}
\caption{Algunas obstrucciones mínimas de la clase $(1,2)$-$M_2$.}
\label{obsts_1_2_M2}
\end{figure}
\begin{figure}[ht!]

\begin{subfigure}{\textwidth}
\begin{center}
\begin{tikzpicture}

\begin{scope}[xshift=0cm,scale=1]

\node [style=vertex] (1) at (0,0.5) {};
\node [style=vertex] (2) at (1,0.5) {};
\node [style=vertex] (3) at (0,1) {};
\node [style=vertex] (4) at (1,1) {};
\node [style=vertex] (5) at (0.5,1.75) {};

\foreach \i/\j in {1/2,3/4,3/5,4/5} \draw [style=edge] (\i) to (\j);
\node at (0.5,0) {\parbox{0.3\linewidth}{\subcaption*{$o_{(1,3),1}$}}};
\end{scope}

\begin{scope}[xshift=2.5cm,scale=1]

\node [style=vertex] (1) at (0.5,0.5) {};
\node [style=vertex] (2) at (1.5,0.5) {};
\node [style=vertex] (3) at (0.5,1.5) {};
\node [style=vertex] (4) at (1.5,1.5) {};
\node [style=vertex] (5) at (0,2) {};
\node [style=vertex] (6) at (2,2) {};

\foreach \i/\j in {1/2,1/3,1/4,2/3,2/4,3/4,5/1,5/3,5/6,6/2,6/4} \draw [style=edge] (\i) to (\j);
\node at (1,0) {\parbox{0.3\linewidth}{\subcaption*{$o_{(1,3),2}$}}};
\end{scope}

\begin{scope}[xshift=6cm,scale=1]

\node [style=vertex] (1) at (0.25,0.5) {};
\node [style=vertex] (2) at (1.25,0.5) {};
\node [style=vertex] (3) at (0,1.15) {};
\node [style=vertex] (4) at (1.5,1.15) {};
\node [style=vertex] (5) at (0.75,1.75) {};

\foreach \i/\j in {1/2,1/3,1/4,1/5,2/3,2/4,2/5,3/4,3/5,4/5} \draw [style=edge] (\i) to (\j);
\node at (0.75,0) {\parbox{0.3\linewidth}{\subcaption*{$o_{(1,3),3}$}}};
\end{scope}

\end{tikzpicture}
\end{center}
\end{subfigure}

%\setlength{\abovecaptionskip}{-15pt}
\caption{Algunas obstrucciones mínimas de la clase $(1,3)$-$M_2$.}
\label{obsts_1_3_M2}
\end{figure}
\begin{figure}[ht!]

\begin{subfigure}{\textwidth}
\begin{center}
\begin{tikzpicture}

\begin{scope}[xshift=0cm,scale=1]

\node [style=vertex] (1) at (0,0.5) {};
\node [style=vertex] (2) at (1,0.5) {};
\node [style=vertex] (3) at (0,1) {};
\node [style=vertex] (4) at (1,1) {};
\node [style=vertex] (5) at (0.5,1.75) {};

\foreach \i/\j in {1/2,3/4,3/5,4/5} \draw [style=edge] (\i) to (\j);
\node at (0.5,0) {\parbox{0.3\linewidth}{\subcaption*{$o_{(1,4),1}$}}};
\end{scope}

\begin{scope}[xshift=2.5cm,scale=1]

\node [style=vertex] (1) at (0.5,0.5) {};
\node [style=vertex] (2) at (1.5,0.5) {};
\node [style=vertex] (3) at (0.5,1.5) {};
\node [style=vertex] (4) at (1.5,1.5) {};
\node [style=vertex] (5) at (0,2) {};
\node [style=vertex] (6) at (2,2) {};

\foreach \i/\j in {1/2,1/3,1/4,2/3,2/4,3/4,5/1,5/3,5/6,6/2,6/4} \draw [style=edge] (\i) to (\j);
\node at (1,0) {\parbox{0.3\linewidth}{\subcaption*{$o_{(1,4),2}$}}};
\end{scope}

\begin{scope}[xshift=6cm,scale=1]

\node [style=vertex] (1) at (0.5,0.5) {};
\node [style=vertex] (2) at (1.5,0.5) {};
\node [style=vertex] (3) at (0,1.25) {};
\node [style=vertex] (4) at (2,1.25) {};
\node [style=vertex] (5) at (0.5,2) {};
\node [style=vertex] (6) at (1.5,2) {};

\foreach \i/\j in {1/2,1/3,1/4,1/5,1/6,2/3,2/4,2/5,2/6,3/4,3/5,3/6,4/5,4/6,5/6} \draw [style=edge] (\i) to (\j);
\node at (1,0) {\parbox{0.3\linewidth}{\subcaption*{$o_{(1,4),3}$}}};
\end{scope}

\end{tikzpicture}
\end{center}
\end{subfigure}

%\setlength{\abovecaptionskip}{-15pt}
\caption{Algunas obstrucciones mínimas de la clase $(1,4)$-$M_2$.}
\label{obsts_1_4_M2}
\end{figure}
\begin{figure}[ht!]

\begin{subfigure}{\textwidth}
\begin{center}
\begin{tikzpicture}

\begin{scope}[xshift=0cm,scale=1]

\node [style=vertex] (1) at (0,0.5) {};
\node [style=vertex] (2) at (1,0.5) {};
\node [style=vertex] (3) at (0,1) {};
\node [style=vertex] (4) at (1,1) {};
\node [style=vertex] (5) at (0.5,1.75) {};

\foreach \i/\j in {1/2,3/4,3/5,4/5} \draw [style=edge] (\i) to (\j);
\node at (0.5,0) {\parbox{0.3\linewidth}{\subcaption*{$o_{(1,5),1}$}}};
\end{scope}

\begin{scope}[xshift=2.5cm,scale=1]

\node [style=vertex] (1) at (0.5,0.5) {};
\node [style=vertex] (2) at (1.5,0.5) {};
\node [style=vertex] (3) at (0.5,1.5) {};
\node [style=vertex] (4) at (1.5,1.5) {};
\node [style=vertex] (5) at (0,2) {};
\node [style=vertex] (6) at (2,2) {};

\foreach \i/\j in {1/2,1/3,1/4,2/3,2/4,3/4,5/1,5/3,5/6,6/2,6/4} \draw [style=edge] (\i) to (\j);
\node at (1,0) {\parbox{0.3\linewidth}{\subcaption*{$o_{(1,5),2}$}}};
\end{scope}

\begin{scope}[xshift=6cm,scale=1]

\node [style=vertex] (1) at (0.5,0.5) {};
\node [style=vertex] (2) at (1.5,0.5) {};
\node [style=vertex] (3) at (0,1) {};
\node [style=vertex] (4) at (2,1) {};
\node [style=vertex] (5) at (0,2) {};
\node [style=vertex] (6) at (2,2) {};
\node [style=vertex] (7) at (1,2.5) {};

\foreach \i/\j in {1/2,1/3,1/4,1/5,1/6,1/7,2/3,2/4,2/5,2/6,2/7,3/4,3/5,3/6,3/7,4/5,4/6,4/7,5/6,5/7,6/7} \draw [style=edge] (\i) to (\j);
\node at (1,0) {\parbox{0.3\linewidth}{\subcaption*{$o_{(1,5),3}$}}};
\end{scope}

\end{tikzpicture}
\end{center}
\end{subfigure}

%\setlength{\abovecaptionskip}{-15pt}
\caption{Algunas obstrucciones mínimas de la clase $(1,5)$-$M_2$.}
\label{obsts_1_5_M2}
\end{figure}

Podemos observar que el algoritmo encuentra que $K_3$ es una obstrucción mínima de la clase $(1,1)$-$M_2$, lo cuál sabemos que es correcto, dado que ésta es la clase de las cográficas bipartitas. 

Sea $\beta$ un entero mayor o igual a uno. Si $2<\beta\le 5$, notemos que el algoritmo encuentra tres obstrucciones mínimas para la clase $(1,\beta)$-$M_2$. Dos de éstas son las obstrucciones mínimas de la clase $(1,\infty)$-$M_2$. Por otra parte, si $0\le\beta\le 5$, se cumple que $K_{\beta+2}$ es una obstrucción  mínima de la clase $(1,\beta)$-$M_2$. Esto nos lleva a pensar en la clase $(1,\infty)$-$M_2$ como el límite cuando $\beta$ tiende a infinito de las clases $(1,\beta)$-$M_2$. Es decir que esperamos que, si una gráfica $G$ es una obstrucción mínima de cada una de las clases $(\alpha,\beta)$-$M_2$ con $\alpha$ fijo y $\beta>n$ para un entero $n$, entonces $G$ sea una obstrucción mínima de la clase $(\alpha,\infty)$-$M_2$.

\subsubsection{Clases $(2,\beta)$-$M_2$}

Las Figuras \ref{obsts_2_2_M2},\ref{obsts_2_3_M2} y \ref{obsts_2_4_M2} muestran los conjuntos de obstrucciones mínimas encontradas con el Algoritmo \ref{alg_obstrucciones_alfabeta} para las clases $(2,2)$-$M_2$, $(2,3)$-$M_2$ y $(2,4)$-$M_2$ respectivamente. Además de éstas, enlistamos las obstrucciones mínimas de las clases $(2,5)$-$M_2$, $(2,6)$-$M_2$ y $(2,7)$-$M_2$.

\begin{figure}[ht!]

\begin{subfigure}{\textwidth}
\begin{center}
\begin{tikzpicture}

\begin{scope}[xshift=0cm,scale=1]

\node [style=vertex] (1) at (0.25,0.5) {};
\node [style=vertex] (2) at (1.25,0.5) {};
\node [style=vertex] (3) at (0,1.15) {};
\node [style=vertex] (4) at (1.5,1.15) {};
\node [style=vertex] (5) at (0.75,1.75) {};

\foreach \i/\j in {1/2,1/3,1/4,1/5,2/3,2/4,2/5,3/4,3/5,4/5} \draw [style=edge] (\i) to (\j);
\node at (0.75,0) {\parbox{0.3\linewidth}{\subcaption*{$o_{(2,2),1}$}}};
\end{scope}

\begin{scope}[xshift=3cm,scale=1]

\node [style=vertex] (1) at (0,0.5) {};
\node [style=vertex] (2) at (1,0.5) {};
\node [style=vertex] (3) at (0.5,0.85) {};
\node [style=vertex] (4) at (0.5,1.5) {};
\node [style=vertex] (5) at (0.5,2) {};

\foreach \i/\j in {1/2,1/3,1/4,2/3,2/4,3/4} \draw [style=edge] (\i) to (\j);
\node at (0.5,0) {\parbox{0.3\linewidth}{\subcaption*{$o_{(2,2),2}$}}};
\end{scope}

\begin{scope}[xshift=5.5cm,scale=1]

\node [style=vertex] (1) at (0,0.5) {};
\node [style=vertex] (2) at (1,0.5) {};
\node [style=vertex] (3) at (0,1) {};
\node [style=vertex] (4) at (1,1) {};
\node [style=vertex] (5) at (0.5,1.75) {};

\foreach \i/\j in {1/2,3/4,3/5,4/5} \draw [style=edge] (\i) to (\j);
\node at (0.5,0) {\parbox{0.3\linewidth}{\subcaption*{$o_{(2,2),3}$}}};
\end{scope}

\end{tikzpicture}
\end{center}
\end{subfigure}

%\setlength{\abovecaptionskip}{-15pt}
\caption{Algunas obstrucciones mínimas de la clase $(2,2)$-$M_2$.}
\label{obsts_2_2_M2}
\end{figure}

Las obstrucciones mínimas de la clase $(2,2)$-$M_2$, ilustradas en la Figura \ref{obsts_2_2_M2} se pueden expresar de la siguiente manera:
\begin{itemize}
    \item $o_{(2,2),1}=K_5$.
    \item $o_{(2,2),2}=K_1+K_4$.
    \item $o_{(2,2),3}=K_2+K_3$.
\end{itemize}

\begin{figure}[ht!]

\begin{subfigure}{\textwidth}
\begin{center}
\begin{tikzpicture}

\begin{scope}[xshift=0cm,scale=1]

\node [style=vertex] (1) at (0,0.5) {};
\node [style=vertex] (2) at (1,0.5) {};
\node [style=vertex] (3) at (0,1.5) {};
\node [style=vertex] (4) at (1,1.5) {};
\node [style=vertex] (5) at (0,2.5) {};
\node [style=vertex] (6) at (1,2.5) {};

\foreach \i/\j in {1/2,1/3,2/3,4/5,4/6,5/6} \draw [style=edge] (\i) to (\j);
\node at (0.5,0) {\parbox{0.3\linewidth}{\subcaption*{$o_{(2,3),1}$}}};
\end{scope}

\begin{scope}[xshift=2cm,scale=1]

\node [style=vertex] (1) at (0.5,0.5) {};
\node [style=vertex] (2) at (1.5,0.5) {};
\node [style=vertex] (3) at (0,1.25) {};
\node [style=vertex] (4) at (2,1.25) {};
\node [style=vertex] (5) at (0.5,2) {};
\node [style=vertex] (6) at (1.5,2) {};

\foreach \i/\j in {1/2,1/3,1/4,1/5,1/6,2/3,2/4,2/5,2/6,3/4,3/5,3/6,4/5,4/6,5/6} \draw [style=edge] (\i) to (\j);
\node at (1,0) {\parbox{0.3\linewidth}{\subcaption*{$o_{(2,3),2}$}}};
\end{scope}

\begin{scope}[xshift=5cm,scale=1]

\node [style=vertex] (1) at (0.25,0.5) {};
\node [style=vertex] (2) at (1.25,0.5) {};
\node [style=vertex] (3) at (0,1.15) {};
\node [style=vertex] (4) at (1.5,1.15) {};
\node [style=vertex] (5) at (0.75,1.75) {};
\node [style=vertex] (6) at (0.75,2.25) {};

\foreach \i/\j in {1/2,1/3,1/4,1/5,2/3,2/4,2/5,3/4,3/5,4/5} \draw [style=edge] (\i) to (\j);
\node at (0.75,0) {\parbox{0.3\linewidth}{\subcaption*{$o_{(2,3),3}$}}};
\end{scope}

\begin{scope}[xshift=7.5cm,scale=1]

\node [style=vertex] (1) at (0,0.5) {};
\node [style=vertex] (2) at (1,0.5) {};
\node [style=vertex] (3) at (0.5,0.85) {};
\node [style=vertex] (4) at (0.5,1.5) {};
\node [style=vertex] (5) at (0,2) {};
\node [style=vertex] (6) at (1,2) {};

\foreach \i/\j in {1/2,1/3,1/4,2/3,2/4,3/4,4/5,4/6,5/6} \draw [style=edge] (\i) to (\j);
\node at (0.5,0) {\parbox{0.3\linewidth}{\subcaption*{$o_{(2,3),4}$}}};
\end{scope}

\begin{scope}[xshift=9.5cm,scale=1]

\node [style=vertex] (1) at (0,0.5) {};
\node [style=vertex] (2) at (1,0.5) {};
\node [style=vertex] (3) at (0.5,0.85) {};
\node [style=vertex] (4) at (0.5,1.5) {};
\node [style=vertex] (5) at (0,2) {};
\node [style=vertex] (6) at (1,2) {};

\foreach \i/\j in {1/2,1/3,1/4,2/3,2/4,3/4,5/6} \draw [style=edge] (\i) to (\j);
\node at (0.5,0) {\parbox{0.3\linewidth}{\subcaption*{$o_{(2,3),5}$}}};
\end{scope}

\end{tikzpicture}
\end{center}
\end{subfigure}

%\setlength{\abovecaptionskip}{-15pt}
\caption{Algunas obstrucciones mínimas de la clase $(2,3)$-$M_2$.}
\label{obsts_2_3_M2}
\end{figure}

Las obstrucciones mínimas de la clase $(2,3)$-$M_2$, ilustradas en la Figura \ref{obsts_2_3_M2}, se pueden expresar de la siguiente manera:
\begin{itemize}
    \item $o_{(2,3),1}=2K_3$.
    \item $o_{(2,3),2}=K_6$.
    \item $o_{(2,3),3}=K_1+K_5$.
    \item $o_{(2,3),4}=K_1\oplus(K_2+K_3)$.
    \item $o_{(2,3),5}=K_2+K_4$.
\end{itemize}

\begin{figure}[ht!]

\begin{subfigure}{\textwidth}
\begin{center}
\begin{tikzpicture}

\begin{scope}[xshift=0cm,scale=1]

\node [style=vertex] (1) at (0,0.5) {};
\node [style=vertex] (2) at (1,0.5) {};
\node [style=vertex] (3) at (0,1.5) {};
\node [style=vertex] (4) at (1,1.5) {};
\node [style=vertex] (5) at (0,2.5) {};
\node [style=vertex] (6) at (1,2.5) {};

\foreach \i/\j in {1/2,1/3,2/3,4/5,4/6,5/6} \draw [style=edge] (\i) to (\j);
\node at (0.5,0) {\parbox{0.3\linewidth}{\subcaption*{$o_{(2,4),1}$}}};
\end{scope}

\begin{scope}[xshift=2.5cm,scale=1]

\node [style=vertex] (1) at (0.625,0.5) {};
\node [style=vertex] (2) at (1.375,0.5) {};
\node [style=vertex] (3) at (0,1) {};
\node [style=vertex] (4) at (2,1) {};
\node [style=vertex] (5) at (0,1.75) {};
\node [style=vertex] (6) at (2,1.75) {};
\node [style=vertex] (7) at (0.25,2.5) {};
\node [style=vertex] (8) at (1.75,2.5) {};
\node [style=vertex] (9) at (1,2.75) {};

\foreach \i/\j in {1/3,1/4,1/5,1/6,1/7,1/8,1/9,2/4,2/5,2/6,2/7,2/8,2/9,3/4,3/5,3/6,3/7,3/8,3/9,4/5,4/6,4/7,4/9,5/6,5/8,6/7,6/9,7/8,7/9,8/9} \draw [style=edge] (\i) to (\j);
\node at (1,0) {\parbox{0.3\linewidth}{\subcaption*{$o_{(2,4),2}$}}};
\end{scope}

\begin{scope}[xshift=6cm,scale=1]

\node [style=vertex] (1) at (0.5,0.5) {};
\node [style=vertex] (2) at (1.5,0.5) {};
\node [style=vertex] (3) at (0,1) {};
\node [style=vertex] (4) at (2,1) {};
\node [style=vertex] (5) at (0,2) {};
\node [style=vertex] (6) at (2,2) {};
\node [style=vertex] (7) at (1,2.5) {};

\foreach \i/\j in {1/2,1/3,1/4,1/5,1/6,1/7,2/3,2/4,2/5,2/6,2/7,3/4,3/5,3/6,3/7,4/5,4/6,4/7,5/6,5/7,6/7} \draw [style=edge] (\i) to (\j);
\node at (1,0) {\parbox{0.3\linewidth}{\subcaption*{$o_{(2,4),3}$}}};
\end{scope}

\begin{scope}[xshift=9.5cm,scale=1]

\node [style=vertex] (4) at (0.25,1.75) {};
\node [style=vertex] (5) at (1.25,1.75) {};
\node [style=vertex] (2) at (0,1.10) {};
\node [style=vertex] (3) at (1.5,1.10) {};
\node [style=vertex] (1) at (0.75,0.5) {};
\node [style=vertex] (6) at (0.25,2.75) {};
\node [style=vertex] (7) at (1.25,2.75) {};

\foreach \i/\j in {1/2,1/3,1/4,1/5,2/3,2/4,2/5,3/4,3/5,4/5,4/6,4/7,5/6,5/7,6/7} \draw [style=edge] (\i) to (\j);
\node at (0.75,0) {\parbox{0.3\linewidth}{\subcaption*{$o_{(2,4),4}$}}};
\end{scope}

\end{tikzpicture}
\end{center}
\end{subfigure}

\begin{subfigure}{\textwidth}
\begin{center}
\begin{tikzpicture}

\begin{scope}[xshift=0cm,scale=1]

\node [style=vertex] (1) at (0.5,0.5) {};
\node [style=vertex] (2) at (1.5,0.5) {};
\node [style=vertex] (3) at (0,1.25) {};
\node [style=vertex] (4) at (2,1.25) {};
\node [style=vertex] (5) at (0.5,2) {};
\node [style=vertex] (6) at (1.5,2) {};
\node [style=vertex] (7) at (1,2.5) {};

\foreach \i/\j in {1/2,1/3,1/4,1/5,1/6,2/3,2/4,2/5,2/6,3/4,3/5,3/6,4/5,4/6,5/6} \draw [style=edge] (\i) to (\j);
\node at (1,0) {\parbox{0.3\linewidth}{\subcaption*{$o_{(2,4),5}$}}};
\end{scope}

\begin{scope}[xshift=3.5cm,scale=1]

\node [style=vertex] (1) at (0.25,0.5) {};
\node [style=vertex] (2) at (1.25,0.5) {};
\node [style=vertex] (3) at (0,1.15) {};
\node [style=vertex] (4) at (1.5,1.15) {};
\node [style=vertex] (5) at (0.75,1.75) {};
\node [style=vertex] (6) at (0.25,2.5) {};
\node [style=vertex] (7) at (1.25,2.5) {};

\foreach \i/\j in {1/2,1/3,1/4,1/5,2/3,2/4,2/5,3/4,3/5,4/5,5/6,5/7,6/7} \draw [style=edge] (\i) to (\j);
\node at (0.75,0) {\parbox{0.3\linewidth}{\subcaption*{$o_{(2,4),6}$}}};
\end{scope}

\begin{scope}[xshift=6.5cm,scale=1]

\node [style=vertex] (1) at (0.25,0.5) {};
\node [style=vertex] (2) at (1.25,0.5) {};
\node [style=vertex] (3) at (0,1.15) {};
\node [style=vertex] (4) at (1.5,1.15) {};
\node [style=vertex] (5) at (0.75,1.75) {};
\node [style=vertex] (6) at (0.25,2.5) {};
\node [style=vertex] (7) at (1.25,2.5) {};

\foreach \i/\j in {1/2,1/3,1/4,1/5,2/3,2/4,2/5,3/4,3/5,4/5,6/7} \draw [style=edge] (\i) to (\j);
\node at (0.75,0) {\parbox{0.3\linewidth}{\subcaption*{$o_{(2,4),7}$}}};
\end{scope}

\end{tikzpicture}
\end{center}
\end{subfigure}

%\setlength{\abovecaptionskip}{-15pt}
\caption{Algunas obstrucciones mínimas de la clase $(2,4)$-$M_2$.}
\label{obsts_2_4_M2}
\end{figure}

Las obstrucciones mínimas de la clase $(2,4)$-$M_2$, ilustradas en la Figura \ref{obsts_2_4_M2}, se pueden expresar de la siguiente manera:
\begin{itemize}
    \item $o_{(2,4),1}=2K_3$.
    \item $o_{(2,4),2}=\overline{P_3}\oplus\overline{P_3}\oplus\overline{P_3}$.
    \item $o_{(2,4),3}=K_7$.
    \item $o_{(2,4),4}=K_2\oplus(K_2+K_3)$.
    \item $o_{(2,4),5}=K_1+K_6$.
    \item $o_{(2,4),6}=K_1\oplus(K_2+K_4)$.
    \item $o_{(2,4),7}=K_2+K_5$.
\end{itemize}


Las obstrucciones mínimas de la clase $(2,5)$-$M_2$ encontradas con el Algoritmo \ref{alg_obstrucciones_alfabeta} se pueden expresar de la siguiente manera:
\begin{itemize}
    \item $o_{(2,5),1}=2K_3$.
    \item $o_{(2,5),2}=\overline{P_3}\oplus\overline{P_3}\oplus\overline{P_3}$.
    \item $o_{(2,5),3}=\overline{P_3}\oplus(K_2+K_3)=(K_1+K_2)\oplus(K_2+K_3)$.
    \item $o_{(2,5),4}=K_8$.
    \item $o_{(2,5),5}=K_3\oplus(K_2+K_3)$.
    \item $o_{(2,5),6}=K_2\oplus(K_2+K_4)$.
    \item $o_{(2,5),7}=K_1+K_7$.
    \item $o_{(2,5),8}=K_1\oplus(K_2+K_5)$.
    \item $o_{(2,5),9}=K_2+K_6$.
\end{itemize}

Las obstrucciones mínimas de la clase $(2,6)$-$M_2$ encontradas con el Algoritmo \ref{alg_obstrucciones_alfabeta} se pueden expresar de la siguiente manera:
\begin{itemize}
    \item $o_{(2,6),1}=2K_3$.
    \item $o_{(2,6),2}=\overline{P_3}\oplus\overline{P_3}\oplus\overline{P_3}$.
    \item $o_{(2,6),3}=\overline{P_3}\oplus(K_2+K_3)=(K_1+K_2)\oplus(K_2+K_3)$.
    \item $o_{(2,6),4}=K_9$.
    \item $o_{(2,6),5}=K_4\oplus(K_2+K_3)$.
    \item $o_{(2,6),6}=K_3\oplus(K_2+K_4)$.
    \item $o_{(2,6),7}=K_2\oplus(K_2+K_5)$.
    \item $o_{(2,6),8}=K_1+K_8$.
    \item $o_{(2,6),9}=K_1\oplus(K_2+K_6)$.
    \item $o_{(2,6),10}=K_2+K_7$.
\end{itemize}

Las obstrucciones mínimas de la clase $(2,7)$-$M_2$ encontradas con el Algoritmo \ref{alg_obstrucciones_alfabeta} se pueden expresar de la siguiente manera:
\begin{itemize}
    \item $o_{(2,7),1}=2K_3$.
    \item $o_{(2,7),2}=\overline{P_3}\oplus\overline{P_3}\oplus\overline{P_3}$.
    \item $o_{(2,7),3}=\overline{P_3}\oplus(K_2+K_3)=(K_1+K_2)\oplus(K_2+K_3)$.
    \item $o_{(2,7),4}=K_{10}$.
    \item $o_{(2,7),5}=K_5\oplus(K_2+K_3)$.
    \item $o_{(2,7),6}=K_4\oplus(K_2+K_4)$.
    \item $o_{(2,7),7}=K_3\oplus(K_2+K_5)$.
    \item $o_{(2,7),8}=K_2\oplus(K_2+K_6)$.
    \item $o_{(2,7),9}=K_1+K_9$.
    \item $o_{(2,7),10}=K_1\oplus(K_2+K_7)$.
    \item $o_{(2,7),11}=K_2+K_8$.
\end{itemize}

Como podemos ver, la gráfica $2K_3$ es una obstrucción mínima de las clases $(2,\beta)$-$M_2$ para $3\le\beta\le7$, la gráfica $\overline{P_3}\oplus\overline{P_3}\oplus\overline{P_3}$ es una obstrucción mínima de las clases $(2,\beta)$-$M_2$ para $4\le\beta\le7$ y la gráfica $\overline{P_3}\oplus(K_2+K_3)$ es una obstrucción mínima de las clases $(2,\beta)$-$M_2$ para $5\le\beta\le7$. Esto nos lleva a pensar que estas tres gráficas pueden ser obstrucciones mínimas de la clase $(2,\infty)$-$M_2$. Además, podemos identificar algunas posibles familias de obstrucciones para cualquier clase $(2,\beta)$-$M_2$.  

%Éstas familias son bastante fáciles de identificar. También es fácil demostrar que son familias de obstrucciones mínimas. Tal vez podamos hacerlo al final si nos queda tiempo.

\subsubsection{Obstrucciones de algunas otras clases  $(\alpha,\beta)$-$M_2$}

%Tengo las obstrucciones mínimas de las clases (3,3), (3,4), (3,5) y (4,4). Se pueden generar más. Tal vez podemos incluirlas en los apéndices.