En esta tesis estudiamos a la clase de cográficas que aceptan una partición en un número $i$ de gráficas multipartitas completas, para distintos valores de $i$. Denotamos a estas clases por $M_i$. Nuestra investigación tiene como punto de partida la clase $M_2$, a la cual estudiamos a profundidad tomando como referencia la investigación realizada sobre las cográficas polares.

A partir del estudio de la clase $M_2$, encontramos un algoritmo (Algoritmo \ref{alg_esta_en_clase}) que es capaz de determinar si una cográfica $G$, representada a través de su coárbol, es elemento de una clase hereditaria de cográficas fija en tiempo lineal con respecto al orden de $G$. Este algoritmo funciona utilizando una variación de los coárboles, los coárboles binarios y el concepto de subcoárbol binario,
que introducimos en este trabajo.

Se puede caracterizar a la clase $M_2$ a través de su conjunto de obstrucciones mínimas que se muestra en la Figura \ref{obsts_M2}. De igual manera, se presentan dos algoritmos para determinar si una cográfica $G$, representada a través de su coárbol, pertenece a la clase $M_2$. El primero de \'estos (Algoritmo \ref{alg_decision}) es una instancia del algoritmo mencionado en el párrafo anterior. Éste funciona generando un coárbol binario $B$ de $G$ y determinando si alguno de los coárboles binarios de las obstrucciones mínimas de $M_2$ es subárbol binario de $B$. El segundo algoritmo (Algoritmo \ref{alg_cert_m2}) es un algoritmo certificador. Es decir que no sólo determina si $G$ pertenece o no a la clase $M_2$, sino que devuelve un sí-certificado o un no-certificado. En el caso de que $G$, pertenezca a la clase $M_2$, el algoritmo colorea las hojas del coárbol de $G$ de dos colores que conforman una $M_2$-partición de $G$. En el caso contrario, el algoritmo colorea algunas hojas del coárbol de $G$ que representan vértices de $G$ que forman una obstrucción mínima de la clase. Este algoritmo se realizó siguiendo la demostración del Teorema \ref{teo_obsts_m2}, en el que se presentan las obstrucciones mínimas de la clase $M_2$. La ejecución de ambos algoritmos toma una cantidad de tiempo que crece de forma lineal con respecto al orden de $G$. Sin embargo, una vez implementados, el algoritmo certificador tuvo un desempeño mucho mejor. Se comprobó con éxito que los resultados de ambos algoritmos son coherentes.

Siguiendo la investigación realizada sobre las cográficas polares, estudiamos las clases $(\alpha, \beta)$-$M_2$, que son las subclases de $M_2$ que contienen a una gráfica si ésta acepta una partición en dos gráficas multipartitas completas tales que una de ellas está formada por a lo más $\alpha$ conjuntos independientes y la otra está formada por a lo más $\beta$ conjuntos independientes con $\alpha$ y $\beta$ enteros mayores o iguales a cero tales que $\alpha \le \beta$. Presentamos un algoritmo (Algoritmo \ref{alg_obstrucciones_alfabeta}) que, dados dos enteros $\alpha$ y $\beta$, genera todas las obstrucciones mínimas de la clase $(\alpha, \beta)$-$M_2$ hasta cierto orden. Este algoritmo utiliza el algoritmo certificador de la clase $M_2$. Utilizando este algoritmo, generamos las obstrucciones mínimas de hasta 15 vértices de varias clases $(\alpha, \beta)$-$M_2$. Si fijamos $\alpha$, podemos observar que las obstrucciones generadas para clases consecutivas siguen un comportamiento predecible que podría ser descrito definiendo familias de obstrucciones como lo hicimos con la clase $(1,\infty)$-$M_2$.

El siguiente paso en nuestra investigación fue caracterizar a la clase $M_3$ a través de su conjunto de obstrucciones mínimas (Teorema \ref{teo_obsts_m3}). Esto nos ayudó a encontrar dos familias de obstrucciones mínimas para cualquier clase $M_i$. Llamamos a estas familias la familia $O$ y la familia $P$ de obstrucciones. Todas las obstrucciones mínimas de las clases $M_1$ y $M_2$ pertenecen a las familias $O$ y $P$, mientras que 14 de 21 obstrucciones mínimas de la clase $M_3$ pertenecen a estas dos familias. Las otras 7 fueron agrupadas en otras 3 familias que podrían generalizarse.

\section{Trabajo a futuro}

En nuestra tesis encontramos varios temas con los que se puede continuar el estudio de las clases $M_i$.

Uno de los resultados principales de nuestra tesis, el Algoritmo \ref{alg_esta_en_clase}, funciona determinando si cada uno de los coárboles binarios de un conjunto es subcoárbol binario de otro coárbol binario. El número de coárboles binarios de una gráfica (en este caso la obstrucción mínima de alguna clase) crece de forma exponencial, por lo que, aunque el algoritmo funcione en tiempo lineal, éste es poco eficiente. A partir de esto, nos preguntamos si una definición más general de sucoárbol pueda dar lugar a un algoritmo más eficiente.

El Algoritmo \ref{alg_obstrucciones_alfabeta} recibe como entrada tres enteros $\alpha$, $\beta$ y $n$ y genera todas las cográficas de hasta $n$ vértices que son obstrucciones mínimas de la clase $(\alpha,\beta)$-$M_2$. Los resultados que generamos con este algoritmo son obstrucciones mínimas de hasta 15 vértices. Si bien, creemos que los conjuntos de obstrucciones mínimas generados para clases que no tienen obstrucciones mínimas ni con 13 ni con 14 vértices son exhaustivos, no podemos estar seguros de ello ya que no contamos con una cota superior para el número de vértices de una obstrucción mínima lo suficientemente pequeña. Así, un resultado que tendría gran impacto en la investigación de las clases $M_i$ sería encontrar dicha cota superior.

El el Apéndice \ref{apéndiceListaAlfaBeta} se listan las obstrucciones mínimas de varias clases $(\alpha,\beta)$-$M_2$. Si fijamos $\alpha$, podemos observar un comportamiento bastante predecible para valores de $\beta$ consecutivos. Tomando como ejemplo el análisis que realizamos de las clases $(1,\beta)$-$M_2$, nos preguntamos si podemos encontrar resultados parecidos para valores de $\alpha$ diferentes de 1.

Así como encontrar el conjunto de obstrucciones mínimas de la clase $M_3$ nos llevó a identificar familias de obstrucciones mínimas para cualquier clase $M_i$, seguir el estudio de la clase $M_3$ como lo hicimos con la clase $M_2$ podría llevarnos a resultados más generales. Así, quedan pendientes las siguientes tareas:
\begin{itemize}
    \item Encontrar un algoritmo para identificar a los elementos de $M_3$.
    \item Encontrar un algoritmo certificador para la clase $M_3$.
    \item El estudio de las clases $(\alpha,\beta,\gamma)$-$M_3$.
\end{itemize}

Otra forma de darle continuidad a los resultados presentados en esta tesis es encontrar un algoritmo para computar a los elementos de la familia $O_n$ de obstrucciones para un entero $n$.

Finalmente, un resultado importante sería identificar otras familias de obstrucciones para cualquier clase $M_i$.
