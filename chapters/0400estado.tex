En el presente capítulo hacemos una revisión de los artículos que dan forma a la investigación que realizamos. El problema de determinar si una cográfica acepta una partición tal que cada una de sus partes es una gráfica multipartita completa no ha sido estudiado con anterioridad. Sin embargo, la investigación realizada sobre las cográficas polares nos sirve como referencia acerca de cómo encontrar las obstrucciones mínimas de una clase hereditaria de cográficas en donde cada uno de sus elementos acepta una partición tal que cada parte cumple con una condición. Esta investigación también nos muestra que se pueden estudiar subclases de las clases anteriormente descritas al agregar restricciones a cada una de las partes. Los primeros tres artículos de los que hablamos abordan el tema de las cográficas polares. %%Tenemos que hablar del otro artículo.

\section{Cográficas polares}
    En este artículo de T.Ekim \cite{Ekim} se introduce la clase de las cográficas polares a la vez que se presenta su conjunto de obstrucciones mínimas (que existe dado que las cográficas polares son una clase hereditaria de gráficas). La metodología con la que se realiza la demostración de que el conjunto de gráficas que se proporciona (Figura \ref{obsts_cografics_polares}) es, en efecto, el conjunto de obstrucciones mínimas de las cográficas polares sirve como base de nuestra demostración de que el conjunto de gráficas que mostramos como conjunto de obstrucciones mínimas de la clase de cográficas que acepta una partición en dos gráficas multipartitas completas (a la que llamamos $M_2$) es correcto. En este artículo también se presentan las cográficas monopolares, una subclase de las cográficas polares, cuyo estudio sirve de inspiración para el estudio de algunas subclases de $M_2$.

Sea $G$ una cográfica, decimos que $G$ es una \emph{\textbf{cográfica polar}} si su conjunto de vértices $V$ acepta una partición $(A,B)$ tal que $A$ induce una gráfica multipartita completa y $B$ induce una unión ajena de clanes.
Decimos que $G$ es ($s,k$)-polar si existe una partición ($A,B$) de los vértices de $G$ en donde $A$ induce en $G$ una unión completa de a lo más $s$ conjuntos independientes (Es decir una gráfica $s$-partita) y $B$ induce en $G$ una unión ajena de a lo más $k$ clanes. Notemos que las cográficas polares son las gráficas ($\infty, \infty$)-polares.

\begin{theorem}
    Sea $G$ una cográfica, decimos que $G$ es una cográfica polar si y sólo si no contiene como subgráfica inducida a ninguna de las gráficas de la Figura \ref{obsts_cografics_polares}.
\end{theorem}

\begin{figure}[H]
\begin{center}
\begin{tikzpicture}

\begin{scope}[xshift=0cm,scale=1]

\node [style=vertex] (1) at (0.5,0) {};
\node [style=vertex] (2) at (0.2,0.5) {};
\node [style=vertex] (3) at (0.8,0.5) {};
\node [style=vertex] (4) at (0,1) {};
\node [style=vertex] (5) at (1,1) {};
\node [style=vertex] (6) at (0.5,1.5) {};
\node [style=vertex] (7) at (0,2) {};
\node [style=vertex] (8) at (1,2) {};
\foreach \i/\j in {1/2,1/3,4/5,4/6,4/7,5/6,5/8,6/7,6/8,7/8}
  \draw [style=edge] (\i) to (\j);
\node [below of=1] {\parbox{0.3\linewidth}{\subcaption*{$H_1$}}};

\end{scope}

\begin{scope}[xshift=2cm,scale=1]

\node [style=vertex] (1) at (0.75,0) {};
\node [style=vertex] (2) at (0.45,0.5) {};
\node [style=vertex] (3) at (1.05,0.5) {};
\node [style=vertex] (4) at (0.5,1) {};
\node [style=vertex] (5) at (1.5,1) {};
\node [style=vertex] (6) at (0,1.5) {};
\node [style=vertex] (7) at (1,1.5) {};
\node [style=vertex] (8) at (0.5,2) {};
\node [style=vertex] (9) at (1.5,2) {};

\foreach \i/\j in {1/2,1/3,4/6,4/7,5/7,5/9,6/7,6/8,7/9,7/8}
  \draw [style=edge] (\i) to (\j);
\node [below of=1] {\parbox{0.3\linewidth}{\subcaption*{$H_2$}}};

\end{scope}

\begin{scope}[xshift=4.5cm,scale=1]

\node [style=vertex] (1) at (0.75,0) {};
\node [style=vertex] (2) at (0.45,0.5) {};
\node [style=vertex] (3) at (1.05,0.5) {};
\node [style=vertex] (4) at (0,1) {};
\node [style=vertex] (5) at (1.5,1) {};
\node [style=vertex] (6) at (0.45,1.5) {};
\node [style=vertex] (7) at (1.05,1.5) {};
\node [style=vertex] (8) at (0,2) {};
\node [style=vertex] (9) at (1.5,2) {};

\foreach \i/\j in {1/2,1/3,4/5,4/6,4/7,4/8,5/6,5/7,5/9,6/7,6/8,7/9,8/9}
  \draw [style=edge] (\i) to (\j);
\node [below of=1] {\parbox{0.3\linewidth}{\subcaption*{$H_3$}}};

\end{scope}

\begin{scope}[xshift=7cm,scale=1]

\node [style=vertex] (1) at (1,0) {};
\node [style=vertex] (2) at (0.7,0.5) {};
\node [style=vertex] (3) at (1.3,0.5) {};
\node [style=vertex] (4) at (0,1) {};
\node [style=vertex] (5) at (1,1) {};
\node [style=vertex] (6) at (2,1) {};
\node [style=vertex] (7) at (0,2) {};
\node [style=vertex] (8) at (1,2) {};
\node [style=vertex] (9) at (2,2) {};
\foreach \i/\j in {1/2,1/3,4/5,4/7,4/8,5/6,5/7,5/8,5/9,6/8,6/9,7/8,8/9}
  \draw [style=edge] (\i) to (\j);
\node [below of=1] {\parbox{0.3\linewidth}{\subcaption*{$H_4$}}};
\end{scope}

\end{tikzpicture}
\end{center}
\caption{Obstrucciones mínimas para las gráficas polares.}
\label{obsts_cografics_polares}
\end{figure}

La demostración de que las gráficas de la Figura \ref{obsts_cografics_polares} forman el conjunto de obstrucciones mínimas de las cográficas polares se realiza describiendo a cada una de éstas en términos de la unión ajena y la unión completa de gráficas más pequeñas.

\begin{enumerate}[(1)]
    \item $H_1 = P_3 + ( \overline{K_2} \oplus P_3) = P_3+ (K_1 \oplus P_4)$
    \item $H_2 = P_3 + (K_1 \oplus (P_3 + K_2))$
    \item $H_3 = P_3 + ( \overline{P_3} \oplus \overline{P_3})$
    \item $H_4 = P_3 + (K_2 \oplus 2K_2)$
\end{enumerate}

Primero se demuestra que ninguna de estas gráficas es una cográfica polar y que, por lo tanto, cualquier gráfica que tenga a alguna de ellas como subgráfica inducida tampoco es una cográfica polar. Por último se muestra que si una cográfica $G$ no es polar, entonces debe de tener a alguna de estas gráficas como subgráfica inducida.

En este artículo también se presenta y se caracteriza la clase de las cográficas monopolares a través de su conjunto de obstrucciones mínimas. Éstas se muestran en la Figura \ref{obsts_cografics_monopolares}

%Sea $G$ una cográfica, decimos que $G$ es una cográfica monopolar si $G$ es una gráfica ($s,k$)-polar con $s\leq 1$ o $k \leq 1$.


\begin{figure}[H]
\begin{center}
\begin{subfigure}{\textwidth}
\begin{tikzpicture}

\begin{scope}[xshift=0cm,scale=1]
\node [style=vertex] (2) at (0.2,0.5) {};
\node [style=vertex] (3) at (0.8,0.5) {};
\node [style=vertex] (4) at (0,1) {};
\node [style=vertex] (5) at (1,1) {};
\node [style=vertex] (6) at (0.5,1.5) {};
\node [style=vertex] (7) at (0,2) {};
\node [style=vertex] (8) at (1,2) {};
\foreach \i/\j in {4/5,4/6,4/7,5/6,5/8,6/7,6/8,7/8}
  \draw [style=edge] (\i) to (\j);
\node at (0.5,-0.25) {\parbox{0.3\linewidth}{\subcaption*{$G_1$}}};
\end{scope}

\begin{scope}[xshift=2cm,scale=1]
\node [style=vertex] (2) at (0.5,0.5) {};
\node [style=vertex] (3) at (1,0.5) {};
\node [style=vertex] (4) at (0,1) {};
\node [style=vertex] (5) at (1,1) {};
\node [style=vertex] (6) at (0.5,1.5) {};
\node [style=vertex] (7) at (0,2) {};
\node [style=vertex] (8) at (1,2) {};
\foreach \i/\j in {2/6,4/5,4/6,4/7,5/6,5/8,6/7,6/8,7/8}
  \draw [style=edge] (\i) to (\j);
\node at (0.5,-0.25) {\parbox{0.3\linewidth}{\subcaption*{$G_2$}}};
\end{scope}

\begin{scope}[xshift=4cm, yshift=0.5cm,scale=1]
\node [style=vertex] (1) at (1,0) {};
\node [style=vertex] (2) at (0.5,0.5) {};
\node [style=vertex] (3) at (1.5,0.5) {};
\node [style=vertex] (4) at (0,1) {};
\node [style=vertex] (5) at (1,1) {};
\node [style=vertex] (6) at (0.5,1.5) {};
\node [style=vertex] (7) at (1.5,1.5) {};
\foreach \i/\j in {2/4,2/5,3/5,3/7,4/5,4/6,5/6,5/7}
  \draw [style=edge] (\i) to (\j);
\node at (0.75,-0.75) {\parbox{0.3\linewidth}{\subcaption*{$G_3$}}};
\end{scope}

\begin{scope}[xshift=6.5cm, yshift=0.5cm,scale=1]
\node [style=vertex] (1) at (0.75,0) {};
\node [style=vertex] (2) at (0,0.5) {};
\node [style=vertex] (3) at (0.75,0.5) {};
\node [style=vertex] (4) at (1.5,0.5) {};
\node [style=vertex] (5) at (0,1.5) {};
\node [style=vertex] (6) at (0.75,1.5) {};
\node [style=vertex] (7) at (1.5,1.5) {};
\foreach \i/\j in {2/3,2/5,2/6,3/4,3/5,3/6,3/7,4/6,4/7,5/6,6/7}
  \draw [style=edge] (\i) to (\j);
\node at (0.75,-0.75) {\parbox{0.3\linewidth}{\subcaption*{$G_4$}}};
\end{scope}

\begin{scope}[xshift=9cm, yshift=0.5cm,scale=1]
\node [style=vertex] (1) at (0.75,0) {};
\node [style=vertex] (2) at (0.25,0.5) {};
\node [style=vertex] (3) at (1.25,0.5) {};
\node [style=vertex] (4) at (0,1) {};
\node [style=vertex] (5) at (0.75,1) {};
\node [style=vertex] (6) at (1.5,1) {};
\node [style=vertex] (7) at (0.75,1.5) {};
\foreach \i/\j in {2/4,2/5,2/6,3/4,3/5,3/6,4/5,4/7,5/6,6/7}
  \draw [style=edge] (\i) to (\j);
\node at (0.75,-0.75) {\parbox{0.3\linewidth}{\subcaption*{$G_5$}}};
\end{scope}

\end{tikzpicture}
\end{subfigure}

\begin{subfigure}{\textwidth}
\begin{center}
\begin{tikzpicture}

\begin{scope}[xshift=0cm, yshift=0.5cm,scale=1]
\node [style=vertex] (1) at (0.75,0) {};
\node [style=vertex] (2) at (0.25,0.5) {};
\node [style=vertex] (3) at (1.25,0.5) {};
\node [style=vertex] (4) at (0,1) {};
\node [style=vertex] (5) at (0.75,1) {};
\node [style=vertex] (6) at (1.5,1) {};
\node [style=vertex] (7) at (0.75,1.5) {};
\foreach \i/\j in {2/3,2/4,2/5,2/6,3/4,3/5,3/6,4/5,4/7,5/6,6/7}
  \draw [style=edge] (\i) to (\j);
\node at (0.75,-0.75) {\parbox{0.3\linewidth}{\subcaption*{$G_6$}}};
\end{scope}

\begin{scope}[xshift=2.5cm, yshift=0.5cm,scale=1]
\node [style=vertex] (1) at (0.75,0) {};
\node [style=vertex] (2) at (0.25,0.5) {};
\node [style=vertex] (3) at (1.25,0.5) {};
\node [style=vertex] (4) at (0,1) {};
\node [style=vertex] (5) at (0.75,1) {};
\node [style=vertex] (6) at (1.5,1) {};
\node [style=vertex] (7) at (0.75,1.5) {};
\foreach \i/\j in {2/3,2/4,2/5,2/6,3/4,3/5,3/6,4/5,4/7,5/6,5/7,6/7}
  \draw [style=edge] (\i) to (\j);
\node at (0.75,-0.75) {\parbox{0.3\linewidth}{\subcaption*{$G_7$}}};
\end{scope}

\begin{scope}[xshift=5cm, yshift=0.5cm,scale=1]
\node [style=vertex] (1) at (0.75,0) {};
\node [style=vertex] (2) at (0,0.5) {};
\node [style=vertex] (3) at (1.5,0.5) {};
\node [style=vertex] (4) at (0.25,1) {};
\node [style=vertex] (5) at (1.25,1) {};
\node [style=vertex] (6) at (0,1.5) {};
\node [style=vertex] (7) at (1.5,1.5) {};
\foreach \i/\j in {2/3,2/4,2/5,2/6,3/4,3/5,3/7,4/5,4/6,5/7,6/7}
  \draw [style=edge] (\i) to (\j);
\node at (0.75,-0.75) {\parbox{0.3\linewidth}{\subcaption*{$G_8$}}};
\end{scope}

\begin{scope}[xshift=7.5cm, yshift=0.5cm,scale=1]
\node [style=vertex] (1) at (0.75,0) {};
\node [style=vertex] (2) at (0,0.5) {};
\node [style=vertex] (3) at (1.5,0.5) {};
\node [style=vertex] (4) at (0.25,1) {};
\node [style=vertex] (5) at (1.25,1) {};
\node [style=vertex] (6) at (0,1.5) {};
\node [style=vertex] (7) at (1.5,1.5) {};
\foreach \i/\j in {2/3,2/4,2/5,2/6,3/4,3/5,3/7,4/5,4/6,4/7,5/6,5/7,6/7}
  \draw [style=edge] (\i) to (\j);
\node at (0.75,-0.75) {\parbox{0.3\linewidth}{\subcaption*{$G_9$}}};
\end{scope}

\end{tikzpicture}
\end{center}
\end{subfigure}

\end{center}
\caption{Obstrucciones mínimas para las gráficas monopolares.}
\label{obsts_cografics_monopolares}
\end{figure}


Como podemos observar, varias de las obstrucciones mínimas de las gráficas monopolares se asemejan a obstrucciones mínimas de las gráficas polares. Por ejemplo, las gráficas $G_1$, $G_3$, $G_4$ y $G_8$ son subgráficas de $H_1$, $H_2$, $H_3$ y $H_4$ respectivamente. Notemos que en los cuatro casos se obtiene una obstrucción mínima para las gráficas monopolares $G$ a partir de una obstrucción mínima para las gráficas polares $H$ al restar vértices de la componente conexa de $H$ que forma un $P_3$. En los últimos tres casos, el $P_3$ es reemplazado por un vértice aislado. Esto nos lleva a pensar en que se puede encontrar una relación entre el conjunto de obstrucciones mínimas de una clase y los conjuntos de obstrucciones mínimas de sus subclases.


\section{Obstrucciones mínimas para cográficas ($s$, 1)-polares}
    En \cite{Fernando}, Contreras-Mendoza y Hern\'andez-Cruz
exhiben el conjunto de obstrucciones mínimas de las
cográficas $(\infty, 1)$-polares. Este conjunto es
utilizado para describir cómo se puede obtener el conjunto
de obstrucciones mínimas de cualquiera de las clases de
cográficas $(s,1)$-polares dado un entero $s \ge 2$. En
nuestra investigación encontramos un resultado similar,
el Lema \ref{lema_1infM2}, donde describimos a las obstrucciones
m\'inimas para las gráficas que aceptan una partición en
un conjunto independiente y una gráfica multipartita completa.

A continuaci\'on reproducimos el resultado antes mencionado.

\begin{theorem}[\cite{Fernando}]
\label{thm:s,1-ess}
  Sea $G$ una cográfica. Entonces $G$ es $(\infty,1)$-polar
  si y sólo si no contiene alguna de las gráficas de la Figura
  \ref{obsts_cografics_esenciales_1spolares} como subgráfica
  inducida. Este conjunto es llamado el conjunto de obstrucciones
  esenciales.
\end{theorem}

\begin{figure}[ht!]
\begin{center}
\begin{tikzpicture}

\begin{scope}[xshift=0cm,scale=1]
\node [style=vertex] (1) at (0,0) {};
\node [style=vertex] (2) at (1,0) {};
\node [style=vertex] (3) at (0,1) {};
\node [style=vertex] (4) at (1,1) {};
\node [style=vertex] (5) at (0.5,2) {};
\foreach \i/\j in {1/2,3/4}
  \draw [style=edge] (\i) to (\j);
\node at (0.5,-0.75) {\parbox{0.3\linewidth}{\subcaption*{$G_1$}}};
\end{scope}

\begin{scope}[xshift=2cm,scale=1]
\node [style=vertex] (1) at (0,0) {};
\node [style=vertex] (2) at (1,0) {};
\node [style=vertex] (3) at (0,1) {};
\node [style=vertex] (4) at (1,1) {};
\node [style=vertex] (5) at (0,2) {};
\node [style=vertex] (6) at (1,2) {};
\foreach \i/\j in {1/2,1/3,2/4,3/4}
  \draw [style=edge] (\i) to (\j);
\node at (0.5,-0.75) {\parbox{0.3\linewidth}{\subcaption*{$G_2$}}};
\end{scope}

\begin{scope}[xshift=4cm,scale=1]
\node [style=vertex] (1) at (0,0) {};
\node [style=vertex] (2) at (1,0) {};
\node [style=vertex] (3) at (0,1) {};
\node [style=vertex] (4) at (1,1) {};
\node [style=vertex] (5) at (0,2) {};
\node [style=vertex] (6) at (1,2) {};
\foreach \i/\j in {1/3,2/4,3/5,4/6}
  \draw [style=edge] (\i) to (\j);
\node at (0.5,-0.75) {\parbox{0.3\linewidth}{\subcaption*{$G_3$}}};
\end{scope}

\begin{scope}[xshift=6cm,scale=1]
\node [style=vertex] (1) at (0.75,0) {};
\node [style=vertex] (2) at (0,0.75) {};
\node [style=vertex] (3) at (0.75,0.75) {};
\node [style=vertex] (4) at (1.5,0.75) {};
\node [style=vertex] (5) at (0.75,1.5) {};
\node [style=vertex] (6) at (0.75,2) {};
\foreach \i/\j in {1/2,1/3,1/4,2/3,2/5,3/4,4/5}
  \draw [style=edge] (\i) to (\j);
\node at (0.75,-0.75) {\parbox{0.3\linewidth}{\subcaption*{$G_4$}}};
\end{scope}

\end{tikzpicture}
\end{center}
\caption{Obstrucciones mínimas para las gráficas polares.}
\label{obsts_cografics_esenciales_1spolares}
\end{figure}

Haciendo uso de estas obstrucciones esenciales,
los autores caracterizan, para cualquier entero
$s$, con $s \ge 2$, a las obstrucciones m\'inimas
para la clase de cogr\'aficas $(s,1)$-polares.
Notemos que, adem\'as de las
obstrucciones esenciales, existen cuatro familias
espor\'adicas de obstrucciones m\'inimas, y se
plantea una regla recursiva para generar
obstrucciones m\'inimas para las cogr\'aficas
$(s,1)$-polares utilizando obstrucciones
m\'inimas no esenciales para las cogr\'aficas
$(t,1)$-polares, con $t < s$.

\begin{theorem}
\label{thm:s,1}
  Sea $G$ una cográfica y $s \ge 2$ un entero. Entonces
  $G$ es una obstrucción mínima de las cográficas
  $(s,1)$-polares si y sólo si es una de las siguientes
  gráficas:

  \begin{itemize}
    \item Una de las cuatro obstrucciones esenciales.

    \item $2K_{s+1}$.

    \item $K_2 + (\overline{K_2}\oplus K_s)$.

    \item $K_1 + (C_4 \oplus K_{s-1})$.

    \item $\overline{(s+1)K_2}$.

    \item El complemento de $G$ es inconexo con componentes
      $G_1, \dots, G_t$ tales que $t \leq s$, y cada $G_i$
      es el complemento de una obstrucción mínima no esencial
      de la clase de cográficas $(s_i, 1)$-polares con
      $\sum^{t}_{i=1}s_i = s-t+1$.
  \end{itemize}
\end{theorem}

%Si $s$ y $k$ son enteros positivos arbitrarios, determinar
%las obstrucciones m\'inimas para las cogr\'aficas
%$(s,k)$-polares parece ser un problema dif\'icil de
%resolver.

El resultado presentado en el Teorema
\ref{thm:s,1} es un ejemplo de c\'omo, al restringir
un problema a un subproblema bien definido, es posible
encontrar soluciones parciales al problema general.   En
particular, se obtuvo una caracterizaci\'on por obstrucciones
m\'inimas para una subclase infinita de las cogr\'aficas
$(s,k)$-polares.

M\'as a\'un, una interpretaci\'on de los Teoremas \ref{thm:s,1-ess}
y \ref{thm:s,1} es que resulta posible pensar a las gr\'aficas
$(\infty,1)$-polares como el l\'imite cuando $s$ tiende al
infinito de las gr\'aficas $(s,1)$-polares. Podemos observar que
las gráficas que son obstrucciones mínimas de las gráficas
$(s,1)$-polares para cualquier entero $s \ge 2$ también son
obstrucciones mínimas de las gráficas $(\infty,1)$-polares. En
nuestra investigación encontramos un resultado similar para las
gráficas $(\alpha,\beta)$-$M_2$\footnote{La clase
$(\alpha,\beta)$-$M_2$ es la clase constituida por todas las
gráficas que aceptan una partición en dos gráficas multipartitas
completas, una formada por a lo más $\alpha$ conjuntos estables
y la otra formada por a lo más $\beta$ conjuntos estables.}, ya
que las obstrucciones mínimas de la clase $(1,\infty)$-$M_2$ son
obstrucciones mínimas de la clase $(1,m)$-$M_2$, para cualquier
valor de $m$ con $m \ge 2$.  Con base en la idea anterior,
a partir de las obstrucciones mínimas de las clases
$(2,m)$-$M_2$, con $m \in \{ 3, \dots, 7 \}$ que generamos
computacionalmente, encontramos algunas obstrucciones mínimas
de la clase $(2,\infty)$-$M_2$. Aplicando este mismo proceso,
podemos encontrar obstrucciones mínimas para la clase $\alpha,
\infty$-$M_2$ dado un entero $\alpha \ge 3$.


\section{Obstrucciones mínimas para cográficas 2-polares}
    %%Por definir: k-polares
En \cite{Hell03}, Hell, Hern\'andez-Cruz y Linhares-Sales
exhiben el conjunto de obstrucciones mínimas de la clase
de las cográficas 2-polares. Para construir este conjunto
primero se presentan resultados preliminares sobre la
estructura de las obstrucciones mínimas de las cográficas
$(k,k)$-polares (llamadas simplemente cográficas $k$-polares)
para cualquier entero positivo $k$. Posteriormente se presenta
el complemento parcial, una operación que conserva la
$2$-polaridad, para construir las 24 obstrucciones m\'inimas
inconexas de las cogr\'aficas $2$-polares a partir de un
conjunto de cuatro obtrucciones m\'inimas.

El siguiente lema describe la estructura de las
obstrucciones m\'inimas para las cogr\'aficas
$k$-polares con el m\'aximo n\'umero posible de
componentes conexas.

\begin{lemma}
\label{lema_2polares_01}
Sean $l$ y $k$ enteros tales que $1 \le l \le k+1$. Salvo isomorfismo, hay exactamente una obstrucción mínima para las cográficas $k$-polares con $k+2$ componentes en total y $l$ componentes triviales. Esta obstrucción mínima es isomorfa a
$$lk_1+(k-l+1)k_2+k_{l,l}$$
\end{lemma}

Aplicando el Lema \ref{lema_2polares_01}, podemos encontrar
tres obstrucciones mínimas para las cográficas $2$-polares.
Éstas se muestran en la Figura \ref{obsts_2polares_01}.

\begin{figure}[ht!]
\begin{center}
\begin{tikzpicture}

\begin{scope}[xshift=0cm,scale=1]
\node [vertex] (1) at (0,0) {};
\node [vertex] (2) at (1,0) {};
\node [vertex] (3) at (0,0.5) {};
\node [vertex] (4) at (1,0.5) {};
\node [vertex] (5) at (0,1) {};
\node [vertex] (6) at (1,1) {};
\node [vertex] (7) at (0.5,1.5) {};
\foreach \i/\j in {1/2,3/4,5/6}
  \draw [edge] (\i) to (\j);
\node at(0.5,-1) {\parbox{0.3\linewidth}{\subcaption*{$F_{1}$}}};
\end{scope}

\begin{scope}[xshift=2.5cm,scale=1]
\node [vertex] (1) at (0,0) {};
\node [vertex] (2) at (1,0) {};
\node [vertex] (3) at (0,0.5) {};
\node [vertex] (4) at (1,0.5) {};
\node [vertex] (5) at (0,1) {};
\node [vertex] (6) at (1,1) {};
\node [vertex] (7) at (0.25,1.5) {};
\node [vertex] (8) at (0.75,1.5) {};
\foreach \i/\j in {1/2,1/4,2/3,3/4,5/6}
  \draw [edge] (\i) to (\j);
\node at(0.5,-1) {\parbox{0.3\linewidth}{\subcaption*{$F_{13}$}}};
\end{scope}

\begin{scope}[xshift=5cm,scale=1]
\node [vertex] (1) at (0,0) {};
\node [vertex] (2) at (1,0) {};
\node [vertex] (3) at (0,0.5) {};
\node [vertex] (4) at (1,0.5) {};
\node [vertex] (5) at (0,1) {};
\node [vertex] (6) at (1,1) {};
\node [vertex] (7) at (0,1.5) {};
\node [vertex] (8) at (0.5,1.5) {};
\node [vertex] (9) at (1,1.5) {};
\foreach \i/\j in {1/2,1/4,2/3,3/4,3/6,4/5,5/6}
  \draw [edge] (\i) to (\j);
\node at(0.5,-1) {\parbox{0.3\linewidth}{\subcaption*{$F_{21}$}}};
\end{scope}

\end{tikzpicture}
\end{center}
\caption{Obstrucciones mínimas para las cográficas 2-polaes obtenidas con el Lema \ref{lema_2polares_01}.}
\label{obsts_2polares_01}
\end{figure}

Ahora introducimos una operaci\'on que preserva la
$2$-polaridad y la propiedad de ser cogr\'afica, por lo
que resulta bastante \'util en el estudio de las
obstrucciones m\'inimas para las cogr\'aficas $2$-polares.
Sea $H$ una gráfica, un \textbf{\emph{complemento parcial}}
de $H$ es una gráfica obtenida de $H$ al dividir a sus
componentes conexas en dos gráficas $H'$ y $H''$, y tomando
de forma separada el complemento de cada una.

Tomando todos los posibles complementos parciales de las
gráficas $F_1$, $F_{13}$ y $F_{21}$ (Figura
\ref{obsts_2polares_01}), encontramos tres familias de
obstrucciones mínimas; \'estas se muestran en las Figuras
\ref{obsts_2polares_02}, \ref{obsts_2polares_03} y
\ref{obsts_2polares_04} respectivamente. Podemos encontrar
una cuarta familia de obstrucciones mínimas tomar todos los
posibles complementos parciales de la gráfica $F_7$ (Figura
\ref{obsts_2polares_05}) que también es una obstrucción mínima
de las gráficas 2-polares. La gráfica $F_7$ se puede construir
de forma natural agregando un $K_2$ a una de las obstrucciones
m\'inimas para $(2,1)$-polaridad en cogr\'aficas.

\begin{figure}[ht!]
\begin{subfigure}{\textwidth}
\begin{center}
\begin{tikzpicture}

\begin{scope}[xshift=0cm,scale=1]
\node [vertex] (1) at (0,0) {};
\node [vertex] (2) at (1,0) {};
\node [vertex] (3) at (0,0.5) {};
\node [vertex] (4) at (1,0.5) {};
\node [vertex] (5) at (0,1) {};
\node [vertex] (6) at (1,1) {};
\node [vertex] (7) at (0.5,1.5) {};
\foreach \i/\j in {1/2,3/4,5/6}
  \draw [edge] (\i) to (\j);
\node at(0.5,-1) {\parbox{0.3\linewidth}{\subcaption*{$F_{1}$}}};
\end{scope}

\begin{scope}[xshift=2.5cm,scale=1]
\node [vertex] (1) at (0,0) {};
\node [vertex] (2) at (0.5,0) {};
\node [vertex] (3) at (1,0) {};
\node [vertex] (4) at (0.5,0.5) {};
\node [vertex] (5) at (0,1) {};
\node [vertex] (6) at (1,1) {};
\node [vertex] (7) at (0.5,1.5) {};
\foreach \i/\j in {1/2,2/3,4/5,4/6,5/7,6/7}
  \draw [edge] (\i) to (\j);
\node at(0.5,-1) {\parbox{0.3\linewidth}{\subcaption*{$F_{2}$}}};
\end{scope}

\begin{scope}[xshift=5cm,scale=1]
\node [vertex] (1) at (0.5,0) {};
\node [vertex] (2) at (0,0.5) {};
\node [vertex] (3) at (0.5,0.5) {};
\node [vertex] (4) at (1,0.5) {};
\node [vertex] (5) at (0.5,1) {};
\node [vertex] (6) at (0,1.5) {};
\node [vertex] (7) at (1,1.5) {};
\foreach \i/\j in {2/5,2/6,3/5,4/5,4/7,5/6,5/7}
  \draw [edge] (\i) to (\j);
\node at(0.5,-1) {\parbox{0.3\linewidth}{\subcaption*{$F_{3}$}}};
\end{scope}

\begin{scope}[xshift=7.5cm,scale=1]
\node [vertex] (1) at (0,0) {};
\node [vertex] (2) at (1,0) {};
\node [vertex] (3) at (0.5,0.5) {};
\node [vertex] (4) at (0,1) {};
\node [vertex] (5) at (0.5,1) {};
\node [vertex] (6) at (1,1) {};
\node [vertex] (7) at (0.5,1.5) {};
\foreach \i/\j in {3/4,3/5,3/6,4/5,4/7,5/6,5/7,6/7}
  \draw [edge] (\i) to (\j);
\node at(0.5,-1) {\parbox{0.3\linewidth}{\subcaption*{$F_{4}$}}};
\end{scope}

\begin{scope}[xshift=10.25cm,scale=1]
\node [vertex] (1) at (0,0) {};
\node [vertex] (2) at (1,0) {};
\node [vertex] (3) at (-0.25,0.5) {};
\node [vertex] (4) at (1.25,0.5) {};
\node [vertex] (5) at (0,1) {};
\node [vertex] (6) at (1,1) {};
\node [vertex] (7) at (0.5,1.5) {};
\foreach \i/\j in {1/3,1/4,1/5,1/6,2/3,2/4,2/5,2/6,3/5,3/6,4/5,4/6}
  \draw [edge] (\i) to (\j);
\node at(0.5,-1) {\parbox{0.3\linewidth}{\subcaption*{$F_{5}$}}};
\end{scope}

\end{tikzpicture}
\end{center}
\end{subfigure}

\caption{Obstrucciones mínimas de las cográficas $2$-polares con 7 vértices.}
\label{obsts_2polares_02}
\end{figure}


\begin{figure}[ht!]
\begin{subfigure}{\textwidth}
\begin{center}
\begin{tikzpicture}

\begin{scope}[xshift=0cm,scale=1]
\node [vertex] (1) at (0,0) {};
\node [vertex] (2) at (1,0) {};
\node [vertex] (3) at (0,0.5) {};
\node [vertex] (4) at (1,0.5) {};
\node [vertex] (5) at (0,1) {};
\node [vertex] (6) at (1,1) {};
\node [vertex] (7) at (0.25,1.5) {};
\node [vertex] (8) at (0.75,1.5) {};
\foreach \i/\j in {1/2,1/4,2/3,3/4,5/6}
  \draw [edge] (\i) to (\j);
\node at(0.5,-1) {\parbox{0.3\linewidth}{\subcaption*{$F_{13}$}}};
\end{scope}

\begin{scope}[xshift=2.5cm,scale=1]
\node [vertex] (1) at (0,0) {};
\node [vertex] (2) at (1,0) {};
\node [vertex] (3) at (0,0.5) {};
\node [vertex] (4) at (1,0.5) {};
\node [vertex] (5) at (0,1) {};
\node [vertex] (6) at (1,1) {};
\node [vertex] (7) at (0,1.5) {};
\node [vertex] (8) at (1,1.5) {};
\foreach \i/\j in {1/2,1/3,1/4,2/4,3/4,5/6,7/8}
  \draw [edge] (\i) to (\j);
\node at(0.5,-1) {\parbox{0.3\linewidth}{\subcaption*{$F_{14}$}}};
\end{scope}

\begin{scope}[xshift=5cm,scale=1]
\node [vertex] (1) at (0,0) {};
\node [vertex] (2) at (1,0) {};
\node [vertex] (3) at (0,0.5) {};
\node [vertex] (4) at (1,0.5) {};
\node [vertex] (5) at (0,1) {};
\node [vertex] (6) at (1,1) {};
\node [vertex] (7) at (0,1.5) {};
\node [vertex] (8) at (1,1.5) {};
\foreach \i/\j in {1/2,1/3,1/4,2/3,2/4,3/5,3/6,4/5,4/6,5/6,7/8}
  \draw [edge] (\i) to (\j);
\node at(0.5,-1) {\parbox{0.3\linewidth}{\subcaption*{$F_{15}$}}};
\end{scope}

\begin{scope}[xshift=7.5cm,scale=1]
\node [vertex] (1) at (0,0) {};
\node [vertex] (2) at (0,0.5) {};
\node [vertex] (3) at (0.5,0) {};
\node [vertex] (4) at (1,0.5) {};
\node [vertex] (5) at (0.5,1) {};
\node [vertex] (6) at (0,1.5) {};
\node [vertex] (7) at (1,1.5) {};
\node [vertex] (8) at (1,0) {};
\foreach \i/\j in {2/5,2/6,4/5,4/7,5/6,5/7,1/3,3/8}
  \draw [edge] (\i) to (\j);
\node at(0.5,-1) {\parbox{0.3\linewidth}{\subcaption*{$F_{16}$}}};
\end{scope}

\begin{scope}[xshift=10cm,scale=1]
\node [vertex] (1) at (0,0) {};
\node [vertex] (2) at (1,0) {};
\node [vertex] (3) at (0,0.5) {};
\node [vertex] (4) at (1,0.5) {};
\node [vertex] (5) at (0,1) {};
\node [vertex] (6) at (1,1) {};
\node [vertex] (7) at (0,1.5) {};
\node [vertex] (8) at (1,1.5) {};
\foreach \i/\j in {1/2,1/3,1/4,2/3,2/4,3/4,3/5,3/6,4/5,4/6,5/6} \draw [edge] (\i) to (\j);
\node at(0.5,-1) {\parbox{0.3\linewidth}{\subcaption*{$F_{17}$}}};
\end{scope}

\end{tikzpicture}
\end{center}
\end{subfigure}

\begin{subfigure}{\textwidth}
\begin{center}
\begin{tikzpicture}

\begin{scope}[xshift=0cm,scale=1]
\node [vertex] (1) at (0.75,0) {};
\node [vertex] (2) at (0.25,0.5) {};
\node [vertex] (3) at (1.25,0.5) {};
\node [vertex] (4) at (0,1) {};
\node [vertex] (5) at (0.75,1) {};
\node [vertex] (6) at (1.5,1) {};
\node [vertex] (7) at (0.25,1.5) {};
\node [vertex] (8) at (1.25,1.5) {};
\foreach \i/\j in {2/3,2/4,2/5,2/6,3/4,3/5,3/6,4/5,4/7,4/8,5/6,5/7,5/8,6/7,6/8,7/8}
  \draw [edge] (\i) to (\j);
\node at(0.75,-1) {\parbox{0.3\linewidth}{\subcaption*{$F_{18}$}}};
\end{scope}

\begin{scope}[xshift=3cm,scale=1]
\node [vertex] (1) at (0,0) {};
\node [vertex] (2) at (1,0) {};
\node [vertex] (3) at (0,0.5) {};
\node [vertex] (4) at (1,0.5) {};
\node [vertex] (5) at (0,1) {};
\node [vertex] (6) at (1,1) {};
\node [vertex] (7) at (0,1.5) {};
\node [vertex] (8) at (1,1.5) {};
\foreach \i/\j in {1/2,1/3,1/4,2/3,2/4,3/5,3/6,4/5,4/6}
  \draw [edge] (\i) to (\j);
\node at(0.5,-1) {\parbox{0.3\linewidth}{\subcaption*{$F_{19}$}}};
\end{scope}

\begin{scope}[xshift=5.5cm,scale=1]
\node [vertex] (1) at (0,0) {};
\node [vertex] (2) at (0.75,0) {};
\node [vertex] (3) at (0,0.75) {};
\node [vertex] (4) at (0.75,0.75) {};
\node [vertex] (5) at (1.5,0.75) {};
\node [vertex] (6) at (0.5,1) {};
\node [vertex] (7) at (1,1) {};
\node [vertex] (8) at (0.75,1.5) {};
\foreach \i/\j in {2/3,2/4,2/5,3/4,3/8,4/5,4/6,4/7,4/8,5/8}
  \draw [edge] (\i) to (\j);
\node at(0.75,-1) {\parbox{0.3\linewidth}{\subcaption*{$F_{20}$}}};
\end{scope}

\end{tikzpicture}
\end{center}
\end{subfigure}

\caption{Familia A de obstrucciones mínimas de las cográficas $2$-polares con 8 vértices.}
\label{obsts_2polares_03}
\end{figure}

\begin{figure}[ht!]
\begin{subfigure}{\textwidth}
\begin{center}
\begin{tikzpicture}

\begin{scope}[xshift=0cm,scale=1]
\node [vertex] (1) at (0,0) {};
\node [vertex] (2) at (1,0) {};
\node [vertex] (3) at (0,0.5) {};
\node [vertex] (4) at (1,0.5) {};
\node [vertex] (5) at (0,1) {};
\node [vertex] (6) at (1,1) {};
\node [vertex] (7) at (0,1.5) {};
\node [vertex] (8) at (0.5,1.5) {};
\node [vertex] (9) at (1,1.5) {};
\foreach \i/\j in {1/2,1/4,2/3,3/4,3/6,4/5,5/6}
  \draw [edge] (\i) to (\j);
\node at(0.5,-1) {\parbox{0.3\linewidth}{\subcaption*{$F_{21}$}}};
\end{scope}

\begin{scope}[xshift=2.5cm,scale=1]
\node [vertex] (1) at (0,0) {};
\node [vertex] (2) at (0.5,0.25) {};
\node [vertex] (3) at (0,0.5) {};
\node [vertex] (4) at (1,0.5) {};
\node [vertex] (5) at (0.5,0.75) {};
\node [vertex] (6) at (1,1) {};
\node [vertex] (7) at (0,1) {};
\node [vertex] (8) at (0.5,1.25) {};
\node [vertex] (9) at (0,1.5) {};
\foreach \i/\j in {1/2,1/3,2/3,4/5,4/6,5/6,7/8,7/9,8/9}
  \draw [edge] (\i) to (\j);
\node at(0.5,-1) {\parbox{0.3\linewidth}{\subcaption*{$F_{22}$}}};
\end{scope}

\begin{scope}[xshift=5cm,scale=1]
\node [vertex] (1) at (0.5,0) {};
\node [vertex] (2) at (1.5,0) {};
\node [vertex] (3) at (0.5,0.5) {};
\node [vertex] (4) at (1.5,0.5) {};
\node [vertex] (5) at (0,1) {};
\node [vertex] (6) at (1,1) {};
\node [vertex] (7) at (2,1) {};
\node [vertex] (8) at (0.5,1.5) {};
\node [vertex] (9) at (1.5,1.5) {};
\foreach \i/\j in {1/2,3/5,3/6,3/8,4/6,4/7,4/9,5/6,5/8,6/7,6/8,6/9,7/9}
  \draw [edge] (\i) to (\j);
\node at(0.75,-1) {\parbox{0.3\linewidth}{\subcaption*{$F_{23}$}}};
\end{scope}

\begin{scope}[xshift=8.5cm,scale=1]
\node [vertex] (1) at (0.75,0) {};
\node [vertex] (2) at (0,0.5) {};
\node [vertex] (3) at (0.75,0.5) {};
\node [vertex] (4) at (1.5,0.5) {};
\node [vertex] (5) at (0.25,1) {};
\node [vertex] (6) at (1.25,1) {};
\node [vertex] (7) at (0,1.5) {};
\node [vertex] (8) at (0.75,1.5) {};
\node [vertex] (9) at (1.5,1.5) {};
\foreach \i/\j in {2/3,2/5,2/6,3/4,3/5,3/6,4/5,4/6,5/6,5/7,5/8,5/9,6/7,6/8,6/9,7/8,8/9}
  \draw [edge] (\i) to (\j);
\node at(0.75,-1) {\parbox{0.3\linewidth}{\subcaption*{$F_{24}$}}};
\end{scope}

\end{tikzpicture}
\end{center}
\end{subfigure}

\caption{Obstrucciones mínimas de las cográficas $2$-polares con 9 vértices.}
\label{obsts_2polares_04}
\end{figure}


\begin{figure}[ht!]
\begin{subfigure}{\textwidth}
\begin{center}
\begin{tikzpicture}

\begin{scope}[xshift=0cm,scale=1]
\node [vertex] (1) at (0.25,0) {};
\node [vertex] (2) at (0.75,0) {};
\node [vertex] (3) at (0,0.75) {};
\node [vertex] (4) at (0.5,0.75) {};
\node [vertex] (5) at (1,0.75) {};
\node [vertex] (6) at (0,1.5) {};
\node [vertex] (7) at (0.5,1.5) {};
\node [vertex] (8) at (1,1.5) {};
\foreach \i/\j in {1/2,3/4,4/5,6/7,7/8}
  \draw [edge] (\i) to (\j);
\node at(0.5,-1) {\parbox{0.3\linewidth}{\subcaption*{$F_{6}$}}};
\end{scope}

\begin{scope}[xshift=2.5cm,scale=1]
\node [vertex] (1) at (0,0) {};
\node [vertex] (8) at (0.5,0) {};
\node [vertex] (2) at (1,0) {};
\node [vertex] (3) at (0.5,0.5) {};
\node [vertex] (4) at (0,1) {};
\node [vertex] (5) at (0.5,1) {};
\node [vertex] (6) at (1,1) {};
\node [vertex] (7) at (0.5,1.5) {};
\foreach \i/\j in {3/4,3/5,3/6,4/5,4/7,5/6,6/7,8/2}
  \draw [edge] (\i) to (\j);
\node at(0.5,-1) {\parbox{0.3\linewidth}{\subcaption*{$F_{7}$}}};
\end{scope}

\begin{scope}[xshift=5cm,scale=1]
\node [vertex] (1) at (0,0) {};
\node [vertex] (2) at (1,0) {};
\node [vertex] (3) at (0,0.5) {};
\node [vertex] (4) at (1,0.5) {};
\node [vertex] (5) at (0,1) {};
\node [vertex] (6) at (1,1) {};
\node [vertex] (7) at (0,1.5) {};
\node [vertex] (8) at (1,1.5) {};
\foreach \i/\j in {1/2,1/3,1/4,2/3,2/4,3/4,3/6,4/5,5/6} \draw [edge] (\i) to (\j);
\node at(0.5,-1) {\parbox{0.3\linewidth}{\subcaption*{$F_{8}$}}};
\end{scope}

\begin{scope}[xshift=7.5cm,scale=1]
\node [vertex] (1) at (0.5,0) {};
\node [vertex] (2) at (0,0.5) {};
\node [vertex] (3) at (1,0.5) {};
\node [vertex] (4) at (0,1) {};
\node [vertex] (5) at (1,1) {};
\node [vertex] (6) at (0,1.5) {};
\node [vertex] (7) at (0.5,1.5) {};
\node [vertex] (8) at (1,1.5) {};
\foreach \i/\j in {2/3,2/4,2/5,3/4,3/5,4/6,4/7,4/8,5/6,5/7,5/8,6/7,7/8} \draw [edge] (\i) to (\j);
\node at(0.5,-1) {\parbox{0.3\linewidth}{\subcaption*{$F_{9}$}}};
\end{scope}

\begin{scope}[xshift=10cm,scale=1]
\node [vertex] (1) at (0.5,0) {};
\node [vertex] (2) at (1,0) {};
\node [vertex] (3) at (0.5,0.5) {};
\node [vertex] (4) at (1.5,0.5) {};
\node [vertex] (5) at (0,1) {};
\node [vertex] (6) at (1,1) {};
\node [vertex] (7) at (0.5,1.5) {};
\node [vertex] (8) at (1.5,1.5) {};
\foreach \i/\j in {1/2,3/5,3/6,4/6,4/8,5/6,5/7,6/7,6/8} \draw [edge] (\i) to (\j);
\node at(0.5,-1) {\parbox{0.3\linewidth}{\subcaption*{$F_{10}$}}};
\end{scope}

\end{tikzpicture}
\end{center}
\end{subfigure}

\begin{subfigure}{\textwidth}
\begin{center}
\begin{tikzpicture}

\begin{scope}[xshift=5cm,scale=1]
\node [vertex] (1) at (1,0) {};
\node [vertex] (3) at (0.5,0.5) {};
\node [vertex] (4) at (1.5,0.5) {};
\node [vertex] (5) at (0,1) {};
\node [vertex] (6) at (1,1) {};
\node [vertex] (7) at (2,1) {};
\node [vertex] (8) at (0.5,1.5) {};
\node [vertex] (9) at (1.5,1.5) {};
\foreach \i/\j in {3/5,3/6,4/6,4/7,5/6,5/8,6/7,6/8,6/9,7/9}
  \draw [edge] (\i) to (\j);
\node at(0.75,-1) {\parbox{0.3\linewidth}{\subcaption*{$F_{11}$}}};
\end{scope}

\begin{scope}[xshift=8.5cm,scale=1]
\node [vertex] (1) at (0,0) {};
\node [vertex] (2) at (1.25,0) {};
\node [vertex] (3) at (2,0.25) {};
\node [vertex] (4) at (0,0.75) {};
\node [vertex] (5) at (0.75,0.75) {};
\node [vertex] (6) at (2,1.25) {};
\node [vertex] (7) at (0,1.5) {};
\node [vertex] (8) at (1.25,1.5) {};
\foreach \i/\j in {2/3,2/4,2/5,2/6,3/5,3/8,3/6,4/5,4/8,5/7,5/8,5/6,8/6}
  \draw [edge] (\i) to (\j);
\node at(0.75,-1) {\parbox{0.3\linewidth}{\subcaption*{$F_{12}$}}};
\end{scope}

\end{tikzpicture}
\end{center}
\end{subfigure}

\caption{Familia B de obstrucciones mínimas de las cográficas $2$-polares con 8 vértices.}
\label{obsts_2polares_05}
\end{figure}

En este artículo podemos observar un ejemplo en el que, dada
una clase hereditaria de cográficas $C$ y una operación
$o$ tales que $C$ es cerrada bajo $o$, es posible encontrar
familias de obstrucciones mínimas para $C$ a partir de
obstrucciones mínimas de $C$ ya conocidas aplic\'andoles
la operación $o$. Las clases de cográficas que estudiamos en
nuestra investigación son cerradas bajo la unión completa y,
de manera parecida a lo que sucede en \cite{Hell03},
encontramos reglas para generar obstrucciones mínimas para
una clase hereditaria $D$ al aplicar la unión completa a
obstrucciones mínimas de subclases de $D$.

% Quizá valdría la pena ser más específicos en la última
% oración de este párrafo.


\section{Un algoritmo lineal para el reconocimiento de cográficas}
    En \cite{Corneil02}, Corneil, Perl y Stewart presentan un algoritmo
que, dada una gráfica, determina si ésta es una cográfica y, si lo
es, devuelve su coárbol. Éste es un algoritmo incremental, se
construye un coárbol agregando los vértices de la gráfica recibida
como entrada uno a uno. Para agregar un nuevo vértice $x$, de la
gráfica de entrada $G$, al coárbol $T$ que se va construyendo, el
algoritmo ejecuta primero una subrutina que se encarga de marcar los
nodos de $T$ empezando por sus hojas. Una hoja de $T$ se marca sólo
si ésta representa un vértice de $G$ adyacente a $x$. La subrutina
continúa marcando nodos de $G$ subiendo hacia la raíz. En nuestra
investigación presentamos un algoritmo similar, el Algoritmo
\ref{alg_subcoarbol}, que ayuda a determinar si una cográfica
$H$ es subcográfica de otra cográfica $I$ marcando los vértices de
un coárbol binario\footnote{Un coárbol binario es un coárbol en el
que cada nodo tiene necesariamente dos hijos y pueden haber nodos
adyacentes con la misma etiqueta.} de $I$ partiendo desde las hojas
y subiendo hacia la raíz.

A continuación se reproducen los resultados de \cite{ Corneil02},
empezando por el algoritmo de marcado antes mencionado.

\begin{algorithm}[!htbp]
\caption{Marcar}
\label{alg_mark_corneil}
\DontPrintSemicolon % Some LaTeX compilers require you to use \dontprintsemicolon instead
\KwIn{$T$, un coárbol con raíz $R$ cuyas hojas son vértices de una gráfica $G$; $x$, el vértice de $G$ que se busca agregar a $T$}
\KwOut{Se marcan y se desmarcan algunos nodos de $T$ }

$D\gets \{\}$ \tcp*[h]{El conjunto de los nodos que han sido marcados y desmarcados} \; 

Marcar todas las hojas de $T$ que sean adyacentes a $x$\;

\ForEach{\emph{nodo marcado $u$ de $T$ tal que todos sus hijos están en el conjunto $D$}}{
    Desmarcar a $u$\;
    Agregar $u$ a $D$\;

    \If{$u \neq R$}{
        \emph{Marcar al padre de $u$}\;
    }
}

\end{algorithm}

Sean $T$ un coárbol, $x$ un vértice y $M$ el conjunto de nodos
de $T$ que se encuentran marcados al terminar la ejecución del
Algoritmo \ref{alg_mark_corneil} al recibir como entrada a $T$
y a $x$. El Teorema \ref{teo_teo1_corneil} puede ser utilizado
para determinar si la gráfica que se obtiene al agregar $x$ a
la gráfica representada por $T$ es una cográfica.
Las siguientes definiciones son necesarias para presentar dicho
teorema. Sea $\alpha$ un nodo de $M$ de profundidad máxima en $T$
y sea $\beta$ un nodo en $M-\{a\}$ de profundidad máxima en $T$.
Decimos que un nodo $\gamma$ de $T$ con etiqueta 1 está
correctamente marcado si y sólo si todos sus hijos, excepto uno,
fueron marcados y desmarcados. Un camino legítimamente alternante
en un coárbol marcado es un camino alternante de nodos correctamente
marcados y nodos sin marcar, con etiqueta 0, cuyos extremos son nodos
con etiqueta 1.


\begin{theorem}
    \label{teo_teo1_corneil}
    Si $G$ es una cográfica con árbol $T$, entonces $G+x$ es una cográfica si y sólo si:
    \begin{itemize}
        \item $M$ es vacío o
        \item se cumplen las siguientes dos condiciones
        \begin{itemize}
            \item $M-\{\alpha\}$ consiste exactamente de los nodos
              con etiqueta 1 de un camino legítimamente alternante
              que termina en $R$.

            \item $\alpha$ es un nodo con etiqueta 0 cuyo padre es
              $\beta$ o $\alpha$ es un nodo con etiqueta 1 cuyo
              abuelo, si existe, es $\beta$.
        \end{itemize}
    \end{itemize}
\end{theorem}

El Algoritmo \ref{alg_principal_corneil}, que constituye el resultado
principal del artículo, utiliza el Algoritmo \ref{alg_mark_corneil} y
el Teorema \ref{teo_teo1_corneil} para determinar si una gráfica es
una cográfica, y construir su coárbol en caso de que lo sea.

\begin{algorithm}[!htbp]
\caption{AlgoritmoDeReconocimiento}
\label{alg_principal_corneil}
\DontPrintSemicolon % Some LaTeX compilers require you to use \dontprintsemicolon instead
\KwIn{$G$, una gráfica cuyo conunto de vértices es $V=\{v_1, v_2, \dots, v_n\}$}
\KwOut{El coárbol de $G$ si $G$ es una cográfica, $null$ en el caso contrario}

Crear el coárbol $T$ cuya raíz $R$ tiene etiqueta 1 si $v_1$ y $v_2$ son adyacentes o etiqueta 0 en el caso contrario\;

\ForEach{x \emph{\textbf{en}} $V(G)-\{v_1,v_2\}$}{
    Marcar($T$, $x$)\;
    \If{Todos los nodos de $T$ fueron marcados y desmarcados}{
        \If{$R$ tiene etiqueta 1}{
            Agregar a $x$ como hijo de $R$\;
        }
        \Else{
            $R\gets$ un nuevo nodo con etiqueta 1 cuyos hijos sean $x$ y $R$\;
        }
    }
    \ElseIf{Ninguno de los nodos de $T$ fue marcado}{
        \If{$R$ tiene etiqueta 0}{
            Agregar a $x$ como hijo de $R$\;
        }
        \Else{
            $R\gets$ un nuevo nodo con etiqueta 0 cuyos hijos sean $x$ y $R$\;
        }
    }
    \ElseIf{La gráfica representada por $T$ agregando $x$ es una cográfica}{
        $u\gets$ el nodo marcado más profundo de $T$\;
        Encontrar en qué nodo del coárbol con raíz $u$ se debe insertar $x$\;
    }
    \Else{
        \Return $null$\;
    }
	Desmarcar todos los nodos de $T$\;
}

\end{algorithm}

%Cierre
El Algoritmo \ref{alg_mark_corneil} consolida un ejemplo
de un algoritmo que funciona marcando los nodos de un coárbol
empezando por las hojas y dirigiéndose hacia la raíz\footnote{En
ingl\'es un {\em bottom-up} algorithm.}. Por otro lado, el Algoritmo
\ref{alg_mark_corneil} es un ejemplo de un algoritmo que, al ser
ejecutado, no sólo resuelve un problema de decisión, sino que
proporciona una prueba (o {\em certificado}) de que su respuesta
es correcta. En este caso, la prueba es el coárbol de la
gráfica recibida como entrada. Sin embargo, cuando la gr\'afica
que se recibe como entrada no es una cogr\'afica, el Algoritmo
\ref{alg_mark_corneil} s\'olo devuelve {\em null}. Si desconfi\'aramos
de una implementaci\'on de este algoritmo, y quisi\'eramos comprobar
que la salida es correcta, podr\'iamos utilizar el co\'arbol en un
caso, pero en el otro, no tendr\'iamos informaci\'on adicional.

En nuestra investigación, el Algoritmo \ref{alg_cert_m2} es un
algoritmo certificador, es decir, devuelve certificados para
verificar, de manera eficiente, que la salida del algoritmo es
correcta, en cualquier caso. Nuestro algoritmo no sólo es capaz de
determinar si una cográfica $G$ pertenece a la clase $M_2$, sino que
devuelve una coloración de las hojas de su coárbol $T$ tal que si
$G \in M_2$, las hojas de $T$ tienen uno de dos colores y las hojas
del mismo color inducen una gráfica multipartita completa en $G$
(un {\em s\'i-certificado}). En el caso contrario, algunas de las
hojas de $T$ tendrán un color distintivo que indica que esos
vértices inducen una obstrucción mínima de $M_2$ en $G$ (un
{\em no-certificado}).


