A continuación presentamos los conceptos de subcoárbol y subcoárbol binario
que serán utilizados para determinar si una cográfica $H$ es subgráfica
inducida de una cográfica $G$ en el Algoritmo \ref{alg_subgraph}.

Sean $T$ y $U$ dos coárboles y $u_1$, $u_2$ y $u_3$ nodos de $U$, decimos
que $U$ es un \emph{\textbf{subcoárbol}} de $T$ si existe una función
inyectiva $f:V(U)\rightarrow V(T)$ tal que, si $u_1$ es una hoja, entonces
$f(u_1)$ tambi\'en es una hoja; si no, entonces $u_1$ y $f(u_1)$ tienen la
misma etiqueta y, si $u_3$ es el ancestro común más profundo de $u_1$ y
$u_2$, entonces $f(u_3)$ es el ancestro común más profundo de $f(u_1)$ y
$f(u_2)$. Llamamos a $f$ la \textbf{\emph{función de coasignación}} de $U$ a
$T$.

El concepto de subcoárbol es diferente del de subárbol dado que, si $T$ y $U$
son coárboles con $U$ subcoárbol de $T$, entonces tenemos que los nodos de $U$
se pueden encontrar dispersos entre los nodos de $T$ a diferencia de lo que se
tendría si $U$ fuera subárbol de $T$. Esto se puede apreciar en la Figura
\ref{fig_subcoarbol01}. N\'otese que esta definición funciona también para
coárboles binarios.

\begin{figure}[h!]
\begin{center}
\begin{tikzpicture}

\begin{scope}[xshift=0cm,scale=1]
\node [style=cotreenode] (1) at (1,1) {0};
\node [style=cotreenode] (2) at (-0.5,0) {1};
\node [style=cotreenode] (3) at (2.5,0) {1};
\node [style=cotreenode] (4) at (-1.25,-1) {0};
\node [style=cotreenode] (5) at (0.25,-1) {0};
\node [style=cotreenode] (6) at (1.75,-1) {0};
\node [style=cotreenode] (7) at (3.25,-1) {0};
\node [style=vertex] (8) at (-1.5,-2) {};
\node [style=vertex] (9) at (-1,-2) {};
\node [style=vertex] (10) at (0,-2) {};
\node [style=vertex] (11) at (0.5,-2) {};
\node [style=vertex] (12) at (1.5,-2) {};
\node [style=vertex] (13) at (2,-2) {};
\node [style=vertex] (14) at (3,-2) {};
\node [style=vertex] (15) at (3.5,-2) {};

\node (16) at (0.25,1) {$f(a)$};
\node (17) at (-1.6,-2.4) {$f(b)$};
\node (18) at (3.25,0) {$f(c)$};
\node (19) at (1.4,-2.4) {$f(d)$};
\node (20) at (3.6,-2.4) {$f(e)$};

\foreach \i/\j in {1/2,1/3,2/4,2/5,3/6,3/7,4/8,4/9,5/10,5/11,6/12,6/13,7/14,7/15}
  \draw [style=edge] (\i) to (\j);
\node [below of=19,xshift=-0.25cm] {\parbox{0.3\linewidth}{\subcaption{}}};
\end{scope}

\begin{scope}[xshift=6cm,scale=1]
\node [style=cotreenode] (1) at (1,1) {0};
\node [style=vertex] (2) at (0,0) {};
\node [style=cotreenode] (3) at (2,0) {1};
\node [style=vertex] (4) at (1.5,-1) {};
\node [style=vertex] (5) at (2.5,-1) {};

\node (6) at (0.5,1) {$a$};
\node (7) at (-0.3,0) {$b$};
\node (8) at (2.5,0) {$c$};
\node (9) at (1.5,-1.3) {$d$};
\node (10) at (2.5,-1.3) {$e$};

\foreach \i/\j in {1/2,1/3,3/4,3/5}
  \draw [style=edge] (\i) to (\j);
\node [below of=9,xshift=-0.25cm] {\parbox{0.3\linewidth}{\subcaption{}}};
\end{scope}

\end{tikzpicture}
\end{center}
\setlength{\abovecaptionskip}{-10pt}
\caption{El coárbol (b) es subcoárbol del coárbol (a). Las etiquetas en los nodos de ambos coárboles indican la asignación de los nodos de (b) a los nodos de (a).}\label{fig_subcoarbol01}
\end{figure}

\begin{lemma}\label{lema_subcoa_01}
    Sean $G$ y $H$ cográficas y $T_G$ y $T_H$ sus coárboles correspondientes, entonces $H$ es una subgráfica inducida de $G$ si y sólo si $T_H$ es subcoárbol de $T_G$.
\end{lemma}

\begin{proof}

  Supongamos primero que $H$ es una subgráfica inducida de $G$. Sabemos que
  $V(H) \subset V(G)$ y que dos vértices son adyacentes en $H$ si y sólo si
  también son adyacentes en $G$. Sean $v$ y $w$ vértices diferentes de $H$,
  $n_H$ el ancestro común más profundo de $v$ y $w$ en $T_H$, $n_G$ el
  ancestro común más profundo de $v$ y $w$ en $T_G$ y $f$ un subconjunto de
  $V(T_H) \times V(T_G)$ tal que $(v,v) \in f$ y $(n_H, n_G)\in f$, veamos
  que $f$ es una función de coasignación de $T_H$ a $T_G$. Empecemos por
  mostrar que $f$ es una función. Dado que todo nodo de $T_H$ es una hoja o,
  en su defecto, es el ancestro común más profundo de dos hojas, cada uno de
  \'estos será el primer elemento de al menos una pareja elemento de $f$. Como
  $n_G$ y $n_H$ son únicos para cualesquiera $v$ y $w$, entonces si una pareja
  $(g,h)$ es elemento de $f$, entonces ésta es la única pareja que tiene a
  $g$ como primer elemento y a $h$ como segundo elemento. Así,
  $f \colon V(T_H)\rightarrow V(T_G)$ es una función.

  Sean $x,y,z \in V(T_H)$. Si $x$ es una hoja de $T_H$, es claro que $f(x) =
  x$ es una hoja de $T_G$. Si no, entonces $x$ es el ancestro común más
  profundo de algún par de hojas de $T_H$ y tiene etiqueta 1 si y sólo si
  éstas son adyacentes en $H$, lo que ocurre si y sólo si son adyacentes en
  $G$, que a su vez ocurre si y sólo si su ancestro común más profundo en
  $T_G$ tiene etiqueta etiqueta 1. Así $x$ y $f(x)$ tienen necesariamente la
  misma etiqueta. Finalmente mostremos que si $x$ es el ancestro común más
  profundo de $y$ y $z$, entonces $f(x)$ es el ancestro común más profundo de
  $f(y)$ y $f(z)$. Si $y$ y $z$ son hojas, esto se cumple trivialmente por la
  definición de $f$. Si $y$ es un nodo interno y $z$ es una hoja, entonces
  existen dos hojas $y'$ y $y''$ de $T_H$ tales que $y$ es su ancestro común
  más profundo. Tenemos que $f(y)$ es el ancestro común más profundo de
  $f(y')$ y $f(y'')$. A su vez, dado que $x$ es el ancestro común más profundo
  de $y'$ y $z$, entonces $f(x)$ es el ancestro común mas profundo de $f(y')$
  y $f(z)$. De igual manera $f(x)$ es el ancestro común mas profundo de
  $f(y'')$ y $f(z)$ Luego, $f(y)$ se encuentra en el camino desde $f(y')$
  hasta $f(x)$ y por lo tanto $f(x)$ es el nodo común más profundo de $f(y)$ y
  $f(z)$. Análogamente si $z$ es un nodo interno y $y$ es una hoja. Si tanto
  $y$ como $z$ son nodos internos, el argumento es prácticamente el mismo,
  pero utilizando dos hojas $y'$ y $y''$ cuyo ancestro común más profundo es
  $y$ y otras dos hojas $z'$ y $z''$ cuyo ancestro común más profundo es $z$.
  Así, $f$ es una función de coasignación de $T_H$ a $T_G$ y $T_H$ es
  subcoárbol de $T_G$.

  Rec\'iprocamente, si $T_H$ es subcoárbol de $T_G$, entonces existe
  una función de coasignación, $f$, de $T_H$ a $T_G$. Luego, sean $h_1$ y
  $h_2$ hojas de $T_H$ y $h_3$ el ancestro común más profundo de $h_1$ y
  $h_2$, tenemos que $f(h_1)$ y $f(h_2)$ son hojas de $T_G$ y que las
  etiquetas de $h_3$ y $f(h_3)$ coinciden. Así, $h_1$ y $h_2$ son adyacentes
  en $H$ si y sólo si $f(h_1)$ y $f(h_2)$ son adyacentes en $G$. Luego,
  $G[f[V(H)]]$ es una subgráfica de $G$ que es isomorfa a $H$. Así, $H$
  es subgráfica de $G$.

\end{proof}

Notemos que esta demostración funciona únicamente para los coárboles y no para los coárboles binarios. En la Figura \ref{fig_coar_bin01} se observan dos coárboles binarios que representan a la misma cográfica. Sin embargo ninguno de los dos es subcoárbol del otro.

\begin{lemma}
    Sean $G$ y $H$ cográficas y $T_G$ y $T_H$ coárboles binarios de $G$ y $H$ respectivamente. Si $T_H$ es subcoárbol de $T_G$, entonces $H$ es subgráfica de $G$.
\end{lemma}

\begin{proof}
    La demostración es igual a la segunda parte de la demostración del Lema \ref{lema_subcoa_01}.
\end{proof}
