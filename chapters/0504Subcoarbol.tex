A continuación presentamos el concepto de subcoárbol binario
que será utilizado para determinar si una cográfica $H$ es subgráfica
inducida de una cográfica $G$ en el Algoritmo \ref{alg_subgraph}.

Sean $T$ y $U$ dos coárboles binarios y $u_1$, $u_2$ y $u_3$ nodos de $U$, decimos
que $U$ es un \emph{\textbf{subcoárbol binario}} de $T$ si existe una función
inyectiva $f:V(U)\rightarrow V(T)$ tal que, si $u_1$ es una hoja, entonces $f(u_1)$
es una hoja también; si no, entonces $u_1.etiqueta = f(u_1).etiqueta$ y, si $u_3$ es
el ancestro común más profundo de $u_1$ y $u_2$, entonces $f(u_3)$ es el ancestro
común más profundo de $f(u_1)$ y $f(u_2)$. Llamamos a $f$ la \textbf{\emph{función de
coasignación}} de $U$ a $T$.

El concepto de subcoárbol binario es diferente del de subárbol dado que, si $T$ y $U$
son coárboles con $U$ subcoárbol binario de $T$, entonces tenemos que los nodos de
$U$ se pueden encontrar dispersos entre los nodos de $T$ a diferencia de lo que se
tendría si $U$ fuera subárbol de $T$. Esto se puede apreciar en la Figura
\ref{fig_subcoarbol01}.

\begin{figure}[h!]
\begin{center}
\begin{tikzpicture}

\begin{scope}[xshift=0cm,scale=1]
\node [style=cotreenode] (1) at (1,1) {0};
\node [style=cotreenode] (2) at (-0.5,0) {1};
\node [style=cotreenode] (3) at (2.5,0) {1};
\node [style=cotreenode] (4) at (-1.25,-1) {0};
\node [style=cotreenode] (5) at (0.25,-1) {0};
\node [style=cotreenode] (6) at (1.75,-1) {0};
\node [style=cotreenode] (7) at (3.25,-1) {0};
\node [style=vertex] (8) at (-1.5,-2) {};
\node [style=vertex] (9) at (-1,-2) {};
\node [style=vertex] (10) at (0,-2) {};
\node [style=vertex] (11) at (0.5,-2) {};
\node [style=vertex] (12) at (1.5,-2) {};
\node [style=vertex] (13) at (2,-2) {};
\node [style=vertex] (14) at (3,-2) {};
\node [style=vertex] (15) at (3.5,-2) {};

\node (16) at (0.25,1) {$f(a)$};
\node (17) at (-1.6,-2.4) {$f(b)$};
\node (18) at (3.25,0) {$f(c)$};
\node (19) at (1.4,-2.4) {$f(d)$};
\node (20) at (3.6,-2.4) {$f(e)$};

\foreach \i/\j in {1/2,1/3,2/4,2/5,3/6,3/7,4/8,4/9,5/10,5/11,6/12,6/13,7/14,7/15}
  \draw [style=edge] (\i) to (\j);
\node [below of=19,xshift=-0.25cm] {\parbox{0.3\linewidth}{\subcaption{}}};
\end{scope}

\begin{scope}[xshift=6cm,scale=1]
\node [style=cotreenode] (1) at (1,1) {0};
\node [style=vertex] (2) at (0,0) {};
\node [style=cotreenode] (3) at (2,0) {1};
\node [style=vertex] (4) at (1.5,-1) {};
\node [style=vertex] (5) at (2.5,-1) {};

\node (6) at (0.5,1) {$a$};
\node (7) at (-0.3,0) {$b$};
\node (8) at (2.5,0) {$c$};
\node (9) at (1.5,-1.3) {$d$};
\node (10) at (2.5,-1.3) {$e$};

\foreach \i/\j in {1/2,1/3,3/4,3/5}
  \draw [style=edge] (\i) to (\j);
\node [below of=9,xshift=-0.25cm] {\parbox{0.3\linewidth}{\subcaption{}}};
\end{scope}

\end{tikzpicture}
\end{center}
\setlength{\abovecaptionskip}{-10pt}
\caption{El coárbol (b) es subcoárbol binario del coárbol (a). Las etiquetas en los nodos de ambos coárboles binarios indican la asignación de los nodos de (b) a los nodos de (a).}\label{fig_subcoarbol01}
\end{figure}

\begin{lemma}
Sean $G$ y $H$ cográficas, $T_G$ un coárbol binario de $G$ y $T_H$ un coárbol binario de $H$. Si $T_H$ es subcoárbol binario de $T_G$, entonces $H$ es una subgráfica inducida de $G$.
\end{lemma}

\begin{proof}
Sean $h_1$ y $h_2$ hojas diferentes de $T_H$ y $h_3$ el ancestro común más profundo
de $h_1$ y $h_2$. Como $T_H$ es subcoárbol binario de $T_G$, entonces existe una
función de coasignación, $f$, de $T_H$ a $T_G$. Luego, tenemos que $f(h_1)$ y
$f(h_2)$ son hojas de $T_G$ y que las etiquetas de $h_3$ y $f(h_3)$ coinciden. Así,
$h_1$ y $h_2$ son adyacentes en $H$ si y sólo si $f(h_1)$ y $f(h_2)$ son adyacentes
en $G$. Luego, $G[f[V(H)]]$ es una subgráfica indicida de $G$ que es isomorfa a $H$.
Así, $H$ es una subgráfica inducida de $G$.
\end{proof}


\begin{lemma}
Sean $G$ y $H$ cográficas y $B_G$ un coárbol binario de $G$. Si $H$ es una subgráfica inducida de $G$, entonces existe un coárbol binario $B_H$ de $H$ tal que $B_H$ es subcoárbol binario de $B_G$.
\end{lemma}

\begin{proof}
Como $H$ es una subgráfica inducida de $G$, $V(H)$ es un subconjunto de las hojas de $B_G$. Consideremos el siguiente proceso recursivo para crear el árbol $B_H$.

\begin{algorithm}[H]
\DontPrintSemicolon
\KwIn{$V(H)$}
\KwOut{$r$, la raíz de $B_H$}
    \If{$V(H)$ tiene un solo elemento}{
        $r\gets v$, el único elemento de $V(H)$\;
    }
    \Else{
        $v\gets$ el nodo más profundo de $B_G$ que es ancestro de cada elemento de $V(H)$\;
        $r\gets$ nuevo nodo de coárbol binario con la misma etiqueta que $v$\;
        Asignar al hijo izquierdo de $r$ el resultado de procesar recursivamente los elementos de $V(H)$ que estén en la rama izquierda de $v$\;
        Asignar al hijo derecho de $r$ el resultado de procesar recursivamente los elementos de $V(H)$ que estén en la rama derecha de $v$\;
    }
    \Return $r$,
\end{algorithm}

Notemos que cada nodo de $B_H$ se construye tomando como base a un nodo de $B_G$. Esto ocurre en la línea 2 si dicho nodo es una hoja o en la línea 4 si es un nodo interno. Sea $x$ un nodo de $B_H$, denotamos por $x'$ al nodo de $B_G$ que se toma como base para construir a $x$.
% Aquí hay un abuso del lenguaje que puede confundir al lector, cuando
% usas $x$ para referirte al nodo y al coárbol con raíz $x$.

Sea $x$ un nodo de $B_H$, notemos lo siguiente. Si $x$ es una hoja de $B_H$, entonces $x$ es un vértice de $H$, por lo que $x = x'$, y si $x$ es un nodo interno, entonces la etiqueta de $x$ es la misma que la etiqueta de $x'$. Además, el hijo derecho de $x$ es ancestro de los mismos vértices de $H$ que el hijo derecho de $x'$. Análogamente para el hijo izquierdo. De esto se sigue que para cualesquiera dos vértices de $H$, su ancestro común más profundo en $B_H$ tiene la misma etiqueta que su ancestro común más profundo en $B_G$.

Veamos que $B_H$ es un coárbol binario de $H$. Notemos que las hojas de $B_H$ son todos los vértices de $H$. Dado que $H$ es una subgráfica inducida de $G$, cualesquiera dos vértices de $H$ son adyacentes si y sólo si su ancestro común más profundo en $B_G$ tiene etiqueta 1. Luego, dos vértices de $H$ son adyacentes si y sólo si su ancestro común más profundo en $B_H$ tiene etiqueta 1. Así, $B_H$ es un subcoárbol binario de $H$.

Sea $f$ un subconjunto de $V(B_H) \times V(B_G)$ tal que $(x,x')\in f$ para cualquier nodo $x$ de $V(B_H)$, veamos que $f$ es una función de coasignación de $B_H$ a $B_G$. Como cada pareja en $f$ tiene como primer elemento a un nodo único de $B_H$ y como segundo elemento un nodo único de $B_G$, $f$ es una función inyectiva. Sean $x_1$, $x_2$ y $x_3$ nodos de $B_H$. Es claro que si $x$ es una hoja, entonces $f(x)=x'$ es una hoja. Y si $x$ es un nodo interno, entonces $f(x)=x'$ tiene la misma etiqueta que $x$. Si $x_3$ es el ancestro común más profundo de $x_1$ y $x_2$, notemos lo siguiente:
\begin{itemize}
    \item el conjunto de vértices de $H$ que son descendientes de $x_1$ es igual al conjunto de vértices de $H$ que son descendientes de $f(x_1)$.
    \item el conjunto de vértices de $H$ que son descendientes de $x_2$ es igual al conjunto de vértices de $H$ que son descendientes de $f(x_2)$.
    \item el hijo derecho de $x_3$ es ancestro de los mismos vértices de $H$ que el hijo derecho de $f(x_3)$.
\end{itemize}
Supongamos sin pérdida de generalidad que $x_1$ es descendiente del hijo izquierdo de $x_3$ y que $x_2$ es descendiente de su hijo derecho. Luego,  $f(x_1)$ es descendiente del hijo izquierdo de $f(x_3)$ y $f(x_2)$ es descendiente de su hijo derecho. Así, $f(x_3)$ es el ancestro común más profundo de $f(x_1)$ y $f(x_2)$. Luego, $f$ es una función de coasignación de $B_H$ a $B_G$.

\end{proof}
