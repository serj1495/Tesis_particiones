Dada una cogr\'afica $G$, el Algoritmo \ref{alg_coa_bin01} construye
recursivamente un co\'arbol binario de $G$ a partir de un co\'arbol de $G$.
Las hojas del co\'arbol son la base de la recursi\'on; al procesar una hoja
del co\'arbol original, simplemente se crea un nuevo nodo para el co\'arbol
binario, inicialmente sin etiqueta, y se devuelve.  Los nodos internos de
un co\'arbol siempre tienen al menos dos hijos.   Si $r$ es un nodo interno
del co\'arbol original con dos hijos, entonces se crea un nuevo nodo en
el co\'arbol binario con la misma etiqueta que $r$, y sus sub\'arboles
izquierdo y derecho son los obtenidos al procesar recursivamente a su
hijo izquierdo y derecho, respectivamente.   Si $r$ tiene al menos tres
hijos, se procesa creando un nuevo coárbol binario de la siguiente forma:
La raíz del coárbol binario tiene la misma etiqueta que $r$, como
sub\'arbol izquierdo de la ra\'iz, se tiene al coárbol binario resultante
de procesar recursivamente al primer hijo de $r$, y como sub\'arbol derecho,
se tiene un nodo con la misma etiqueta que $r$, que a su vez tiene como
sub\'arbol izquierdo al árbol binario resultante de procesar recursivamente
al segundo hijo de $r$ y como sub\'arbol derecho un nuevo nodo con la misma
etiqueta que \'el. El árbol binario resultante es un árbol cargado a la
derecha. La Figura \ref{fig_alg_coa_bin01} muestra
una ejecución ilustrativa del algoritmo.

\begin{algorithm}[ht!]
\caption{CrearArbolBinario}
\label{alg_coa_bin01}
\DontPrintSemicolon % Some LaTeX compilers require you to use \dontprintsemicolon instead
\KwIn{$r$ la raíz del coárbol}
\KwOut{$r'$ la raíz del coárbol binario}

$r' \gets $ nuevo nodo de árbol binario\;

\If{$r$ es un nodo interno}{
    $r'.etiqueta = r.etiqueta$\;
    $s \gets r'$\;
    $i \gets 0$\;
    \While{$i < r.children.size - 2$}{
        $s.izquierda \gets \text{CrearArbolBinario}(r.hijos[i])$\;
        $s.derecha \gets $ nuevo nodo de árbol binario\;
        $s \gets s.derecha$\;
        $s.etiqueta \gets r.etiqueta$\;
        $i \gets i+1$\;
    }
    $s.izquierda \gets \text{CrearArbolBinario}(r.hijos[i])$\;
    $s.derecha \gets \text{CrearArbolBinario}(r.hijos[i+1])$\;
}
\Return $r'$\;

\end{algorithm}

\begin{figure}[ht!]
\centering

\begin{subfigure}{0.7\textwidth}
\begin{tikzpicture}
\begin{scope}[xshift=0cm,scale=1]
\node [style=cotreenode, fill=lightgray] (1) at (2,1) {1};
\node [style=vertex] (2) at (0.5,0) {};
\node [style=vertex] (3) at (1.5,0) {};
\node [style=vertex] (4) at (2.5,0) {};
\node [style=cotreenode] (5) at (3.5,0) {0};
\node [style=vertex] (6) at (3,-1) {};
\node [style=vertex] (7) at (4,-1) {};
\foreach \i/\j in {1/2,1/3,1/4,1/5,5/6,5/7}
  \draw [style=edge] (\i) to (\j);
\end{scope}
\begin{scope}[xshift=6.5cm,scale=1]
\node [style=cotreenode] (1) at (1,1) {1};
\node [style=vertex] (2) at (0.5,0) {};
\node [style=cotreenode] (3) at (1.5,0) {1};
\foreach \i/\j in {1/2,1/3}
  \draw [style=edge] (\i) to (\j);
\end{scope}
\end{tikzpicture}
\end{subfigure}

\par\bigskip

\begin{subfigure}{0.7\textwidth}
\begin{tikzpicture}
\begin{scope}[xshift=0cm,scale=1]
\node [style=cotreenode, fill=lightgray] (1) at (2,1) {1};
\node [style=vertex] (2) at (0.5,0) {};
\node [style=vertex] (3) at (1.5,0) {};
\node [style=vertex] (4) at (2.5,0) {};
\node [style=cotreenode] (5) at (3.5,0) {0};
\node [style=vertex] (6) at (3,-1) {};
\node [style=vertex] (7) at (4,-1) {};
\foreach \i/\j in {1/2,1/3,1/4,1/5,5/6,5/7}
  \draw [style=edge] (\i) to (\j);
\end{scope}
\begin{scope}[xshift=6.5cm,scale=1]
\node [style=cotreenode] (1) at (1,1) {1};
\node [style=vertex] (2) at (0.5,0) {};
\node [style=cotreenode] (3) at (1.5,0) {1};
\node [style=vertex] (4) at (1,-1) {};
\node [style=cotreenode] (5) at (2,-1) {1};
\foreach \i/\j in {1/2,1/3,3/4,3/5}
  \draw [style=edge] (\i) to (\j);
\end{scope}
\end{tikzpicture}
\end{subfigure}

\par\bigskip

\begin{subfigure}{0.7\textwidth}
\begin{tikzpicture}
\begin{scope}[xshift=0cm,scale=1]
\node [style=cotreenode, fill=lightgray] (1) at (2,1) {1};
\node [style=vertex] (2) at (0.5,0) {};
\node [style=vertex] (3) at (1.5,0) {};
\node [style=vertex] (4) at (2.5,0) {};
\node [style=cotreenode] (5) at (3.5,0) {0};
\node [style=vertex] (6) at (3,-1) {};
\node [style=vertex] (7) at (4,-1) {};
\foreach \i/\j in {1/2,1/3,1/4,1/5,5/6,5/7}
  \draw [style=edge] (\i) to (\j);
\end{scope}
\begin{scope}[xshift=6.5cm,scale=1]
\node [style=cotreenode] (1) at (1,1) {1};
\node [style=vertex] (2) at (0.5,0) {};
\node [style=cotreenode] (3) at (1.5,0) {1};
\node [style=vertex] (4) at (1,-1) {};
\node [style=cotreenode] (5) at (2,-1) {1};
\node [style=vertex] (6) at (1.5,-2) {};
\node [style=cotreenode] (7) at (2.5,-2) {0};
\foreach \i/\j in {1/2,1/3,3/4,3/5,5/6,5/7}
  \draw [style=edge] (\i) to (\j);
\end{scope}
\end{tikzpicture}
\end{subfigure}

\par\bigskip

\begin{subfigure}{0.7\textwidth}
\begin{tikzpicture}
\begin{scope}[xshift=0cm,scale=1]
\node [style=cotreenode] (1) at (2,1) {1};
\node [style=vertex] (2) at (0.5,0) {};
\node [style=vertex] (3) at (1.5,0) {};
\node [style=vertex] (4) at (2.5,0) {};
\node [style=cotreenode, fill=lightgray] (5) at (3.5,0) {0};
\node [style=vertex] (6) at (3,-1) {};
\node [style=vertex] (7) at (4,-1) {};
\foreach \i/\j in {1/2,1/3,1/4,1/5,5/6,5/7}
  \draw [style=edge] (\i) to (\j);
\end{scope}
\begin{scope}[xshift=6.5cm,scale=1]
\node [style=cotreenode] (1) at (1,1) {1};
\node [style=vertex] (2) at (0.5,0) {};
\node [style=cotreenode] (3) at (1.5,0) {1};
\node [style=vertex] (4) at (1,-1) {};
\node [style=cotreenode] (5) at (2,-1) {1};
\node [style=vertex] (6) at (1.5,-2) {};
\node [style=cotreenode] (7) at (2.5,-2) {0};
\node [style=vertex] (8) at (2,-3) {};
\node [style=vertex] (9) at (3,-3) {};
\foreach \i/\j in {1/2,1/3,3/4,3/5,5/6,5/7,7/8,7/9}
  \draw [style=edge] (\i) to (\j);
\end{scope}
\end{tikzpicture}
\end{subfigure}


\caption{Ejemplo de la ejecución del Algoritmo \ref{alg_coa_bin01}. A la izquierda se muestra el coárbol original, mienrtras se marca con gris el nodo que se está procesando. A la derecha aparece el coárbol binario que se va construyendo.}\label{fig_alg_coa_bin01}


\end{figure}


En términos de las particiones de las componentes conexas de la gráfica,
el algoritmo realiza lo siguiente. Si la etiqueta de $r$ es $0$, entonces
el coárbol con raíz en $r$ representa una cográfica inconexa y se elige
la partición de sus vértices en la que la primera parte es una componente
conexa y la segunda parte es el resto. Si la etiqueta de $r$ es uno, como
la cográfica representada es conexa, se elige una componente conexa del
complemento de la cográfica para la primera parte y el resto de las
componentes del complemento para la segunda.

Dado que el Algoritmo \ref{alg_coa_bin01} recorre a lo más una vez cada nodo de $r$, su tiempo de ejecución es $O(n)$ en donde $n$ es el número total de nodos del árbol con raíz $r$.
