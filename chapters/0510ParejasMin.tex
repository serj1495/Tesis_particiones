En esta subsección, presentamos el concepto de conjunto de parejas mínimas, que nos ayudará a determinar si una cográfica $G$ pertenece a una clase $(\alpha, \beta)$-$M_2$ sin tener que revisar todas las posibles $M_2$-particiones de $G$.

Sean $P_1=(\alpha_1, \beta_1)$ y $P_2=(\alpha_2, \beta_2)$ parejas de enteros tales que $0 < \alpha_1 \le \beta_1$ y $0 < \alpha_2 \le \beta_2$. Decimos que $P_1$ \emph{\textbf{domina}} a $P_2$ si $\alpha_1 \geq \alpha_2$ y $\beta_1 \geq \beta_2$. Notemos que ésta es una relación transitiva.

Sea $G$ una gráfica en $M_2$, denotamos por $S(G)$ al conjunto de todas las parejas ordenadas de enteros $(\alpha, \beta)$ tales que $G$ acepta una $M_2$-partición de tamaño $(\alpha, \beta)$.

Sea $G$ una gráfica en $M_2$, el \textbf{\emph{conjunto de parejas mínimas}} de $G$, denotado por $\mu(G)$, es el conjunto de todas las parejas $(\alpha, \beta)\in S(G)$ tales que, para cualquier otro $(\alpha', \beta')\in S(G)$, $(\alpha, \beta)$ no domina a $(\alpha', \beta')$.

\begin{lemma}\label{lema_parejas_principal}
Si $G$ es una gráfica en $M_2$ y $(\alpha, \beta)$ una pareja de enteros mayores a cero tales que $\alpha \le \beta$, entonces $G$ pertenece a la clase $(\alpha, \beta)$-$M_2$ si y sólo si existe una pareja de enteros $(\alpha', \beta')\in \mu(G)$ tal que $(\alpha, \beta)$ domina a $(\alpha', \beta')$.
\end{lemma}

\begin{proof}
Supongamos primero que $G\in (\alpha, \beta)\text{-}M_2$. Sabemos que $G$
acepta una $M_2$-partición $(A,B)$ tal que $G[A]$ es una gráfica multipartita
completa formada por $\alpha'$ conjuntos independientes con $\alpha' \le
\alpha$ y $G[B]$ es una gráfica multipartita completa formada por $\beta'$
conjuntos independientes con $\beta' \le \beta$. Como $(A,B)$ es una
$M_2$-partición de $G$, entonces $(\alpha', \beta')\in S(G)$. Supongamos sin
pérdida de generalidad (dado que la dominación es una relación transitiva)
que $(\alpha', \beta')\in \mu(G)$. Luego, como $\alpha' \le \alpha$ y $\beta'
\le \beta$, entonces $(\alpha, \beta)$ domina a $(\alpha', \beta') \in \mu(G)$.

Recíprocamente. Como $(\alpha', \beta')\in \mu(G)$, entonces $(\alpha', \beta')\in S(G)$. Luego, existe una $M_2$-partición $(A,B)$ de $G$ de tamaño $(\alpha', \beta')$. Como $(\alpha, \beta)$ domina a $(\alpha', \beta')$, entonces $G[A]$ es una gráfica multipartita completa formada por a lo más $\alpha$ conjuntos independientes y $G[B]$ es una gráfica multipartita completa formada por a lo más $\beta$ conjuntos independientes. Así, $G\in (\alpha, \beta)\text{-}M_2$.
\end{proof}
