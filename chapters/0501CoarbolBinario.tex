Tomando como base el concepto de coárbol, podemos imaginar otra estructura de tipo árbol para la representación de las cográficas en la que cada nodo tenga a lo más un número $k$ de hijos. Esta limitante resulta útil para formular algoritmos rápidos en cográficas. El menor valor que puede tomar $k$ es de 2. Este valor será utilizado a lo largo de este capítulo para representar a las cográficas a través de árboles binarios.


\begin{definition}{}

    Sean $G=(V,E)$ una cográfica, $C = c_1, c_2, \dots, c_n$ el conjunto de las componentes conexas de $G$, $D = {d_1, d_2, \dots, d_m}$ el conjunto de las componentes conexas de $\overline{G}$, $(C_1, C_2)$ una partición en dos partes de $C$ y $(D_1, D_2)$ una partición en dos partes de $D$, decimos que el árbol binario arraigado etiquetado, $(T,r)$, es un \textbf{\emph{coárbol binario}} de $G$ si se puede construir de la siguiente manera: Si $G$ consta de un sólo vértice, entonces $T$ sólo contiene a $r$. De lo contrario, si $G$ es conexa, entonces $r$ tiene la etiqueta $1$, uno de los hijos de $r$ es el coárbol binario de $G-D_1$ y el otro es el coárbol binario de $G-D_2$. Y finalmente, si $G$ es inconexa, entonces $r$ tiene la etiqueta $0$, uno de sus hijos es el coárbol binario de $G-C_1$ y el otro el coárbol binario de $G-C_2$.


\end{definition}

Claramente, una cográfica puede ser representada por más de un coárbol binario diferente como se muestra en la Figura \ref{fig_coar_bin01}. Sin embargo, la propiedad de que dos vértices son adyacentes si y sólo si su ancestro común más profundo tiene la etiqueta 1 se mantiene.

\begin{figure}[ht!]
\begin{center}
\begin{tikzpicture}

\begin{scope}[xshift=0cm,scale=1]

\node [vertex] (1) at (0,0) {};
\node [vertex] (2) at (1,0) {};
\node [vertex] (3) at (0,1) {};
\node [vertex] (4) at (1,1) {};
\foreach \i/\j in {1/2,1/3,1/4,2/3,2/4,3/4}
  \draw [edge] (\i) to (\j);
\node [below of=1,xshift=.5cm] {\parbox{0.3\linewidth}{\subcaption{}}};

\end{scope}

\begin{scope}[xshift=3.5cm,scale=1]

\node [cotreenode] (1) at (1,1) {1};
\node [cotreenode] (2) at (0,0) {1};
\node [cotreenode] (3) at (2,0) {1};
\node [vertex] (4) at (-0.5,-1) {};
\node [vertex] (5) at (0.5,-1) {};
\node [vertex] (6) at (1.5,-1) {};
\node [vertex] (7) at (2.5,-1) {};
\foreach \i/\j in {1/2,1/3,2/4,2/5,2/4,3/6,3/7}
  \draw [edge] (\i) to (\j);
\node [below of=5,xshift=.5cm] {\parbox{0.3\linewidth}{\subcaption{}}};

\end{scope}

\begin{scope}[xshift=7.5cm,scale=1]

\node [cotreenode] (1) at (1,1) {1};
\node [vertex] (2) at (0,0) {};
\node [cotreenode] (3) at (2,0) {1};
\node [vertex] (6) at (1,-1) {};
\node [cotreenode] (7) at (3,-1) {1};
\node [vertex] (8) at (2,-2) {};
\node [vertex] (9) at (4,-2) {};

\foreach \i/\j in {1/2,1/3,3/6,3/7,7/8,7/9}
  \draw [edge] (\i) to (\j);
\node [below of=8] {\parbox{0.3\linewidth}{\subcaption{}}};

\end{scope}


\end{tikzpicture}
\end{center}
\setlength{\abovecaptionskip}{-10pt}
\caption{(b) y (c) son dos coárboles binarios diferentes que representan a la cográfica (a).}\label{fig_coar_bin01}
\end{figure}
