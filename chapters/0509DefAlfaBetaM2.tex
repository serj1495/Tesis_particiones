Sean $\alpha$ y $\beta$ enteros tales que $0 < \alpha \le \beta$, una gr\'afica $G$ est\'a en la \emph{\textbf{clase $(\alpha, \beta)-M_2$}}  si y sólo si su conjunto de vértices acepta una partición $(A,B)$ tal que $G[A]$ es una gráfica multipartita completa formada por a lo más $\alpha$ conjuntos independientes y $G[B]$ es una gráfica multipartita completa formada por a lo más $\beta$ conjuntos independientes. Decimos que $(A,B)$ es una $M_2$-partición de $G$ de tamaño $(\alpha, \beta)$.

Notemos que la clase $M_2$ es la clase $(\infty, \infty)$-$M_2$ y que la clase $(1,1)$-$M_2$ es la clase de todas las cográficas bipartitas. En la demostración del Teorema \ref{teo_obsts_m2} hablamos de las cográficas que aceptan una partición en un conjunto independiente y una gráfica multipartita completa. Es decir, las gráficas de la clase $(1,\infty)$-$M_2$. A partir de esta demostración, podemos encontrar las obstrucciones mínimas de la clase; éstas nos serán de ayuda para entender el comportamiento de las clases $(1,\beta)$-$M_2$.

\begin{lemma}
\label{lema_1infM2}
Si $G$ es una cográfica, entonces $G$ est\'a en la clase $(1,\infty)$-$M_2$
si y sólo si $G$ es libre de las gráficas de la Figura \ref{obsts_1infM2}.

\end{lemma}

\begin{figure}[ht!]
\begin{center}
\begin{tikzpicture}

\begin{scope}[xshift=0cm,scale=1]

\node [style=vertex] (1) at (0,0) {};
\node [style=vertex] (2) at (1,0) {};
\node [style=vertex] (3) at (0,0.5) {};
\node [style=vertex] (4) at (1,0.5) {};
\node [style=vertex] (5) at (0.5,1.25) {};
\foreach \i/\j in {1/2,3/4,3/5,4/5}
  \draw [style=edge] (\i) to (\j);
\node [below of=1,xshift=.5cm]
{\parbox{0.3\linewidth}{\subcaption*{$H'$}}};

\end{scope}

\begin{scope}[xshift=3cm,scale=1]

\node [style=vertex] (1) at (0,0) {};
\node [style=vertex] (2) at (0.5,0.5) {};
\node [style=vertex] (3) at (1.5,0.5) {};
\node [style=vertex] (4) at (0.5,1.5) {};
\node [style=vertex] (5) at (1.5,1.5) {};
\node [style=vertex] (6) at (0,2) {};

\foreach \i/\j in {1/2,1/3,1/6,2/3,2/4,2/5,3/4,3/5,4/5,4/6,5/6}
  \draw [style=edge] (\i) to (\j);
\node [below of=1,xshift=0.75cm] {\parbox{0.3\linewidth}{\subcaption*{$J'$}}};

\end{scope}
\end{tikzpicture}
\end{center}
\setlength{\abovecaptionskip}{-15pt}
\caption{Obstrucciones mínimas para la clase $(1,\infty)$-$M_2$.}
\label{obsts_1infM2}
\end{figure}

\begin{proof}

Notemos que $H'=K_2 + K_3$ y $J'=\overline{P_3} \oplus \overline{P_3}$.

Supongamos primero que $G \in (1,\infty)$-$M_2$. Veamos que $H'\notin (1,\infty)$-$M_2$. Procedamos por contradicción. Supongamos que $H' \in (1,\infty)$-$M_2$. Luego, $H$ (Figura \ref{obsts_M2}) también está en $(1,\infty)$-$M_2$ y por lo tanto $H\in M_2$, lo que es una contradicción. Así,  $H'\notin (1,\infty)$-$M_2$. Análogamente para $J'$. Como ni $H'$ ni $J'$ están en $(1,\infty)$-$M_2$ y toda subgráfica inducida de $G$ sí lo está, $G$ es libre de $H'$ y $J'$.

Recíprocamente, si $G$ es libre de $H'$ y $J'$, mostremos que $G\in (1,\infty)$-$M_2$. Sea $r$ la raíz del coárbol de $G$. Consideremos los siguientes casos que son exhaustivos.

Consideremos el caso en el que $G$ es inconexa. Si $G$ es una gráfica vacía, es claro que $G \in (1,\infty)$-$M_2$. Si $G$ tiene exactamente una componente no trivial y al menos una trivial. Entonces, del Caso 2 de la segunda parte de la demostración del Teorema \ref{teo_obsts_m2}, se sigue que $G \in (1,\infty)$-$M_2$. Si $G$ tiene al menos dos componentes no triviales. Como $G$ es libre de $H'$, entonces cada componente de $G$ es libre de $K_3$. Es decir que cada componente de $G$ es bipartita. Luego, $G$ es bipartita. Así, $G \in (1,1)$-$M_2$, por lo que $G \in (1,\infty)$-$M_2$. 

Por otra parte, consideremos el caso en el que $G$ es conexa.  Si $G$ es
isomorfa a $K_1$, es claro que $G \in (1,\infty)$-$M_2$. En el caso contrario,
como $G$ es libre de $J'$, todos los hijos de $r$ representan gráficas
multipartitas completas excepto a lo m\'as uno. Si todos sus hijos representan
gráficas multipartitas completas, entonces $G$ es la unión completa de varias
gráficas multipartitas completas, por lo que $G$ es multipartita completa. Si
$r$ tiene un hijo $s$ que no representa una gráfica multipartita completa,
como $G[s]$ es inconexa y libre de $H'$ y $J'$, entonces, procediendo
inductivamente, $G[s]$ acepta una partición $(A,B)$ en un conjunto
independiente $G[A]$ y una gráfica multipartita completa $G[B]$. Notemos que
$G' = G-V(G[s])$ es una gráfica multipartita completa dado que todos los hijos
de $r$ diferentes de $s$ representan gráficas multipartitas completas. Como
todos los vértices de $G'$ son adyacentes a los vértices de $G[s]$, entonces
$G' \oplus G[B]=G-A$ es una gráfica multipartita completa. Luego, $(A,V(G)-A)$
es una partición de $G$ en un conjunto independiente y una gráfica multipartita
completa. Así, $G \in (1,\infty)$-$M_2$.

\end{proof}
