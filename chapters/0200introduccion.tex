Una gráfica $G$ es una pareja ordenada de conjuntos ajenos $(V,E)$ tal que $E$ es un conjunto de parejas no ordenadas de elementos de $V$. Llamamos a $V$ el conjunto de vértices de $G$ y a $E$ el conjunto de aristas de $G$. Decimos que dos vértices de $G$ son adyacentes si la pareja formada por estos está en $E$. Sea $V'$ un subconjunto de $V$, la subgráfica de $G$ inducida por $V'$ es la gráfica que tiene como conjunto de vértices a $V'$ y como conjunto de aristas al subconjunto de $E$ de las parejas no ordenadas de elementos de $V'$. Una gráfica multipartita completa es una gráfica cuyo conjunto de vértices acepta una partición $(A_1, A_2,\dots, A_n)$ tal que, para cualesquiera enteros diferentes $1\le i,j \le n$, los vértices en $A_i$ no son adyacentes entre sí pero sí son adyacentes a cada uno de los vértices en $A_j$. Un clan es un conjunto de vértices de $G$ tal que todos sus elementos son adyacentes entre sí.

Una clase hereditaria de gráficas $C$ es un conjunto de gráficas tal que, si $C$ contiene a $G=(V,E)$, entonces, para cualquier subconjunto $V'$ de $V$, $C$ contiene a la subgráfica de $G$ inducida por $V'$. Toda clase hereditaria de gráficas $C$ puede ser caracterizada a través de un conjunto de gráficas $S$ tal que toda gráfica en $C$ no tiene a ninguna gráfica de $S$ como subgráfica inducida. Llamamos a $S$ el conjunto de obstrucciones mínimas de $C$. Las cográficas son la clase hereditaria de gráficas definida recursivamente de la siguiente manera:

\begin{itemize}
    \item Una gráfica con un solo vértice es una cográfica.
    \item Si $G=(V_G,E_G)$ y $H=(V_H,E_H)$ son cográficas sin vértices en común, entonces $(V_G\cup V_H, E_G \cup E_H)$ es una cográfica.
    \item Si $G=(V,E)$ es una cográfica y $W$ el conjunto de todas las parejas no ordenadas de elementos de $V$, entonces $(V_G, W-E)$ es una cográfica.
\end{itemize}

En 1990, Damascke, P. \cite{Damaschke} mostró que cualquier clase hereditaria de cográficas puede ser caracterizada por un conjunto finito de obstrucciones mínimas.

En esta tesis estudiamos a las clases de cográficas que se definen de la siguiente manera. Sea $i$ un entero mayor o igual a uno, la clase $M_i$ es la clase de las cográficas cuyo conjunto de vértices acepta una partición en $i$ partes tal que cada parte induce una gráfica multipartita completa. Nos referimos a estas clases en conjunto como las clases $M_i$. Este problema no ha sido estudiado con anterioridad. Sin embargo, utilizamos la investigación realizada acerca de las cográficas polares como guía para el estudio de la clase $M_2$, que sirve como base de nuestra investigación.

Los resultados principales de nuestra tesis son los siguientes. Presentamos un algoritmo que, dada una clase de cográficas fija, representada a través de su conjunto de obstrucciones mínimas, es capaz de determinar si una cográfica pertenece a dicha clase en tiempo lineal con respecto al orden de la gráfica. Caracterizamos a la clase $M_2$ a través de su conjunto de obstrucciones mínimas. Presentamos un algoritmo de tiempo lineal para el reconocimiento de los elementos de la clase $M_2$. Presentamos un algoritmo certificador de tiempo lineal para la clase $M_2$. Tomando como base el estudio de las cográficas $(s,k)$-polares, estudiamos un conjunto de subclases de la clase $M_2$ a las que llamamos clases $(\alpha,\beta)$-$M_2$. Como resultado principal de este estudio, presentamos un algoritmo que, dados tres enteros $\alpha$, $\beta$ y $n$ mayores o iguales a uno, genera las obstrucciones mínimas de la clase $(\alpha,\beta)$-$M_2$ con hasta $n$ vértices. Presentamos el conjunto de obstrucciones mínimas de la clase $M_3$. Finalmente, presentamos dos familias de obstrucciones mínimas para cualquier clase $M_i$.

La tesis está organizada de la siguiente manera. En el Capítulo 2 se presentan de manera detallada y formal las definiciones necesarias para entender el problema que resolvemos con nuestro trabajo de tesis. Estas definiciones proporcionan una introducción a la Teoría de Gráficas y, dentro de esta área, a las cográficas, que son el tipo de gráficas con las que trabajamos en todo el documento. En el Capítulo 3 presentamos los resultados en la investigación de las cográficas polares que dan forma a nuestro estudio de la clase $M_2$. En el Capítulo 4, presentamos los resultados de nuestra investigación que comprenden un conjunto de lemas, teoremas, algoritmos y listas de obstrucciones mínimas para varias clases hereditarias de cográficas. La correctitud de cada uno de nuestros resultados se demuestra de manera formal. Finalmente, en el Capítulo 5, presentamos las conclusiones de nuestra investigación a la vez que proponemos varias formas de continuar con la investigación que iniciamos con este trabajo de tesis.
