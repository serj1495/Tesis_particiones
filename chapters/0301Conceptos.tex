%A continuación se presentan los conceptos básicos de la Teoría de Gráficas. Todas las gráficas que se consideran en este trabajo son gráficas simples, es decir que no contienen aristas de un vértice a sí mismo o varias aristas diferentes que conecten al mismo par de vértices. En términos generales se siguen las definiciones de \cite{Bondy}. Sin embargo, algunas podrían diferir.

%Una \textit{\textbf{gráfica}} $G$ es una tercia ordenada $(V(G), E(G), \psi_G)$ formada por un conjunto no vacío $V(G)$ de \textit{vértices}, un conjunto $E(G)$, ajeno a $V(G)$, de \textit{aristas} y una \textit{función de incidencia} $\psi_G$ que asocia a cada arista de $G$ una pareja no ordenada de vértices (no necesariamente distintos)  de $G$. Si $e$ es una arista y tanto $u$ como $v$ son vértices tales que $\psi_G(e) = uv$, entonces decimos que $e$ une a $u$ y a $v$; llamamos a los vértices $u$ y $v$ los \textit{extremos} de $e$. Decimos que los extremos de una arista son  \textit{incidentes} con ésta y viceversa. Dos vértices que son incidentes con un mismo arista son \textit{adyacentes}.  Una arista cuyos extremos son idénticos es llamada un \textit{lazo}.

%Una gráfica es \textit{finita} si tanto su conjunto de vértices como su conjunto de aristas son finitos.

%Una gráfica es \textit{simple si no tiene lazos}

A continuación proporcionamos conceptos básicos del área de Teoría de Gráficas. Dado que nuestra investigación pertenece a dicha área, estos conceptos serán utilizados a lo largo de todo el documento. De manera general, las definiciones que se presentan fueron tomadas de Bollobás \cite{Bollobas}. Algunas de éstas pueden diferir de las proporcionadas en Bollobás \cite{Bollobas} de forma no sustancial.

\subsection{Conceptos básicos y notación}

Una \textbf{\emph{gráfica}} $G$ es una pareja ordenada de conjuntos ajenos $(V,E)$ tal que $E$ es subconjunto del conjunto de parejas no ordenadas de elementos de $V$. Decimos que $V$ es el conjunto de \textbf{\emph{vértices}} de $G$ y que $E$ es el conjunto de \textbf{\emph{aristas}} de $G$.

Si $G$ es una gráfica, denotamos su conjunto de vértices como $V(G)$ y su conjunto de aristas como $E(G)$. Sin embargo, cuando $G$ es la única gráfica de la que estamos hablando, podemos referirnos a su conjunto de vértices simplemente como $V$ y a su conjunto de aristas como $E$. Si $x$ es un vértice de $G$, podemos escribir $x\in G$ en lugar de $x \in V(G)$.

Dada una arista $\{x,y\}$ de una gráfica, decimos que $\{x,y\}$ \textbf{\emph{une}} a los vértices $x$ y $y$ y la denotamos como $xy$. Los vértices $x$ y $y$ son los \textbf{\emph{extremos}} de dicha arista. Si $xy \in E(G)$, entonces $x$ y $y$ son \textbf{\emph{adyacentes}} en $G$. Si el contexto es claro, simplemente decimos que $x$ y $y$ son adyacentes. Notemos que la arista $xy$ es exactamente la misma arista que $yx$.

Decimos que una gráfica $G$ es una \textbf{\textit{gráfica finita}} si tanto $V$ como $E$ son finitos. A lo largo de este documento, trabajaremos exclusivamente con gráficas finitas. Si $G$ es una gráfica finita, decimos que el \textbf{\emph{orden}} de $G$, denotado como $|V(G)|$, es el número de vértices de $G$. El \textbf{\emph{tamaño}} de $G$ es el número de aristas de $G$, denotado como $|E(G)|$.

Las gráficas suelen ser representadas por medio de dibujos. En esta representación los vértices aparecen como pequeños círculos y las aristas como líneas que conectan a los vértices que son sus extremos. Podemos ver un ejemplo de esto en la Figura \ref{fig_ejemplo_graph}. Las etiquetas que aparecen a lado de cada vértice indican el nombre con el que se denota dicho vértice. Usualmente estas etiquetas no son incluidas.

\begin{figure}[ht!]
\begin{center}
\begin{tikzpicture}
\begin{scope}[xshift=0cm,scale=1]
\node [vertex] (1) at (0,0) {};
\node [vertex] (2) at (1,0) {};
\node [vertex] (3) at (0,1) {};
\node [vertex] (4) at (1,1) {};
\node [vertex] (5) at (-1,-1) {};
\node [vertex] (6) at (2,-1) {};
\node [vertex] (7) at (-1,2) {};
\node [vertex] (8) at (2,2) {};
\node at (-0.5,0) {$e$};
\node at (1.5,0) {$f$};
\node at (-0.5,1) {$c$};
\node at (1.5,1) {$d$};
\node at (-1.5,-1) {$g$};
\node at (2.5,-1) {$h$};
\node at (-1.5,2) {$a$};
\node at (2.5,2) {$b$};
\foreach \i/\j in {1/4,2/3,5/1,6/2,7/3,8/4,5/6,5/7,8/6,8/7}
  \draw [edge] (\i) to (\j);
\end{scope}
\end{tikzpicture}
\end{center}
\caption{Ejemplo de la representación de una gráfica.}\label{fig_ejemplo_graph}
\end{figure}

Dadas dos gráficas $G=(V,E)$ y $G'=(V',E')$, decimos que $G'$ es \textbf{\emph{subgráfica}} de $G$ si $V' \subseteq V$ y $E' \subseteq E$. Denotamos esto como $G' \subseteq G$. La subgráfica de $G$ \emph{\textbf{inducida}} por $V'\in V$, a la que denotamos como $G[V']$, es la gráfica cuyo conjunto de vértices es $V'$ y cuyas aristas unen a dos vértices si y sólo si estos son adyacentes en $G$.

En la Figura \ref{fig_ejemplo_subgraph} se muestran dos gráficas. Ambas gráficas son subgráficas de la gráfica de la Figura \ref{fig_ejemplo_graph}, a la que llamaremos $G$ para los fines de este ejemplo. Dado que la gráfica (a) no incluye las aristas $ab$ y $ac$, ésta no es una subgráfica inducida de $G$. Por otra parte, la gráfica (b) es la subgráfica de $G$ inducida por el conjunto de vértices $\{a,b,c,d,g,h\}$.


\begin{figure}[!htbp]
\centering
\begin{tikzpicture}

\begin{scope}[xshift=0cm,scale=1]
\node [vertex] (3) at (0,1) {};
\node [vertex] (4) at (1,1) {};
\node [vertex] (5) at (-1,-1) {};
\node [vertex] (6) at (2,-1) {};
\node [vertex] (7) at (-1,2) {};
\node [vertex] (8) at (2,2) {};
\node at (-0.5,1) {$c$};
\node at (1.5,1) {$d$};
\node at (-1.5,-1) {$g$};
\node at (2.5,-1) {$h$};
\node at (-1.5,2) {$a$};
\node at (2.5,2) {$b$};
\foreach \i/\j in {8/4,5/6,5/7,8/6}
  \draw [edge] (\i) to (\j);
\node [below of=5,xshift=1.5cm] {\parbox{0.3\linewidth}{\subcaption{Subgráfica}}};
\end{scope}

\begin{scope}[xshift=5cm,scale=1]
\node [vertex] (3) at (0,1) {};
\node [vertex] (4) at (1,1) {};
\node [vertex] (5) at (-1,-1) {};
\node [vertex] (6) at (2,-1) {};
\node [vertex] (7) at (-1,2) {};
\node [vertex] (8) at (2,2) {};
\node at (-0.5,1) {$c$};
\node at (1.5,1) {$d$};
\node at (-1.5,-1) {$g$};
\node at (2.5,-1) {$h$};
\node at (-1.5,2) {$a$};
\node at (2.5,2) {$b$};
\foreach \i/\j in {7/3,8/4,5/6,5/7,8/6,8/7}
  \draw [edge] (\i) to (\j);
\node [below of=5,xshift=1.5cm] {\parbox{0.3\linewidth}{\subcaption{Subgráfica inducida}}};
\end{scope}

\end{tikzpicture}
\caption{Ejemplo de una subgráfica y de una subgráfica inducida de la gráfica de la Figura \ref{fig_ejemplo_graph}.}
\label{fig_ejemplo_subgraph}
\end{figure}

Decimos que dos gráficas $G=(V,E)$ y $G'=(V',E')$ son \textbf{\emph{isomorfas}} si existe una biyección $\phi:V\rightarrow V'$ tal que $xy\in E$ si y sólo si $\phi(x)\phi(y) \in E' $. A lo largo de este documento no hacemos distinción entre gráficas isomorfas. Es decir que, si dos gráficas $G$ y $G'$ son isomorfas, las consideramos como la misma gráfica.

\subsection{Algunas gráficas distinguidas}

A continuación se presentan algunas familias de gráficas a cuyos elementos nos referimos por medio de un nombre específico. A lo largo del documento usaremos dicho nombre para referirnos a cualquier copia isomorfa de estas gráficas.

El tamaño de una gráfica de orden $n$ es al menos $0$ y a lo más $\binom{n}{2}$. Una gráfica de orden $n$ y tamaño $\binom{n}{2}$ es llamada \textbf{\emph{gráfica completa}} de orden $n$ y se denota como $K_n$. En $K_n$ cualesquiera dos vértices son adyacentes. Decimos que la gráfica $K_1$ es una gráfica \textbf{\emph{trivial}}.

Un vértice que no es adyacente a ningún otro vértice es llamado un \textbf{\emph{vértice aislado}}. Un conjunto de vértices es un \textbf{\emph{conjunto independiente}} si no contiene elementos que sean adyacentes.

Una \textbf{\emph{trayectoria}} es una gráfica $P$ tal que $V(P) = {x_0,x_1,\dots,x_l}$ y $E(P) = {x_0x_1,x_1x_2,\dots,x_{l-1}x_l}$.
La trayectoria $P$ se denota usualmente como $x_0x_1\dots x_l$. Decimos que $P$ es una trayectoria de $x_0$ a $x_l$. Denotamos
como $P_l$ a una trayectoria arbitraria de orden $l$.

Un \textbf{\emph{camino}} $W$ en una gráfica $G$ es una sucesi\'on alternante de vértices y aristas de $G$, digamos $x_0, e_1, x_1, e_2, \dots, e_l, x_l$ en donde $e_i=x_{i-1}x_i$, $0<i\leq l$. De acuerdo con la terminología anterior, $W$ es un $(x_0 x_l)$-camino y es denotado como $x_0x_1\dots x_l$; la \textbf{\emph{longitud}} de $W$ es $l$. Este camino es llamado un  \textbf{\emph{paseo}} si todas sus aristas son distintas. Notemos que una trayectoria es un camino en el que todos sus vértices son distintos. Un paseo cuyos vértices de inicio y fin coinciden es llamado un \textbf{\emph{circuito}}. Para ser más precisos, un circuito es un paseo cerrado sin vértices de inicio y fin distinguidos. Si un camino $W = x_0x_1\dots x_l$ es tal que $l\geq 3$, $x_0=x_l$ y los vértices $x_i$, $0<i<l$ son distintos los unos de los otros y de $x_0$, decimos que $W$ es un \textbf{\emph{ciclo}}.

Con frecuencia usamos la notación $P_l$ para referirnos a una trayectoria arbitraria de longitud $l$ y la notación $C_l$ para referirnos a un ciclo de longitud $l$. En la Figura \ref{fig_tipos_graficas} podemos ver ejemplos de una gráfica completa, un camino y un ciclo.

\begin{figure}[!htbp]
\centering

\begin{tikzpicture}
\begin{scope}[xshift=0cm,scale=1]
\node [vertex] (1) at (0,0) {};
\node [vertex] (2) at (2,0) {};
\node [vertex] (3) at (1,0.75) {};
\node [vertex] (4) at (1,2) {};
\foreach \i/\j in {1/2,1/3,1/4,2/3,2/4,3/4}
  \draw [edge] (\i) to (\j);
\end{scope}

\begin{scope}[xshift=3cm,scale=1]
\node [vertex] (1) at (0,0.5) {};
\node [vertex] (2) at (1,1.5) {};
\node [vertex] (3) at (2,0.5) {};
\node [vertex] (4) at (3,1.5) {};
\foreach \i/\j in {1/2,2/3,3/4}
  \draw [edge] (\i) to (\j);
\end{scope}

\begin{scope}[xshift=7cm,scale=1]
\node [vertex] (1) at (0,0) {};
\node [vertex] (2) at (2,0) {};
\node [vertex] (3) at (2,2) {};
\node [vertex] (4) at (0,2) {};
\foreach \i/\j in {1/2,2/3,3/4,4/1}
  \draw [edge] (\i) to (\j);
\end{scope}
\end{tikzpicture}
\caption{Las gráficas $K_4$, $P_4$ y $C_4$.}
\label{fig_tipos_graficas}
\end{figure}

Una gráfica es \textbf{\emph{conexa}} si para cada par de vértices distintos existe un camino de entre ellos. Una gráfica que no es conexa es llamada \textbf{\emph{inconexa}}. Dada una gráfica $G$, una subgráfica de $G$ conexa y m\'axima por contenci\'on (es decir que se vuelve inconexa si se le agrega cualquier otro vértice) es llamada una \textbf{\emph{componente conexa}} de $G$.

Una gráfica $G$ es \textbf{\emph{bipartita}} con clases $V_1$ y $V_2$ si $V(G) = V_1 \cup V_2$, $V_1 \cap V_2 = \emptyset$ y cada arista en $E(G)$ une a un vértice de $V_1$ con un vértice de $V_2$. Decimos que $G$ tiene una \textbf{\emph{bipartición}} $(V_1,V_2)$. De manera similar, $G$ es \textbf{\emph{r-partita}} (\textbf{\emph{multipartita}}) con clases $V_1, V_2, \dots, V_r$ si $V(G) = \bigcup_{i=1}^r V_i$, $V_i \cap V_j = \emptyset$ para todos $1\leq i < j \leq r$ y ninguna arista une vértices de la misma clase. Una gráfica es \textbf{\emph{r-partita completa}} (o bien, \textbf{\emph{multipartita completa}}) si es multipartita y si todo par de vértices de clases distintas son adyacentes.

\subsection{Árboles}

A continuación se exponen los conceptos de árbol y de árbol arraigado (también conocido como árbol enraizado), dos tipos de gráficas que serán ampliamente utilizadas en el documento, en particular, en los algoritmos que presentamos.

Una gráfica que no contiene ciclo alguno es un \emph{\textbf{bosque}}; un \emph{\textbf{árbol}} es un bosque conexo. Notemos que un bosque es un conjunto de árboles sin aristas entre ellos.

Si bien los textos sobre algoritmos mantienen una noción uniforme de lo que es un árbol arraigado, es difícil encontrar una definición formal. Por lo que a continuación se presenta una definición y notación propias.

Un \textbf{\emph{árbol arraigado}} es un árbol con un vértice distinguido al que llamamos \textbf{\emph{raíz}}. Un árbol arraigado se puede denotar como una pareja ordenada $(T,r)$, en donde $T$ es un árbol y $r \in V(T)$. Nos referiremos al árbol arraigado $(T,r)$ simplemente como $T$, indicando que $r$ es su raíz únicamente cuando sea necesario.

\begin{figure}[!htbp]
\centering
\begin{subfigure}{\textwidth}
\centering
\begin{tikzpicture}
\begin{scope}[xshift=8.5cm,scale=1]
\node [vertex] (1) at (1.5,4) {};
\node [vertex] (2) at (0.5,3) {};
\node [vertex] (3) at (1.5,3) {};
\node [vertex] (4) at (2.5,3) {};
\node [vertex] (5) at (0,2) {};
\node [vertex] (6) at (1,2) {};
\node [vertex] (7) at (1.75,2) {};
\node [vertex] (8) at (3.25,2) {};
\node [vertex] (12) at (2.75,1) {};
\node [vertex] (14) at (3.75,1) {};
\node [vertex] (15) at (3.5,0) {};
\node [vertex] (16) at (4,0) {};
\foreach \i/\j in {1/2,1/3,1/4,2/5,2/6,4/7,4/8,8/12,8/14,14/15,14/16}
  \draw [edge] (\i) to (\j);
\node [right of=1] {\parbox{0.1\linewidth}{$raíz$}};
\end{scope}
\end{tikzpicture}
\end{subfigure}
\caption{Ejemplo de un árbol arraigado. El vértice distinguido como raíz aparece como tal.}
\label{fig_ejemplo_arbol}
\end{figure}

En la Figura \ref{fig_ejemplo_arbol} podemos observar la representación de un árbol arraigado. A partir de ahora supondremos que el nodo que se dibuja con mayor altura en la representación es la raíz, sin la necesidad de incluir una etiqueta para identificarlo.
%Como podemos observar en la Figura \ref{fig_ejemplo_arbol}, un árbol arraigado se representa usualmente dibujando su nodo raíz en la parte superior. A partir de ahora, .

Sea $T$ un árbol arraigado con raíz $r$, y sean $u$ y $v$ nodos de $T$. Denotamos a la única trayectoria desde $v$ hasta $r$ como $P_T(v)$, y establecemos un orden parcial en $V(T)$ con la relación $\geq$ tal que $u \geq v$ si y sólo si $u$ es un vértice en $P_T(v)$. Si $u$ y $v$ son vértices de T distintos tales que $u \geq v$, decimos que $u$ es \textbf{\emph{ancestro}} de $v$, y que $v$ es \textbf{\emph{descendiente}} de $u$. El \textbf{\emph{ancestro común más profundo}} de dos vértices de $T$ es el vértice que está tanto en $P_T(u)$ como en $P_T(v)$ y que se encuentra a mayor distancia de $r$. Si $v$ es un vértice de $T$, decimos que $v$ es una \textbf{\emph{hoja}} si no tiene descendientes, y decimos que es un \textbf{\emph{nodo interno}} en el caso contrario. Si $x$ y $y$ son nodos de un árbol tales que $x$ es ancestro de $y$ y estos son adyacentes, decimos que $x$ es el padre de $y$ y que $y$ es hijo de $x$. Podemos ver estos conceptos ilustrados en la Figura \ref{fig_ejemplo_arbol_2}.

\begin{figure}[!htbp]
\centering
\begin{subfigure}{\textwidth}
\centering
\begin{tikzpicture}
\begin{scope}[xshift=8.5cm,scale=1]
\node [vertex] (1) at (1.5,4) {};
\node [vertex] (2) at (0.5,3) {};
\node [vertex] (3) at (1.5,3) {};
\node [vertex] (4) at (2.5,3) {};
\node [vertex] (5) at (0,2) {};
\node [vertex] (6) at (1,2) {};
\node [vertex] (7) at (1.75,2) {};
\node [vertex] (8) at (3.25,2) {};
\node [vertex] (12) at (2.75,1) {};
\node [vertex] (14) at (3.75,1) {};
\node [vertex] (15) at (3.5,0) {};
\node [vertex] (16) at (4,0) {};
\foreach \i/\j in {1/2,1/3,1/4,2/5,2/6,4/7,4/8,8/12,8/14,14/15,14/16}
  \draw [edge] (\i) to (\j);
\node [right of=1] {\parbox{0.1\linewidth}{$a$}};
\node [right of=4] {\parbox{0.1\linewidth}{$b$}};
\node [left of=5, xshift=1cm] {\parbox{0.1\linewidth}{$c$}};
\node [left of=15, xshift=0.5cm] {$d$};
\end{scope}
\end{tikzpicture}
\end{subfigure}
\caption{El nodo $a$ es ancestro de los nodos $b,c $ y $d$. Los nodos $b,c $ y $d$ son descendientes de $a$. El ancestro común más profundo de los nodos $c$ y $d$ es el nodo $a$. El nodo $b$ es hijo del nodo $a$ y éste a su vez es padre del nodo $b$. Los nodos $c$ y $d$ son hojas del árbol. Los nodos $a$ y $b$ son nodos internos.}
\label{fig_ejemplo_arbol_2}
\end{figure}


\subsection{Operaciones en gráficas}

A continuación presentamos algunas operaciones que se pueden aplicar a las gráficas. Éstas son útiles para construir gráficas a partir de otras gráficas. También nos sirven para describir algunas gráficas y estructuras que se utilizan en el documento.

Dada una gráfica $G$, el \textbf{\emph{complemento}} de $G$, denotado como $\overline{G}$, es la gráfica tal que $V(\overline{G}) = G$ y $E(\overline{G})$ es el conjunto de todas las parejas no ordenadas de elementos de $V$ menos las aristas de $E(G)$; así, dos vértices son adyacentes en $\overline{G}$ si y sólo si no son adyacentes en $G$.

 Dada una gráfica $G$ y uno de sus vértices $x$, denotamos como $G - x $ a la gráfica $G[V-\{x\}]$. Es decir, la subgráfica de $G$ inducida por el conjunto de vértices $V-\{x\}$.

Decimos que la gráfica $G + H = (V(G)\cup V(H), E(G)\cup E(H))$ es la \textbf{\emph{unión ajena}} de $G$ y $H$. Por otra parte, obtenemos la \textbf{\emph{unión completa}} de $G$ y $H$, denotada como $G \oplus H$, a partir de $G + H$ al agregar todas las aristas entre los vértices de $G$ y los vértices de $H$.

\subsection{Clases de gráficas hereditarias}

A continuación se presenta el concepto de clase hereditaria de gráficas y otras definiciones relacionadas. Todas las clases de gráficas que estudiamos en este trabajo son clases hereditarias.  Las definiciones fueron tomadas de Kitaev \cite{Kitaev}.

Una clase $X$ de gráficas que contiene una gráfica $G$ si y sólo si contiene también a todas las subgráicas inducidas de $G$ es llamada una \textbf{\emph{clase hereditaria}} de gráficas.

Dado un conjunto de gráficas $M$ (finito o infinito), denotamos como $Free(M)$ a la clase de gráficas que contiene todas las gráficas que no tienen a ninguna gráfica de $M$ como subgráfica inducida. Decimos que las gráficas en $M$ son \textbf{\emph{subgráficas inducidas prohibidas}} para la clase $Free(M)$, y que las gráficas en $Free(M)$ son libres de $M$. Diremos también que las gráficas en $Free(M)$ son libres de $G$ para cualquier gráfica $G$ en el conjunto $M$. Por convención, llamamos a los elementos de $M$ \textbf{\emph{obstrucciones}} de $Free(M)$.

Una clase $X$ de gráficas es hereditaria si y sólo si existe un conjunto $M$ tal que $X = Free(M)$.

Una gráfica es una \textbf{\emph{subgráfica inducida  prohibida mínima}}, también llamada una \textbf{\emph{obstrucción mínima}} de una clase hereditaria $X$ si $G$ no pertenece a $X$ pero toda subgráfica inducida de $G$ (con excepción de $G$ misma) pertenece a $X$. Denotamos como $Forb(X)$ al conjunto de todas las obstrucciones mínimas de la clase hereditaria $X$.

Para cualquier clase hereditaria $X$, tenemos que $X = Free(Forb(X))$. Además, $Forb(X)$ es el único conjunto mínimo con esta propiedad.
