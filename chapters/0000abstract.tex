Graph Theory is the branch of Discrete Mathematics that studies the mathematical structures known as graphs. A graph $G$ is an ordered pair of disjoint sets $(V,E)$ such that $E$ is a set of unordered pairs of elements of $V$. We call $V$ the set of vertices of $G$ and $E$ the set of edges of $G$. If an element of $E$ contains two vertices, we say that those vertices are adjacent. In this thesis, we work with a particular kind of graphs known as cographs, that can be characterized in several ways. One of this ways is that cographs are the graphs that does not contain the path on 4 vertices, $P_4$, as an induced subgraph.

A clasic problem in Graph Theory is graph coloring, which consists in deciding if it is possible to label the vertices of a graph with a given number of different labels, also known as colors, in a way such that if two vertices are adjacent, then those two vertices have different labels. A generalization of the graph coloring problem is the matrix partition problem, which consists in deciding if the vertices of a graph can be labeled with a given number of labels in a way such that vertices with the same label satisfy a homogeneity property. In this thesis we address a problem of matrix partitions. We study the classes of cographs defined in the following way. Given an integer $i$ greater than or equal to one, the class $M_i$ is the class of the cographs whose set of vertices accepts a partition in $i$ parts such that each part induces a multipartite complete graph. The class $M_1$, which is the class of multipartite complete cographs, has been widely studied. However, the classes $M_i$ for values of $i$ greater than one have not been previously studied. The base of our research is the study of the class $M_2$. For this study, we use the research done about polar cographs as guide. We characterize the class $M_2$ through its set of minimal obstructions, we present an algorithm to recognize its elements and a certifying algorithm. We also study a set of subclasses of the class $M_2$ that we call the classes $(\alpha,\beta)$-$M_2$. Similarly, we characterize the class $M_3$ through its set of minimal obstructions and present two families of minimal obstructions for any class $M_i$.
