%%Por definir: k-polares
En \cite{Hell03}, Hell, Hern\'andez-Cruz y Linhares-Sales
exhiben el conjunto de obstrucciones mínimas de la clase
de las cográficas 2-polares. Para construir este conjunto
primero se presentan resultados preliminares sobre la
estructura de las obstrucciones mínimas de las cográficas
$(k,k)$-polares (llamadas simplemente cográficas $k$-polares)
para cualquier entero positivo $k$. Posteriormente se presenta
el complemento parcial, una operación que conserva la
$2$-polaridad, para construir las 24 obstrucciones m\'inimas
inconexas de las cogr\'aficas $2$-polares a partir de un
conjunto de cuatro obtrucciones m\'inimas.

El siguiente lema describe la estructura de las
obstrucciones m\'inimas para las cogr\'aficas
$k$-polares con el m\'aximo n\'umero posible de
componentes conexas.

\begin{lemma}
\label{lema_2polares_01}
Sean $l$ y $k$ enteros tales que $1 \le l \le k+1$. Salvo isomorfismo, hay exactamente una obstrucción mínima para las cográficas $k$-polares con $k+2$ componentes en total y $l$ componentes triviales. Esta obstrucción mínima es isomorfa a
$$lk_1+(k-l+1)k_2+k_{l,l}$$
\end{lemma}

Aplicando el Lema \ref{lema_2polares_01}, podemos encontrar
tres obstrucciones mínimas para las cográficas $2$-polares.
Éstas se muestran en la Figura \ref{obsts_2polares_01}.

\begin{figure}[ht!]
\begin{center}
\begin{tikzpicture}

\begin{scope}[xshift=0cm,scale=1]
\node [vertex] (1) at (0,0) {};
\node [vertex] (2) at (1,0) {};
\node [vertex] (3) at (0,0.5) {};
\node [vertex] (4) at (1,0.5) {};
\node [vertex] (5) at (0,1) {};
\node [vertex] (6) at (1,1) {};
\node [vertex] (7) at (0.5,1.5) {};
\foreach \i/\j in {1/2,3/4,5/6}
  \draw [edge] (\i) to (\j);
\node at(0.5,-1) {\parbox{0.3\linewidth}{\subcaption*{$F_{1}$}}};
\end{scope}

\begin{scope}[xshift=2.5cm,scale=1]
\node [vertex] (1) at (0,0) {};
\node [vertex] (2) at (1,0) {};
\node [vertex] (3) at (0,0.5) {};
\node [vertex] (4) at (1,0.5) {};
\node [vertex] (5) at (0,1) {};
\node [vertex] (6) at (1,1) {};
\node [vertex] (7) at (0.25,1.5) {};
\node [vertex] (8) at (0.75,1.5) {};
\foreach \i/\j in {1/2,1/4,2/3,3/4,5/6}
  \draw [edge] (\i) to (\j);
\node at(0.5,-1) {\parbox{0.3\linewidth}{\subcaption*{$F_{13}$}}};
\end{scope}

\begin{scope}[xshift=5cm,scale=1]
\node [vertex] (1) at (0,0) {};
\node [vertex] (2) at (1,0) {};
\node [vertex] (3) at (0,0.5) {};
\node [vertex] (4) at (1,0.5) {};
\node [vertex] (5) at (0,1) {};
\node [vertex] (6) at (1,1) {};
\node [vertex] (7) at (0,1.5) {};
\node [vertex] (8) at (0.5,1.5) {};
\node [vertex] (9) at (1,1.5) {};
\foreach \i/\j in {1/2,1/4,2/3,3/4,3/6,4/5,5/6}
  \draw [edge] (\i) to (\j);
\node at(0.5,-1) {\parbox{0.3\linewidth}{\subcaption*{$F_{21}$}}};
\end{scope}

\end{tikzpicture}
\end{center}
\caption{Obstrucciones mínimas para las cográficas 2-polaes obtenidas con el Lema \ref{lema_2polares_01}.}
\label{obsts_2polares_01}
\end{figure}

Ahora introducimos una operaci\'on que preserva la
$2$-polaridad y la propiedad de ser cogr\'afica, por lo
que resulta bastante \'util en el estudio de las
obstrucciones m\'inimas para las cogr\'aficas $2$-polares.
Sea $H$ una gráfica, un \textbf{\emph{complemento parcial}}
de $H$ es una gráfica obtenida de $H$ al dividir a sus
componentes conexas en dos gráficas $H'$ y $H''$, y tomando
de forma separada el complemento de cada una.

Tomando todos los posibles complementos parciales de las
gráficas $F_1$, $F_{13}$ y $F_{21}$ (Figura
\ref{obsts_2polares_01}), encontramos tres familias de
obstrucciones mínimas; \'estas se muestran en las Figuras
\ref{obsts_2polares_02}, \ref{obsts_2polares_03} y
\ref{obsts_2polares_04} respectivamente. Podemos encontrar
una cuarta familia de obstrucciones mínimas tomar todos los
posibles complementos parciales de la gráfica $F_7$ (Figura
\ref{obsts_2polares_05}) que también es una obstrucción mínima
de las gráficas 2-polares. La gráfica $F_7$ se puede construir
de forma natural agregando un $K_2$ a una de las obstrucciones
m\'inimas para $(2,1)$-polaridad en cogr\'aficas.

\begin{figure}[ht!]
\begin{subfigure}{\textwidth}
\begin{center}
\begin{tikzpicture}

\begin{scope}[xshift=0cm,scale=1]
\node [vertex] (1) at (0,0) {};
\node [vertex] (2) at (1,0) {};
\node [vertex] (3) at (0,0.5) {};
\node [vertex] (4) at (1,0.5) {};
\node [vertex] (5) at (0,1) {};
\node [vertex] (6) at (1,1) {};
\node [vertex] (7) at (0.5,1.5) {};
\foreach \i/\j in {1/2,3/4,5/6}
  \draw [edge] (\i) to (\j);
\node at(0.5,-1) {\parbox{0.3\linewidth}{\subcaption*{$F_{1}$}}};
\end{scope}

\begin{scope}[xshift=2.5cm,scale=1]
\node [vertex] (1) at (0,0) {};
\node [vertex] (2) at (0.5,0) {};
\node [vertex] (3) at (1,0) {};
\node [vertex] (4) at (0.5,0.5) {};
\node [vertex] (5) at (0,1) {};
\node [vertex] (6) at (1,1) {};
\node [vertex] (7) at (0.5,1.5) {};
\foreach \i/\j in {1/2,2/3,4/5,4/6,5/7,6/7}
  \draw [edge] (\i) to (\j);
\node at(0.5,-1) {\parbox{0.3\linewidth}{\subcaption*{$F_{2}$}}};
\end{scope}

\begin{scope}[xshift=5cm,scale=1]
\node [vertex] (1) at (0.5,0) {};
\node [vertex] (2) at (0,0.5) {};
\node [vertex] (3) at (0.5,0.5) {};
\node [vertex] (4) at (1,0.5) {};
\node [vertex] (5) at (0.5,1) {};
\node [vertex] (6) at (0,1.5) {};
\node [vertex] (7) at (1,1.5) {};
\foreach \i/\j in {2/5,2/6,3/5,4/5,4/7,5/6,5/7}
  \draw [edge] (\i) to (\j);
\node at(0.5,-1) {\parbox{0.3\linewidth}{\subcaption*{$F_{3}$}}};
\end{scope}

\begin{scope}[xshift=7.5cm,scale=1]
\node [vertex] (1) at (0,0) {};
\node [vertex] (2) at (1,0) {};
\node [vertex] (3) at (0.5,0.5) {};
\node [vertex] (4) at (0,1) {};
\node [vertex] (5) at (0.5,1) {};
\node [vertex] (6) at (1,1) {};
\node [vertex] (7) at (0.5,1.5) {};
\foreach \i/\j in {3/4,3/5,3/6,4/5,4/7,5/6,5/7,6/7}
  \draw [edge] (\i) to (\j);
\node at(0.5,-1) {\parbox{0.3\linewidth}{\subcaption*{$F_{4}$}}};
\end{scope}

\begin{scope}[xshift=10.25cm,scale=1]
\node [vertex] (1) at (0,0) {};
\node [vertex] (2) at (1,0) {};
\node [vertex] (3) at (-0.25,0.5) {};
\node [vertex] (4) at (1.25,0.5) {};
\node [vertex] (5) at (0,1) {};
\node [vertex] (6) at (1,1) {};
\node [vertex] (7) at (0.5,1.5) {};
\foreach \i/\j in {1/3,1/4,1/5,1/6,2/3,2/4,2/5,2/6,3/5,3/6,4/5,4/6}
  \draw [edge] (\i) to (\j);
\node at(0.5,-1) {\parbox{0.3\linewidth}{\subcaption*{$F_{5}$}}};
\end{scope}

\end{tikzpicture}
\end{center}
\end{subfigure}

\caption{Obstrucciones mínimas de las cográficas $2$-polares con 7 vértices.}
\label{obsts_2polares_02}
\end{figure}


\begin{figure}[ht!]
\begin{subfigure}{\textwidth}
\begin{center}
\begin{tikzpicture}

\begin{scope}[xshift=0cm,scale=1]
\node [vertex] (1) at (0,0) {};
\node [vertex] (2) at (1,0) {};
\node [vertex] (3) at (0,0.5) {};
\node [vertex] (4) at (1,0.5) {};
\node [vertex] (5) at (0,1) {};
\node [vertex] (6) at (1,1) {};
\node [vertex] (7) at (0.25,1.5) {};
\node [vertex] (8) at (0.75,1.5) {};
\foreach \i/\j in {1/2,1/4,2/3,3/4,5/6}
  \draw [edge] (\i) to (\j);
\node at(0.5,-1) {\parbox{0.3\linewidth}{\subcaption*{$F_{13}$}}};
\end{scope}

\begin{scope}[xshift=2.5cm,scale=1]
\node [vertex] (1) at (0,0) {};
\node [vertex] (2) at (1,0) {};
\node [vertex] (3) at (0,0.5) {};
\node [vertex] (4) at (1,0.5) {};
\node [vertex] (5) at (0,1) {};
\node [vertex] (6) at (1,1) {};
\node [vertex] (7) at (0,1.5) {};
\node [vertex] (8) at (1,1.5) {};
\foreach \i/\j in {1/2,1/3,1/4,2/4,3/4,5/6,7/8}
  \draw [edge] (\i) to (\j);
\node at(0.5,-1) {\parbox{0.3\linewidth}{\subcaption*{$F_{14}$}}};
\end{scope}

\begin{scope}[xshift=5cm,scale=1]
\node [vertex] (1) at (0,0) {};
\node [vertex] (2) at (1,0) {};
\node [vertex] (3) at (0,0.5) {};
\node [vertex] (4) at (1,0.5) {};
\node [vertex] (5) at (0,1) {};
\node [vertex] (6) at (1,1) {};
\node [vertex] (7) at (0,1.5) {};
\node [vertex] (8) at (1,1.5) {};
\foreach \i/\j in {1/2,1/3,1/4,2/3,2/4,3/5,3/6,4/5,4/6,5/6,7/8}
  \draw [edge] (\i) to (\j);
\node at(0.5,-1) {\parbox{0.3\linewidth}{\subcaption*{$F_{15}$}}};
\end{scope}

\begin{scope}[xshift=7.5cm,scale=1]
\node [vertex] (1) at (0,0) {};
\node [vertex] (2) at (0,0.5) {};
\node [vertex] (3) at (0.5,0) {};
\node [vertex] (4) at (1,0.5) {};
\node [vertex] (5) at (0.5,1) {};
\node [vertex] (6) at (0,1.5) {};
\node [vertex] (7) at (1,1.5) {};
\node [vertex] (8) at (1,0) {};
\foreach \i/\j in {2/5,2/6,4/5,4/7,5/6,5/7,1/3,3/8}
  \draw [edge] (\i) to (\j);
\node at(0.5,-1) {\parbox{0.3\linewidth}{\subcaption*{$F_{16}$}}};
\end{scope}

\begin{scope}[xshift=10cm,scale=1]
\node [vertex] (1) at (0,0) {};
\node [vertex] (2) at (1,0) {};
\node [vertex] (3) at (0,0.5) {};
\node [vertex] (4) at (1,0.5) {};
\node [vertex] (5) at (0,1) {};
\node [vertex] (6) at (1,1) {};
\node [vertex] (7) at (0,1.5) {};
\node [vertex] (8) at (1,1.5) {};
\foreach \i/\j in {1/2,1/3,1/4,2/3,2/4,3/4,3/5,3/6,4/5,4/6,5/6} \draw [edge] (\i) to (\j);
\node at(0.5,-1) {\parbox{0.3\linewidth}{\subcaption*{$F_{17}$}}};
\end{scope}

\end{tikzpicture}
\end{center}
\end{subfigure}

\begin{subfigure}{\textwidth}
\begin{center}
\begin{tikzpicture}

\begin{scope}[xshift=0cm,scale=1]
\node [vertex] (1) at (0.75,0) {};
\node [vertex] (2) at (0.25,0.5) {};
\node [vertex] (3) at (1.25,0.5) {};
\node [vertex] (4) at (0,1) {};
\node [vertex] (5) at (0.75,1) {};
\node [vertex] (6) at (1.5,1) {};
\node [vertex] (7) at (0.25,1.5) {};
\node [vertex] (8) at (1.25,1.5) {};
\foreach \i/\j in {2/3,2/4,2/5,2/6,3/4,3/5,3/6,4/5,4/7,4/8,5/6,5/7,5/8,6/7,6/8,7/8}
  \draw [edge] (\i) to (\j);
\node at(0.75,-1) {\parbox{0.3\linewidth}{\subcaption*{$F_{18}$}}};
\end{scope}

\begin{scope}[xshift=3cm,scale=1]
\node [vertex] (1) at (0,0) {};
\node [vertex] (2) at (1,0) {};
\node [vertex] (3) at (0,0.5) {};
\node [vertex] (4) at (1,0.5) {};
\node [vertex] (5) at (0,1) {};
\node [vertex] (6) at (1,1) {};
\node [vertex] (7) at (0,1.5) {};
\node [vertex] (8) at (1,1.5) {};
\foreach \i/\j in {1/2,1/3,1/4,2/3,2/4,3/5,3/6,4/5,4/6}
  \draw [edge] (\i) to (\j);
\node at(0.5,-1) {\parbox{0.3\linewidth}{\subcaption*{$F_{19}$}}};
\end{scope}

\begin{scope}[xshift=5.5cm,scale=1]
\node [vertex] (1) at (0,0) {};
\node [vertex] (2) at (0.75,0) {};
\node [vertex] (3) at (0,0.75) {};
\node [vertex] (4) at (0.75,0.75) {};
\node [vertex] (5) at (1.5,0.75) {};
\node [vertex] (6) at (0.5,1) {};
\node [vertex] (7) at (1,1) {};
\node [vertex] (8) at (0.75,1.5) {};
\foreach \i/\j in {2/3,2/4,2/5,3/4,3/8,4/5,4/6,4/7,4/8,5/8}
  \draw [edge] (\i) to (\j);
\node at(0.75,-1) {\parbox{0.3\linewidth}{\subcaption*{$F_{20}$}}};
\end{scope}

\end{tikzpicture}
\end{center}
\end{subfigure}

\caption{Familia A de obstrucciones mínimas de las cográficas $2$-polares con 8 vértices.}
\label{obsts_2polares_03}
\end{figure}

\begin{figure}[ht!]
\begin{subfigure}{\textwidth}
\begin{center}
\begin{tikzpicture}

\begin{scope}[xshift=0cm,scale=1]
\node [vertex] (1) at (0,0) {};
\node [vertex] (2) at (1,0) {};
\node [vertex] (3) at (0,0.5) {};
\node [vertex] (4) at (1,0.5) {};
\node [vertex] (5) at (0,1) {};
\node [vertex] (6) at (1,1) {};
\node [vertex] (7) at (0,1.5) {};
\node [vertex] (8) at (0.5,1.5) {};
\node [vertex] (9) at (1,1.5) {};
\foreach \i/\j in {1/2,1/4,2/3,3/4,3/6,4/5,5/6}
  \draw [edge] (\i) to (\j);
\node at(0.5,-1) {\parbox{0.3\linewidth}{\subcaption*{$F_{21}$}}};
\end{scope}

\begin{scope}[xshift=2.5cm,scale=1]
\node [vertex] (1) at (0,0) {};
\node [vertex] (2) at (0.5,0.25) {};
\node [vertex] (3) at (0,0.5) {};
\node [vertex] (4) at (1,0.5) {};
\node [vertex] (5) at (0.5,0.75) {};
\node [vertex] (6) at (1,1) {};
\node [vertex] (7) at (0,1) {};
\node [vertex] (8) at (0.5,1.25) {};
\node [vertex] (9) at (0,1.5) {};
\foreach \i/\j in {1/2,1/3,2/3,4/5,4/6,5/6,7/8,7/9,8/9}
  \draw [edge] (\i) to (\j);
\node at(0.5,-1) {\parbox{0.3\linewidth}{\subcaption*{$F_{22}$}}};
\end{scope}

\begin{scope}[xshift=5cm,scale=1]
\node [vertex] (1) at (0.5,0) {};
\node [vertex] (2) at (1.5,0) {};
\node [vertex] (3) at (0.5,0.5) {};
\node [vertex] (4) at (1.5,0.5) {};
\node [vertex] (5) at (0,1) {};
\node [vertex] (6) at (1,1) {};
\node [vertex] (7) at (2,1) {};
\node [vertex] (8) at (0.5,1.5) {};
\node [vertex] (9) at (1.5,1.5) {};
\foreach \i/\j in {1/2,3/5,3/6,3/8,4/6,4/7,4/9,5/6,5/8,6/7,6/8,6/9,7/9}
  \draw [edge] (\i) to (\j);
\node at(0.75,-1) {\parbox{0.3\linewidth}{\subcaption*{$F_{23}$}}};
\end{scope}

\begin{scope}[xshift=8.5cm,scale=1]
\node [vertex] (1) at (0.75,0) {};
\node [vertex] (2) at (0,0.5) {};
\node [vertex] (3) at (0.75,0.5) {};
\node [vertex] (4) at (1.5,0.5) {};
\node [vertex] (5) at (0.25,1) {};
\node [vertex] (6) at (1.25,1) {};
\node [vertex] (7) at (0,1.5) {};
\node [vertex] (8) at (0.75,1.5) {};
\node [vertex] (9) at (1.5,1.5) {};
\foreach \i/\j in {2/3,2/5,2/6,3/4,3/5,3/6,4/5,4/6,5/6,5/7,5/8,5/9,6/7,6/8,6/9,7/8,8/9}
  \draw [edge] (\i) to (\j);
\node at(0.75,-1) {\parbox{0.3\linewidth}{\subcaption*{$F_{24}$}}};
\end{scope}

\end{tikzpicture}
\end{center}
\end{subfigure}

\caption{Obstrucciones mínimas de las cográficas $2$-polares con 9 vértices.}
\label{obsts_2polares_04}
\end{figure}


\begin{figure}[ht!]
\begin{subfigure}{\textwidth}
\begin{center}
\begin{tikzpicture}

\begin{scope}[xshift=0cm,scale=1]
\node [vertex] (1) at (0.25,0) {};
\node [vertex] (2) at (0.75,0) {};
\node [vertex] (3) at (0,0.75) {};
\node [vertex] (4) at (0.5,0.75) {};
\node [vertex] (5) at (1,0.75) {};
\node [vertex] (6) at (0,1.5) {};
\node [vertex] (7) at (0.5,1.5) {};
\node [vertex] (8) at (1,1.5) {};
\foreach \i/\j in {1/2,3/4,4/5,6/7,7/8}
  \draw [edge] (\i) to (\j);
\node at(0.5,-1) {\parbox{0.3\linewidth}{\subcaption*{$F_{6}$}}};
\end{scope}

\begin{scope}[xshift=2.5cm,scale=1]
\node [vertex] (1) at (0,0) {};
\node [vertex] (8) at (0.5,0) {};
\node [vertex] (2) at (1,0) {};
\node [vertex] (3) at (0.5,0.5) {};
\node [vertex] (4) at (0,1) {};
\node [vertex] (5) at (0.5,1) {};
\node [vertex] (6) at (1,1) {};
\node [vertex] (7) at (0.5,1.5) {};
\foreach \i/\j in {3/4,3/5,3/6,4/5,4/7,5/6,6/7,8/2}
  \draw [edge] (\i) to (\j);
\node at(0.5,-1) {\parbox{0.3\linewidth}{\subcaption*{$F_{7}$}}};
\end{scope}

\begin{scope}[xshift=5cm,scale=1]
\node [vertex] (1) at (0,0) {};
\node [vertex] (2) at (1,0) {};
\node [vertex] (3) at (0,0.5) {};
\node [vertex] (4) at (1,0.5) {};
\node [vertex] (5) at (0,1) {};
\node [vertex] (6) at (1,1) {};
\node [vertex] (7) at (0,1.5) {};
\node [vertex] (8) at (1,1.5) {};
\foreach \i/\j in {1/2,1/3,1/4,2/3,2/4,3/4,3/6,4/5,5/6} \draw [edge] (\i) to (\j);
\node at(0.5,-1) {\parbox{0.3\linewidth}{\subcaption*{$F_{8}$}}};
\end{scope}

\begin{scope}[xshift=7.5cm,scale=1]
\node [vertex] (1) at (0.5,0) {};
\node [vertex] (2) at (0,0.5) {};
\node [vertex] (3) at (1,0.5) {};
\node [vertex] (4) at (0,1) {};
\node [vertex] (5) at (1,1) {};
\node [vertex] (6) at (0,1.5) {};
\node [vertex] (7) at (0.5,1.5) {};
\node [vertex] (8) at (1,1.5) {};
\foreach \i/\j in {2/3,2/4,2/5,3/4,3/5,4/6,4/7,4/8,5/6,5/7,5/8,6/7,7/8} \draw [edge] (\i) to (\j);
\node at(0.5,-1) {\parbox{0.3\linewidth}{\subcaption*{$F_{9}$}}};
\end{scope}

\begin{scope}[xshift=10cm,scale=1]
\node [vertex] (1) at (0.5,0) {};
\node [vertex] (2) at (1,0) {};
\node [vertex] (3) at (0.5,0.5) {};
\node [vertex] (4) at (1.5,0.5) {};
\node [vertex] (5) at (0,1) {};
\node [vertex] (6) at (1,1) {};
\node [vertex] (7) at (0.5,1.5) {};
\node [vertex] (8) at (1.5,1.5) {};
\foreach \i/\j in {1/2,3/5,3/6,4/6,4/8,5/6,5/7,6/7,6/8} \draw [edge] (\i) to (\j);
\node at(0.5,-1) {\parbox{0.3\linewidth}{\subcaption*{$F_{10}$}}};
\end{scope}

\end{tikzpicture}
\end{center}
\end{subfigure}

\begin{subfigure}{\textwidth}
\begin{center}
\begin{tikzpicture}

\begin{scope}[xshift=5cm,scale=1]
\node [vertex] (1) at (1,0) {};
\node [vertex] (3) at (0.5,0.5) {};
\node [vertex] (4) at (1.5,0.5) {};
\node [vertex] (5) at (0,1) {};
\node [vertex] (6) at (1,1) {};
\node [vertex] (7) at (2,1) {};
\node [vertex] (8) at (0.5,1.5) {};
\node [vertex] (9) at (1.5,1.5) {};
\foreach \i/\j in {3/5,3/6,4/6,4/7,5/6,5/8,6/7,6/8,6/9,7/9}
  \draw [edge] (\i) to (\j);
\node at(0.75,-1) {\parbox{0.3\linewidth}{\subcaption*{$F_{11}$}}};
\end{scope}

\begin{scope}[xshift=8.5cm,scale=1]
\node [vertex] (1) at (0,0) {};
\node [vertex] (2) at (1.25,0) {};
\node [vertex] (3) at (2,0.25) {};
\node [vertex] (4) at (0,0.75) {};
\node [vertex] (5) at (0.75,0.75) {};
\node [vertex] (6) at (2,1.25) {};
\node [vertex] (7) at (0,1.5) {};
\node [vertex] (8) at (1.25,1.5) {};
\foreach \i/\j in {2/3,2/4,2/5,2/6,3/5,3/8,3/6,4/5,4/8,5/7,5/8,5/6,8/6}
  \draw [edge] (\i) to (\j);
\node at(0.75,-1) {\parbox{0.3\linewidth}{\subcaption*{$F_{12}$}}};
\end{scope}

\end{tikzpicture}
\end{center}
\end{subfigure}

\caption{Familia B de obstrucciones mínimas de las cográficas $2$-polares con 8 vértices.}
\label{obsts_2polares_05}
\end{figure}

En este artículo podemos observar un ejemplo en el que, dada
una clase hereditaria de cográficas $C$ y una operación
$o$ tales que $C$ es cerrada bajo $o$, es posible encontrar
familias de obstrucciones mínimas para $C$ a partir de
obstrucciones mínimas de $C$ ya conocidas aplic\'andoles
la operación $o$. Las clases de cográficas que estudiamos en
nuestra investigación son cerradas bajo la unión completa y,
de manera parecida a lo que sucede en \cite{Hell03},
encontramos reglas para generar obstrucciones mínimas para
una clase hereditaria $D$ al aplicar la unión completa a
obstrucciones mínimas de subclases de $D$.

% Quizá valdría la pena ser más específicos en la última
% oración de este párrafo.
