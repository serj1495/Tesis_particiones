En \cite{Corneil02}, Corneil, Perl y Stewart presentan un algoritmo
que, dada una gráfica, determina si ésta es una cográfica y, si lo
es, devuelve su coárbol. Éste es un algoritmo incremental, se
construye un coárbol agregando los vértices de la gráfica recibida
como entrada uno a uno. Para agregar un nuevo vértice $x$, de la
gráfica de entrada $G$, al coárbol $T$ que se va construyendo, el
algoritmo ejecuta primero una subrutina que se encarga de marcar los
nodos de $T$ empezando por sus hojas. Una hoja de $T$ se marca sólo
si ésta representa un vértice de $G$ adyacente a $x$. La subrutina
continúa marcando nodos de $G$ subiendo hacia la raíz. En nuestra
investigación presentamos un algoritmo similar, el Algoritmo
\ref{alg_subcoarbol}, que ayuda a determinar si una cográfica
$H$ es subcográfica de otra cográfica $I$ marcando los vértices de
un coárbol binario\footnote{Un coárbol binario es un coárbol en el
que cada nodo tiene necesariamente dos hijos y pueden haber nodos
adyacentes con la misma etiqueta.} de $I$ partiendo desde las hojas
y subiendo hacia la raíz.

A continuación se reproducen los resultados de \cite{ Corneil02},
empezando por el algoritmo de marcado antes mencionado.

\begin{algorithm}[!htbp]
\caption{Marcar}
\label{alg_mark_corneil}
\DontPrintSemicolon % Some LaTeX compilers require you to use \dontprintsemicolon instead
\KwIn{$T$, un coárbol con raíz $R$ cuyas hojas son vértices de una gráfica $G$; $x$, el vértice de $G$ que se busca agregar a $T$}
\KwOut{Se marcan y se desmarcan algunos nodos de $T$ }

$D\gets \{\}$ \tcp*[h]{El conjunto de los nodos que han sido marcados y desmarcados} \; 

Marcar todas las hojas de $T$ que sean adyacentes a $x$\;

\ForEach{\emph{nodo marcado $u$ de $T$ tal que todos sus hijos están en el conjunto $D$}}{
    Desmarcar a $u$\;
    Agregar $u$ a $D$\;

    \If{$u \neq R$}{
        \emph{Marcar al padre de $u$}\;
    }
}

\end{algorithm}

Sean $T$ un coárbol, $x$ un vértice y $M$ el conjunto de nodos
de $T$ que se encuentran marcados al terminar la ejecución del
Algoritmo \ref{alg_mark_corneil} al recibir como entrada a $T$
y a $x$. El Teorema \ref{teo_teo1_corneil} puede ser utilizado
para determinar si la gráfica que se obtiene al agregar $x$ a
la gráfica representada por $T$ es una cográfica.
Las siguientes definiciones son necesarias para presentar dicho
teorema. Sea $\alpha$ un nodo de $M$ de profundidad máxima en $T$
y sea $\beta$ un nodo en $M-\{a\}$ de profundidad máxima en $T$.
Decimos que un nodo $\gamma$ de $T$ con etiqueta 1 está
correctamente marcado si y sólo si todos sus hijos, excepto uno,
fueron marcados y desmarcados. Un camino legítimamente alternante
en un coárbol marcado es un camino alternante de nodos correctamente
marcados y nodos sin marcar, con etiqueta 0, cuyos extremos son nodos
con etiqueta 1.


\begin{theorem}
    \label{teo_teo1_corneil}
    Si $G$ es una cográfica con árbol $T$, entonces $G+x$ es una cográfica si y sólo si:
    \begin{itemize}
        \item $M$ es vacío o
        \item se cumplen las siguientes dos condiciones
        \begin{itemize}
            \item $M-\{\alpha\}$ consiste exactamente de los nodos
              con etiqueta 1 de un camino legítimamente alternante
              que termina en $R$.

            \item $\alpha$ es un nodo con etiqueta 0 cuyo padre es
              $\beta$ o $\alpha$ es un nodo con etiqueta 1 cuyo
              abuelo, si existe, es $\beta$.
        \end{itemize}
    \end{itemize}
\end{theorem}

El Algoritmo \ref{alg_principal_corneil}, que constituye el resultado
principal del artículo, utiliza el Algoritmo \ref{alg_mark_corneil} y
el Teorema \ref{teo_teo1_corneil} para determinar si una gráfica es
una cográfica, y construir su coárbol en caso de que lo sea.

\begin{algorithm}[!htbp]
\caption{AlgoritmoDeReconocimiento}
\label{alg_principal_corneil}
\DontPrintSemicolon % Some LaTeX compilers require you to use \dontprintsemicolon instead
\KwIn{$G$, una gráfica cuyo conunto de vértices es $V=\{v_1, v_2, \dots, v_n\}$}
\KwOut{El coárbol de $G$ si $G$ es una cográfica, $null$ en el caso contrario}

Crear el coárbol $T$ cuya raíz $R$ tiene etiqueta 1 si $v_1$ y $v_2$ son adyacentes o etiqueta 0 en el caso contrario\;

\ForEach{x \emph{\textbf{en}} $V(G)-\{v_1,v_2\}$}{
    Marcar($T$, $x$)\;
    \If{Todos los nodos de $T$ fueron marcados y desmarcados}{
        \If{$R$ tiene etiqueta 1}{
            Agregar a $x$ como hijo de $R$\;
        }
        \Else{
            $R\gets$ un nuevo nodo con etiqueta 1 cuyos hijos sean $x$ y $R$\;
        }
    }
    \ElseIf{Ninguno de los nodos de $T$ fue marcado}{
        \If{$R$ tiene etiqueta 0}{
            Agregar a $x$ como hijo de $R$\;
        }
        \Else{
            $R\gets$ un nuevo nodo con etiqueta 0 cuyos hijos sean $x$ y $R$\;
        }
    }
    \ElseIf{La gráfica representada por $T$ agregando $x$ es una cográfica}{
        $u\gets$ el nodo marcado más profundo de $T$\;
        Encontrar en qué nodo del coárbol con raíz $u$ se debe insertar $x$\;
    }
    \Else{
        \Return $null$\;
    }
	Desmarcar todos los nodos de $T$\;
}

\end{algorithm}

%Cierre
El Algoritmo \ref{alg_mark_corneil} consolida un ejemplo
de un algoritmo que funciona marcando los nodos de un coárbol
empezando por las hojas y dirigiéndose hacia la raíz\footnote{En
ingl\'es un {\em bottom-up} algorithm.}. Por otro lado, el Algoritmo
\ref{alg_mark_corneil} es un ejemplo de un algoritmo que, al ser
ejecutado, no sólo resuelve un problema de decisión, sino que
proporciona una prueba (o {\em certificado}) de que su respuesta
es correcta. En este caso, la prueba es el coárbol de la
gráfica recibida como entrada. Sin embargo, cuando la gr\'afica
que se recibe como entrada no es una cogr\'afica, el Algoritmo
\ref{alg_mark_corneil} s\'olo devuelve {\em null}. Si desconfi\'aramos
de una implementaci\'on de este algoritmo, y quisi\'eramos comprobar
que la salida es correcta, podr\'iamos utilizar el co\'arbol en un
caso, pero en el otro, no tendr\'iamos informaci\'on adicional.

En nuestra investigación, el Algoritmo \ref{alg_cert_m2} es un
algoritmo certificador, es decir, devuelve certificados para
verificar, de manera eficiente, que la salida del algoritmo es
correcta, en cualquier caso. Nuestro algoritmo no sólo es capaz de
determinar si una cográfica $G$ pertenece a la clase $M_2$, sino que
devuelve una coloración de las hojas de su coárbol $T$ tal que si
$G \in M_2$, las hojas de $T$ tienen uno de dos colores y las hojas
del mismo color inducen una gráfica multipartita completa en $G$
(un {\em s\'i-certificado}). En el caso contrario, algunas de las
hojas de $T$ tendrán un color distintivo que indica que esos
vértices inducen una obstrucción mínima de $M_2$ en $G$ (un
{\em no-certificado}).
