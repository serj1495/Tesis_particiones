En este capítulo presentamos el producto de nuestra investigación. La primera
sección del capítulo tiene como resultado principal un algoritmo capaz de
determinar si una cográfica pertenece a una clase hereditaria de gráficas fija
en tiempo lineal. A partir de la segunda sección abordamos el problema
principal de nuestra tesis, determinar si una cográfica acepta una partición en
un número dado de gráficas multipartitas completas. Comenzamos realizando un
estudio detallado de la clase de cográficas que aceptan una partición en dos
gráficas multipartitas completas, a la que llamamos $M_2$. En la segunda
sección del capítulo caracterizamos a la clase $M_2$ a través de su conjunto de
obstrucciones mínimas, proporcionamos un algoritmo para reconocer a sus
elementos (utilizando el resultado principal de la primera sección) y
presentamos un algoritmo certificador que no sólo es capaz de reconocer si una
cográfica $G$ pertenece a  $M_2$, sino que encuentra una partición de $G$ en
dos gráficas multipartitas completas o una obstrucción mínima de la clase como
subgráfica inducida de $G$. En la tercera sección estudiamos a las clases
$(\alpha, \beta)-M_2$, subclases de $M_2$ cuyos elementos aceptan una partición
en dos gráficas multipartitas completas de tamaños restringidos. El resultado
principal de esta sección es un algoritmo para encontrar obstrucciones mínimas
para cualquier clase $(\alpha, \beta)-M_2$. La cuarta y última sección del
capítulo da un paso en la generalización de los resultados de las secciones
anteriores. En éste se proporcionan algunas familias de obstrucciones mínimas
para caracterizar a la clase de cográficas que aceptan una partición en $i$
gráficas multipartitas completas dado un entero $i \geq 2$.

\section{Términos y algoritmos generales}
    En esta sección se presenta un conjunto de conceptos y algoritmos
    \'utiles para cogr\'aficas en general. El resultado
    principal de la sección es un algoritmo de tiempo lineal capaz de
    determinar si una cográfica representada por su coárbol pertenece a una
    clase hereditaria fija de cográficas caracterizada por su conjunto de
    obstrucciones mínimas.

    \subsection{Coárbol binario}
        Tomando como base el concepto de coárbol, podemos imaginar otra estructura de
tipo árbol para la representación de las cográficas en la que cada nodo tenga a
lo más un número $k$ de hijos. Esta limitante resulta útil para formular
algoritmos rápidos en cográficas. El menor valor que puede tomar $k$ es 2,
lo que resulta en una representaci\'on de cualquier cogr\'afica
mediante un co\'arbol binario.   Esta representaci\'on es la
que utilizaremos principalmente en el presente cap\'itulo.

Sean $G=(V,E)$ una cográfica, $C = \{c_1, c_2, \dots, c_n$\} el conjunto de
las componentes conexas de $G$, $D = \{d_1, d_2, \dots, d_m\}$ el conjunto
de las componentes conexas de $\overline{G}$, $(C_1, C_2)$ una partición en
dos partes de $C$ y $(D_1, D_2)$ una partición en dos partes de $D$. Decimos
que el árbol binario arraigado etiquetado, $(T,r)$, es un
\textbf{\emph{coárbol binario}} de $G$ si se puede construir de la siguiente
manera: Si $G$ consta de un sólo vértice, entonces $T$ sólo contiene a $r$,
que es igual al único vértice de $G$.
De lo contrario, si $G$ es conexa, entonces $r$ tiene la etiqueta $1$, uno
de los hijos de $r$ es el coárbol binario de $G-D_1$ y el otro es el coárbol
binario de $G-D_2$. Y finalmente, si $G$ es inconexa, entonces $r$ tiene la
etiqueta $0$, uno de sus hijos es el coárbol binario de $G-C_1$ y el otro el
coárbol binario de $G-C_2$.

Claramente, una cográfica puede ser representada por más de un coárbol binario
diferente como se muestra en la Figura \ref{fig_coar_bin01}. Sin embargo, la
propiedad de que dos vértices son adyacentes si y sólo si su ancestro común más
profundo tiene la etiqueta 1 se mantiene.

\begin{figure}[ht!]
\begin{center}
\begin{tikzpicture}

\begin{scope}[xshift=0cm,scale=1]

\node [vertex] (1) at (0,0) {};
\node [vertex] (2) at (1,0) {};
\node [vertex] (3) at (0,1) {};
\node [vertex] (4) at (1,1) {};
\foreach \i/\j in {1/2,1/3,1/4,2/3,2/4,3/4}
  \draw [edge] (\i) to (\j);
\node [below of=1,xshift=.5cm] {\parbox{0.3\linewidth}{\subcaption{}}};

\end{scope}

\begin{scope}[xshift=3.5cm,scale=1]

\node [cotreenode] (1) at (1,1) {1};
\node [cotreenode] (2) at (0,0) {1};
\node [cotreenode] (3) at (2,0) {1};
\node [vertex] (4) at (-0.5,-1) {};
\node [vertex] (5) at (0.5,-1) {};
\node [vertex] (6) at (1.5,-1) {};
\node [vertex] (7) at (2.5,-1) {};
\foreach \i/\j in {1/2,1/3,2/4,2/5,2/4,3/6,3/7}
  \draw [edge] (\i) to (\j);
\node [below of=5,xshift=.5cm] {\parbox{0.3\linewidth}{\subcaption{}}};

\end{scope}

\begin{scope}[xshift=7.5cm,scale=1]

\node [cotreenode] (1) at (1,1) {1};
\node [vertex] (2) at (0,0) {};
\node [cotreenode] (3) at (2,0) {1};
\node [vertex] (6) at (1,-1) {};
\node [cotreenode] (7) at (3,-1) {1};
\node [vertex] (8) at (2,-2) {};
\node [vertex] (9) at (4,-2) {};

\foreach \i/\j in {1/2,1/3,3/6,3/7,7/8,7/9}
  \draw [edge] (\i) to (\j);
\node [below of=8] {\parbox{0.3\linewidth}{\subcaption{}}};

\end{scope}

\end{tikzpicture}
\end{center}
\setlength{\abovecaptionskip}{-10pt}
\caption{(b) y (c) son dos coárboles binarios diferentes que representan a la cográfica (a).}\label{fig_coar_bin01}
\end{figure}

    \subsection{Algoritmo para generar un coárbol binario}
        Podemos obtener un coárbol binario a partir de un coárbol con el Algoritmo \ref{alg_coa_bin01}. En este algoritmo un nodo interno con al menos tres hijos, $r$, de un coárbol, se procesa creando un nuevo coárbol binario de la siguiente forma: La raíz del coárbol binario tiene como primer hijo al coárbol binario resultante de procesar al primer hijo de $r$ y como segundo hijo un nodo con la misma etiqueta de $r$ que a su vez tiene como primer hijo al árbol binario resultante de procesar al segundo hijo de $r$ y como segundo hijo un nuevo nodo con la misma etiqueta y así sucesivamente. Cuando sólo quedan los últimos dos hijos de $r$, estos se procesan y los árboles binarios resultantes son los hijos del último nodo creado. El árbol binario resultante es un árbol cargado a la derecha. La Figura \ref{fig_alg_coa_bin01} muestra una ejecución ilustrativa del algoritmo.

\begin{algorithm}[h]
\caption{CrearArbolBinario}
\label{alg_coa_bin01}
\DontPrintSemicolon % Some LaTeX compilers require you to use \dontprintsemicolon instead
\KwIn{$r$ la raíz del coárbol}
\KwOut{$r'$ la raíz del coárbol binario}

$r' \gets \text{nuevo nodo de árbol binario}$\;

\If{$r\ \emph{es un nodo interno} $}{
    $r'.etiqueta = r.etiqueta$\;
    $s \gets r'$\;
    $i \gets 0$\;
    \While{$i < r.children.size - 2$}{
        $s.primerHijo \gets \text{CrearArbolBinario}(r.hijos[i])$\;
        $s.segundoHijo \gets \text{nuevo nodo de árbol binario}$\;
        $s \gets s.segundoHijo$\;
        $s.etiqueta \gets r.etiqueta$\;
        $i \gets i+1$\;
    }
    $s.primerHijo \gets \text{CrearArbolBinario}(r.hijos[i])$\;
    $s.segundoHijo \gets \text{CrearArbolBinario}(r.hijos[i+1])$\;
}
\Return $r'$\;

\end{algorithm}

\begin{figure}[h!]
\centering

\begin{subfigure}{0.7\textwidth}
\begin{tikzpicture}
\begin{scope}[xshift=0cm,scale=1]
\node [style=cotreenode, fill=lightgray] (1) at (2,1) {1};
\node [style=vertex] (2) at (0.5,0) {};
\node [style=vertex] (3) at (1.5,0) {};
\node [style=vertex] (4) at (2.5,0) {};
\node [style=cotreenode] (5) at (3.5,0) {0};
\node [style=vertex] (6) at (3,-1) {};
\node [style=vertex] (7) at (4,-1) {};
\foreach \i/\j in {1/2,1/3,1/4,1/5,5/6,5/7}
  \draw [style=edge] (\i) to (\j);
\end{scope}
\begin{scope}[xshift=6.5cm,scale=1]
\node [style=cotreenode] (1) at (1,1) {1};
\node [style=vertex] (2) at (0.5,0) {};
\node [style=cotreenode] (3) at (1.5,0) {1};
\foreach \i/\j in {1/2,1/3}
  \draw [style=edge] (\i) to (\j);
\end{scope}
\end{tikzpicture}
\end{subfigure}

\par\bigskip

\begin{subfigure}{0.7\textwidth}
\begin{tikzpicture}
\begin{scope}[xshift=0cm,scale=1]
\node [style=cotreenode, fill=lightgray] (1) at (2,1) {1};
\node [style=vertex] (2) at (0.5,0) {};
\node [style=vertex] (3) at (1.5,0) {};
\node [style=vertex] (4) at (2.5,0) {};
\node [style=cotreenode] (5) at (3.5,0) {0};
\node [style=vertex] (6) at (3,-1) {};
\node [style=vertex] (7) at (4,-1) {};
\foreach \i/\j in {1/2,1/3,1/4,1/5,5/6,5/7}
  \draw [style=edge] (\i) to (\j);
\end{scope}
\begin{scope}[xshift=6.5cm,scale=1]
\node [style=cotreenode] (1) at (1,1) {1};
\node [style=vertex] (2) at (0.5,0) {};
\node [style=cotreenode] (3) at (1.5,0) {1};
\node [style=vertex] (4) at (1,-1) {};
\node [style=cotreenode] (5) at (2,-1) {1};
\foreach \i/\j in {1/2,1/3,3/4,3/5}
  \draw [style=edge] (\i) to (\j);
\end{scope}
\end{tikzpicture}
\end{subfigure}

\par\bigskip

\begin{subfigure}{0.7\textwidth}
\begin{tikzpicture}
\begin{scope}[xshift=0cm,scale=1]
\node [style=cotreenode, fill=lightgray] (1) at (2,1) {1};
\node [style=vertex] (2) at (0.5,0) {};
\node [style=vertex] (3) at (1.5,0) {};
\node [style=vertex] (4) at (2.5,0) {};
\node [style=cotreenode] (5) at (3.5,0) {0};
\node [style=vertex] (6) at (3,-1) {};
\node [style=vertex] (7) at (4,-1) {};
\foreach \i/\j in {1/2,1/3,1/4,1/5,5/6,5/7}
  \draw [style=edge] (\i) to (\j);
\end{scope}
\begin{scope}[xshift=6.5cm,scale=1]
\node [style=cotreenode] (1) at (1,1) {1};
\node [style=vertex] (2) at (0.5,0) {};
\node [style=cotreenode] (3) at (1.5,0) {1};
\node [style=vertex] (4) at (1,-1) {};
\node [style=cotreenode] (5) at (2,-1) {1};
\node [style=vertex] (6) at (1.5,-2) {};
\node [style=cotreenode] (7) at (2.5,-2) {0};
\foreach \i/\j in {1/2,1/3,3/4,3/5,5/6,5/7}
  \draw [style=edge] (\i) to (\j);
\end{scope}
\end{tikzpicture}
\end{subfigure}

\par\bigskip

\begin{subfigure}{0.7\textwidth}
\begin{tikzpicture}
\begin{scope}[xshift=0cm,scale=1]
\node [style=cotreenode] (1) at (2,1) {1};
\node [style=vertex] (2) at (0.5,0) {};
\node [style=vertex] (3) at (1.5,0) {};
\node [style=vertex] (4) at (2.5,0) {};
\node [style=cotreenode, fill=lightgray] (5) at (3.5,0) {0};
\node [style=vertex] (6) at (3,-1) {};
\node [style=vertex] (7) at (4,-1) {};
\foreach \i/\j in {1/2,1/3,1/4,1/5,5/6,5/7}
  \draw [style=edge] (\i) to (\j);
\end{scope}
\begin{scope}[xshift=6.5cm,scale=1]
\node [style=cotreenode] (1) at (1,1) {1};
\node [style=vertex] (2) at (0.5,0) {};
\node [style=cotreenode] (3) at (1.5,0) {1};
\node [style=vertex] (4) at (1,-1) {};
\node [style=cotreenode] (5) at (2,-1) {1};
\node [style=vertex] (6) at (1.5,-2) {};
\node [style=cotreenode] (7) at (2.5,-2) {0};
\node [style=vertex] (8) at (2,-3) {};
\node [style=vertex] (9) at (3,-3) {};
\foreach \i/\j in {1/2,1/3,3/4,3/5,5/6,5/7,7/8,7/9}
  \draw [style=edge] (\i) to (\j);
\end{scope}
\end{tikzpicture}
\end{subfigure}


\caption{Ejemplo de la ejecución del Algoritmo \ref{alg_coa_bin01}. A la izquierda se muestra el coárbol original, mienrtras se marca con gris el nodo que se está procesando. A la derecha aparece el coárbol binario que se va construyendo.}\label{fig_alg_coa_bin01}


\end{figure}


En términos de las particiones de las componentes conexas de la gráfica, el algoritmo realiza lo siguiente. Si la etiqueta de $r$ es $0$, entonces el coárbol con raíz en $r$ representa una cográfica inconexa y se elige la partición de sus vértices en la que la primera parte es una componente conexa y la segunda parte es el resto. Sucede lo mismo si la etiqueta de $r$ es uno, pero como la cográfica representada es conexa, en su lugar se toman una componente conexa del complemento de la cográfica representada en la primera parte y el resto en la segunda.

Dado que el Algoritmo \ref{alg_coa_bin01} recorre a lo más una vez cada nodo de $r$, su tiempo de ejecución es $O(n)$ en donde $n$ es el número total de nodos del árbol con raíz $r$.

    \subsection{Algoritmo para generar todos los coárboles binarios de una gráfica}
        Podemos obtener todos los coárboles binarios correspondientes a un coárbol
haciendo uso del Algoritmo \ref{alg_coa_bin02}. Este algoritmo recibe como
entrada la raíz del coárbol, $r$, y devuelve un conjunto de nodos, $S$, cada uno
de cuyos elementos es la raíz de un coárbol binario. Los nodos internos son
procesados creando un nuevo coárbol binario para cada posible partición del
conjunto de hijos de dicho nodo. Al procesar las hojas, simplemente se crea un
nuevo nodo que será una hoja en los árboles binarios. 

\begin{algorithm}[ht!]
\caption{CrearÁrbolesBinarios}
\label{alg_coa_bin02}
\DontPrintSemicolon % Some LaTeX compilers require you to use \dontprintsemicolon instead
\KwIn{$r$ la raíz de un coárbol $T$}
\KwOut{$S = \{r'_1, r'_2, \dots, r'_n\}$ con $r_i$ la raíz de un coárbol binario de $T$ para todo $1\geq i \geq n$}

$S \gets \emptyset$\;
\If{$r$ es una hoja}{
    Agregar un nuevo nodo de coárbol binario a $S$\;    
}
\Else{
    \ForEach{partición en dos partes $(A,B)$ del conjunto de hijos de $r$ tal que ni $A$ ni $B$ son vacíos}{
        \If{$A$ tiene sólo un elemento $a$}{
            $L \gets \text{CrearÁrbolesBinarios(\emph{a})}$\;
        }
        \Else{
            $a'\gets$ nuevo nodo de coárbol con la etiqueta de $r$\;
            Agregar todos los elementos de $A$ como hijos de $a'$\;
            $L \gets \text{CrearÁrbolesBinarios(\emph{a'})}$\;
        }
        \If{$B$ tiene sólo un elemento $b$}{
            $R \gets \text{CrearÁrbolesBinarios(\emph{b})}$\;
        }
        \Else{
            $b'\gets$ nuevo nodo de coárbol con la etiqueta de $r$\;
            Agregar todos los elementos de $B$ como hijos de $b'$\;
            $R \gets \text{CrearÁrbolesBinarios(\emph{b'})}$\;
        }
        \ForEach{$l\in L$ \textbf{y cada} $r \in R$}{
            $s\gets $ nuevo nodo de coárbol binario con la etiqueta de $r$\;
            $s.izquierda \gets l$\;
            $s.derecha \gets r$\;
            Agregar $s$ a $S$\;
        }
    }
}
\Return $S$\;
    
\end{algorithm}

Dado que el número de coárboles binarios que representan a una gráfica crece de forma exponencial, el tiempo en el que se pueden generar dichos coárboles binarios crece al menos de forma exponencial. El Algoritmo \ref{alg_coa_bin02} no es óptimo, ya que todos los nodos con excepción de la raíz y las hojas se procesan múltiples veces. Esto no repercute en el resultado principal de la sección, el Algoritmo \ref{alg_esta_en_clase}, que sirve para identificar a los elementos de una clase hereditaria de cográficas $C$, ya que, aunque se requiere encontrar todos los coárboles binarios de cada una de las obstrucciones mínimas de $C$, se contempla que este cómputo se realice antes de la ejecución del algoritmo.
    \subsection{Subcoárbol}
        A continuación presentamos el concepto de subcoárbol binario
que será utilizado para determinar si una cográfica $H$ es subgráfica
inducida de una cográfica $G$ en el Algoritmo \ref{alg_subgraph}.

Sean $T$ y $U$ dos coárboles binarios y $u_1$, $u_2$ y $u_3$ nodos de $U$, decimos
que $U$ es un \emph{\textbf{subcoárbol binario}} de $T$ si existe una función
inyectiva $f:V(U)\rightarrow V(T)$ tal que, si $u_1$ es una hoja, entonces $f(u_1)$
es una hoja también; si no, entonces $u_1.etiqueta = f(u_1).etiqueta$ y, si $u_3$ es
el ancestro común más profundo de $u_1$ y $u_2$, entonces $f(u_3)$ es el ancestro
común más profundo de $f(u_1)$ y $f(u_2)$. Llamamos a $f$ la \textbf{\emph{función de
coasignación}} de $U$ a $T$.

El concepto de subcoárbol binario es diferente del de subárbol dado que, si $T$ y $U$
son coárboles con $U$ subcoárbol binario de $T$, entonces tenemos que los nodos de
$U$ se pueden encontrar dispersos entre los nodos de $T$ a diferencia de lo que se
tendría si $U$ fuera subárbol de $T$. Esto se puede apreciar en la Figura
\ref{fig_subcoarbol01}.

\begin{figure}[h!]
\begin{center}
\begin{tikzpicture}

\begin{scope}[xshift=0cm,scale=1]
\node [style=cotreenode] (1) at (1,1) {0};
\node [style=cotreenode] (2) at (-0.5,0) {1};
\node [style=cotreenode] (3) at (2.5,0) {1};
\node [style=cotreenode] (4) at (-1.25,-1) {0};
\node [style=cotreenode] (5) at (0.25,-1) {0};
\node [style=cotreenode] (6) at (1.75,-1) {0};
\node [style=cotreenode] (7) at (3.25,-1) {0};
\node [style=vertex] (8) at (-1.5,-2) {};
\node [style=vertex] (9) at (-1,-2) {};
\node [style=vertex] (10) at (0,-2) {};
\node [style=vertex] (11) at (0.5,-2) {};
\node [style=vertex] (12) at (1.5,-2) {};
\node [style=vertex] (13) at (2,-2) {};
\node [style=vertex] (14) at (3,-2) {};
\node [style=vertex] (15) at (3.5,-2) {};

\node (16) at (0.25,1) {$f(a)$};
\node (17) at (-1.6,-2.4) {$f(b)$};
\node (18) at (3.25,0) {$f(c)$};
\node (19) at (1.4,-2.4) {$f(d)$};
\node (20) at (3.6,-2.4) {$f(e)$};

\foreach \i/\j in {1/2,1/3,2/4,2/5,3/6,3/7,4/8,4/9,5/10,5/11,6/12,6/13,7/14,7/15}
  \draw [style=edge] (\i) to (\j);
\node [below of=19,xshift=-0.25cm] {\parbox{0.3\linewidth}{\subcaption{}}};
\end{scope}

\begin{scope}[xshift=6cm,scale=1]
\node [style=cotreenode] (1) at (1,1) {0};
\node [style=vertex] (2) at (0,0) {};
\node [style=cotreenode] (3) at (2,0) {1};
\node [style=vertex] (4) at (1.5,-1) {};
\node [style=vertex] (5) at (2.5,-1) {};

\node (6) at (0.5,1) {$a$};
\node (7) at (-0.3,0) {$b$};
\node (8) at (2.5,0) {$c$};
\node (9) at (1.5,-1.3) {$d$};
\node (10) at (2.5,-1.3) {$e$};

\foreach \i/\j in {1/2,1/3,3/4,3/5}
  \draw [style=edge] (\i) to (\j);
\node [below of=9,xshift=-0.25cm] {\parbox{0.3\linewidth}{\subcaption{}}};
\end{scope}

\end{tikzpicture}
\end{center}
\setlength{\abovecaptionskip}{-10pt}
\caption{El coárbol (b) es subcoárbol binario del coárbol (a). Las etiquetas en los nodos de ambos coárboles binarios indican la asignación de los nodos de (b) a los nodos de (a).}\label{fig_subcoarbol01}
\end{figure}

\begin{lemma}
Sean $G$ y $H$ cográficas, $T_G$ un coárbol binario de $G$ y $T_H$ un coárbol binario de $H$. Si $T_H$ es subcoárbol binario de $T_G$, entonces $H$ es una subgráfica inducida de $G$.
\end{lemma}

\begin{proof}
Sean $h_1$ y $h_2$ hojas diferentes de $T_H$ y $h_3$ el ancestro común más profundo
de $h_1$ y $h_2$. Como $T_H$ es subcoárbol binario de $T_G$, entonces existe una
función de coasignación, $f$, de $T_H$ a $T_G$. Luego, tenemos que $f(h_1)$ y
$f(h_2)$ son hojas de $T_G$ y que las etiquetas de $h_3$ y $f(h_3)$ coinciden. Así,
$h_1$ y $h_2$ son adyacentes en $H$ si y sólo si $f(h_1)$ y $f(h_2)$ son adyacentes
en $G$. Luego, $G[f[V(H)]]$ es una subgráfica indicida de $G$ que es isomorfa a $H$.
Así, $H$ es una subgráfica inducida de $G$.
\end{proof}


\begin{lemma}
Sean $G$ y $H$ cográficas y $B_G$ un coárbol binario de $G$. Si $H$ es una subgráfica inducida de $G$, entonces existe un coárbol binario $B_H$ de $H$ tal que $B_H$ es subcoárbol binario de $B_G$.
\end{lemma}

\begin{proof}
Como $H$ es una subgráfica inducida de $G$, $V(H)$ es un subconjunto de las hojas de $B_G$. Consideremos el siguiente proceso recursivo para crear el árbol $B_H$.

\begin{algorithm}[H]
\DontPrintSemicolon
\KwIn{$V(H)$}
\KwOut{$r$, la raíz de $B_H$}
    \If{$V(H)$ tiene un solo elemento}{
        $r\gets v$, el único elemento de $V(H)$\;
    }
    \Else{
        $v\gets$ el nodo más profundo de $B_G$ que es ancestro de cada elemento de $V(H)$\;
        $r\gets$ nuevo nodo de coárbol binario con la misma etiqueta que $v$\;
        Asignar al hijo izquierdo de $r$ el resultado de procesar recursivamente los elementos de $V(H)$ que estén en la rama izquierda de $v$\;
        Asignar al hijo derecho de $r$ el resultado de procesar recursivamente los elementos de $V(H)$ que estén en la rama derecha de $v$\;
    }
    \Return $r$,
\end{algorithm}

Notemos que cada nodo de $B_H$ se construye tomando como base a un nodo de $B_G$. Esto ocurre en la línea 2 si dicho nodo es una hoja o en la línea 4 si es un nodo interno. Sea $x$ un nodo de $B_H$, denotamos por $x'$ al nodo de $B_G$ que se toma como base para construir a $x$.
% Aquí hay un abuso del lenguaje que puede confundir al lector, cuando
% usas $x$ para referirte al nodo y al coárbol con raíz $x$.

Sea $x$ un nodo de $B_H$, notemos lo siguiente. Si $x$ es una hoja de $B_H$, entonces $x$ es un vértice de $H$, por lo que $x = x'$, y si $x$ es un nodo interno, entonces la etiqueta de $x$ es la misma que la etiqueta de $x'$. Además, el hijo derecho de $x$ es ancestro de los mismos vértices de $H$ que el hijo derecho de $x'$. Análogamente para el hijo izquierdo. De esto se sigue que para cualesquiera dos vértices de $H$, su ancestro común más profundo en $B_H$ tiene la misma etiqueta que su ancestro común más profundo en $B_G$.

Veamos que $B_H$ es un coárbol binario de $H$. Notemos que las hojas de $B_H$ son todos los vértices de $H$. Dado que $H$ es una subgráfica inducida de $G$, cualesquiera dos vértices de $H$ son adyacentes si y sólo si su ancestro común más profundo en $B_G$ tiene etiqueta 1. Luego, dos vértices de $H$ son adyacentes si y sólo si su ancestro común más profundo en $B_H$ tiene etiqueta 1. Así, $B_H$ es un subcoárbol binario de $H$.

Sea $f$ un subconjunto de $V(B_H) \times V(B_G)$ tal que $(x,x')\in f$ para cualquier nodo $x$ de $V(B_H)$, veamos que $f$ es una función de coasignación de $B_H$ a $B_G$. Como cada pareja en $f$ tiene como primer elemento a un nodo único de $B_H$ y como segundo elemento un nodo único de $B_G$, $f$ es una función inyectiva. Sean $x_1$, $x_2$ y $x_3$ nodos de $B_H$. Es claro que si $x$ es una hoja, entonces $f(x)=x'$ es una hoja. Y si $x$ es un nodo interno, entonces $f(x)=x'$ tiene la misma etiqueta que $x$. Si $x_3$ es el ancestro común más profundo de $x_1$ y $x_2$, notemos lo siguiente:
\begin{itemize}
    \item el conjunto de vértices de $H$ que son descendientes de $x_1$ es igual al conjunto de vértices de $H$ que son descendientes de $f(x_1)$.
    \item el conjunto de vértices de $H$ que son descendientes de $x_2$ es igual al conjunto de vértices de $H$ que son descendientes de $f(x_2)$.
    \item el hijo derecho de $x_3$ es ancestro de los mismos vértices de $H$ que el hijo derecho de $f(x_3)$.
\end{itemize}
Supongamos sin pérdida de generalidad que $x_1$ es descendiente del hijo izquierdo de $x_3$ y que $x_2$ es descendiente de su hijo derecho. Luego,  $f(x_1)$ es descendiente del hijo izquierdo de $f(x_3)$ y $f(x_2)$ es descendiente de su hijo derecho. Así, $f(x_3)$ es el ancestro común más profundo de $f(x_1)$ y $f(x_2)$. Luego, $f$ es una función de coasignación de $B_H$ a $B_G$.

\end{proof}

    \subsection{Algoritmo para encontrar obstrucciones mínimas} \label{sec_AlgoSub}
        La presente sección aborda el problema de determinar si una cográfica $G$ tiene
a otra cográfica $H$ como subgráfica inducida haciendo uso de los conceptos de
coárbol binario y subcoárbol. Se proporciona un algoritmo (Algoritmo
\ref{alg_subgraph}) para resolver este problema tal que, si se fija el tamaño de
$H$, su tiempo de ejecución crece de forma lineal con respecto al tamaño de $G$.
Este algoritmo es útil para identificar a las gráficas pertenecientes a una clase 
caracterizada a través de su conjunto de obstrucciones mínimas de forma rápida.

\subsubsection{Algoritmo para determinar si un coárbol binario es subcoárbol binario de otro}

\begin{definition}
    Sean $T$ y $U$ coárboles (binarios) y $u$ un nodo de $U$, decimos que
    $f:V(U)\rightarrow\{marcado, no\_marcado\}$ es una \textbf{\emph{función de
    verificación}} de $T$ para $U$ si $f(u) = marcado$ si y sólo si el coárbol
    (binario) con raíz $u$ es subcoárbol (binario) de $T$. Si $f(u) = marcado$,
    decimos que $f$ \textbf{\emph{marca}} a $u$.
\end{definition}

El Algoritmo \ref{alg_subcoarbol} recibe como entradas dos coárboles binarios,
$G$ y $H$ representados por sus raíces $g$ y $h$ respectivamente, y devuelve una
función de verificación, $f_g$, de $G$ para $H$.

Este algoritmo funciona creando la función de verificación de cada subárbol de $G$ para $H$, empezando aquellos cuya raíz es más profunda. De esta manera, si la función de verificación de $G$ para $H$ evaluada en $h$ es $marcado$, entonces  $H$ es subcoárbol de $G$.

\begin{algorithm}[ht!]
\caption{Función\_de\_coasignación}
\label{alg_subcoarbol}
\DontPrintSemicolon % Some LaTeX compilers require you to use \dontprintsemicolon instead
\KwIn{$g$ y $h$, las raíces de dos coárboles binarios para las gráficas $G$ y $H$ respectivamente}
\KwOut{$func$, la función de verificación de $G$ para $H$}

 $func \gets \text{nueva función de coasignación tal que} func(x)=no\_marcado \text{ para todo } x\in V(H)$\;

 \If{$g$ es una hoja}{
    $func \text{ marca a todas las hojas de } H$\;
 }
 \Else{
    $v_{izq} \gets \text{Función\_de\_coasignación}(g.izquierda, h)$\;
    $v_{der} \gets \text{Función\_de\_coasignación}(g.derecha, h)$\;

    \ForEach{nodo \textbf{\emph{de}} H}{
        \If{$v_{izq}(nodo) = marcado \emph{ \textbf{o} } v_{der}(nodo) = marcado$}{
            $func(nodo) \gets marcado$\;
        }
        \ElseIf{nodo.etiqueta = g.etiqueta \emph{\textbf{y}} $v_{izq}$ \emph{marca a uno de los hijos de} nodo \emph{y} $v_{der}$ \emph{al otro}}{
            $func(nodo) \gets marcado$\;
        }
    }

 }

$\Return func$

\end{algorithm}

\begin{theorem}
    La ejecución del Algoritmo \ref{alg_subcoarbol}, \emph{Función\_de\_coasignación($g$, $h$)} regresa una función, $func$, tal que $func$ es una función de verificación de $\acute{a}rbol(g)$ para $\acute{a}rbol(h)$.
\end{theorem}

\begin{proof}

    Sea $n$ un nodo de $\acute{a}rbol(h)$. Para probar que $func$ es una función
    de verificación de $\acute{a}rbol(g)$ para $\acute{a}rbol(h)$, tenemos que
    probar que $func(n) = marcado$ si y sólo si $\acute{a}rbol(n)$ es subcoárbol
    de $\acute{a}rbol(g)$.

    \textbf{Necesidad}: En esta parte de la demostración, se supone que el algoritmo ha sido ejecutado y que $func$ marca a $n$. Procedamos por inducción sobre la altura de $g$.

    \emph{Caso base:} Si $g$ tiene altura 0, entonces $g$ es una hoja, por lo
    que $func$ marca únicamente a las hojas de $\acute{a}rbol(h)$. Como $func$
    marca a $n$, entonces $n$ es una hoja. Luego, la función $f=\{(n,g)\}$ es
    una función de coasignación de $\acute{a}rbol(n)$ a $\acute{a}rbol(g)$, por
    lo que $\acute{a}rbol(n)$ es subcoárbol de $\acute{a}rbol(n)$.

    \emph{Paso inductivo:} Si $g$ tiene altura $k > 0$. Supongamos como
    hipotesis inductiva (H.I.) que, para todo nodo de un coárbol binario, $g'$,
    de altura $k' < k$ se cumple que, si $func' = $
    Función\_de\_coasignación$(g',h)$ marca a un nodo $n'$ de
    $\acute{a}rbol(h)$, entonces $\acute{a}rbol(n')$ es subcoárbol de
    $\acute{a}rbol(g')$. Como $g$ no es una hoja, el algoritmo debió de entrar
    al bloque de instrucciones de las líneas 5 a 11. En las líneas 5 y 6 se
    crean dos funciones que cumplen con la H.I., ya que $g.izquierda$ y
    $g.derecha$ tienen ambas una altura menor a $k$. Como $func$ marca a $n$,
    entonces $n$ debe de cumplir la condición de la línea 8 o la condición de la
    línea 10. Si se cumple la condición de la línea 8, entonces $v_{izq}(n) =
    marcado$ o $v_{der}(n) = marcado$, por lo que $\acute{a}rbol(n)$ es
    subcoárbol de $\acute{a}rbol(g.izquierda)$ o de $\acute{a}rbol(g.derecha)$,
    y por lo tanto es subcoárbol de $\acute{a}rbol(g)$. De lo contrario, se
    cumple la condición de la línea 10, entonces $v_{izq}$ marca a $n.izquierda$
    o a $n.derecha$ y $v_{der}$ marca al otro. Supongamos sin pérdida de
    generalidad que $v_{izq}$ marca a $n.izquierda$ y $v_{der}$ marca a
    $n.derecha$. Sean $f_i:V(\acute{a}rbol(n.izquierda))\rightarrow
    V(\acute{a}rbol(g.izquierda))$ la función de coasignación de
    $\acute{a}rbol(n.izquierda)$ a $\acute{a}rbol(g.izquierda)$ y
    $f_d:V(\acute{a}rbol(n.derecha))\rightarrow V(\acute{a}rbol(g.derecha))$ la
    función de coasignación de $\acute{a}rbol(n.derecha)$ a
    $\acute{a}rbol(g.derecha)$, mostremos que la función $f = f_i \cup f_d \cup
    \{(n,g)\}$ es una función de coasignación de $\acute{a}rbol(n)$ a
    $\acute{a}rbol(g)$. Como los dominios de $f_i$ y $f_d$ son ajenos y ninguno
    contiene a $n$, entonces $f$ es una función. Como los rangos de $f_i$ y
    $f_d$ son ajenos, ninguno contiene a $g$ y tanto $f_i$ como $f_d$ son
    inyectivas, entonces $f$ es inyectiva. Por otra parte, por la condición de
    la línea 10, sabemos que $n.etiqueta = g.etiqueta$. También sabemos que, sea
    $x \in V(\acute{a}rbol(n.izquierda))$, si $x$ es una hoja, entonces $f(x) =
    f_i(x)$ es una hoja y si no, entonces $x.etiqueta = f_i(x).etiqueta =
    f(x).etiqueta$. Análogamente para un $y \in V(\acute{a}rbol(n.derecha))$ y
    $f_d$. Finalmente, si $n$ es el ancestro común más profundo de dos nodos
    $z_1$ y $z_2$, entonces $z_1$ es descendiente de $n.derecha$ y $z_2$ es
    descendiente de $n.izquierda$ o viceversa. Supongamos lo primero sin pérdida
    de generalidad. Luego, por la condición de la línea 10, $v_{izq}$ marca a
    uno y $v_{der}$ marca al otro. Supongamos sin pérdida de generalidad que
    $v_{izq}$ marca a $z_1$ y $v_{der}$ marca a $z_2$. Entonces, $f(z_1) =
    f_i(z_1) \in V(\acute{a}rbol(g.izquierda))$ y $f(z_2) = f_i(z_2) \in
    V(\acute{a}rbol(g.derecha))$, por lo que el ancestro común más profundo de
    $f(z_1)$ y $f(z_2)$ es $g = f(n)$. Así, $f$ es una función de coasignación
    de $\acute{a}rbol(n)$ a $\acute{a}rbol(g)$ y $\acute{a}rbol(n)$ es
    subcoárbol de $\acute{a}rbol(g)$.

     \textbf{Suficiencia}: En esta parte de la demostración se supone que
     $\acute{a}rbol(n)$ es subcoárbol de $\acute{a}rbol(g)$ y se sigue la
     ejecución del algoritmo para mostrar que, al final de la misma, $func$
     marcará a $n$. Sea $f$ la función de cosignación de $\acute{a}rbol(n)$ a
     $\acute{a}rbol(g)$, procedamos por inducción sobre la altura de $g$.

    \emph{Caso base:} Si la altura de $g$ es 0, entonces $g$ es una hoja, por lo
    que se cumple con la condición de la línea 2 y se ejecuta la línea 3,
    haciendo que $func$ marque todas las hojas de $H$. Como $\acute{a}rbol(n)$
    es subcoárbol de $\acute{a}rbol(g)$ y $\acute{a}rbol(g)$ sólo tiene un nodo,
    entonces $n$ debe de ser una hoja. Luego, $func$ marca a $n$.

    \emph{Paso inductivo:} Si $g$ tiene altura $k > 0$. Supongamos como H.I. que
    todo coárbol, $g'$, con altura $k' < k$ cumple con que, siendo $n'$ un nodo
    de $\acute{a}rbol(h)$, si $\acute{a}rbol(n')$ es subcoárbol de
    $\acute{a}rbol(g')$, entonces $func'=$Función\_de\_coasignación $(g',h)$
    marca a $n'$. Como $g$ no es una hoja, el algoritmo ejecuta las líneas 5 y 6
    y posteriormente el bloque de las líneas 8 a 11 para cada nodo de $H$. Si
    $n$ es marcada por $v_{izq}$ o $v_{der}$, entonces se ejecuta la línea 9 y
    $func$ marca a $n$. En el caso contrario, probemos que se cumple la
    condición de la línea 10. Mostremos primero que $f(n) = g$ procediendo por
    contradicción. Supongamos que $f(n) = x$ para algún $x\in
    V(\acute{a}rbol(g))-\{g\}$. Como $x$ es descendiente de $g$, tiene altura
    menor a $k$. También sabemos que $f$ es una función de coasignación de
    $\acute{a}rbol(n)$ a $\acute{a}rbol(x)$, por lo que, por H.I., $n$ debería
    de ser marcado ya sea por $v_{izq}$ o por $v_{der}$, lo que es una
    contradicción. Luego, $f(n) = g$, y por lo tanto $n$ no es una hoja y
    $f(n).etiqueta = g.etiqueta$. Mostremos ahora que tanto $n.izquierda$ como
    $n.derecha$ son marcados cada uno ya sea por $v_{izq}$ o por $v_{der}$.
    Sabemos que $f(n.izquierda)$ y $f(n.derecha)$ son descendientes de $r$. Como
    $f\mid_{V(\acute{a}rbol(n.izquierda))}$ es una función de coasignación de
    $\acute{a}rbol(n.izquierda)$ a $\acute{a}rbol(f(n.izquierda))$ y $f(n.izquierda)
    \neq g$ ya que $f$ es inyectiva, entonces $\acute{a}rbol(n.izquierda)$ es
    subcoárbol de algún descendiente de $g$, al que llamaremos $y$. Como $y$
    tiene altura menor a $k$, su función de verificación correspondiente marca a
    $n.izquierda$ (por H.I.), y por la condición de la línea 8, sus ancestros
    también lo marcan. Luego $v_{izq}$ o $v_{der}$ marcan a $n.izquierda$.
    Análogamente para $n.derecha$. Así, tanto $n.izquierda$ como $n.derecha$
    están marcados cada uno ya sea en $v_{izq}$ o en $v_{der}$. Mostremos, por
    último, que uno es marcado por $v_{izq}$ y el otro es marcado por $v_{der}$.
    Como el ancestro común más profundo de $n.izquierda$ y $n.derecha$ es $n$, y
    $f(n)=g$, entonces el ancestro común más profundo de $f(n.izquierda)$ y
    $f(n.derecha)$ debe de ser $g$. Luego, $f(n.izquierda)$ está en una rama de
    $g$ y $f(n.derecha)$ está en la otra. Supongamos sin pérdida de generalidad
    que $f(n.izquierda)$ está en la rama izquierda de $g$ y $f(n.derecha)$ está
    en la rama derecha. Como $f\mid_{V(\acute{a}rbol(n.izquierda))}$ es una
    función de coasignación de $\acute{a}rbol(n.izquierda)$ a
    $\acute{a}rbol(g.izquierda)$ y por H.I., entonces $v_{izq}$ marca a
    $n.izquierda$. De forma análoga, $v_{der}$ marca a $n.derecha$. Concluyendo,
    como $n.etiqueta = r.etiqueta$ y tanto $n.izquierda$ como $n.derecha$ son
    marcados uno por $v_{izq}$ y el otro por $v_{der}$, se cumple la condición
    de la línea 10 y $func$ marca a $n$. Así, al final de la ejecución del
    algoritmo, $n$ estará marcado.

\end{proof}

Dado que, para cada nodo de $G$, se crea una función de verificación cuyo
dominio es el conjunto de los nodos de $H$, el tiempo de ejecución del algoritmo
crece de la forma $O(\mid V(G) \mid \mid V(H) \mid)$. 

\subsubsection{Determinar si una cográfica es subcográfica de otra}

Haciendo uso del Algoritmo \ref{alg_subcoarbol}, se puede idear otro algoritmo
para determinar si una cográfica, $H$ es subgráfica de otra cográfica, $G$, al
buscar todas las formas del coárbol binario de $H$ en un solo coárbol binario de
$G$.

\begin{algorithm}[ht!]
\caption{Es\_subgráfica}
\label{alg_subgraph}
\DontPrintSemicolon % Some LaTeX compilers require you to use \dontprintsemicolon instead
\KwIn{$g$ y $h$, las raíces de dos coárboles, $G$ y $H$ respectivamente.}
\KwOut{$verdadero$ si la cográfica representada por $H$ es subgráfica de la cográfica representada por $G$. $falso$ en el caso contrario.}

$g\_bin \gets \text{CrearÁrbolBinario}(g)$\;
$h\_bins \gets$ las raíces de todos los coárboles binarios correspondientes a $H$\;

\ForEach{bin \textbf{\emph{en}} h\_bins}{
    $f = \text{Función\_de\_coasignación}(g\_bin,bin)$\;
    \If{f(bin) = marcado}{
        $\Return\ verdadero$\;
    }
}

$\Return\ falso$\;

\end{algorithm}

Como la línea 1 Algoritmo \ref{alg_subgraph} se ejecuta en tiempo $O(\mid V(G)
\mid)$, la complejidad temporal de éste depende del número de coárboles binarios
correspondientes a $H$ (que crece con mayor rapidez). Sin embargo, si se fija
$H$, la complejidad temporal de éste es simplemente  $O(\mid V(G) \mid)$. Fijar
$H$ resultará útil cuando se esté resolviendo un problema específico como el de
encontrar una obstrucción mínima en una gráfica. La aplicación de este algoritmo
en el presente trabajo de tesis es desarrollar algoritmos que nos permitan
identificar a los elementos de una clase hereditaria de cográficas en tiempo
lineal. Esto se muestra en el Algoritmo \ref{alg_esta_en_clase}. 

\begin{algorithm}[ht!]
\caption{Pertenece_a_la_clase}
\label{alg_esta_en_clase}
\DontPrintSemicolon % Some LaTeX compilers require you to use \dontprintsemicolon instead
\KwIn{$g$, la raíz del coárbol de una cográfica $G$.}
\KwOut{$verdadero$ si $G$ pertenece a la clase hereditaria de cográficas $C$}

$g\_bin \gets \text{CrearÁrbolBinario}(g)$\;
$C\_bins \gets$ las raíces de todos los coárboles binarios de todas las obstrucciones mínimas de la clase $C$\;

\ForEach{$bin$ \textbf{\emph{en}} $C\_bins$}{
    $f = \text{Función\_de\_coasignación}(g\_bin,bin)$\;
    \If{$f(bin) = marcado$}{
        $\Return\ verdadero$\;
    }
}

$\Return\ falso$\;

\end{algorithm}

El Algoritmo \ref{alg_esta_en_clase} es una variación del Algoritmo
\ref{alg_subgraph} en el que se fija una clase $C$ cuyos elementos se desea
identificar. Al computar el conjunto de coárboles binarios de cada una de las
obstrucciones mínimas de $C$ antes de la ejecución del algoritmo
\ref{alg_esta_en_clase}, la línea 2 del algoritmo se puede ejecutar en tiempo
constante. Gracias a esto obtenemos un algoritmo que determina si una gráfica
$G$ pertenece a la clase $C$ en tiempo $O(\mid V(G) \mid)$.



\section{La clase $M_2$}

    A partir de esta sección abordamos el problema principal de la tesis, determinar si una cográfica acepta una partición en un número dado de gráficas multipartitas completas. Empezamos por dar nombre a las clases de cográficas que estudiamos, y procedemos con el análisis de una de estas clases que nos servirá de base para estudiar el resto. 
    
    \begin{definition}
        Dado un entero $i\geq 1$, la \textbf{\emph{clase $M_i$}} es la clase de cográficas que contiene una gráfica $G$ si y sólo si $V(G)$ acepta una partición $(A_1, A_2, \dots, A_i)$ tal que $G[A_j]$ es una gráfica multipartita completa para cada $1\leq j \leq i$.  Decimos que $(A_1, A_2, \dots, A_i)$ es una $M_i$-partición de $G$.
    \end{definition}

    Dado que las gráficas multipartitas completas conforman una clase hereditaria de gráficas, es claro toda clase $M_i$ es una clase hereditaria de gráficas, pues si tomamos uno de sus elementos $G$ que tiene una $M_i$-partición $(A_1, A_2, \dots, A_i)$ y sustraemos uno de sus vértices $x$, dicho vértice será elemento de algún $A_j$ con $1\leq j \leq i$. Como $G[A_j]-\{x\}$ sigue siendo una gráfica multipartita completa, entonces $G-\{x\}$ sigue siendo un elemento de $M_i$. 
    
    Dado que toda clase $M_i$ es una clase hereditaria de gráficas, cada una de éstas puede se caracterizadas por un conjunto de obstrucciones mínimas y se puede identificar a sus elementos en tiempo lineal con el Algoritmo \ref{alg_esta_en_clase}. Una característica de estas clases que nos será de utilidad es que son cerradas bajo la unión completa (\ref{lema_union_completa}). 
    
    \begin{lemma}
    \label{lema_union_completa}
    Sea $i$ un entero mayor o igual a 2, la clase $M_i$ es cerrada bajo la unión completa.
    \end{lemma}
    Sean $G,H\in M_i$, sabemos que existen $M_i$-particiones $(A_1,A_2,\dots, A_i)$ y $(B_1,B_2, \dots, B_i)$ de $G$ y $H$ respectivamente. Mostremos que $P = (A_1\cup B_1, A_2\cup B_2, \dots, A_i \cup B_i)$ es una $M_i$-partición de $G\oplus H$.
    \begin{proof}
    Claramente, cada uno de los vértices de $G \oplus H$ se encuentra en exactamente uno de los elementos de $P$. Así, $P$ es una partición de los vértices de $G \oplus H$. Luego, como la unión completa de gráficas multipartitas completas es una gráfica multipartita, tenemos que $(G\oplus H)[A_j\cup B_j] = G[A_j]\oplus G[B_j]$ es una gráfica multipartita completa para cualquier $1\le j \le i$. Así, $P$ es una $M_i$ partición de $G \oplus H$ y $G \oplus H\in M_i$.
    \end{proof}
    
    Notemos que la clase $M_1$ es la clase de las gráficas multipartitas completas. En la presente sección nos enfocaremos en la clase $M_2$, que es la más pequeña de las clases $M_i$ después de $M_1$ (que ha sido ampliamente estudiada). Caracterizamos a $M_2$ a través de su conjunto de obstrucciones mínimas, proporcionamos un algoritmo de tiempo lineal para identificar a sus elementos y presentamos un algoritmo certificador que no sólo determina si una gráfica $G$ pertenece a $M_2$, sino que colorea las hojas de su coárbol indicando si se encontró una $M_2$-partición de $G$ o una obstrucción mínima de la clase como subgráfica inducida de $G$.

    \subsection{Obstrucciones mínimas}
        \begin{theorem} \label{teo_obsts_m2}

    Para una cográfica $G$, las siguientes afirmaciones son equivalentes.
    \begin{enumerate}[(a)]
        \item $G \in M_2$.
        \item $G$ no contiene a ninguna de las gráficas de las Figuras \ref{obsts_O_M3} como subgráficas inducidas. $H, I$ ni a $J$ como subgráficas inducidas.
    \end{enumerate}

\end{theorem}

\begin{figure}[ht!]
\begin{center}
\begin{tikzpicture}

\begin{scope}[xshift=0cm,scale=1]

\node [style=vertex] (1) at (0,0) {};
\node [style=vertex] (2) at (1,0) {};
\node [style=vertex] (3) at (0,0.5) {};
\node [style=vertex] (4) at (1,0.5) {};
\node [style=vertex] (5) at (0.5,1.25) {};
\node [style=vertex] (6) at (0.5,2) {};
\foreach \i/\j in {1/2,3/4,3/5,4/5}
  \draw [style=edge] (\i) to (\j);
\node [below of=1,xshift=.5cm]
{\parbox{0.3\linewidth}{\subcaption*{$H$}}};

\end{scope}

\begin{scope}[xshift=3cm,scale=1]

\node [style=vertex] (1) at (0,0) {};
\node [style=vertex] (2) at (1,0) {};
\node [style=vertex] (3) at (0,0.5) {};
\node [style=vertex] (4) at (1,0.5) {};
\node [style=vertex] (5) at (0.5,1.25) {};
\node [style=vertex] (6) at (0.5,2) {};
\foreach \i/\j in {1/2,3/4,3/5,4/5,5/6}
  \draw [style=edge] (\i) to (\j);
\node [below of=1,xshift=.5cm]  {\parbox{0.3\linewidth}{\subcaption*{$I$}}};

\end{scope}

\begin{scope}[xshift=6cm,scale=1]

\node [style=vertex] (1) at (0,0) {};
\node [style=vertex] (2) at (0.5,0.5) {};
\node [style=vertex] (3) at (1.5,0.5) {};
\node [style=vertex] (4) at (0.5,1.5) {};
\node [style=vertex] (5) at (1.5,1.5) {};
\node [style=vertex] (6) at (0,2) {};
\node [style=vertex] (7) at (2,1) {};

\foreach \i/\j in {1/2,1/3,1/6,2/3,2/4,2/5,3/4,3/5,4/5,4/6,5/6}
  \draw [style=edge] (\i) to (\j);
\node [below of=1,xshift=1cm] {\parbox{0.3\linewidth}{\subcaption*{$J$}}};

\end{scope}
\end{tikzpicture}
\end{center}
\setlength{\abovecaptionskip}{-15pt}
\caption{Obstrucciones mínimas para la clase $M_2$.}
\label{obsts_O_M3}
\end{figure}

\begin{proof}

    Notemos que las subgráficas $H, I$ y $J$ pueden ser descritas de la siguiente manera:

    \begin{enumerate}[(1)]
        \item $I = K_1 + K_2 + K_3$.
        \item $H = Paw + K_2$.
        \item $J = (\overline{P_3} \oplus \overline{P_3}) + K_1$.
    \end{enumerate}

    \textbf{\emph{Necesidad}}: Dado que $M_2$ es una clase hereditaria, todas las subgráficas inducidas en $G$ deben de estar en $M_2$. Procedamos por contrapositiva mostrando que ni $H$ ni $I$ ni $J$ están en $M_2$. Si ambos vértices del $K_2$ en $H$ se encuentran en la misma parte, entonces no puede haber ningún vértice adicional en dicha parte, o ésta contendría un $\overline{P_3}$. Como los vértices restantes inducen una gráfica que no es multipartita completa, esto no puede suceder. Así, un vértice del $K_2$ debe estar en una parte y el otro en la otra. Como son vértices independientes, cada parte debe de ser un conjunto independiente. Como la gráfica inducida por los vértices restantes contiene un $K_3$, no es bipartita y no se puede dividir en dos conjuntos independientes. Así, $H$ no pertenece a $M_2$. Análogamente, se muestra que $I$ no está en $M_2$. Por otra parte, como $J$ tiene un vértice aislado, el resto de sus vértices (mismos que forman un $\overline{P_3} \oplus \overline{P_3}$) deben de poder dividirse en dos partes de manera tal que una induzca un conjunto independiente y la otra una gráfica multipartita completa. Siempre que tomamos uno de los vértices de uno de los dos $\overline{P_3}$ para agregarlo al conjunto independiente, ninguno los vértices del otro $\overline{P_3}$ puede ser agregado al conjunto independiente, pues es adyacente al vértice que agregamos primero. Así, la subgráfica inducida $\overline{P_3} \oplus \overline{P_3}$ no acepta una partición en un conjunto independiente y una gráfica multipartita completa. Luego, $J$ no está en $M_2$.

    \textbf{\emph{Suficiencia}}: Consideramos los siguientes casos.

    \emph{Caso 1:} $G$ tiene al menos dos componentes conexas no triviales.

    Consideremos la partición de $V = (A,B)$ tal que $A$ contiene únicamente una componente no trivial y $B$ el resto. Como $G[A]$ y $G[B]$ contienen ambas componentes no triviales, las dos poseen un $K_2$. Luego, ni $G[B]$ ni $G[A]$ contienen un $Paw$, o $G$ tendría a $I$ como subgráfica inducida. Dado que $G[A]$ y $G[B]$ son cográficas, son también gráficas perfectas, y al ninguna tener un $Paw$ como subgráfica inducida, cada una es bipartita o multipartita completa. Si ambas son multipartitas, entonces $G \in M_2$.

    Si ambas son bipartitas, entonces $G$ es bipartita también y acepta una partición en dos conjuntos independientes, cada uno de los cuales es una gráfica multipartita completa, por lo que $G \in M_2$.

    Si $G[A]$ es bipartita y $G[B]$ es multipartita completa, como $G[A]$ es una cográfica conexa, entonces es una gráfica multipartita completa y $G \in M_2$.

    Finalmente, si $G[A]$ es multipartita completa y $G[B]$ es bipartita. Si $G[B]$ tiene una sola componente, $G[B]$ es bipartita completa y $G \in M_2$. Si $G[B]$ tiene más de una componente, debe tener a $\overline{P_3}$ como subgráfica inducida. Luego, $G[A]$ debe ser libre de $K_3$ o $G$ tendría a $H$ como subráfica inducida. Así, $G[A]$ es bipartita. Como ambas son bipartitas, $G \in M_2$.

    \emph{Caso 2:} $G$ tiene exactamente una componente conexa no trivial y al menos una trivial.

    Como $G$ contiene al menos una componente trivial, la única partición que puede aceptar en dos gráficas multipartitas completas es una partición en un conjunto independiente y una gráfica multipartita completa. Luego, la componente no trivial de $G$, a la que llamaremos $G'$, debe de aceptar una partición en un conjunto independiente y una gráfica multipartita completa.

    Si $G'$ es bipartita, entonces acepta una partición en dos conjuntos independientes, y por lo tanto $G \in M_2$. Si $G'$ es una gráfica multipartita completa, $G \in M_2$. Si $G$ no es una gráfica bipartita ni multipartita completa, dado que es una cográfica, y por lo tanto una gráfica perfecta, contiene un $Paw$. Sea $y$ la raíz del coárbol de $G'$ y sea $z$, descendiente de $y$, el nodo más profundo que tiene un $Paw$ como subgráfica inducida, probemos por inducción sobre la distancia desde $y$ hasta $z$, $d$, que $G'[y]$ acepta una partición en un conjunto independiente y una gráfica multipartita completa.

    \textbf{Caso base}: $d = 0$. O bien, $y = z$.

    Notemos que $z$ tiene etiqueta 1, pues $Paw$ es una gráfica conexa. Dado que $z$ tiene etiqueta 1, todos sus hijos inducen gráficas multipartitas completas menos uno, $w$, que tiene etiqueta 0. Mostremos por contradicción que todos los hijos de $w$ inducen gráficas multipartitas completas. Supongamos que alguno de los hijos de $w$ contiene un $\overline{P_3}$. Como el nodo más profundo que contiene un $\overline{P_3}$ debe tener etiqueta 0, $w$ tiene un hijo de etiqueta 1 que tiene al menos 2 hijos, uno de los cuales contiene al $\overline{P_3}$ y el otro que tiene al menos un $K_1$. Luego, dicho hijo contiene un $Paw$, lo que es una contradicción. Si $w$ tiene un sólo hijo que no es un vértice, el resto de sus hijos forman un conjunto independiente, $C$. Si eliminamos este conjunto independiente, el único hijo de $w$ induce una gráfica multipartita completa, luego $G'[w] - C$ es una gráfica multipartita completa, entonces $G'[z] - C$ es una unión completa de de gráficas multipartitas completas y por lo tanto una gráfica multipartita completa. De esto se sigue que $G[z]$ acepta una partición en un conjunto independiente, $C$, y una gráfica multipartita completa. Si $w$ tiene al menos dos hijos no triviales, notemos que ninguno de ellos puede contener a $K_3$, o de lo contrario $w$ contendría a $K_2+K_3$ y $G$ no sería libre de $I$. Luego, todos los hijos de $w$ han de inducir gráficas bipartitas, es decir que $w$ induce también una gráfica bipartita. En otras palabras, $G'[w]$ acepta una partición en dos conjuntos independientes. Si sustraemos uno de estos conjuntos independientes, denoatdo como $D$, entonces $G'[w]-D$ es un conjunto independiente. Luego $G'[z]-D$ es la unión completa de gráficas multipartitas completas y un conjunto independiente, así $G'[z]-D$ es una gráfica multipartita completa. Luego, $G'[z]$ acepta una partición en un conjunto independiente, $D$ y una gráfica multipartita completa, $G'[z] - D$.

    Como en todos los casos $z$ acepta una partición en un conjunto independiente y una gráfica multipartita completa y $y = z$, entonces $y$ acepta la misma partición.

    \textbf{Paso inductivo}: $d \geq 2$.

    Notemos que $d$ siempre será par, ya que tanto $y$ como $z$ son nodos con etiqueta 1. Sea $k$ un entero tal que $k \geq 2$. Supongamos, como hipótesis inductiva, que si $G''$ es una cográfica conexa libre de $H, I$ y $J$ tal que la distancia, $d'$, entre la raíz, $y'$ de su coárbol y el nodo más profundo que contiene un $Paw$ es igual a $k-2$, entonces $G''$ acepta una partición en un conjunto independiente y una gráfica multipartita completa.

    Dado que $G'$ es libre de $J$, todos los hijos de $y$ menos uno inducen gráficas multipartitas completas. Dicho hijo, $v$, tiene etiqueta 0 y al menos uno de sus hijos debe de contener un $Paw$. Denotemos a dicho hijo como $u$. El resto de los hijos de $v$ deben de ser vértices, o de lo contrario, $G'[v]$ contendría a $K_2 + K_3$ como subgráfica inducida y por lo tanto $G$ contendría a $I$. Denotemos a este conjunto de vértices como $E$. Luego, $G'[u]$  es una cográfica que cumple con las condiciones de la hipótesis inductiva, por lo que acepta una partición en un conjunto independiente, $D$ y una gráfica multipartita completa. Si eliminamos de $u$ los vértices de $D$, entonces teneos que $G'[u] - D$ es una gráfica multipartita completa. Luego, si eliminamos también los vértices de $E$ de $G'[v]$, tenemos que $G'[v]-D-E$ es una gráfica multipartita completa. Luego, $G'-D-E$ es una unión completa de gráficas multipartitas completas por lo que también es una gráfica multipartita completa. Notemos que, dado que $v$ tiene etiqueta 0, no existen aristas entre los vértices en $D$ y los vértices en $E$, es decir que $D \cup E$ es un conjunto independiente. Así, $G'$ acepta una partición en un conjunto independiente, $D \cup E$ y una gráfica multipartita completa, $G' - D - E$.


    Como $G'$ acepta una partición en un conjunto independiente y una gráfica multipartita completa, entonces $G \in M_2$.


    \emph{Caso 3:} $G$ es un conjunto independiente con al menos dos vértices.

    Dado que $G$ es una gráfica multipartita completa, se sigue inmediatamente que está en $M_2$.

    \emph{Caso 4:} $G$ es conexa.

    Dado que toda cográfica inconexa libre de $H$, $I$ y $J$ acepta una partición en dos gráficas multipartitas completas, una cográfica conexa es o un vértice aislado o una unión de cográficas inconexas y la clase $M_2$ es cerrada bajo la unión completa, $G \in M_2$.


\end{proof}


    \subsection{Reconocimiento de la clase $M_2$}
        Haciendo uso  del Algoritmo \ref{alg_esta_en_clase} se puede determinar si una cográfica pertenece o no a la clase $M_2$. Como se especificó en la Sección \ref{sec_AlgoSub}, el tiempo de este algoritmo crece de forma lineal de acuerdo con el tamaño de la gráfica de entrada si encontramos primero todos los coárboles binarios de las obstrucciones de la clase. Como conocemos las obstrucciones mínimas de la clase $M_2$, que son finitas, se puede buscar cada una en tiempo lineal y por lo tanto se puede reconocer si una cográfica pertenece a la clase $M_2$ en tiempo lineal. El Algoritmo \ref{alg_decision}, que es una instancia del Algoritmo \ref{alg_esta_en_clase} corresponde a este proceso. Los árboles binarios de cada una de las obstrucciones de la clase $M_2$ se muestran en la Figura \ref{fig_obsts_bin}.

\begin{figure}[ht!]
\centering

\begin{subfigure}{0.85\textwidth}
\begin{tikzpicture}

\begin{scope}[xshift=0cm,scale=1]
\node [style=cotreenode] (1) at (1,1) {0};
\node [style=cotreenode] (2) at (0,0) {0};
\node [style=cotreenode] (3) at (2,0) {1};
\node [style=vertex] (4) at (-0.5,-1) {};
\node [style=cotreenode] (5) at (0.5,-1) {1};
\node [style=vertex] (6) at (1.5,-1) {};
\node [style=cotreenode] (7) at (2.5,-1) {1};
\node [style=vertex] (8) at (0.25,-2) {};
\node [style=vertex] (9) at (0.75,-2) {};
\node [style=vertex] (10) at (2.25,-2) {};
\node [style=vertex] (11) at (2.75,-2) {};
\foreach \i/\j in {1/2,1/3,2/4,2/5,3/6,3/7,5/8,5/9,7/10,7/11}
  \draw [style=edge] (\i) to (\j);
\node [below of=9,xshift=0.25cm] {\parbox{0.3\linewidth}{\subcaption*{$H_1$}}};
\end{scope}

\begin{scope}[xshift=4.5cm,scale=1]
\node [style=cotreenode] (1) at (1,1) {0};
\node [style=cotreenode] (2) at (0,0) {0};
\node [style=cotreenode] (3) at (2,0) {1};
\node [style=vertex] (4) at (-0.5,-1) {};
\node [style=cotreenode] (5) at (0.5,-1) {1};
\node [style=vertex] (6) at (1.5,-1) {};
\node [style=vertex] (7) at (2.5,-1) {};
\node [style=vertex] (8) at (0.125,-2) {};
\node [style=cotreenode] (9) at (0.875,-2) {1};
\node [style=vertex] (10) at (0.625,-3) {};
\node [style=vertex] (11) at (1.125,-3) {};
\foreach \i/\j in {1/2,1/3,2/4,2/5,3/6,3/7,5/8,5/9,9/10,9/11}
  \draw [style=edge] (\i) to (\j);
\node [below of=11] {\parbox{0.3\linewidth}{\subcaption*{$H_2$}}};
\end{scope}

\begin{scope}[xshift=9cm,scale=1]
\node [style=cotreenode] (1) at (1,1) {0};
\node [style=cotreenode] (2) at (0,0) {0};
\node [style=vertex] (3) at (2,0) {};
\node [style=cotreenode] (4) at (-0.5,-1) {1};
\node [style=cotreenode] (5) at (0.5,-1) {1};
\node [style=vertex] (8) at (0.125,-2) {};
\node [style=cotreenode] (9) at (0.875,-2) {1};
\node [style=vertex] (10) at (0.625,-3) {};
\node [style=vertex] (11) at (1.125,-3) {};
\node [style=vertex] (12) at (-0.75,-2) {};
\node [style=vertex] (13) at (-0.25,-2) {};
\foreach \i/\j in {1/2,1/3,2/4,2/5,5/8,5/9,9/10,9/11,4/12,4/13}
  \draw [style=edge] (\i) to (\j);
\node [below of=11] {\parbox{0.3\linewidth}{\subcaption*{$H_3$}}};
\end{scope}


\end{tikzpicture}
\end{subfigure}


\begin{subfigure}{0.6\textwidth}
\begin{tikzpicture}

\begin{scope}[xshift=0cm,scale=1]
\node [style=cotreenode] (1) at (1,1) {0};
\node [style=cotreenode] (2) at (0,0) {1};
\node [style=cotreenode] (3) at (2,0) {1};
\node [style=vertex] (4) at (-0.5,-1) {};
\node [style=vertex] (5) at (0.5,-1) {};
\node [style=vertex] (6) at (1.5,-1) {};
\node [style=cotreenode] (7) at (2.5,-1) {0};
\node [style=vertex] (8) at (2.125,-2) {};
\node [style=cotreenode] (9) at (2.875,-2) {1};
\node [style=vertex] (10) at (2.625,-3) {};
\node [style=vertex] (11) at (3.125,-3) {};

\foreach \i/\j in {1/2,1/3,2/4,2/5,3/6,3/7,7/8,7/9,9/10,9/11}
  \draw [style=edge] (\i) to (\j);
\node [below of=10,xshift=-1.625cm] {\parbox{0.3\linewidth}{\subcaption*{$I_1$}}};
\end{scope}

\begin{scope}[xshift=5cm,scale=1]
\node [style=cotreenode] (1) at (1,1) {0};
\node [style=vertex] (2) at (0,0) {};
\node [style=cotreenode] (3) at (2,0) {1};
\node [style=cotreenode] (4) at (1.25,-1) {0};
\node [style=cotreenode] (5) at (2.75,-1) {0};
\node [style=cotreenode] (6) at (0.825,-2) {1};
\node [style=vertex] (7) at (1.625,-2) {};
\node [style=vertex] (8) at (2.375,-2) {};
\node [style=cotreenode] (9) at (3.125,-2) {1};
\node [style=vertex] (10) at (0.575,-3) {};
\node [style=vertex] (11) at (1.075,-3) {};
\node [style=vertex] (12) at (2.875,-3) {};
\node [style=vertex] (13) at (3.375,-3) {};


\foreach \i/\j in {1/2,1/3,3/4,3/5,4/6,4/7,5/8,5/9,6/10,6/11,9/12,9/13}
  \draw [style=edge] (\i) to (\j);
\node [below of=12,xshift=-1.625cm] {\parbox{0.3\linewidth}{\subcaption*{$J_1$}}};
\end{scope}
\end{tikzpicture}
\end{subfigure}
\setlength{\abovecaptionskip}{5pt}
\caption{$H_1, H_2$ y $H_3$ son los coárboles binarios correspondientes a la obstrucción $H$. El coárbol binario $I_1$ es el único que corresponde a la obstrucción $I$. El coárbol binario $J_1$ es el único que corresponde a la obstrucción $J$.}\label{fig_obsts_bin}


\end{figure}


\begin{algorithm}[ht!]
\caption{Pertenece\_a\_M2}
\label{alg_decision}
\DontPrintSemicolon % Some LaTeX compilers require you to use \dontprintsemicolon instead
\KwIn{$g$, la raíz del coárbol de una cográfica $G$.}
\KwOut{$verdadero$ si $G$ pertenece a la clase hereditaria de cográficas $C$}

$g\_bin \gets \text{CrearÁrbolBinario}(g)$\;
$C\_bins \gets \{H_1, H_2, H_3, I_1, J_1\}$\;

\ForEach{$bin$ \textbf{\emph{en}} $C\_bins$}{
    $f = \text{Función\_de\_coasignación}(g\_bin,bin)$\;
    \If{$f(bin) = marcado$}{
        $\Return\ verdadero$\;
    }
}

$\Return\ falso$\;

\end{algorithm}




    \subsection{Algoritmo certificador}
        Si bien, el Algoritmo \ref{alg_decision} es capaz de identificar a las cográficas que pertenece a la clase $M_2$, no es posible determinar a partir de éste cuáles son las dos partes en las que se puede dividir una gráfica de la clase. La presente sección muestra un algoritmo (Algoritmo \ref{alg_cert_m2}) que, dada una cográfica representada por su coárbol, devuelve una coloración de las hojas de este último. Si la gráfica pertenece a $M_2$, las hojas tendrán dos colores, $verde$ y $azul$, cada uno de los cuales corresponde a una parte de la partición en dos gráficas multipartitas completas. En el caso contrario, las hojas correspondientes a los vértices que forman una obstrucción mínima tendrán un color que indique de qué obstrucción mínima se trata ($amarillo$ para $H$, $anaranjado$ para $I$ y $rojo$ para $J$). El Algoritmo \ref{alg_cert_m2} hace uso de los Algoritmos \ref{alg_cert_caso1} y \ref{alg_cert_caso2}, que funcionan de la misma manera para casos específicos del problema. La correctitud de estos algoritmo se sigue de la demostración del Teorema \ref{teo_obsts_m2}

\subsubsection{Algoritmo para reconocer gráficas bipartitas completas conexas}

El Algoritmo \ref{alg_bpc} es un algoritmo que resulta útil para los algoritmos subsecuentes. Éste recibe la raíz de un coárbol, $g$, y devuelve $verdadero$ si la gráfica representada por dicho coárbol es una gráfica bipartita completa conexa, coloreando los vértices de una parte de color $azul$ y los de la otra parte de $verde$. En el caso contrario, se colorean con $amarillo$  tres hojas cuyo ancestro común más profundo sea un nodo con etiqueta 1. Es decir que se colorean los vértices que inducen un $K_3$ en la gráfica. El bloque de la línea 9 a la 28 se ejecuta sólo si $g$ tiene exactamente dos hijos. En las líneas 10 a 18 se busca un $K_3$ en el primer hijo de $g$ y en las líneas 19 a 27 se busca en el segundo hijo. La Figura \ref{fig_bipartita} muestra el resultado de la ejecución de este algoritmo para algunos coárboles.

\begin{algorithm}[!htbp]
\caption{Es\_bipartita_completa}
\label{alg_bpc}

\DontPrintSemicolon % Some LaTeX compilers require you to use \dontprintsemicolon instead
\KwIn{$g$, la raíz de un coárbol, $G$}
\KwOut{Verdadero si la gráfica representada por $G$ es bipartita completa. Falso en el caso contrario. Las hojas de $árbol(g)$ se colorean.}

\If{g \text{es una hoja}}{
    $g.color \gets azul$\;
    $\Return\ verdadero$\;
}
\ElseIf{$g.etiqueta = 0$}{
    $\Return\ falso$\;
}
\ElseIf{$g.hijos.tamaño > 2$}{
    Marcar con amarillo: una hoja en cada uno de tres hijos diferentes de $g$\;
    $\Return\ falso$\;
}
\Else(\tcp*[h]{Hay exactamente dos hijos}){
    \If{g.hijos\emph{[0]} \text{es una hoja}}{
        $g.hijos[0].color \gets azul$\;
    }
    \Else{
        \For{gchild \textbf{\emph{en}} g.hijos\emph{[0]}.hijos}{
            \If{gchild \text{es una hoja}}{
                $gchild.color \gets azul$\;
            }
            \Else{
                Marcar con amarillo: dos hojas que tengan como ancestro común más profundo a $gchild$ y una hoja descendiente de $g.hijos[1]$\;
                $\Return\ falso$\;
            }
        }
    }
    \If{g.hijos\emph{[1]} \text{es una hoja}}{
        $g.hijos[1].color \gets verde$\;
    }
    \Else{
        \For{gchild \textbf{\emph{en}} g.hijos\emph{[1]}.hijos}{
            \If{gchild \text{es una hoja}}{
                $gchild.color \gets verde$\;
            }
            \Else{
                Marcar con amarillo: dos hojas que tengan como ancestro común más profundo a $gchild$ y una hoja descendiente de $g.hijos[0]$\;
                $\Return\ falso$\;
            }
        }
    }
}

$\Return\ verdadero$\;

\end{algorithm}

\begin{figure}[!htbp]
\begin{center}
\begin{tikzpicture}

\begin{scope}[xshift=0cm,scale=1]
\node [style=cotreenode] (1) at (1,3) {1};
\node [style=vertex, fill=blue] (2) at (0.5,2) {};
\node [style=vertex, fill=green] (3) at (1.5,2) {};
\foreach \i/\j in {1/2,1/3}
  \draw [style=edge] (\i) to (\j);
\end{scope}

\begin{scope}[xshift=2cm,scale=1]
\node [style=cotreenode] (1) at (1,3) {1};
\node [style=vertex, fill=blue] (2) at (0.5,2) {};
\node [style=cotreenode] (3) at (1.5,2) {0};
\node [style=vertex, fill=green] (4) at (1,1) {};
\node [style=vertex, fill=green] (5) at (1.5,1) {};
\node [style=vertex, fill=green] (6) at (2,1) {};
\foreach \i/\j in {1/2,1/3,3/4,3/5,3/6}
  \draw [style=edge] (\i) to (\j);
\end{scope}

\begin{scope}[xshift=4.75cm,scale=1]
\node [style=cotreenode] (1) at (1,3) {1};
\node [style=cotreenode] (2) at (0.5,2) {0};
\node [style=cotreenode] (3) at (1.5,2) {0};
\node [style=vertex, fill=blue] (4) at (0.25,1) {};
\node [style=vertex, fill=blue] (5) at (0.5,1) {};
\node [style=vertex, fill=blue] (6) at (0.75,1) {};
\node [style=vertex, fill=green] (7) at (1.25,1) {};
\node [style=vertex, fill=green] (8) at (1.5,1) {};
\node [style=vertex, fill=green] (9) at (1.75,1) {};

\foreach \i/\j in {1/2,1/3,2/4,2/5,2/6,3/7,3/8,3/9}
  \draw [style=edge] (\i) to (\j);
\end{scope}

\begin{scope}[xshift=7cm,scale=1]
\node [style=cotreenode] (1) at (1,3) {1};
\node [style=vertex, fill=yellow] (2) at (0.5,2) {};
\node [style=vertex, fill=yellow] (3) at (1,2) {};
\node [style=vertex, fill=yellow] (4) at (1.5,2) {};
\foreach \i/\j in {1/2,1/3,1/4}
  \draw [style=edge] (\i) to (\j);
\end{scope}

\begin{scope}[xshift=9.25cm,scale=1]
\node [style=cotreenode] (1) at (1,3) {1};
\node [style=cotreenode] (2) at (0.5,2) {0};
\node [style=cotreenode] (3) at (1.5,2) {0};
\node [style=vertex, fill=yellow] (4) at (0.25,1) {};
\node [style=vertex, fill=blue] (6) at (0.75,1) {};
\node [style=vertex, fill=green] (7) at (1.25,1) {};
\node [style=cotreenode] (9) at (1.75,1) {1};
\node [style=vertex, fill=yellow] (10) at (1.5,0) {};
\node [style=vertex, fill=yellow] (11) at (2,0) {};

\foreach \i/\j in {1/2,1/3,2/4,2/6,3/7,3/9,9/10,9/11}
  \draw [style=edge] (\i) to (\j);
\end{scope}

\end{tikzpicture}
\end{center}
\caption{Ejemplos del resultado de la ejecución del Algoritmo \ref{alg_bpc}.}
\label{fig_bipartita}
\end{figure}



\subsubsection{Caso 1}

El algoritmo \ref{alg_cert_caso1} corresponde al $Caso\ 1$ de la demostración del Teorema \ref{teo_obsts_m2}. Éste recibe como entrada la raíz de un coárbol que representa una cográfica inconexa que tiene al menos dos componentes conexas no triviales. En el bloque de las líneas 1 a 12 se aborda el caso en el que la gráfica tiene exactamente dos componentes conexas y se busca un $Paw$ que pueda formar la obstrucción $I$. En el bloque de las líneas 13 a 17 se aborda el caso en el que hay al menos 3 componente conexas y se busca un $K_3$ en cada componente para formar la obstrucción $H$. Si no se encuentra ninguna de las obstrucciones mínimas, se devuelve $verdadero$ y cada una de las hojas del coárbol tendrán color $azul$ o $verde$. Las Figuras \ref{fig_certificador_caso1_01} y \ref{fig_certificador_caso1_02} muestran la ejecución del algoritmo para gráficas sin ninguna de las obstrucciones. La Figura \ref{fig_certificador_caso1_03} muestra el resultado de la ejecución para tres gráficas, cada una de las cuales contiene una obstrucción.

\begin{algorithm}[!htbp]
\small
\caption{M2\_Caso\_1}
\label{alg_cert_caso1}

\DontPrintSemicolon % Some LaTeX compilers require you to use \dontprintsemicolon instead
\KwIn{$g$, la raíz de un coárbol con etiqueta 0 y al menos dos hijos que no son hojas}
\KwOut{Verdadero si $G$ pertenece a la clase $M_2$. Falso en el caso contrario. Las hojas de $árbol(g)$ se colorean.}

\If{g.hijos.tamaño = 2}{
    \For{gchild \textbf{\emph{en}} g.hijos\emph{[0]}}{
        \If{gchild \text{es una hoja}}{
            $gchild.color \gets azul$\;
        }
        \Else{
            \For{ggchild \textbf{\emph{en}} gchild.hijos}{
                \If{ggchild \text{es una hoja}}{
                    $gchild.color \gets azul$\;
                }
                \Else(\tcp*[h]{Se marca la obstrucción $I$}){
                    Marcar con anaranjado: una hoja en $ggchild.hijos[0]$, una hoja en $ggchild.hijos[1]$, una hoja en un hermano de $ggchild$, una hoja en un hermano de $gchild$ y dos hojas cuyo ancestro común más profundo sea el hermano de $g.hijos[0]$\;

                    $\Return\ falso$\;
                }
            }
        }
    }
    Repetir el procedimiento de las líneas 2 a 11 para $g.hijos[1]$, pero marcando con color $verde$ en vez de $azul$\;
}
\Else{
    \For{child \textbf{\emph{en}} g.hijos}{
        \If{\emph{Es\_bipartita\_completa(}$child$\emph{)} = falso}{
             Marcar con amarillo: dos hojas cuyo ancestro común más profundo sea un hermano de $child$ que no sea una hoja y una hoja en un hermano diferente\;
                        $\Return\ falso$\;
        }
    }
}


$\Return\ verdadero$\;

\end{algorithm}


\begin{figure}[!htbp]
\centering

\begin{subfigure}{\textwidth}
\centering
\begin{tikzpicture}
\begin{scope}[xshift=0cm,scale=1]
\node [style=cotreenode, fill=lightgray] (1) at (1.5,4) {0};
\node [style=cotreenode] (2) at (0.5,3) {1};
\node [style=cotreenode] (3) at (2.5,3) {1};
\node [style=vertex] (4) at (0,2) {};
\node [style=vertex] (6) at (1,2) {};
\node [style=vertex] (7) at (2,2) {};
\node [style=cotreenode] (8) at (3,2) {0};
\node [style=vertex] (9) at (2.5,1) {};
\node [style=vertex] (10) at (3,1) {};
\node [style=vertex] (11) at (3.5,1) {};
\foreach \i/\j in {1/2,1/3,2/4,2/6,3/7,3/8,8/9,8/10,8/11}
  \draw [style=edge] (\i) to (\j);
\end{scope}
\begin{scope}[xshift=4cm,scale=1]
\node [style=cotreenode, fill=lightgray] (1) at (1.5,4) {0};
\node [style=cotreenode, fill=lightgray] (2) at (0.5,3) {1};
\node [style=cotreenode] (3) at (2.5,3) {1};
\node [style=vertex] (4) at (0,2) {};
\node [style=vertex] (6) at (1,2) {};
\node [style=vertex] (7) at (2,2) {};
\node [style=cotreenode] (8) at (3,2) {0};
\node [style=vertex] (9) at (2.5,1) {};
\node [style=vertex] (10) at (3,1) {};
\node [style=vertex] (11) at (3.5,1) {};
\foreach \i/\j in {1/2,1/3,2/4,2/6,3/7,3/8,8/9,8/10,8/11}
  \draw [style=edge] (\i) to (\j);
\end{scope}
\begin{scope}[xshift=8cm,scale=1]
\node [style=cotreenode, fill=lightgray] (1) at (1.5,4) {0};
\node [style=cotreenode, fill=lightgray] (2) at (0.5,3) {1};
\node [style=cotreenode] (3) at (2.5,3) {1};
\node [style=vertex, fill=lightgray] (4) at (0,2) {};
\node [style=vertex] (6) at (1,2) {};
\node [style=vertex] (7) at (2,2) {};
\node [style=cotreenode] (8) at (3,2) {0};
\node [style=vertex] (9) at (2.5,1) {};
\node [style=vertex] (10) at (3,1) {};
\node [style=vertex] (11) at (3.5,1) {};
\foreach \i/\j in {1/2,1/3,2/4,2/6,3/7,3/8,8/9,8/10,8/11}
  \draw [style=edge] (\i) to (\j);
\end{scope}
\end{tikzpicture}
\end{subfigure}

\begin{subfigure}{\textwidth}
\centering
\begin{tikzpicture}
\begin{scope}[xshift=0cm,scale=1]
\node [style=cotreenode, fill=lightgray] (1) at (1.5,4) {0};
\node [style=cotreenode, fill=lightgray] (2) at (0.5,3) {1};
\node [style=cotreenode] (3) at (2.5,3) {1};
\node [style=vertex, fill=blue] (4) at (0,2) {};
\node [style=vertex, fill=lightgray] (6) at (1,2) {};
\node [style=vertex] (7) at (2,2) {};
\node [style=cotreenode] (8) at (3,2) {0};
\node [style=vertex] (9) at (2.5,1) {};
\node [style=vertex] (10) at (3,1) {};
\node [style=vertex] (11) at (3.5,1) {};
\foreach \i/\j in {1/2,1/3,2/4,2/6,3/7,3/8,8/9,8/10,8/11}
  \draw [style=edge] (\i) to (\j);
\end{scope}
\begin{scope}[xshift=4cm,scale=1]
\node [style=cotreenode, fill=lightgray] (1) at (1.5,4) {0};
\node [style=cotreenode] (2) at (0.5,3) {1};
\node [style=cotreenode, fill=lightgray] (3) at (2.5,3) {1};
\node [style=vertex, fill=blue] (4) at (0,2) {};
\node [style=vertex, fill=blue] (6) at (1,2) {};
\node [style=vertex] (7) at (2,2) {};
\node [style=cotreenode] (8) at (3,2) {0};
\node [style=vertex] (9) at (2.5,1) {};
\node [style=vertex] (10) at (3,1) {};
\node [style=vertex] (11) at (3.5,1) {};
\foreach \i/\j in {1/2,1/3,2/4,2/6,3/7,3/8,8/9,8/10,8/11}
  \draw [style=edge] (\i) to (\j);
\end{scope}
\begin{scope}[xshift=8cm,scale=1]
\node [style=cotreenode, fill=lightgray] (1) at (1.5,4) {0};
\node [style=cotreenode] (2) at (0.5,3) {1};
\node [style=cotreenode, fill=lightgray] (3) at (2.5,3) {1};
\node [style=vertex, fill=blue] (4) at (0,2) {};
\node [style=vertex, fill=blue] (6) at (1,2) {};
\node [style=vertex, fill=lightgray] (7) at (2,2) {};
\node [style=cotreenode] (8) at (3,2) {0};
\node [style=vertex] (9) at (2.5,1) {};
\node [style=vertex] (10) at (3,1) {};
\node [style=vertex] (11) at (3.5,1) {};
\foreach \i/\j in {1/2,1/3,2/4,2/6,3/7,3/8,8/9,8/10,8/11}
  \draw [style=edge] (\i) to (\j);
\end{scope}
\end{tikzpicture}
\end{subfigure}

\begin{subfigure}{\textwidth}
\centering
\begin{tikzpicture}
\begin{scope}[xshift=0cm,scale=1]
\node [style=cotreenode, fill=lightgray] (1) at (1.5,4) {0};
\node [style=cotreenode] (2) at (0.5,3) {1};
\node [style=cotreenode, fill=lightgray] (3) at (2.5,3) {1};
\node [style=vertex, fill=blue] (4) at (0,2) {};
\node [style=vertex, fill=blue] (6) at (1,2) {};
\node [style=vertex, fill=green] (7) at (2,2) {};
\node [style=cotreenode, fill=lightgray] (8) at (3,2) {0};
\node [style=vertex] (9) at (2.5,1) {};
\node [style=vertex] (10) at (3,1) {};
\node [style=vertex] (11) at (3.5,1) {};
\foreach \i/\j in {1/2,1/3,2/4,2/6,3/7,3/8,8/9,8/10,8/11}
  \draw [style=edge] (\i) to (\j);
\end{scope}
\begin{scope}[xshift=4cm,scale=1]
\node [style=cotreenode, fill=lightgray] (1) at (1.5,4) {0};
\node [style=cotreenode] (2) at (0.5,3) {1};
\node [style=cotreenode, fill=lightgray] (3) at (2.5,3) {1};
\node [style=vertex, fill=blue] (4) at (0,2) {};
\node [style=vertex, fill=blue] (6) at (1,2) {};
\node [style=vertex, fill=green] (7) at (2,2) {};
\node [style=cotreenode, fill=lightgray] (8) at (3,2) {0};
\node [style=vertex, fill=lightgray] (9) at (2.5,1) {};
\node [style=vertex] (10) at (3,1) {};
\node [style=vertex] (11) at (3.5,1) {};
\foreach \i/\j in {1/2,1/3,2/4,2/6,3/7,3/8,8/9,8/10,8/11}
  \draw [style=edge] (\i) to (\j);
\end{scope}
\begin{scope}[xshift=8cm,scale=1]
\node [style=cotreenode, fill=lightgray] (1) at (1.5,4) {0};
\node [style=cotreenode] (2) at (0.5,3) {1};
\node [style=cotreenode, fill=lightgray] (3) at (2.5,3) {1};
\node [style=vertex, fill=blue] (4) at (0,2) {};
\node [style=vertex, fill=blue] (6) at (1,2) {};
\node [style=vertex, fill=green] (7) at (2,2) {};
\node [style=cotreenode, fill=lightgray] (8) at (3,2) {0};
\node [style=vertex, fill=green] (9) at (2.5,1) {};
\node [style=vertex, fill=lightgray] (10) at (3,1) {};
\node [style=vertex] (11) at (3.5,1) {};
\foreach \i/\j in {1/2,1/3,2/4,2/6,3/7,3/8,8/9,8/10,8/11}
  \draw [style=edge] (\i) to (\j);
\end{scope}
\end{tikzpicture}
\end{subfigure}

\begin{subfigure}{\textwidth}
\centering
\begin{tikzpicture}
\begin{scope}[xshift=0cm,scale=1]
\node [style=cotreenode, fill=lightgray] (1) at (1.5,4) {0};
\node [style=cotreenode] (2) at (0.5,3) {1};
\node [style=cotreenode, fill=lightgray] (3) at (2.5,3) {1};
\node [style=vertex, fill=blue] (4) at (0,2) {};
\node [style=vertex, fill=blue] (6) at (1,2) {};
\node [style=vertex, fill=green] (7) at (2,2) {};
\node [style=cotreenode, fill=lightgray] (8) at (3,2) {0};
\node [style=vertex, fill=green] (9) at (2.5,1) {};
\node [style=vertex, fill=green] (10) at (3,1) {};
\node [style=vertex, fill=lightgray] (11) at (3.5,1) {};
\foreach \i/\j in {1/2,1/3,2/4,2/6,3/7,3/8,8/9,8/10,8/11}
  \draw [style=edge] (\i) to (\j);
\end{scope}
\begin{scope}[xshift=4cm,scale=1]
\node [style=cotreenode, fill=lightgray] (1) at (1.5,4) {0};
\node [style=cotreenode] (2) at (0.5,3) {1};
\node [style=cotreenode, fill=lightgray] (3) at (2.5,3) {1};
\node [style=vertex, fill=blue] (4) at (0,2) {};
\node [style=vertex, fill=blue] (6) at (1,2) {};
\node [style=vertex, fill=green] (7) at (2,2) {};
\node [style=cotreenode, fill=lightgray] (8) at (3,2) {0};
\node [style=vertex, fill=green] (9) at (2.5,1) {};
\node [style=vertex, fill=green] (10) at (3,1) {};
\node [style=vertex, fill=green] (11) at (3.5,1) {};
\foreach \i/\j in {1/2,1/3,2/4,2/6,3/7,3/8,8/9,8/10,8/11}
  \draw [style=edge] (\i) to (\j);
\end{scope}
\end{tikzpicture}
\end{subfigure}

\caption{Ejemplo de la ejecución del Algoritmo \ref{alg_cert_caso1}. Se muestran en color gris los nodos del árbol que están siendo procesados. Los colores de las hojas corresponden a los colores que asigna el algoritmo.}
\label{fig_certificador_caso1_01}
\end{figure}

\begin{figure}[!htbp]
\centering
\begin{subfigure}{\textwidth}
\centering
\begin{tikzpicture}
\begin{scope}[xshift=0cm,scale=1]
\node [style=cotreenode, fill=lightgray] (1) at (2,4) {0};
\node [style=cotreenode] (2) at (0.5,3) {1};
\node [style=vertex] (3) at (2,3) {};
\node [style=cotreenode] (4) at (3.5,3) {1};
\node [style=vertex] (5) at (0,2) {};
\node [style=vertex] (6) at (1,2) {};
\node [style=cotreenode] (7) at (2.75,2) {0};
\node [style=cotreenode] (8) at (4.25,2) {0};
\node [style=vertex] (9) at (2.25,1) {};
\node [style=vertex] (10) at (2.75,1) {};
\node [style=vertex] (11) at (3.25,1) {};
\node [style=vertex] (12) at (3.75,1) {};
\node [style=vertex] (13) at (4.25,1) {};
\node [style=vertex] (14) at (4.75,1) {};
\foreach \i/\j in {1/2,1/3,1/4,2/5,2/6,4/7,4/8,7/9,7/10,7/11,8/12,8/13,8/14}
  \draw [style=edge] (\i) to (\j);
\end{scope}
\begin{scope}[xshift=6cm,scale=1]
\node [style=cotreenode, fill=lightgray] (1) at (2,4) {0};
\node [style=cotreenode, fill=lightgray] (2) at (0.5,3) {1};
\node [style=vertex] (3) at (2,3) {};
\node [style=cotreenode] (4) at (3.5,3) {1};
\node [style=vertex] (5) at (0,2) {};
\node [style=vertex] (6) at (1,2) {};
\node [style=cotreenode] (7) at (2.75,2) {0};
\node [style=cotreenode] (8) at (4.25,2) {0};
\node [style=vertex] (9) at (2.25,1) {};
\node [style=vertex] (10) at (2.75,1) {};
\node [style=vertex] (11) at (3.25,1) {};
\node [style=vertex] (12) at (3.75,1) {};
\node [style=vertex] (13) at (4.25,1) {};
\node [style=vertex] (14) at (4.75,1) {};
\foreach \i/\j in {1/2,1/3,1/4,2/5,2/6,4/7,4/8,7/9,7/10,7/11,8/12,8/13,8/14}
  \draw [style=edge] (\i) to (\j);
\end{scope}
\end{tikzpicture}
\end{subfigure}

\begin{subfigure}{\textwidth}
\centering
\begin{tikzpicture}
\begin{scope}[xshift=0cm,scale=1]
\node [style=cotreenode, fill=lightgray] (1) at (2,4) {0};
\node [style=cotreenode] (2) at (0.5,3) {1};
\node [style=vertex, fill=lightgray] (3) at (2,3) {};
\node [style=cotreenode] (4) at (3.5,3) {1};
\node [style=vertex, fill=blue] (5) at (0,2) {};
\node [style=vertex, fill=green] (6) at (1,2) {};
\node [style=cotreenode] (7) at (2.75,2) {0};
\node [style=cotreenode] (8) at (4.25,2) {0};
\node [style=vertex] (9) at (2.25,1) {};
\node [style=vertex] (10) at (2.75,1) {};
\node [style=vertex] (11) at (3.25,1) {};
\node [style=vertex] (12) at (3.75,1) {};
\node [style=vertex] (13) at (4.25,1) {};
\node [style=vertex] (14) at (4.75,1) {};
\foreach \i/\j in {1/2,1/3,1/4,2/5,2/6,4/7,4/8,7/9,7/10,7/11,8/12,8/13,8/14}
  \draw [style=edge] (\i) to (\j);
\end{scope}
\begin{scope}[xshift=6cm,scale=1]
\node [style=cotreenode, fill=lightgray] (1) at (2,4) {0};
\node [style=cotreenode] (2) at (0.5,3) {1};
\node [style=vertex, fill=blue] (3) at (2,3) {};
\node [style=cotreenode, fill=lightgray] (4) at (3.5,3) {1};
\node [style=vertex, fill=blue] (5) at (0,2) {};
\node [style=vertex, fill=green] (6) at (1,2) {};
\node [style=cotreenode] (7) at (2.75,2) {0};
\node [style=cotreenode] (8) at (4.25,2) {0};
\node [style=vertex] (9) at (2.25,1) {};
\node [style=vertex] (10) at (2.75,1) {};
\node [style=vertex] (11) at (3.25,1) {};
\node [style=vertex] (12) at (3.75,1) {};
\node [style=vertex] (13) at (4.25,1) {};
\node [style=vertex] (14) at (4.75,1) {};
\foreach \i/\j in {1/2,1/3,1/4,2/5,2/6,4/7,4/8,7/9,7/10,7/11,8/12,8/13,8/14}
  \draw [style=edge] (\i) to (\j);
\end{scope}
\end{tikzpicture}
\end{subfigure}

\begin{subfigure}{\textwidth}
\centering
\begin{tikzpicture}
\begin{scope}[xshift=0cm,scale=1]
\node [style=cotreenode, fill=lightgray] (1) at (2,4) {0};
\node [style=cotreenode] (2) at (0.5,3) {1};
\node [style=vertex, fill=blue] (3) at (2,3) {};
\node [style=cotreenode, fill=lightgray] (4) at (3.5,3) {1};
\node [style=vertex, fill=blue] (5) at (0,2) {};
\node [style=vertex, fill=green] (6) at (1,2) {};
\node [style=cotreenode] (7) at (2.75,2) {0};
\node [style=cotreenode] (8) at (4.25,2) {0};
\node [style=vertex, fill=blue] (9) at (2.25,1) {};
\node [style=vertex, fill=blue] (10) at (2.75,1) {};
\node [style=vertex, fill=blue] (11) at (3.25,1) {};
\node [style=vertex, fill=green] (12) at (3.75,1) {};
\node [style=vertex, fill=green] (13) at (4.25,1) {};
\node [style=vertex, fill=green] (14) at (4.75,1) {};
\foreach \i/\j in {1/2,1/3,1/4,2/5,2/6,4/7,4/8,7/9,7/10,7/11,8/12,8/13,8/14}
  \draw [style=edge] (\i) to (\j);
\end{scope}
\end{tikzpicture}
\end{subfigure}

\caption{Ejemplo de la ejecución del Algoritmo \ref{alg_cert_caso1}. Se muestran en color gris los nodos del árbol que están siendo procesados. Los colores de las hojas corresponden a los colores que asigna el algoritmo.}
\label{fig_certificador_caso1_02}
\end{figure}

\begin{figure}[!htbp]
\centering
\begin{subfigure}{\textwidth}
\centering
\begin{tikzpicture}
\begin{scope}[xshift=0cm,scale=1]
\node [style=cotreenode] (1) at (1.5,4) {0};
\node [style=cotreenode] (2) at (0.5,3) {1};
\node [style=cotreenode] (4) at (2.5,3) {1};
\node [style=vertex, fill=orange] (5) at (0,2) {};
\node [style=vertex, fill=blue] (6) at (1,2) {};
\node [style=vertex, fill=orange] (7) at (1.75,2) {};
\node [style=cotreenode] (8) at (3.25,2) {0};
\node [style=vertex, fill=orange] (12) at (2.75,1) {};
\node [style=cotreenode] (14) at (3.75,1) {1};
\node [style=vertex, fill=orange] (15) at (3.5,0) {};
\node [style=vertex, fill=orange] (16) at (4,0) {};
\foreach \i/\j in {1/2,1/4,2/5,2/6,4/7,4/8,8/12,8/14,14/15,14/16}
  \draw [style=edge] (\i) to (\j);
\end{scope}
\begin{scope}[xshift=4.5cm,scale=1]
\node [style=cotreenode] (1) at (1.5,4) {0};
\node [style=cotreenode] (2) at (0.5,3) {1};
\node [style=vertex, fill=yellow] (3) at (1.5,3) {};
\node [style=cotreenode] (4) at (2.5,3) {1};
\node [style=vertex, fill=yellow] (5) at (0,2) {};
\node [style=vertex, fill=yellow] (6) at (1,2) {};
\node [style=vertex, fill=yellow] (7) at (2,2) {};
\node [style=vertex, fill=yellow] (8) at (2.5,2) {};
\node [style=vertex, fill=yellow] (9) at (3,2) {};
\foreach \i/\j in {1/2,1/3,1/4,2/5,2/6,4/7,4/8,4/9}
  \draw [style=edge] (\i) to (\j);
\end{scope}
\begin{scope}[xshift=8.5cm,scale=1]
\node [style=cotreenode] (1) at (1.5,4) {0};
\node [style=cotreenode] (2) at (0.5,3) {1};
\node [style=vertex, fill=yellow] (3) at (1.5,3) {};
\node [style=cotreenode] (4) at (2.5,3) {1};
\node [style=vertex, fill=yellow] (5) at (0,2) {};
\node [style=vertex, fill=yellow] (6) at (1,2) {};
\node [style=vertex, fill=yellow] (7) at (1.75,2) {};
\node [style=cotreenode] (8) at (3.25,2) {0};
\node [style=vertex, fill=green] (12) at (2.75,1) {};
\node [style=cotreenode] (14) at (3.75,1) {1};
\node [style=vertex, fill=yellow] (15) at (3.5,0) {};
\node [style=vertex, fill=yellow] (16) at (4,0) {};
\foreach \i/\j in {1/2,1/3,1/4,2/5,2/6,4/7,4/8,8/12,8/14,14/15,14/16}
  \draw [style=edge] (\i) to (\j);
\end{scope}
\end{tikzpicture}
\end{subfigure}


\caption{Ejemplos del resultado de la ejecución del Algoritmo \ref{alg_cert_caso1} en los que se encuentra una obstrucción.}
\label{fig_certificador_caso1_03}
\end{figure}

\subsubsection{Caso 2}

El algoritmo \ref{alg_cert_caso2} corresponde al $Caso\ 2$ de la demostración del Teorema \ref{teo_obsts_m2}. Éste recibe como entrada la raíz, $g$, de un coárbol que representa una cográfica inconexa que tiene exactamente una componente conexa no trivial y al menos una trivial. En el bloque de las líneas 6 a 28 se procesa el hijo de $g$ que no es una hoja. En las líneas 7 a 19 se procesan los nietos de $g$ y se registra si alguno tiene un hijo que no sea una hoja (es decir que dicho nieto de $g$ corresponde a una gráfica no multipartita completa) en la variable $aux\_gchild$. La cantidad de hijos diferentes de una hoja de éste se registra en $ggchildren\_no\_hojas$. Si hay más de un nieto que tenga hijos que no son hojas, se marca la obstrucción $J$ (Línea 18). Una vez procesados los nietos de $g$, se decide cómo será procesado el nieto de $g$ que no corresponde a una gráfica multipartita completa. Si tal hijo no existe, la partición ya se habrá hecho (Líneas 20 y 21), esto corresponde a una parte del caso base del $Caso\_2$ de la demostración del Teorema \ref{teo_obsts_m2}. Si dicho nieto tiene un solo hijo que no es una hoja, se procesa recursivamente (Líneas 22 y 23), esto corresponde al paso inductivo del $Caso\_2$ de la demostración ya mencionada. Y finalmente, si tiene más de un hijo que no es una hoja, se busca que todos estos hijos sean bipartitas, esta es la otra parte del caso base del $Caso\_2$ de la demostración. La Figura \ref{fig_certificador_caso2_01} muestra un ejemplo de la ejecución del algoritmo para un coárbol cuya cográfica no contiene a ninguna de las obstrucciones mínimas de $M_2$. La Figura \ref{fig_certificador_caso2_02} muestra el resultado de la ejecución para coárboles que contienen una obstrucción.




\begin{algorithm}[!htbp]
\SetInd{1pt}{10pt}
\footnotesize
\caption{M2\_Caso\_2}
\label{alg_cert_caso2}

\DontPrintSemicolon % Some LaTeX compilers require you to use \dontprintsemicolon instead
\KwIn{$g$, la raíz de un coárbol con etiqueta 0 que tiene exactamente un hijo que no es una hoja y al menos uno que es una hoja}
\KwOut{Verdadero si $G$ peretenece a la clase $M_2$. Falso en el caso contrario. Las hojas de $G$ se colorean.}


    $aux\_gchild \gets null$\;
    $ggchildren\_no\_hojas \gets 0$\;

    \For{child \textbf{\emph{en}} $g.hijos$}{
        \If{child \text{es una hoja}}{
            $child.color \gets azul$\;
        }
        \Else(\tcp*[h]{Sólo se ejecuta una vez}){
            \For{gchild \textbf{\emph{en}} $child.hijos$}{
                \If{gchild \text{es una hoja}}{
                    $gchild.color \gets verde$\;
                }
                \Else{
                    \For{ggchild \textbf{\emph{en}} $gchild.hijos$}{
                        \If{ggchild \text{es una hoja}}{
                            $ggchild.color \gets verde$\;
                        }
                        \ElseIf{$aux\_gchild = null \emph{\textbf{ o }} aux\_gchild = gchild$}{
                            $aux\_gchild \gets gchild$\;
                            $ggchildren\_no\_hojas \gets ggchildren\_no\_hojas + 1$\;
                        }
                        \Else{
                            Marcar con rojo: Un hijo de $g$ que sea una hoja, dos hojas cuyo ancestro común más profundo sea $ggchild$, una hoja en un hermano de $ggchild$, dos hojas cuyo ancestro común más profundo sea un hijo de $aux\_gchild$ que no es una hoja y una hoja en un hijo de $aux\_gchild$ diferente del anterior\;
                            $\Return\ falso$\;
                        }
                    }
                }
            }

            \If{ggchildren\_no\_hojas = 0}{
                $\Return\ verdadero$\;
            }
            \ElseIf{ggchildren\_no\_hojas = 1}{
                $\Return$ M2\_Caso\_2($aux\_gchild$)\;
            }
            \Else{
                \For{ggchild \textbf{\emph{en}} aux\_gchild}{
                    \If{\emph{Es\_bipartita_completa(}$ggchild$\emph{)} = falso}{
                        Marcar con amarillo: dos hojas cuyo ancestro común más profundo sea un hermano de $ggchild$ que no sea una hoja y un hijo de $g$ que sea una hoja\;
                        $\Return\ falso$\;
                    }
                }
            }

        }
    }

    $\Return\ verdadero$\;


\end{algorithm}

\begin{figure}[!htbp]
\centering

\begin{subfigure}{\textwidth}
\centering
\begin{tikzpicture}
\begin{scope}[xshift=0cm,scale=1]
\node [style=cotreenode, fill=lightgray] (1) at (3.5,5) {0};
\node [style=vertex] (2) at (0.5,4) {};
\node [style=vertex] (3) at (1.5,4) {};
\node [style=vertex] (4) at (2.5,4) {};
\node [style=cotreenode] (5) at (3.5,4) {1};
\node [style=vertex] (6) at (0.5,3) {};
\node [style=cotreenode] (7) at (2,3) {0};
\node [style=cotreenode] (8) at (3.5,3) {0};
\node [style=vertex] (9) at (1.75,2) {};
\node [style=vertex] (10) at (2.25,2) {};
\node [style=vertex] (11) at (2.75,2) {};
\node [style=cotreenode] (12) at (3.5,2) {1};
\node [style=cotreenode] (13) at (4.5,2) {1};
\node [style=vertex] (14) at (3.25,1) {};
\node [style=vertex] (15) at (3.75,1) {};
\node [style=vertex] (16) at (4.25,1) {};
\node [style=vertex] (17) at (4.75,1) {};
\foreach \i/\j in {1/2,1/3,1/4,1/5,5/6,5/7,5/8,7/9,7/10,8/11,8/12,8/13,12/14,12/15,13/16,13/17}
  \draw [style=edge] (\i) to (\j);
\end{scope}
\begin{scope}[xshift=5cm,scale=1]
\node [style=cotreenode, fill=lightgray] (1) at (3.5,5) {0};
\node [style=vertex, fill=blue] (2) at (0.5,4) {};
\node [style=vertex, fill=blue] (3) at (1.5,4) {};
\node [style=vertex, fill=blue] (4) at (2.5,4) {};
\node [style=cotreenode, fill=lightgray] (5) at (3.5,4) {1};
\node [style=vertex] (6) at (0.5,3) {};
\node [style=cotreenode] (7) at (2,3) {0};
\node [style=cotreenode] (8) at (3.5,3) {0};
\node [style=vertex] (9) at (1.75,2) {};
\node [style=vertex] (10) at (2.25,2) {};
\node [style=vertex] (11) at (2.75,2) {};
\node [style=cotreenode] (12) at (3.5,2) {1};
\node [style=cotreenode] (13) at (4.5,2) {1};
\node [style=vertex] (14) at (3.25,1) {};
\node [style=vertex] (15) at (3.75,1) {};
\node [style=vertex] (16) at (4.25,1) {};
\node [style=vertex] (17) at (4.75,1) {};
\foreach \i/\j in {1/2,1/3,1/4,1/5,5/6,5/7,5/8,7/9,7/10,8/11,8/12,8/13,12/14,12/15,13/16,13/17}
  \draw [style=edge] (\i) to (\j);
\end{scope}
\begin{scope}[xshift=10cm,scale=1]
\node [style=cotreenode, fill=lightgray] (1) at (3.5,5) {0};
\node [style=vertex, fill=blue] (2) at (0.5,4) {};
\node [style=vertex, fill=blue] (3) at (1.5,4) {};
\node [style=vertex, fill=blue] (4) at (2.5,4) {};
\node [style=cotreenode, fill=lightgray] (5) at (3.5,4) {1};
\node [style=vertex, fill=green] (6) at (0.5,3) {};
\node [style=cotreenode, fill=lightgray] (7) at (2,3) {0};
\node [style=cotreenode] (8) at (3.5,3) {0};
\node [style=vertex] (9) at (1.75,2) {};
\node [style=vertex] (10) at (2.25,2) {};
\node [style=vertex] (11) at (2.75,2) {};
\node [style=cotreenode] (12) at (3.5,2) {1};
\node [style=cotreenode] (13) at (4.5,2) {1};
\node [style=vertex] (14) at (3.25,1) {};
\node [style=vertex] (15) at (3.75,1) {};
\node [style=vertex] (16) at (4.25,1) {};
\node [style=vertex] (17) at (4.75,1) {};
\foreach \i/\j in {1/2,1/3,1/4,1/5,5/6,5/7,5/8,7/9,7/10,8/11,8/12,8/13,12/14,12/15,13/16,13/17}
  \draw [style=edge] (\i) to (\j);
\end{scope}
\end{tikzpicture}
\end{subfigure}
\begin{subfigure}{\textwidth}
\centering
\begin{tikzpicture}
\begin{scope}[xshift=0cm,scale=1]
\node [style=cotreenode, fill=lightgray] (1) at (3.5,5) {0};
\node [style=vertex, fill=blue] (2) at (0.5,4) {};
\node [style=vertex, fill=blue] (3) at (1.5,4) {};
\node [style=vertex, fill=blue] (4) at (2.5,4) {};
\node [style=cotreenode, fill=lightgray] (5) at (3.5,4) {1};
\node [style=vertex, fill=green] (6) at (0.5,3) {};
\node [style=cotreenode] (7) at (2,3) {0};
\node [style=cotreenode, fill=lightgray] (8) at (3.5,3) {0};
\node [style=vertex, fill=green] (9) at (1.75,2) {};
\node [style=vertex, fill=green] (10) at (2.25,2) {};
\node [style=vertex] (11) at (2.75,2) {};
\node [style=cotreenode] (12) at (3.5,2) {1};
\node [style=cotreenode] (13) at (4.5,2) {1};
\node [style=vertex] (14) at (3.25,1) {};
\node [style=vertex] (15) at (3.75,1) {};
\node [style=vertex] (16) at (4.25,1) {};
\node [style=vertex] (17) at (4.75,1) {};
\foreach \i/\j in {1/2,1/3,1/4,1/5,5/6,5/7,5/8,7/9,7/10,8/11,8/12,8/13,12/14,12/15,13/16,13/17}
  \draw [style=edge] (\i) to (\j);
\end{scope}
\begin{scope}[xshift=5cm,scale=1]
\node [style=cotreenode, fill=lightgray] (1) at (3.5,5) {0};
\node [style=vertex, fill=blue] (2) at (0.5,4) {};
\node [style=vertex, fill=blue] (3) at (1.5,4) {};
\node [style=vertex, fill=blue] (4) at (2.5,4) {};
\node [style=cotreenode, fill=lightgray] (5) at (3.5,4) {1};
\node [style=vertex, fill=green] (6) at (0.5,3) {};
\node [style=cotreenode] (7) at (2,3) {0};
\node [style=cotreenode, fill=lightgray] (8) at (3.5,3) {0};
\node [style=vertex, fill=green] (9) at (1.75,2) {};
\node [style=vertex, fill=green] (10) at (2.25,2) {};
\node [style=vertex, fill=green] (11) at (2.75,2) {};
\node [style=cotreenode, fill=lightgray] (12) at (3.5,2) {1};
\node [style=cotreenode] (13) at (4.5,2) {1};
\node [style=vertex] (14) at (3.25,1) {};
\node [style=vertex] (15) at (3.75,1) {};
\node [style=vertex] (16) at (4.25,1) {};
\node [style=vertex] (17) at (4.75,1) {};
\foreach \i/\j in {1/2,1/3,1/4,1/5,5/6,5/7,5/8,7/9,7/10,8/11,8/12,8/13,12/14,12/15,13/16,13/17}
  \draw [style=edge] (\i) to (\j);
\end{scope}
\begin{scope}[xshift=10cm,scale=1]
\node [style=cotreenode, fill=lightgray] (1) at (3.5,5) {0};
\node [style=vertex, fill=blue] (2) at (0.5,4) {};
\node [style=vertex, fill=blue] (3) at (1.5,4) {};
\node [style=vertex, fill=blue] (4) at (2.5,4) {};
\node [style=cotreenode, fill=lightgray] (5) at (3.5,4) {1};
\node [style=vertex, fill=green] (6) at (0.5,3) {};
\node [style=cotreenode] (7) at (2,3) {0};
\node [style=cotreenode, fill=lightgray] (8) at (3.5,3) {0};
\node [style=vertex, fill=green] (9) at (1.75,2) {};
\node [style=vertex, fill=green] (10) at (2.25,2) {};
\node [style=vertex, fill=green] (11) at (2.75,2) {};
\node [style=cotreenode] (12) at (3.5,2) {1};
\node [style=cotreenode, fill=lightgray] (13) at (4.5,2) {1};
\node [style=vertex] (14) at (3.25,1) {};
\node [style=vertex] (15) at (3.75,1) {};
\node [style=vertex] (16) at (4.25,1) {};
\node [style=vertex] (17) at (4.75,1) {};
\foreach \i/\j in {1/2,1/3,1/4,1/5,5/6,5/7,5/8,7/9,7/10,8/11,8/12,8/13,12/14,12/15,13/16,13/17}
  \draw [style=edge] (\i) to (\j);
\end{scope}
\end{tikzpicture}
\end{subfigure}

\begin{subfigure}{\textwidth}
\centering
\begin{tikzpicture}
\begin{scope}[xshift=0cm,scale=1]
\node [style=cotreenode, fill=lightgray] (1) at (3.5,5) {0};
\node [style=vertex, fill=blue] (2) at (0.5,4) {};
\node [style=vertex, fill=blue] (3) at (1.5,4) {};
\node [style=vertex, fill=blue] (4) at (2.5,4) {};
\node [style=cotreenode, fill=lightgray] (5) at (3.5,4) {1};
\node [style=vertex, fill=green] (6) at (0.5,3) {};
\node [style=cotreenode] (7) at (2,3) {0};
\node [style=cotreenode, fill=lightgray] (8) at (3.5,3) {0};
\node [style=vertex, fill=green] (9) at (1.75,2) {};
\node [style=vertex, fill=green] (10) at (2.25,2) {};
\node [style=vertex, fill=green] (11) at (2.75,2) {};
\node [style=cotreenode, fill=lightgray] (12) at (3.5,2) {1};
\node [style=cotreenode] (13) at (4.5,2) {1};
\node [style=vertex] (14) at (3.25,1) {};
\node [style=vertex] (15) at (3.75,1) {};
\node [style=vertex] (16) at (4.25,1) {};
\node [style=vertex] (17) at (4.75,1) {};
\foreach \i/\j in {1/2,1/3,1/4,1/5,5/6,5/7,5/8,7/9,7/10,8/11,8/12,8/13,12/14,12/15,13/16,13/17}
  \draw [style=edge] (\i) to (\j);
\end{scope}
\begin{scope}[xshift=5cm,scale=1]
\node [style=cotreenode, fill=lightgray] (1) at (3.5,5) {0};
\node [style=vertex, fill=blue] (2) at (0.5,4) {};
\node [style=vertex, fill=blue] (3) at (1.5,4) {};
\node [style=vertex, fill=blue] (4) at (2.5,4) {};
\node [style=cotreenode, fill=lightgray] (5) at (3.5,4) {1};
\node [style=vertex, fill=green] (6) at (0.5,3) {};
\node [style=cotreenode] (7) at (2,3) {0};
\node [style=cotreenode, fill=lightgray] (8) at (3.5,3) {0};
\node [style=vertex, fill=green] (9) at (1.75,2) {};
\node [style=vertex, fill=green] (10) at (2.25,2) {};
\node [style=vertex, fill=green] (11) at (2.75,2) {};
\node [style=cotreenode] (12) at (3.5,2) {1};
\node [style=cotreenode, fill=lightgray] (13) at (4.5,2) {1};
\node [style=vertex, fill=blue] (14) at (3.25,1) {};
\node [style=vertex, fill=green] (15) at (3.75,1) {};
\node [style=vertex] (16) at (4.25,1) {};
\node [style=vertex] (17) at (4.75,1) {};
\foreach \i/\j in {1/2,1/3,1/4,1/5,5/6,5/7,5/8,7/9,7/10,8/11,8/12,8/13,12/14,12/15,13/16,13/17}
  \draw [style=edge] (\i) to (\j);
\end{scope}
\begin{scope}[xshift=10cm,scale=1]
\node [style=cotreenode, fill=lightgray] (1) at (3.5,5) {0};
\node [style=vertex, fill=blue] (2) at (0.5,4) {};
\node [style=vertex, fill=blue] (3) at (1.5,4) {};
\node [style=vertex, fill=blue] (4) at (2.5,4) {};
\node [style=cotreenode, fill=lightgray] (5) at (3.5,4) {1};
\node [style=vertex, fill=green] (6) at (0.5,3) {};
\node [style=cotreenode] (7) at (2,3) {0};
\node [style=cotreenode, fill=lightgray] (8) at (3.5,3) {0};
\node [style=vertex, fill=green] (9) at (1.75,2) {};
\node [style=vertex, fill=green] (10) at (2.25,2) {};
\node [style=vertex, fill=green] (11) at (2.75,2) {};
\node [style=cotreenode] (12) at (3.5,2) {1};
\node [style=cotreenode, fill=lightgray] (13) at (4.5,2) {1};
\node [style=vertex, fill=blue] (14) at (3.25,1) {};
\node [style=vertex, fill=green] (15) at (3.75,1) {};
\node [style=vertex, fill=blue] (16) at (4.25,1) {};
\node [style=vertex, fill=green] (17) at (4.75,1) {};
\foreach \i/\j in {1/2,1/3,1/4,1/5,5/6,5/7,5/8,7/9,7/10,8/11,8/12,8/13,12/14,12/15,13/16,13/17}
  \draw [style=edge] (\i) to (\j);
\end{scope}
\end{tikzpicture}
\end{subfigure}
\caption{Ejemplo de la ejecución del Algoritmo \ref{alg_cert_caso2}. Se muestran en color gris los nodos del árbol que están siendo procesados. El procesamiento de las hojas hermanas se realiza en una sola imagen. Los colores de las hojas corresponden a los colores que asigna el algoritmo.}
\label{fig_certificador_caso2_01}
\end{figure}


\begin{figure}[!htbp]
\centering

\begin{subfigure}{\textwidth}
\centering
\begin{tikzpicture}
\begin{scope}[xshift=0cm,scale=1]
\node [style=cotreenode] (1) at (3.5,5) {0};
\node [style=vertex, fill=yellow] (2) at (0.5,4) {};
\node [style=vertex, fill=blue] (3) at (1.5,4) {};
\node [style=vertex, fill=blue] (4) at (2.5,4) {};
\node [style=cotreenode] (5) at (3.5,4) {1};
\node [style=vertex, fill=green] (6) at (0.5,3) {};
\node [style=cotreenode] (7) at (2,3) {0};
\node [style=cotreenode] (8) at (3.5,3) {0};
\node [style=vertex, fill=green] (9) at (1.75,2) {};
\node [style=vertex, fill=green] (10) at (2.25,2) {};
\node [style=vertex, fill=green] (11) at (2.75,2) {};
\node [style=cotreenode] (12) at (3.5,2) {1};
\node [style=cotreenode] (13) at (4.5,2) {1};
\node [style=vertex, fill=yellow] (14) at (3.25,1) {};
\node [style=vertex, fill=yellow] (15) at (3.75,1) {};
\node [style=vertex, fill=yellow] (16) at (4.25,1) {};
\node [style=vertex, fill=yellow] (17) at (4.75,1) {};
\node [style=vertex, fill=yellow] (18) at (4.5,1) {};
\foreach \i/\j in {1/2,1/3,1/4,1/5,5/6,5/7,5/8,7/9,7/10,8/11,8/12,8/13,12/14,12/15,13/16,13/17,13/18}
  \draw [style=edge] (\i) to (\j);
\end{scope}
\begin{scope}[xshift=5cm,scale=1]
\node [style=cotreenode] (1) at (3.5,5) {0};
\node [style=vertex, fill=red] (2) at (0.5,4) {};
\node [style=vertex, fill=blue] (3) at (1.5,4) {};
\node [style=vertex, fill=blue] (4) at (2.5,4) {};
\node [style=cotreenode] (5) at (3.5,4) {1};
\node [style=vertex, fill=green] (6) at (0.5,3) {};
\node [style=vertex, fill=green] (7) at (1.5,3) {};
\node [style=cotreenode] (8) at (2.5,3) {0};
\node [style=cotreenode] (9) at (4.5,3) {0};
\node [style=cotreenode] (10) at (2,2) {1};
\node [style=vertex, fill=red] (11) at (3,2) {};
\node [style=cotreenode] (12) at (4,2) {1};
\node [style=vertex, fill=red] (13) at (5,2) {};
\node [style=vertex, fill=red] (14) at (1.5,1) {};
\node [style=vertex, fill=red] (15) at (2.5,1) {};
\node [style=vertex, fill=red] (16) at (3.5,1) {};
\node [style=vertex, fill=red] (17) at (4.5,1) {};
\foreach \i/\j in {1/2,1/3,1/4,1/5,5/6,5/7,5/8,5/9,8/10,8/11,9/12,9/13,10/14,10/15,12/16,12/17}
  \draw [style=edge] (\i) to (\j);
\end{scope}
\end{tikzpicture}
\end{subfigure}
\caption{Ejemplo del resultado de la ejecución del Algoritmo \ref{alg_cert_caso2} para coárboles que incluyen una obstrucción.}
\label{fig_certificador_caso2_02}
\end{figure}


\subsubsection{Algoritmo certificador}

Finalmente, el Algoritmo \ref{alg_cert_m2} utiliza los algoritmos anteriores para colorear las hojas del coárbol recibido como entrada, $g$. En el caso de que la gráfica sea conexa (líneas 4 a 8), simplemente se llama el algoritmo para cada una de los hijos de $g$. Esto no significa que sea un algoritmo recursivo, dado que, para las gráficas inconexas y las hojas, el algoritmo no vuelve a ser llamado. En el caso de que la gráfica sea inconexa, se ejecuta el bloque de las líneas 10 a 21. En las líneas 10 a 15 se cuenta el número de componentes conexas de la gráfica representada (es decir que se cuentan los hijos de $g$ que no son hojas). Y por último se toma la decisión de qué caso debe llamarse.


\begin{algorithm}[!htbp]
\caption{M2\_Certificador}
\label{alg_cert_m2}

\DontPrintSemicolon % Some LaTeX compilers require you to use \dontprintsemicolon instead
\KwIn{$g$, la raíz de un coárbol, $G$}
\KwOut{Verdadero si la gráfica representada por $G$ pertenece a la clase $M_2$. Falso en el caso contrario. Las hojas de $G$ se colorean.}

    \If{$g$ \text{es una hoja}}{
        $g.color \gets azul$\;
        $\Return\ verdadero$\;
    }
    \ElseIf{$g.etiqueta = 1$}{
        \For{child \textbf{\emph{en}} $g.hijos$}{
            \If{\emph{M2\_Certificador(}child\emph{)} = falso}{
                $\Return\ falso$\;
            }
            $\Return\ verdadero$\;
        }
    }
    \Else{
        $componentes\_no\_triviales \gets 0$\;
        \For{child \textbf{\emph{en}} $g.hijos$}{
            \If{$child$ \text{es una hoja}}{
                $child.color \gets azul$\;
            }
            \Else{
                $componentes\_no\_triviales \gets componentes\_no\_triviales + 1$\;
            }
        }
        \If{componentes\_no\_triviales = 0}{
            $\Return\ verdadero$\;
        }
        \ElseIf{componentes\_no\_triviales = 1}{
            $\Return$ M2\_Caso\_2($g$)\;
        }
        \Else{
            $\Return$ M2\_Caso\_1($g$)\;
        }
    }


$\Return\ verdadero$\;

\end{algorithm}


\section{Subclases de $M_2$}

    \subsection{Clases $(\alpha, \beta)-M_2$}

    \subsection{Conjunto de parejas mínimas}

    \subsection{Reconocimiento de las clases $(\alpha, \beta)-M_2$}

    \subsection{Algoritmo para generar obstrucciones mínimas}

\section{Particiones en más de dos partes}
    \subsection{Las Clases $M_i$}

    \subsection{Obstrucciones mínimas de la clase $M_3$}
        \begin{theorem} \label{teo_obsts_m2}

    Para una cográfica $G$, las siguientes afirmaciones son equivalentes.
    \begin{enumerate}[(a)] 
        \item $G \in M_3$.
        \item $G$ no contiene a ninguna de las gráficas de las Figuras \ref{obsts_O_M3} como subgráfica inducida.
    \end{enumerate}

\end{theorem}

\begin{figure}[h]





\begin{subfigure}{\textwidth}
\begin{center}
\begin{tikzpicture}
\begin{scope}[xshift=0cm,scale=1]
%K4
\node [style=vertex] (1) at (0,1) {};
\node [style=vertex] (2) at (1,1) {};
\node [style=vertex] (3) at (0.5,1.3) {};
\node [style=vertex] (4) at (0.5,1.75) {};
%K3
\node [style=vertex] (5) at (0,2.25) {};
\node [style=vertex] (6) at (1,2.25) {};
\node [style=vertex] (7) at (0.5,3) {};
%K2
\node [style=vertex] (8) at (0,3.5) {};
\node [style=vertex] (9) at (1,3.5) {};
%K1
\node [style=vertex] (10) at (0.5,4) {};

\foreach \i/\j in {1/2,1/3,1/4,2/3,2/4,3/4,5/6,5/7,6/7,8/9} \draw [style=edge] (\i) to (\j);
\node at (0.5,0) {\parbox{0.3\linewidth}{\subcaption*{$O_{3,1}$}}};
\end{scope}

\begin{scope}[xshift=2.5cm,scale=1]
%K4
\node [style=vertex] (1) at (0,1) {};
\node [style=vertex] (2) at (1,1) {};
\node [style=vertex] (3) at (0.5,1.3) {};
\node [style=vertex] (4) at (0.5,1.75) {};
%K3
\node [style=vertex] (5) at (0,2.25) {};
\node [style=vertex] (6) at (1,2.25) {};
\node [style=vertex] (7) at (0.5,3) {};
%K2
\node [style=vertex] (8) at (0,4) {};
\node [style=vertex] (9) at (1,4) {};
%K1
\node [style=vertex] (10) at (0.5,3.5) {};

\foreach \i/\j in {1/2,1/3,1/4,2/3,2/4,3/4,5/6,5/7,6/7,8/9} \draw [style=edge] (\i) to (\j);
\foreach \i/\j in {7/10} \draw [style=edge] (\i) to (\j);
\node at (0.5,0) {\parbox{0.3\linewidth}{\subcaption*{$O_{3,2}$}}};
\end{scope}

\begin{scope}[xshift=5cm,scale=1]
%K4
\node [style=vertex] (1) at (0,1) {};
\node [style=vertex] (2) at (1,1) {};
\node [style=vertex] (3) at (0.5,1.3) {};
\node [style=vertex] (4) at (0.5,1.75) {};
%K3
\node [style=vertex] (5) at (0,2.75) {};
\node [style=vertex] (6) at (1,2.75) {};
\node [style=vertex] (7) at (0.5,3.5) {};
%K2
\node [style=vertex] (8) at (0,4) {};
\node [style=vertex] (9) at (1,4) {};
%K1
\node [style=vertex] (10) at (0.5,2.25) {};

\foreach \i/\j in {1/2,1/3,1/4,2/3,2/4,3/4,5/6,5/7,6/7,8/9} \draw [style=edge] (\i) to (\j);
\foreach \i/\j in {4/10} \draw [style=edge] (\i) to (\j);
\node at (0.5,0) {\parbox{0.3\linewidth}{\subcaption*{$O_{3,3}$}}};
\end{scope}

\begin{scope}[xshift=7.5cm,scale=1]
%K4
\node [style=vertex] (1) at (0,1) {};
\node [style=vertex] (2) at (1,1) {};
\node [style=vertex] (3) at (0.5,1.3) {};
\node [style=vertex] (4) at (0.5,1.75) {};
%K3
\node [style=vertex] (5) at (0,2.75) {};
\node [style=vertex] (6) at (1,2.75) {};
\node [style=vertex] (7) at (0.5,3.5) {};
%K2
\node [style=vertex] (8) at (0,4) {};
\node [style=vertex] (9) at (1,4) {};
%K1
\node [style=vertex] (10) at (0.5,2.25) {};

\foreach \i/\j in {1/2,1/3,1/4,2/3,2/4,3/4,5/6,5/7,6/7,8/9} \draw [style=edge] (\i) to (\j);
\foreach \i/\j in {4/10,1/10} \draw [style=edge] (\i) to (\j);
\node at (0.5,0) {\parbox{0.3\linewidth}{\subcaption*{$O_{3,4}$}}};
\end{scope}

\begin{scope}[xshift=10cm,scale=1]
%K4
\node [style=vertex] (1) at (0,1) {};
\node [style=vertex] (2) at (1,1) {};
\node [style=vertex] (3) at (0.5,1.3) {};
\node [style=vertex] (4) at (0.5,1.75) {};
%K3
\node [style=vertex] (5) at (0,2.75) {};
\node [style=vertex] (6) at (1,2.75) {};
\node [style=vertex] (7) at (0.5,3.5) {};
%K2
\node [style=vertex] (8) at (0,2.25) {};
\node [style=vertex] (9) at (1,2.25) {};
%K1
\node [style=vertex] (10) at (0.5,4) {};

\foreach \i/\j in {1/2,1/3,1/4,2/3,2/4,3/4,5/6,5/7,6/7,8/9} \draw [style=edge] (\i) to (\j);
\foreach \i/\j in {4/8,4/9} \draw [style=edge] (\i) to (\j);
\node at (0.5,0) {\parbox{0.3\linewidth}{\subcaption*{$O_{3,5}$}}};
\end{scope}
\end{tikzpicture}
\end{center}
\end{subfigure}

\begin{subfigure}{\textwidth}
\begin{center}
\begin{tikzpicture}

\begin{scope}[xshift=0cm,scale=1]
%K4
\node [style=vertex] (1) at (0,1) {};
\node [style=vertex] (2) at (1,1) {};
\node [style=vertex] (3) at (0.5,1.3) {};
\node [style=vertex] (4) at (0.5,1.75) {};
%K3
\node [style=vertex] (5) at (0,3.25) {};
\node [style=vertex] (6) at (1,3.25) {};
\node [style=vertex] (7) at (0.5,4) {};
%K2
\node [style=vertex] (8) at (0.5,2.5) {};
\node [style=vertex] (9) at (1,2.5) {};
%K1
\node [style=vertex] (10) at (0,2.5) {};

\foreach \i/\j in {1/2,1/3,1/4,2/3,2/4,3/4,5/6,5/7,6/7,8/9} \draw [style=edge] (\i) to (\j);
\foreach \i/\j in {4/8,4/9,4/10} \draw [style=edge] (\i) to (\j);
\node at (0.5,0) {\parbox{0.3\linewidth}{\subcaption*{$O_{3,6}$}}};
\end{scope}

\begin{scope}[xshift=2.5cm,scale=1]
%K4
\node [style=vertex] (1) at (0,1) {};
\node [style=vertex] (2) at (1,1) {};
\node [style=vertex] (3) at (0.5,1.3) {};
\node [style=vertex] (4) at (0.5,1.75) {};
%K3
\node [style=vertex] (5) at (0,3.25) {};
\node [style=vertex] (6) at (1,3.25) {};
\node [style=vertex] (7) at (0.5,4) {};
%K2
\node [style=vertex] (8) at (0.5,2.5) {};
\node [style=vertex] (9) at (1,2.5) {};
%K1
\node [style=vertex] (10) at (0,2.5) {};

\foreach \i/\j in {1/2,1/3,1/4,2/3,2/4,3/4,5/6,5/7,6/7,8/9} \draw [style=edge] (\i) to (\j);
\foreach \i/\j in {4/8,4/9,4/10,1/10} \draw [style=edge] (\i) to (\j);
\node at (0.5,0) {\parbox{0.3\linewidth}{\subcaption*{$O_{3,7}$}}};
\end{scope}

\begin{scope}[xshift=5cm,scale=1]
%K4
\node [style=vertex] (1) at (0,1) {};
\node [style=vertex] (2) at (1,1) {};
\node [style=vertex] (3) at (0.5,1.3) {};
\node [style=vertex] (4) at (0.5,1.75) {};
%K3
\node [style=vertex] (5) at (0,2.75) {};
\node [style=vertex] (6) at (1,2.75) {};
\node [style=vertex] (7) at (0.5,3.5) {};
%K2
\node [style=vertex] (8) at (0,2.25) {};
\node [style=vertex] (9) at (1,2.25) {};
%K1
\node [style=vertex] (10) at (0.5,4) {};

\foreach \i/\j in {1/2,1/3,1/4,2/3,2/4,3/4,5/6,5/7,6/7,8/9} \draw [style=edge] (\i) to (\j);
\foreach \i/\j in {4/8,4/9,7/10} \draw [style=edge] (\i) to (\j);
\node at (0.5,0) {\parbox{0.3\linewidth}{\subcaption*{$O_{3,5}$}}};
\end{scope}

\end{tikzpicture}
\end{center}
\end{subfigure}

\setlength{\abovecaptionskip}{-15pt}
\caption{Obstrucciones mínimas para la clase $M_2$.}
\label{obsts_O_M3}
\end{figure}

\begin{proof}

\end{proof}


    \subsection{Familia $O$ de obstrucciones}

    \subsection{Familia $P$ de obstrucciones}
