En este capítulo presentamos el producto de nuestra investigación. La primera
sección del capítulo tiene como resultado principal un algoritmo capaz de
determinar si una cográfica pertenece a una clase hereditaria de gráficas fija
en tiempo lineal. A partir de la segunda sección abordamos el problema
principal de nuestra tesis, determinar si una cográfica acepta una partición en
un número dado de gráficas multipartitas completas. Comenzamos realizando un
estudio detallado de la clase de cográficas que aceptan una partición en dos
gráficas multipartitas completas, a la que llamamos $M_2$. En la segunda
sección del capítulo caracterizamos a la clase $M_2$ a través de su conjunto de
obstrucciones mínimas, proporcionamos un algoritmo para reconocer a sus
elementos (utilizando el resultado principal de la primera sección) y
presentamos un algoritmo certificador que no sólo es capaz de reconocer si una
cográfica $G$ pertenece a  $M_2$, sino que encuentra una partición de $G$ en
dos gráficas multipartitas completas o una obstrucción mínima de la clase como
subgráfica inducida de $G$. En la tercera sección estudiamos a las clases
$(\alpha, \beta)-M_2$, subclases de $M_2$ cuyos elementos aceptan una partición
en dos gráficas multipartitas completas de tamaños restringidos. El resultado
principal de esta sección es un algoritmo para encontrar obstrucciones mínimas
para cualquier clase $(\alpha, \beta)-M_2$. La cuarta y última sección del
capítulo da un paso en la generalización de los resultados de las secciones
anteriores. En éste se proporcionan algunas familias de obstrucciones mínimas
para caracterizar a la clase de cográficas que aceptan una partición en $i$
gráficas multipartitas completas dado un entero $i \geq 2$.

\section{Términos y algoritmos generales}

    En esta sección se presenta un conjunto de conceptos y algoritmos
    \'utiles para cogr\'aficas en general. El resultado
    principal de la sección es un algoritmo de tiempo lineal capaz de
    determinar si una cográfica, representada por su coárbol, pertenece a una
    clase hereditaria fija de cográficas caracterizada por su conjunto de
    obstrucciones mínimas.

    \subsection{Coárbol binario}
        Tomando como base el concepto de coárbol, podemos imaginar otra estructura de tipo árbol para la representación de las cográficas en la que cada nodo tenga a lo más un número $k$ de hijos. Esta limitante resulta útil para formular algoritmos rápidos en cográficas. El menor valor que puede tomar $k$ es de 2. Este valor será utilizado a lo largo de este capítulo para representar a las cográficas a través de árboles binarios.


\begin{definition}{}

    Sean $G=(V,E)$ una cográfica, $C = c_1, c_2, \dots, c_n$ el conjunto de las componentes conexas de $G$, $D = {d_1, d_2, \dots, d_m}$ el conjunto de las componentes conexas de $\overline{G}$, $(C_1, C_2)$ una partición en dos partes de $C$ y $(D_1, D_2)$ una partición en dos partes de $D$, decimos que el árbol binario arraigado etiquetado, $(T,r)$, es un \textbf{\emph{coárbol binario}} de $G$ si se puede construir de la siguiente manera: Si $G$ consta de un sólo vértice, entonces $T$ sólo contiene a $r$. De lo contrario, si $G$ es conexa, entonces $r$ tiene la etiqueta $1$, uno de los hijos de $r$ es el coárbol binario de $G-D_1$ y el otro es el coárbol binario de $G-D_2$. Y finalmente, si $G$ es inconexa, entonces $r$ tiene la etiqueta $0$, uno de sus hijos es el coárbol binario de $G-C_1$ y el otro el coárbol binario de $G-C_2$.


\end{definition}

Claramente, una cográfica puede ser representada por más de un coárbol binario diferente como se muestra en la Figura \ref{fig_coar_bin01}. Sin embargo, la propiedad de que dos vértices son adyacentes si y sólo si su ancestro común más profundo tiene la etiqueta 1 se mantiene.

\begin{figure}[ht!]
\begin{center}
\begin{tikzpicture}

\begin{scope}[xshift=0cm,scale=1]

\node [vertex] (1) at (0,0) {};
\node [vertex] (2) at (1,0) {};
\node [vertex] (3) at (0,1) {};
\node [vertex] (4) at (1,1) {};
\foreach \i/\j in {1/2,1/3,1/4,2/3,2/4,3/4}
  \draw [edge] (\i) to (\j);
\node [below of=1,xshift=.5cm] {\parbox{0.3\linewidth}{\subcaption{}}};

\end{scope}

\begin{scope}[xshift=3.5cm,scale=1]

\node [cotreenode] (1) at (1,1) {1};
\node [cotreenode] (2) at (0,0) {1};
\node [cotreenode] (3) at (2,0) {1};
\node [vertex] (4) at (-0.5,-1) {};
\node [vertex] (5) at (0.5,-1) {};
\node [vertex] (6) at (1.5,-1) {};
\node [vertex] (7) at (2.5,-1) {};
\foreach \i/\j in {1/2,1/3,2/4,2/5,2/4,3/6,3/7}
  \draw [edge] (\i) to (\j);
\node [below of=5,xshift=.5cm] {\parbox{0.3\linewidth}{\subcaption{}}};

\end{scope}

\begin{scope}[xshift=7.5cm,scale=1]

\node [cotreenode] (1) at (1,1) {1};
\node [vertex] (2) at (0,0) {};
\node [cotreenode] (3) at (2,0) {1};
\node [vertex] (6) at (1,-1) {};
\node [cotreenode] (7) at (3,-1) {1};
\node [vertex] (8) at (2,-2) {};
\node [vertex] (9) at (4,-2) {};

\foreach \i/\j in {1/2,1/3,3/6,3/7,7/8,7/9}
  \draw [edge] (\i) to (\j);
\node [below of=8] {\parbox{0.3\linewidth}{\subcaption{}}};

\end{scope}


\end{tikzpicture}
\end{center}
\setlength{\abovecaptionskip}{-10pt}
\caption{(b) y (c) son dos coárboles binarios diferentes que representan a la cográfica (a).}\label{fig_coar_bin01}
\end{figure}

    \subsection{Algoritmo para generar un coárbol binario}
        Podemos obtener un coárbol binario a partir de un coárbol con el Algoritmo \ref{alg_coa_bin01}. En este algoritmo un nodo interno con al menos tres hijos, $r$, de un coárbol, se procesa creando un nuevo coárbol binario de la siguiente forma: La raíz del coárbol binario tiene como primer hijo al coárbol binario resultante de procesar al primer hijo de $r$ y como segundo hijo un nodo con la misma etiqueta de $r$ que a su vez tiene como primer hijo al árbol binario resultante de procesar al segundo hijo de $r$ y como segundo hijo un nuevo nodo con la misma etiqueta y así sucesivamente. Cuando sólo quedan los últimos dos hijos de $r$, estos se procesan y los árboles binarios resultantes son los hijos del último nodo creado. El árbol binario resultante es un árbol cargado a la derecha. La Figura \ref{fig_alg_coa_bin01} muestra una ejecución ilustrativa del algoritmo.

\begin{algorithm}[ht!]
\caption{CrearArbolBinario}
\label{alg_coa_bin01}
\DontPrintSemicolon % Some LaTeX compilers require you to use \dontprintsemicolon instead
\KwIn{$r$ la raíz del coárbol}
\KwOut{$r'$ la raíz del coárbol binario}

$r' \gets \text{nuevo nodo de árbol binario}$\;

\If{$r\ \emph{es un nodo interno} $}{
    $r'.etiqueta = r.etiqueta$\;
    $s \gets r'$\;
    $i \gets 0$\;
    \While{$i < r.children.size - 2$}{
        $s.primerHijo \gets \text{CrearArbolBinario}(r.hijos[i])$\;
        $s.segundoHijo \gets \text{nuevo nodo de árbol binario}$\;
        $s \gets s.segundoHijo$\;
        $s.etiqueta \gets r.etiqueta$\;
        $i \gets i+1$\;
    }
    $s.primerHijo \gets \text{CrearArbolBinario}(r.hijos[i])$\;
    $s.segundoHijo \gets \text{CrearArbolBinario}(r.hijos[i+1])$\;
}
\Return $r'$\;

\end{algorithm}

\begin{figure}[ht!]
\centering

\begin{subfigure}{0.7\textwidth}
\begin{tikzpicture}
\begin{scope}[xshift=0cm,scale=1]
\node [style=cotreenode, fill=lightgray] (1) at (2,1) {1};
\node [style=vertex] (2) at (0.5,0) {};
\node [style=vertex] (3) at (1.5,0) {};
\node [style=vertex] (4) at (2.5,0) {};
\node [style=cotreenode] (5) at (3.5,0) {0};
\node [style=vertex] (6) at (3,-1) {};
\node [style=vertex] (7) at (4,-1) {};
\foreach \i/\j in {1/2,1/3,1/4,1/5,5/6,5/7}
  \draw [style=edge] (\i) to (\j);
\end{scope}
\begin{scope}[xshift=6.5cm,scale=1]
\node [style=cotreenode] (1) at (1,1) {1};
\node [style=vertex] (2) at (0.5,0) {};
\node [style=cotreenode] (3) at (1.5,0) {1};
\foreach \i/\j in {1/2,1/3}
  \draw [style=edge] (\i) to (\j);
\end{scope}
\end{tikzpicture}
\end{subfigure}

\par\bigskip

\begin{subfigure}{0.7\textwidth}
\begin{tikzpicture}
\begin{scope}[xshift=0cm,scale=1]
\node [style=cotreenode, fill=lightgray] (1) at (2,1) {1};
\node [style=vertex] (2) at (0.5,0) {};
\node [style=vertex] (3) at (1.5,0) {};
\node [style=vertex] (4) at (2.5,0) {};
\node [style=cotreenode] (5) at (3.5,0) {0};
\node [style=vertex] (6) at (3,-1) {};
\node [style=vertex] (7) at (4,-1) {};
\foreach \i/\j in {1/2,1/3,1/4,1/5,5/6,5/7}
  \draw [style=edge] (\i) to (\j);
\end{scope}
\begin{scope}[xshift=6.5cm,scale=1]
\node [style=cotreenode] (1) at (1,1) {1};
\node [style=vertex] (2) at (0.5,0) {};
\node [style=cotreenode] (3) at (1.5,0) {1};
\node [style=vertex] (4) at (1,-1) {};
\node [style=cotreenode] (5) at (2,-1) {1};
\foreach \i/\j in {1/2,1/3,3/4,3/5}
  \draw [style=edge] (\i) to (\j);
\end{scope}
\end{tikzpicture}
\end{subfigure}

\par\bigskip

\begin{subfigure}{0.7\textwidth}
\begin{tikzpicture}
\begin{scope}[xshift=0cm,scale=1]
\node [style=cotreenode, fill=lightgray] (1) at (2,1) {1};
\node [style=vertex] (2) at (0.5,0) {};
\node [style=vertex] (3) at (1.5,0) {};
\node [style=vertex] (4) at (2.5,0) {};
\node [style=cotreenode] (5) at (3.5,0) {0};
\node [style=vertex] (6) at (3,-1) {};
\node [style=vertex] (7) at (4,-1) {};
\foreach \i/\j in {1/2,1/3,1/4,1/5,5/6,5/7}
  \draw [style=edge] (\i) to (\j);
\end{scope}
\begin{scope}[xshift=6.5cm,scale=1]
\node [style=cotreenode] (1) at (1,1) {1};
\node [style=vertex] (2) at (0.5,0) {};
\node [style=cotreenode] (3) at (1.5,0) {1};
\node [style=vertex] (4) at (1,-1) {};
\node [style=cotreenode] (5) at (2,-1) {1};
\node [style=vertex] (6) at (1.5,-2) {};
\node [style=cotreenode] (7) at (2.5,-2) {0};
\foreach \i/\j in {1/2,1/3,3/4,3/5,5/6,5/7}
  \draw [style=edge] (\i) to (\j);
\end{scope}
\end{tikzpicture}
\end{subfigure}

\par\bigskip

\begin{subfigure}{0.7\textwidth}
\begin{tikzpicture}
\begin{scope}[xshift=0cm,scale=1]
\node [style=cotreenode] (1) at (2,1) {1};
\node [style=vertex] (2) at (0.5,0) {};
\node [style=vertex] (3) at (1.5,0) {};
\node [style=vertex] (4) at (2.5,0) {};
\node [style=cotreenode, fill=lightgray] (5) at (3.5,0) {0};
\node [style=vertex] (6) at (3,-1) {};
\node [style=vertex] (7) at (4,-1) {};
\foreach \i/\j in {1/2,1/3,1/4,1/5,5/6,5/7}
  \draw [style=edge] (\i) to (\j);
\end{scope}
\begin{scope}[xshift=6.5cm,scale=1]
\node [style=cotreenode] (1) at (1,1) {1};
\node [style=vertex] (2) at (0.5,0) {};
\node [style=cotreenode] (3) at (1.5,0) {1};
\node [style=vertex] (4) at (1,-1) {};
\node [style=cotreenode] (5) at (2,-1) {1};
\node [style=vertex] (6) at (1.5,-2) {};
\node [style=cotreenode] (7) at (2.5,-2) {0};
\node [style=vertex] (8) at (2,-3) {};
\node [style=vertex] (9) at (3,-3) {};
\foreach \i/\j in {1/2,1/3,3/4,3/5,5/6,5/7,7/8,7/9}
  \draw [style=edge] (\i) to (\j);
\end{scope}
\end{tikzpicture}
\end{subfigure}


\caption{Ejemplo de la ejecución del Algoritmo \ref{alg_coa_bin01}. A la izquierda se muestra el coárbol original, mienrtras se marca con gris el nodo que se está procesando. A la derecha aparece el coárbol binario que se va construyendo.}\label{fig_alg_coa_bin01}


\end{figure}


En términos de las particiones de las componentes conexas de la gráfica, el algoritmo realiza lo siguiente. Si la etiqueta de $r$ es $0$, entonces el coárbol con raíz en $r$ representa una cográfica inconexa y se elige la partición de sus vértices en la que la primera parte es una componente conexa y la segunda parte es el resto. Sucede lo mismo si la etiqueta de $r$ es uno, pero como la cográfica representada es conexa, en su lugar se toman una componente conexa del complemento de la cográfica representada en la primera parte y el resto en la segunda.

Dado que el Algoritmo \ref{alg_coa_bin01} recorre a lo más una vez cada nodo de $r$, su tiempo de ejecución es $O(n)$ en donde $n$ es el número total de nodos del árbol con raíz $r$.

    \subsection{Algoritmo para generar todos los coárboles binarios de una gráfica}
        \iffalse

Podemos obtener todos los coárboles binarios correspondientes a un coárbol haciendo uso del Algoritmo \ref{alg_coa_bin02}. Este algoritmo recibe como entrada la raíz del coárbol, $r$, y devuelve un conjunto de nodos, $S$, cada uno de cuyos elementos es la raíz de un coárbol binario. Los nodos internos son procesados creando un nuevo coárbol binario para cada posible partición del conjunto de hijos de dicho nodo. Al procesar las hojas, simplemente se crea un nuevo nodo que será una hoja en los árboles binarios. 


\begin{algorithm}[h]
\caption{CrearÁrbolesBinarios}
\label{alg_coa_bin02}
\DontPrintSemicolon % Some LaTeX compilers require you to use \dontprintsemicolon instead
\KwIn{$r$ la raíz de un coárbol}
\KwOut{$S = \{r'_1, r'_2, \dots, r'_n\}$ con $r_i$ la raíz de un coárbol binario}

$r' \gets \text{nuevo nodo de árbol binario}$\;

\If{$r\ \emph{es un nodo interno} $}{
    $r'.etiqueta = r.etiqueta$\;
    $s \gets r'$\;
    $i \gets 0\;
    \While{$i < r.children.size - 2$}{
        $s.primerHijo = \text{CrearArbolBinario}(r.hijos[i])$\;
        $s.segundoHijo \gets \text{nuevo nodo de árbol binario}$\;
        $s \gets s.segundoHijo$\;
        $s.etiqueta = r.etiqueta$\;
        $i = i+1$\;
    }
    $s.primerHijo = \text{CrearArbolBinario}(r.hijos[i])$\;
    $s.segundoHijo = \text{CrearArbolBinario}(r.hijos[i+1])$\;
}
\Return $r'$\;
    
\end{algorithm}

\fi
    \subsection{Subcoárbol}
        A continuación presentamos los conceptos de subcoárbol y subcoárbol binario
que serán utilizados para determinar si una cográfica $H$ es subgráfica
inducida de una cográfica $G$ en el Algoritmo \ref{alg_subgraph}.

Sean $T$ y $U$ dos coárboles y $u_1$, $u_2$ y $u_3$ nodos de $U$, decimos
que $U$ es un \emph{\textbf{subcoárbol}} de $T$ si existe una función
inyectiva $f:V(U)\rightarrow V(T)$ tal que, si $u_1$ es una hoja, entonces
$f(u_1)$ tambi\'en es una hoja; si no, entonces $u_1$ y $f(u_1)$ tienen la
misma etiqueta y, si $u_3$ es el ancestro común más profundo de $u_1$ y
$u_2$, entonces $f(u_3)$ es el ancestro común más profundo de $f(u_1)$ y
$f(u_2)$. Llamamos a $f$ la \textbf{\emph{función de coasignación}} de $U$ a
$T$.

El concepto de subcoárbol es diferente del de subárbol dado que, si $T$ y $U$
son coárboles con $U$ subcoárbol de $T$, entonces tenemos que los nodos de $U$
se pueden encontrar dispersos entre los nodos de $T$ a diferencia de lo que se
tendría si $U$ fuera subárbol de $T$. Esto se puede apreciar en la Figura
\ref{fig_subcoarbol01}. N\'otese que esta definición funciona también para
coárboles binarios.

\begin{figure}[h!]
\begin{center}
\begin{tikzpicture}

\begin{scope}[xshift=0cm,scale=1]
\node [style=cotreenode] (1) at (1,1) {0};
\node [style=cotreenode] (2) at (-0.5,0) {1};
\node [style=cotreenode] (3) at (2.5,0) {1};
\node [style=cotreenode] (4) at (-1.25,-1) {0};
\node [style=cotreenode] (5) at (0.25,-1) {0};
\node [style=cotreenode] (6) at (1.75,-1) {0};
\node [style=cotreenode] (7) at (3.25,-1) {0};
\node [style=vertex] (8) at (-1.5,-2) {};
\node [style=vertex] (9) at (-1,-2) {};
\node [style=vertex] (10) at (0,-2) {};
\node [style=vertex] (11) at (0.5,-2) {};
\node [style=vertex] (12) at (1.5,-2) {};
\node [style=vertex] (13) at (2,-2) {};
\node [style=vertex] (14) at (3,-2) {};
\node [style=vertex] (15) at (3.5,-2) {};

\node (16) at (0.25,1) {$f(a)$};
\node (17) at (-1.6,-2.4) {$f(b)$};
\node (18) at (3.25,0) {$f(c)$};
\node (19) at (1.4,-2.4) {$f(d)$};
\node (20) at (3.6,-2.4) {$f(e)$};

\foreach \i/\j in {1/2,1/3,2/4,2/5,3/6,3/7,4/8,4/9,5/10,5/11,6/12,6/13,7/14,7/15}
  \draw [style=edge] (\i) to (\j);
\node [below of=19,xshift=-0.25cm] {\parbox{0.3\linewidth}{\subcaption{}}};
\end{scope}

\begin{scope}[xshift=6cm,scale=1]
\node [style=cotreenode] (1) at (1,1) {0};
\node [style=vertex] (2) at (0,0) {};
\node [style=cotreenode] (3) at (2,0) {1};
\node [style=vertex] (4) at (1.5,-1) {};
\node [style=vertex] (5) at (2.5,-1) {};

\node (6) at (0.5,1) {$a$};
\node (7) at (-0.3,0) {$b$};
\node (8) at (2.5,0) {$c$};
\node (9) at (1.5,-1.3) {$d$};
\node (10) at (2.5,-1.3) {$e$};

\foreach \i/\j in {1/2,1/3,3/4,3/5}
  \draw [style=edge] (\i) to (\j);
\node [below of=9,xshift=-0.25cm] {\parbox{0.3\linewidth}{\subcaption{}}};
\end{scope}

\end{tikzpicture}
\end{center}
\setlength{\abovecaptionskip}{-10pt}
\caption{El coárbol (b) es subcoárbol del coárbol (a). Las etiquetas en los nodos de ambos coárboles indican la asignación de los nodos de (b) a los nodos de (a).}\label{fig_subcoarbol01}
\end{figure}

\begin{lemma}\label{lema_subcoa_01}
    Sean $G$ y $H$ cográficas y $T_G$ y $T_H$ sus coárboles correspondientes, entonces $H$ es una subgráfica inducida de $G$ si y sólo si $T_H$ es subcoárbol de $T_G$.
\end{lemma}

\begin{proof}
    {\color{red}
    Revisa esta demostraci\'on.   En particular, al principio, cuando
    est'as declarando variables, declaras a $f$ como si ya existiera,
    pero justamente quieres demostrar su existencia.   En lugar de
    declararla con las dem\'as variables, puedes decir que vamos a
    definir una funci\'on $f$ con $f \colon V(T_H)\rightarrow V(T_G)$,
    tal que, etc. y verificaremos que es una coasignaci\'on.   De
    entrada, no es clara la definici\'on de $f$; parece ser una
    definici\'on recursiva, de abajo hacia arriba, pero esto tampoco
    es claro.
    }
    Supongamos primero que $H$ es una subgráfica inducida de $G$. Sabemos que
    $V(H)\subset V(G)$ y que dos vértices son adyacentes en $H$ si y sólo si
    también son adyacentes en $G$. Sean $v$ y $w$ vértices de $H$, $n_H$ el
    ancestro común más profundo de $v$ y $w$ en $T_H$, $n_G$ el ancestro común
    más profundo de $v$ y $w$ en $T_G$ y $f:V(T_H)\rightarrow V(T_G)$ una
    función tal que $f(v)=v$ y $f(n_H) = n_G$, mostremos que $f$ es una función
    de coasignación de $T_H$ a $T_G$. Sean $x,y,z\in V(T_H)$. Si $x$ es una
    hoja de $T_H$, es claro que $f(x) = x$ es una hoja de $T_G$. Si no,
    entonces $x$ es el ancestro común más profundo de algún par de hojas de
    $T_H$ y tiene etiqueta 1 si y sólo si éstas son adyacentes en $H$, lo que
    ocurre si y sólo si son adyacentes en $G$, que a su vez ocurre si y sólo si
    su ancestro común más profundo en $T_G$ tiene etiqueta etiqueta 1. Así $x$
    y $f(x)$ tienen necesariamente la misma etiqueta. Finalmente mostremos que
    si $x$ es el ancestro común más profundo de $y$ y $z$, entonces $f(x)$ es
    el ancestro común más profundo de $f(y)$ y $f(z)$. Si $y$ y $z$ son hojas,
    esto se cumple trivialmente por la definición de $f$. Si $y$ es un nodo
    interno y $z$ es una hoja, entonces existen dos hojas $y'$ y $y''$ de $T_H$
    tales que $y$ es su ancestro común más profundo. Tenemos que $f(y)$ es el
    ancestro común más profundo de $f(y')$ y $f(y'')$. A su vez, dado que $x$
    es el ancestro común más profundo de $y'$ y $z$, entonces $f(x)$ es el
    ancestro común mas profundo de $f(y')$ y $f(z)$. De igual manera $f(x)$ es
    el ancestro común mas profundo de $f(y'')$ y $f(z)$ Luego, $f(y)$ se
    encuentra en el camino desde $f(y')$ hasta $f(x)$ y por lo tanto $f(x)$ es
    el nodo común más profundo de $f(y)$ y $f(z)$. Análogamente si $z$ es un
    nodo interno y $y$ es una hoja. Si tanto $y$ como $z$ son nodos internos,
    el argumento es prácticamente el mismo, pero utilizando dos hojas $y'$ y
    $y''$ cuyo ancestro común más profundo es $y$ y otras dos hojas $z'$ y
    $z''$ cuyo ancestro común más profundo es $z$. Así, $f$ es una función de
    coasignación de $T_H$ a $T_G$ y $T_H$ es subcoárbol de $T_G$.

    Rec\'iprocamente, si $T_H$ es subcoárbol de $T_G$, entonces existe
    una función de coasignación, $f$, de $T_H$ a $T_G$. Luego, sean $h_1$ y
    $h_2$ hojas de $T_H$ y $h_3$ el ancestro común más profundo de $h_1$ y
    $h_2$, tenemos que $f(h_1)$ y $f(h_2)$ son hojas de $T_G$ y que
    $h_3.etiqueta = f(h_3).etiqueta$. Así, $h_1$ y $h_2$ son adyacentes
    en $H$ si y sólo si $f(h_1)$ y $f(h_2)$ son adyacentes en $G$. Luego,
    $G[f[V(H)]]$ es una subgráfica de $G$ que es isomorfa a $H$. Así, $H$
    es subgráfica de $G$.

\end{proof}

Notemos que esta demostración funciona únicamente para los coárboles y no para los coárboles binarios. En la Figura \ref{fig_coar_bin01} se observan dos coárboles binarios que representan a la misma cográfica. Sin embargo ninguno de los dos es subcoárbol del otro.

\begin{lemma}
    Sean $G$ y $H$ cográficas y $T_G$ y $T_H$ coárboles binarios de $G$ y $H$ respectivamente. Si $T_H$ es subcoárbol de $T_G$, entonces $H$ es subgráfica de $G$.
\end{lemma}

\begin{proof}
    La demostración es igual a la segunda parte de la demostración del Lema \ref{lema_subcoa_01}.
\end{proof}

    \subsection{Algoritmo para encontrar obstrucciones mínimas} \label{sec_AlgoSub}
        La presente sección aborda el problema de determinar si una cográfica $G$ tiene a otra cográfica $H$ como subgráfica inducida haciendo uso de los conceptos de coárbol binario y subcoárbol. Se proporciona un algoritmo (Algoritmo \ref{alg_subgraph}) para resolver este problema tal que, si se fija el tamaño de $H$, su tiempo de ejecución crece de forma lineal con respecto al tamaño de $G$. Este algoritmo es útil para identificar a las gráficas pertenecientes a una clase caracterizada a través de su conjunto de obstrucciones mínimas de forma rápida. 

\subsubsection{Algoritmo para determinar si un coárbol binario es subcoárbol binario de otro}

\begin{definition}
    Sean $T$ y $U$ coárboles (binarios) y $u$ un nodo de $U$, decimos que $f:V(U)\rightarrow\{marcado, no\_marcado\}$ es una \textbf{\emph{función de verificación}} de $T$ para $U$ si $f(u) = marcado$ si y sólo si el coárbol (binario) con raíz en $u$ es subcoárbol (binario) de $T$. Si $f(u) = marcado$, decimos que $f$ \textbf{\emph{marca}} a $u$.
\end{definition}

El Algoritmo \ref{alg_subcoarbol} recibe como entradas dos coárboles binarios, $G$ y $H$ representados por sus raíces $g$ y $h$ respectivamente, y devuelve una función de verificación, $f_g$, de $G$ para $H$. 

Este algoritmo funciona creando la función de verificación de cada subárbol de $G$ para $H$, empezando por los más profundos. De esta manera, si la función de verificación de $G$ para $H$ evaluada en $h$ es $marcado$, entonces  $H$ es subcoárbol de $G$.

\begin{algorithm}[h]
\caption{Función\_de\_coasignación}
\label{alg_subcoarbol}
\DontPrintSemicolon % Some LaTeX compilers require you to use \dontprintsemicolon instead
\KwIn{$g$ y $h$, las raíces de dos coárboles binarios para las gráficas $G$ y $H$ respectivamente}
\KwOut{$func$, la función de verificación de $G$ para $H$}

 $func \gets \text{nueva función de coasignación tal que} func(x)=no\_marcado \text{ para todo } x\in V(H)$\;
 
 \If{g \emph{es una hoja}}{
    $func \text{ marca a todas las hojas de } H$\;
 }
 \Else{
    $v_{izq} \gets \text{Función\_de\_coasignación}(g.izquierda, h)$\;
    $v_{der} \gets \text{Función\_de\_coasignación}(g.derecha, h)$\;
    
    \ForEach{nodo \textbf{\emph{de}} H}{
        \If{$v_{izq}(nodo) = marcado \emph{ \textbf{o} } v_{der}(nodo) = marcado$}{
            $func(nodo) \gets marcado$\;
        }
        \ElseIf{nodo.etiqueta = g.etiqueta \emph{\textbf{y}} $v_{izq}$ \emph{marca a uno de los hijos de} nodo \emph{y} $v_{der}$ \emph{al otro}}{
            $func(nodo) \gets marcado$\;
        }
    }
    
 }

$\Return func$
    
\end{algorithm}
    
\begin{theorem}
    La ejecución del Algoritmo \ref{alg_subcoarbol}, Función\_de\_coasignación $(g, h)$ regresa una función, $func$, tal que $func$ es una función de verificación de $árbol(g)$ para $árbol(h)$.
\end{theorem}

\begin{proof}
    
    Sea $n$ un nodo de $árbol(h)$. Para probar que $func$ es una función de verificación de $árbol(g)$ para $árbol(h)$, tenemos que probar que $func(n) = marcado$ si y sólo si $árbol(n)$ es subcoárbol de $árbol(g)$.

    \textbf{Necesidad}: En esta parte de la demostración, se supone que el algoritmo ha sido ejecutado y que $func$ marca a $n$. Procedamos por inducción sobre la altura de $g$.
    
    \emph{Caso base:} Si $g$ tiene altura 0, entonces $g$ es una hoja, por lo que $func$ marca únicamente a las hojas de $árbol(h)$. Como $func$ marca a $n$, entonces $n$ es una hoja. Luego, la función $f=\{(n,g)\}$ es una función de coasignación de $árbol(n)$ a $árbol(g)$, por lo que $árbol(n)$ es subcoárbol de $árbol(n)$.
    
    \emph{Paso inductivo:} Si $g$ tiene altura $k > 0$. Supongamos como hipotesis inductiva (H.I.) que, para todo nodo de un coárbol binario, $g'$, de altura $k' < k$ se cumple que, si $func' = $ Función\_de\_coasignación$(g',h)$ marca a un nodo $n'$ de $árbol(h)$, entonces $árbol(n')$ es subcoárbol de $árbol(g')$. Como $g$ no es una hoja, el algoritmo debió de entrar al bloque de instrucciones de las líneas 5 a 11. En las líneas 5 y 6 se crean dos funciones que cumplen con la H.I., ya que $g.izquierda$ y $g.derecha$ tienen ambas una altura menor a $k$. Como $func$ marca a $n$, entonces $n$ debe de cumplir la condición de la línea 8 o la condición de la línea 10. Si se cumple la condición de la línea 8, entonces $v_{izq}(n) = marcado$ o $v_{der}(n) = marcado$, por lo que $árbol(n)$ es subcoárbol de $árbol(g.izquierda)$ o de $árbol(g.derecha)$, y por lo tanto es subcoárbol de $árbol(g)$. De lo contrario, se cumple la condición de la línea 10, entonces $v_{izq}$ marca a $n.izquierda$ o a $n.derecha$ y $v_{der}$ marca al otro. Supongamos sin pérdida de generalidad que $v_{izq}$ marca a $n.izquierda$ y $v_{der}$ marca a $n.derecha$. Sean $f_i:V(árbol(n.izquierda))\rightarrow V(árbol(g.izquierda))$ la función de coasignación de $árbol(n.izquierda)$ a $árbol(g.izquierda)$ y $f_d:V(árbol(n.derecha))\rightarrow V(árbol(g.derecha))$ la función de coasignación de $árbol(n.derecha)$ a $árbol(g.derecha)$, mostremos que la función $f = f_i \cup f_d \cup \{(n,g)\}$ es una función de coasignación de $árbol(n)$ a $árbol(g)$. Como los dominios de $f_i$ y $f_d$ son ajenos y ninguno contiene a $n$, entonces $f$ es una función. Como los rangos de $f_i$ y $f_d$ son ajenos, ninguno contiene a $g$ y tanto $f_i$ como $f_d$ son inyectivas, entonces $f$ es inyectiva. Por otra parte, por la condición de la línea 10, sabemos que $n.etiqueta = g.etiqueta$. También sabemos que, sea $x \in V(árbol(n.izquierda))$, si $x$ es una hoja, entonces $f(x) = f_i(x)$ es una hoja y si no, entonces $x.etiqueta = f_i(x).etiqueta = f(x).etiqueta$. Análogamente para un $y \in V(árbol(n.derecha))$ y $f_d$. Finalmente, si $n$ es el ancestro común más profundo de dos nodos $z_1$ y $z_2$, entonces $z_1$ es descendiente de $n.derecha$ y $z_2$ es descendiente de $n.izquierda$ o viceversa. Supongamos lo primero sin pérdida de generalidad. Luego, por la condición de la línea 10, $v_{izq}$ marca a uno y $v_{der}$ marca al otro. Supongamos sin pérdida de generalidad que $v_{izq}$ marca a $z_1$ y $v_{der}$ marca a $z_2$. Entonces, $f(z_1) = f_i(z_1) \in V(árbol(g.izquierda))$ y $f(z_2) = f_i(z_2) \in V(árbol(g.derecha))$, por lo que el ancestro común más profundo de $f(z_1)$ y $f(z_2)$ es $g = f(n)$. Así, $f$ es una función de coasignación de $árbol(n)$ a $árbol(g)$ y $árbol(n)$ es subcoárbol de $árbol(g)$.
    
     \textbf{Suficiencia}: En esta parte de la demostración se supone que $árbol(n)$ es subcoárbol de $árbol(g)$ y se sigue la ejecución del algoritmo para mostrar que, al final de la misma, $func$ marcará a $n$. Sea $f$ la función de cosignación de $árbol(n)$ a $árbol(g)$, procedamos por inducción sobre la altura de $g$.
    
    \emph{Caso base:} Si la altura de $g$ es 0, entonces $g$ es una hoja, por lo que se cumple con la condición de la línea 2 y se ejecuta la línea 3, haciendo que $func$ marque todas las hojas de $H$. Como $árbol(n)$ es subcoárbol de $árbol(g)$ y $árbol(g)$ sólo tiene un nodo, entonces $n$ debe de ser una hoja. Luego, $func$ marca a $n$.
    
    \emph{Paso inductivo:} Si $g$ tiene altura $k > 0$. Supongamos como H.I. que todo coárbol, $g'$, con altura $k' < k$ cumple con que, siendo $n'$ un nodo de $árbol(h)$, si $árbol(n')$ es subcoárbol de $árbol(g')$, entonces $func'=$Función\_de\_coasignación $(g',h)$ marca a $n'$. Como $g$ no es una hoja, el algoritmo ejecuta las líneas 5 y 6 y posteriormente el bloque de las líneas 8 a 11 para cada nodo de $H$. Si $n$ es marcada por $v_{izq}$ o $v_{der}$, entonces se ejecuta la línea 9 y $func$ marca a $n$. En el caso contrario, probemos que se cumple la condición de la línea 10. Mostremos primero que $f(n) = g$ procediendo por contradicción. Supongamos que $f(n) = x$ para algún $x\in V(árbol(g))-\{g\}$. Como $x$ es descendiente de $g$, tiene altura menor a $k$. También sabemos que $f$ es una función de coasignación de $árbol(n)$ a $árbol(x)$, por lo que, por H.I., $n$ debería de ser marcado ya sea por $v_{izq}$ o por $v_{der}$, lo que es una contradicción. Luego, $f(n) = g$, y por lo tanto $n$ no es una hoja y $f(n).etiqueta = g.etiqueta$. Mostremos ahora que tanto $n.izquierda$ como $n.derecha$ son marcados cada uno ya sea por $v_{izq}$ o por $v_{der}$. Sabemos que $f(n.izquierda)$ y $f(n.derecha)$ son descendientes de $r$. Como $f\mid_{V(árbol(n.izquierda))}$ es una función de coasignación de $árbol(n.izquierda)$ a $árbol(f(n.izquierda))$ y $f(n.izquierda) \neq g$ ya que $f$ es inyectiva, entonces $árbol(n.izquierda)$ es subcoárbol de algún descendiente de $g$, al que llamaremos $y$. Como $y$ tiene altura menor a $k$, su función de verificación correspondiente marca a $n.izquierda$ (por H.I.), y por la condición de la línea 8, sus ancestros también lo marcan. Luego $v_{izq}$ o $v_{der}$ marcan a $n.izquierda$. Análogamente para $n.derecha$. Así, tanto $n.izquierda$ como $n.derecha$ están marcados cada uno ya sea en $v_{izq}$ o en $v_{der}$. Mostremos, por último, que uno es marcado po $v_{izq}$ y el otro es marcado por $v_{der}$. Como el ancestro común más profundo de $n.izquierda$ y $n.derecha$ es $n$, y $f(n)=g$, entonces el ancestro común más profundo de $f(n.izquierda)$ y $f(n.derecha)$ debe de ser $g$. Luego, $f(n.izquierda)$ está en una rama de $g$ y $f(n.derecha)$ está en la otra. Supongamos sin pérdida de generalidad que $f(n.izquierda)$ está en la rama izquierda de $g$ y $f(n.derecha)$ está en la rama derecha. Como $f\mid_{V(árbol(n.izquierda))}$ es una función de coasignación de $árbol(n.izquierda)$ a $árbol(g.izquierda)$ y por H.I., entonces $v_{izq}$ marca a $n.izquierda$. De forma análoga, $v_{der}$ marca a $n.derecha$. Concluyendo, como $n.etiqueta = r.etiqueta$ y tanto $n.izquierda$ como $n.derecha$ son marcados uno por $v_{izq}$ y el otro por $v_{der}$, se cumple la condición de la línea 10 y $func$ marca a $n$. Así, al final de la ejecución del algoritmo, $n$ estará marcado.
    
\end{proof}

Dado que, para cada nodo de $G$, se crea una función de verificación cuyo dominio es el conjunto de los nodos de $H$, el tiempo de ejecución del algoritmo crece de la forma $O(\mid V(G) \mid \mid V(H) \mid)$. 

\subsubsection{Determinar si una cográfica es subcográfica de otra}

Haciendo uso del Algoritmo \ref{alg_subcoarbol}, se puede idear otro algoritmo para determinar si una cográfica, $H$ es subgráfica de otra cográfica, $G$, al buscar todas las formas del coárbol binario de $H$ en un solo coárbol binario de $G$. 

\begin{algorithm}[h]
\caption{Es\_subgráfica}
\label{alg_subgraph}
\DontPrintSemicolon % Some LaTeX compilers require you to use \dontprintsemicolon instead
\KwIn{$g$ y $h$, las raíces de dos coárboles, $G$ y $H$ respectivamente.}
\KwOut{$verdadero$ si la cográfica representada por $H$ es subgráfica de la cográfica representada por $G$. $falso$ en el caso contrario.}

$g\_bin \gets \text{CrearÁrbolBinario}(g)$\;
$h\_bins \gets \text{las raíces de todos los coárboles binarios correspondientes a } H$\;

\ForEach{bin \textbf{\emph{en}} h\_bins}{
    $f = \text{Función\_de\_coasignación}(g\_bin,bin)$\;
    \If{f(bin) = marcado}{
        $\Return\ verdadero$\;
    }
}

$\Return\ falso$\;
    
\end{algorithm}

Como la línea 1 Algoritmo \ref{alg_subgraph} se ejecuta en tiempo $O(\mid V(G) \mid)$, la complejidad temporal de éste depende del número de coárboles binarios correspondientes a $H$ (que crece con mayor rapidez). Sin embargo, si se fija $H$, la complejidad temporal de éste es simplemente  $O(\mid V(G) \mid)$. Fijar $H$ resultará útil cuando se esté resolviendo un problema específico como el de encontrar una obstrucción mínima en una gráfica.


\section{La clase $M_2$}

A partir de esta sección abordamos el problema principal de la tesis,
determinar si una cográfica acepta una partición en un número dado de
gráficas multipartitas completas. Empezamos por dar nombre a las clases
de cográficas que estudiamos, y procedemos con el análisis de una de estas
clases que nos servirá de base para estudiar el resto.

Dado un entero positivo $i$, la \textbf{\emph{clase $M_i$}} es la clase de
las cogr\'aficas tales que su conjunto de v\'ertices acepta una partición
$(A_1, A_2, \dots, A_i)$, donde $A_j$ induce una gráfica multipartita
completa para cada $1 \le j \le i$.  Decimos que $(A_1, A_2, \dots, A_i)$
es una $M_i$-partición.

Dado que las gráficas multipartitas completas conforman una clase
hereditaria de gráficas, la clase $M_i$ tambi\'en es una clase hereditaria
de gráficas para cada entero positivo $i$.   Si $G$ es una gr\'afica que
tiene una $M_i$-partición $(A_1, \dots, A_i)$ y $x$ es un v\'ertice de $G$,
entonces $x$ es elemento de $A_j$ para alg\'un $1 \le j \le i$. Como
$G[A_j]-x$ es una gráfica multipartita completa, entonces $(A_1, \dots,
A_j - \{x\}, \dots, A_i)$ es una $M_i$-partici\'on de $G-x$, que resulta
ser un elemento de $M_i$.

Sea $i$ un entero positivo. Como $M_i$ es una clase hereditaria de gráficas,
entonces puede ser caracterizada mediante un conjunto de obstrucciones
mínimas, y se puede identificar a sus elementos en tiempo lineal con el
Algoritmo \ref{alg_esta_en_clase}. El siguiente lema contiene una
observaci\'on que resultar\'a de utilidad, la clase $M_i$ es cerrada bajo
la unión completa.

\begin{lemma}
\label{lema_union_completa}
Sea $i$ un entero mayor o igual a 2.  La clase $M_i$ es cerrada bajo
la unión completa.
\end{lemma}

\begin{proof}
Sean $G,H \in M_i$, sabemos que existen $M_i$-particiones $(A_1, \dots,
A_i)$ y $(B_1, \dots, B_i)$ de $G$ y $H$, respectivamente. Veamos que $P
= (A_1 \cup B_1, \dots, A_i \cup B_i)$ es una $M_i$-partición de $G \oplus H$.

Claramente, cada uno de los vértices de $G \oplus H$ se encuentra en
exactamente uno de los elementos de $P$. Así, $P$ es una partición de los
vértices de $G \oplus H$. Luego, como la unión completa de gráficas
multipartitas completas es una gráfica multipartita, tenemos que $(G \oplus
H)[A_j \cup B_j] = G[A_j] \oplus G[B_j]$ es una gráfica multipartita completa
para cualquier $1 \le j \le i$. Así, $P$ es una $M_i$ partición de $G \oplus
H$ y $G \oplus H \in M_i$.
\end{proof}

Notemos que la clase $M_1$ es la clase de las gráficas multipartitas
completas. En la presente sección nos enfocaremos en la clase $M_2$, que
es la más pequeña de las clases $M_i$ después de $M_1$ (que ha sido
ampliamente estudiada). Caracterizamos a $M_2$ a través de su conjunto de
obstrucciones mínimas, proporcionamos un algoritmo de tiempo lineal para
identificar a sus elementos y presentamos un algoritmo certificador que no
sólo determina si una gráfica $G$ pertenece a $M_2$, sino que colorea las
hojas de su coárbol indicando si se encontró una $M_2$-partición de $G$ o
una obstrucción mínima de la clase como subgráfica inducida de $G$.

    \subsection{Obstrucciones mínimas}
        El primer paso en nuestro estudio de la clase $M_2$ es caracterizar a la
misma a través de su conjunto de obstrucciones mínimas, lo cual se realiza 
en el Teorema \ref{teo_obsts_m2}, cuya demostración requiere del Teorema
\ref{teo_paw} y del Lema \ref{lema_bipartitas}. Recordemos que
la gr\'afica conocida como \textbf{\emph{Paw}} es la gr\'afica obtenida de
$K_3$ al agregar un v\'ertice nuevo y hacerlo adyacente a exactamente un
v\'ertice de $K_3$, o bien $K_1 \oplus (K_1 + K_2)$.

\begin{theorem}[\cite{Olariu}] \label{teo_paw}
	Sea $G$ una gráfica perfecta, $G$ es libre de $Paw$ si y sólo si cada componente de $G$ es libre de $K_3$ o multipartita completa.
\end{theorem}

Aplicando este teorema, podemos concluir lo siguiente. Dado que las cográficas son gráficas perfectas y toda cográfica libre de $K_3$ es bipartita, si una cográfica $G$ es libre de $Paw$, entonces $G$ es bipartita o multipartita completa.

\begin{lemma} \label{lema_bipartitas}
Sea $G$ una cográfica conexa. Si $G$ es bipartita, entonces $G$ es bipartita completa.
\end{lemma}

\begin{proof}
Sea $r$ la raíz del coárbol de $G$. Si $G$ es trivial, es claro que $G$ es bipartita completa. En el caso contrario, $r$ tiene etiqueta 1. Como $G$ es bipartita, entonces es libre de $K_3$. Luego, $r$ tiene exactamente dos hijos, ninguno de los cuales puede contener un $K_2$. Así cada uno de los hijos de $r$ representa a un conjunto independiente. Luego, $G$ es la unión completa de dos conjuntos independientes. Es decir, $G$ es una gráfica bipartita completa.
\end{proof}

\begin{theorem} \label{teo_obsts_m2}

    Para una cográfica $G$, las siguientes afirmaciones son equivalentes.
    \begin{enumerate}[(a)]
        \item $G \in M_2$.
        \item $G$ no contiene a ninguna de las gráficas de las Figuras \ref{obsts_O_M3} como subgráficas inducidas.
    \end{enumerate}

\end{theorem}

\begin{figure}[ht!]
\begin{center}
\begin{tikzpicture}

\begin{scope}[xshift=0cm,scale=1]

\node [style=vertex] (1) at (0,0) {};
\node [style=vertex] (2) at (1,0) {};
\node [style=vertex] (3) at (0,0.5) {};
\node [style=vertex] (4) at (1,0.5) {};
\node [style=vertex] (5) at (0.5,1.25) {};
\node [style=vertex] (6) at (0.5,2) {};
\foreach \i/\j in {1/2,3/4,3/5,4/5}
  \draw [style=edge] (\i) to (\j);
\node [below of=1,xshift=.5cm]
{\parbox{0.3\linewidth}{\subcaption*{$H$}}};

\end{scope}

\begin{scope}[xshift=3cm,scale=1]

\node [style=vertex] (1) at (0,0) {};
\node [style=vertex] (2) at (1,0) {};
\node [style=vertex] (3) at (0,0.5) {};
\node [style=vertex] (4) at (1,0.5) {};
\node [style=vertex] (5) at (0.5,1.25) {};
\node [style=vertex] (6) at (0.5,2) {};
\foreach \i/\j in {1/2,3/4,3/5,4/5,5/6}
  \draw [style=edge] (\i) to (\j);
\node [below of=1,xshift=.5cm]  {\parbox{0.3\linewidth}{\subcaption*{$I$}}};

\end{scope}

\begin{scope}[xshift=6cm,scale=1]

\node [style=vertex] (1) at (0,0) {};
\node [style=vertex] (2) at (0.5,0.5) {};
\node [style=vertex] (3) at (1.5,0.5) {};
\node [style=vertex] (4) at (0.5,1.5) {};
\node [style=vertex] (5) at (1.5,1.5) {};
\node [style=vertex] (6) at (0,2) {};
\node [style=vertex] (7) at (2,1) {};

\foreach \i/\j in {1/2,1/3,1/6,2/3,2/4,2/5,3/4,3/5,4/5,4/6,5/6}
  \draw [style=edge] (\i) to (\j);
\node [below of=1,xshift=1cm] {\parbox{0.3\linewidth}{\subcaption*{$J$}}};

\end{scope}
\end{tikzpicture}
\end{center}
\setlength{\abovecaptionskip}{-15pt}
\caption{Obstrucciones mínimas para la clase $M_2$.}
\label{obsts_O_M3}
\end{figure}

\begin{proof}

  Notemos que las gráficas $H, I$ y $J$ pueden ser construidas de la siguiente manera:

  \begin{enumerate}[(1)]
      \item $H = K_1 + K_2 + K_3$.
      \item $I = Paw + K_2$.
      \item $J = (\overline{P_3} \oplus \overline{P_3}) + K_1$.
  \end{enumerate}

  Supongamos primero que $G \in M_2$ y procedamos probando la contrapositiva
  de la afirmación. Es decir, probemos que que si $G$ tiene a $H$, a $I$
  o a $J$ como subgráficas inducidas, entonces $G \notin M_2$. Para ello basta
  con que probemos que ninguna de estas gráficas es un elemento de $M_2$.
  Veamos que ninguna partici\'on en dos partes es una $M_2$-partici\'on.
  Sea $(X,Y)$ una partici\'on de $G$. Si ambos vértices de $K_2$ en $H$
  se encuentran en $X$, entonces la existencia de cualquier vértice adicional
  en $X$, implicar\'ia que $G[X]$ contiene un $\overline{P_3}$. Como los
  vértices restantes inducen una gráfica que no es multipartita completa,
  esta partici\'on no es una $M_2$-partici\'on. Supongamos entonces que un
  v\'ertice de $K_2$ est\'a en $X$ y el otro en $Y$. Como la gráfica inducida
  por los vértices restantes de $H$ contiene un $K_3$, podemos suponer sin
  p\'erdida de generalidad que dos de sus v\'ertices se encuentran en $X$,
  por lo que $X$ contiene una copia inducida de $\overline{P_3}$. Como la
  partici\'on fue elegida arbitrariamente, tenemos que $H$ no pertenece a
  $M_2$. El argumento para probar que $I$ no está en $M_2$ es an\'algo al
  anterior.

  Por otra parte, como $J$ tiene un vértice aislado, para que admitiera
  una $M_2$-partici\'on, el resto de sus vértices (que inducen un
  $\overline{P_3} \oplus \overline{P_3}$) deben de poder dividirse en dos
  partes de manera tales que una induzca un conjunto independiente y la otra
  una gráfica multipartita completa. Siempre que tomamos uno de los vértices
  de uno de los dos $\overline{P_3}$ para formar el conjunto independiente,
  ninguno los vértices del otro $\overline{P_3}$ puede ser agregado al mismo,
  pues es adyacente al vértice que agregamos primero. Así, la subgráfica
  inducida $\overline{P_3} \oplus \overline{P_3}$ no acepta una partición en
  un conjunto independiente y una gráfica multipartita completa. Luego, $J$
  no está en $M_2$.

  Recíprocamente, supongamos que $G$ es libre de $H$, $I$ y $J$ y probemos
  que es elemento de $M_2$. Para ello consideramos
  cuatro casos que son exhaustivos. Los primeros tres casos cubren toda
  situación en la que $G$ es inconexa, mientras que el último caso se cumple
  si $G$ es conexa.

  \emph{Caso 1:} $G$ tiene al menos dos componentes conexas no triviales.

  Consideremos la partición de $V$ en dos partes $(A,B)$ tal que $A$ contiene
  únicamente una componente no trivial y $B$ el resto. Como $G[A]$ y $G[B]$
  contienen ambas componentes no triviales, las dos poseen un $K_2$. Luego,
  ni $G[B]$ ni $G[A]$ pueden contener un $Paw$, o $G$ tendría a $I$ como
  subgráfica inducida. Dado que $G[A]$ y $G[B]$ son cográficas, son también
  gráficas perfectas, y al ninguna tener un $Paw$ como subgráfica inducida,
  cada una es bipartita o multipartita completa \cite{Olariu}.

  % Realmente no se acostumbra a argumentar así cuando usas un resultado
  % de alguien más.  Lo común es escribir el teorema, citando la fuente
  % (algo como \begin{theorem}[\cite{Olariu}] ...), etiquetarlo, y referirte
  % al teorema dentro de tu trabajo.   De otra forma, le estás dejando al
  % lector el trabajo de buscar el artículo, y tratar de adivinar qué
  % resultado estás aplicando.

  Si tanto $G[A]$ como $G[B]$ son gráficas multipartitas completas,
  entonces $G \in M_2$. Si ambas son bipartitas, entonces $G$ es bipartita
  también y acepta una partición en dos conjuntos independientes, cada uno
  de los cuales es una gráfica multipartita completa, por lo que $G \in M_2$.
  Si $G[A]$ es bipartita y $G[B]$ es multipartita completa, como $G[A]$ es
  una cográfica conexa, entonces es una gráfica multipartita completa y $G
  \in M_2$.

  % El último enunciado necesita una referencia al resultado de que toda
  % cográfica bipartita es bipartita completa.   Que por cierto, no recuerdo
  % haber visto.   Creo que eso es algo que podría ir en el capítulo de
  % antecedentes, sección de cográficas.

  Finalmente, si $G[A]$ es multipartita completa y $G[B]$ es bipartita. Si
  $G[B]$ tiene una sola componente, $G[B]$ es bipartita completa y $G \in
  M_2$. Si $G[B]$ tiene más de una componente, como al menos una es no
  trivial, debe tener a $\overline{P_3}$ como subgráfica inducida. Luego,
  $G[A]$ debe ser libre de $K_3$ o $G$ tendría a $H$ como subráfica inducida.
  Así, $G[A]$ es bipartita. Como ambas son bipartitas, $G \in M_2$.


  \emph{Caso 2:} $G$ tiene exactamente una componente conexa no trivial y
  al menos una trivial.

  Como $G$ contiene al menos una componente trivial, la única partición que
  puede aceptar en dos gráficas multipartitas completas es una partición en
  un conjunto independiente y una gráfica multipartita completa. Luego, la
  componente no trivial de $G$, a la que llamaremos $G'$, debe de aceptar una
  partición en un conjunto independiente y una gráfica multipartita completa.

  Si $G'$ es bipartita, entonces acepta una partición en dos conjuntos
  independientes, y por lo tanto $G \in M_2$. Si $G'$ es una gráfica
  multipartita completa, entonces $G \in M_2$. Si $G$ no es una gráfica
  bipartita ni multipartita completa, dado que es una cográfica, y por lo
  tanto una gráfica perfecta, $G$ contiene un $Paw$. Sea $y$ la raíz del
  coárbol de $G'$ y sea $z$, descendiente de $y$, el nodo más profundo que
  tiene un $Paw$ como subgráfica inducida, probemos por inducción sobre la
  distancia desde $y$ hasta $z$, denotada por $d$, que $G'[y]$ acepta una
  partición en un conjunto independiente y una gráfica multipartita completa.

  \textbf{Caso base}: $d = 0$, o bien, $y = z$.

  Notemos que $z$ tiene etiqueta 1, pues $Paw$ es una gráfica conexa. Dado
  que $z$ tiene etiqueta 1, todos sus hijos inducen gráficas multipartitas
  completas menos uno, $w$, que tiene etiqueta 0. Mostremos por contradicción
  que todos los hijos de $w$ inducen gráficas multipartitas completas.
  Supongamos que alguno de los hijos de $w$ contiene un $\overline{P_3}$.
  Como el nodo más profundo que contiene un $\overline{P_3}$ debe tener
  etiqueta 0, $w$ tiene un hijo de etiqueta 1 y éste a su vez tiene al menos
  2 hijos, uno de los cuales contiene a $\overline{P_3}$ y el otro que tiene
  al menos un $K_1$. Luego, dicho hijo contiene un $Paw$, lo que es una
  contradicción.

  Si $w$ tiene un sólo hijo que no es un vértice, el resto de sus hijos
  forman un conjunto independiente, $C$. Si eliminamos este conjunto
  independiente, como el único hijo de $w$ que queda induce una gráfica
  multipartita completa, entonces $G'[w] - C$ es una gráfica multipartita
  completa. Luego, $G'[z] - C$ es la unión completa de varias gráficas
  multipartitas completas y por lo tanto es una gráfica multipartita
  completa. De esto se sigue que $G[z]$ acepta una partición en un conjunto
  independiente, $C$, y una gráfica multipartita completa $G'[z] - C$.

  Si $w$ tiene al menos dos hijos no triviales, notemos que ninguno de
  ellos puede contener a $K_3$, o de lo contrario $w$ contendría a $K_2+K_3$
  y $G$ no sería libre de $I$. Luego, todos los hijos de $w$ inducen gráficas
  bipartitas, es decir que $w$ induce también una gráfica bipartita. En otras
  palabras, $G'[w]$ acepta una partición en dos conjuntos independientes. Si
  sustraemos uno de estos conjuntos independientes, denotado por $D$, entonces
  $G'[w]-D$ es un conjunto independiente. Luego $G'[z]-D$ es la unión completa
  de al menos una gráfica multipartita completa y un conjunto independiente.
  Así, $G'[z]-D$ es una gráfica multipartita completa. Luego, $G'[z]$ acepta
  una partición en un conjunto independiente, $D$ y una gráfica multipartita
  completa, $G'[z] - D$.

  Como en todos los casos $z$ acepta una partición en un conjunto
  independiente y una gráfica multipartita completa y $y = z$, entonces $y$
  acepta la misma partición.

  \textbf{Paso inductivo}: $d \ge 2$.

  Notemos que $d$ siempre será par, ya que tanto $y$ como $z$ son nodos con
  etiqueta 1. Sea $k$ un entero tal que $k \ge 2$. Supongamos, como hipótesis
  inductiva, que si $G''$ es una cográfica conexa libre de $H, I$ y $J$ tal
  que la distancia, $d'$, entre la raíz, $y'$ de su coárbol y el nodo más
  profundo que contiene un $Paw$ es igual a $k-2$, entonces $G''$ acepta una
  partición en un conjunto independiente y una gráfica multipartita completa.

  Dado que $G'$ es libre de $J$, todos los hijos de $y$, menos uno, inducen
  gráficas multipartitas completas. Dicho hijo, $v$, tiene etiqueta 0 y al
  menos uno de sus hijos debe de contener un $Paw$. Denotemos a dicho hijo
  como $u$. El resto de los hijos de $v$ deben de ser vértices, o de lo
  contrario, $G'[v]$ contendría a $K_2 + K_3$ como subgráfica inducida, por
  lo que $G$ contendría a $I$. Denotemos a este conjunto de vértices como
  $E$. Luego, $G'[u]$  es una cográfica que cumple con las condiciones de
  la hipótesis inductiva, por lo que acepta una partición en un conjunto
  independiente, $D$ y una gráfica multipartita completa. Tenemos entonces
  que $G'[u] - D$ es una gráfica multipartita completa. Se sigue que
  $(G'[v]-D)-E$ es una gráfica multipartita completa. Luego, $(G'-D)-E$ es
  una unión completa de gráficas multipartitas completas por lo que también
  es una gráfica multipartita completa. Notemos que, dado que $v$ tiene
  etiqueta 0, no existen aristas entre los vértices en $D$ y los vértices en
  $E$, es decir que $D \cup E$ es un conjunto independiente. Así, $G'$ acepta
  una partición en un conjunto independiente, $D \cup E$ y una gráfica
  multipartita completa, $(G' - D) - E$.

  Como $G'$ acepta una partición en un conjunto independiente y una gráfica
  multipartita completa, entonces $G \in M_2$.


  \emph{Caso 3:} $G$ es un conjunto independiente con al menos dos vértices.

  Dado que $G$ es una gráfica multipartita completa, se sigue inmediatamente
  que está en $M_2$.

  \emph{Caso 4:} $G$ es conexa.

  Dado que toda cográfica inconexa libre de $H$, $I$ y $J$ acepta una
  partición en dos gráficas multipartitas completas, y como una cográfica
  conexa es o un vértice aislado o una unión completa de cográficas
  inconexas se sigue de los casos anteriores y del hecho de que la clase
  $M_2$ es cerrada bajo uniones completas (Lema
  \ref{lema_union_completa}), que $G \in M_2$.

\end{proof}


    \subsection{Reconocimiento de la clase $M_2$}
        Haciendo uso  del Algoritmo \ref{alg_esta_en_clase} se puede determinar si una cográfica pertenece o no a la clase $M_2$. Como se especificó en la Sección \ref{sec_AlgoSub}, el tiempo de este algoritmo crece de forma lineal de acuerdo con el tamaño de la gráfica de entrada si encontramos primero todos los coárboles binarios de las obstrucciones de la clase. Como conocemos las obstrucciones mínimas de la clase $M_2$, que son finitas, se puede buscar cada una en tiempo lineal y por lo tanto se puede reconocer si una cográfica pertenece a la clase $M_2$ en tiempo lineal. El Algoritmo \ref{alg_decision}, que es una instancia del Algoritmo \ref{alg_esta_en_clase} corresponde a este proceso. Los árboles binarios de cada una de las obstrucciones de la clase $M_2$ se muestran en la Figura \ref{fig_obsts_bin}.

\begin{figure}[ht!]
\centering

\begin{subfigure}{0.85\textwidth}
\begin{tikzpicture}

\begin{scope}[xshift=0cm,scale=1]
\node [style=cotreenode] (1) at (1,1) {0};
\node [style=cotreenode] (2) at (0,0) {0};
\node [style=cotreenode] (3) at (2,0) {1};
\node [style=vertex] (4) at (-0.5,-1) {};
\node [style=cotreenode] (5) at (0.5,-1) {1};
\node [style=vertex] (6) at (1.5,-1) {};
\node [style=cotreenode] (7) at (2.5,-1) {1};
\node [style=vertex] (8) at (0.25,-2) {};
\node [style=vertex] (9) at (0.75,-2) {};
\node [style=vertex] (10) at (2.25,-2) {};
\node [style=vertex] (11) at (2.75,-2) {};
\foreach \i/\j in {1/2,1/3,2/4,2/5,3/6,3/7,5/8,5/9,7/10,7/11}
  \draw [style=edge] (\i) to (\j);
\node [below of=9,xshift=0.25cm] {\parbox{0.3\linewidth}{\subcaption*{$H_1$}}};
\end{scope}

\begin{scope}[xshift=4.5cm,scale=1]
\node [style=cotreenode] (1) at (1,1) {0};
\node [style=cotreenode] (2) at (0,0) {0};
\node [style=cotreenode] (3) at (2,0) {1};
\node [style=vertex] (4) at (-0.5,-1) {};
\node [style=cotreenode] (5) at (0.5,-1) {1};
\node [style=vertex] (6) at (1.5,-1) {};
\node [style=vertex] (7) at (2.5,-1) {};
\node [style=vertex] (8) at (0.125,-2) {};
\node [style=cotreenode] (9) at (0.875,-2) {1};
\node [style=vertex] (10) at (0.625,-3) {};
\node [style=vertex] (11) at (1.125,-3) {};
\foreach \i/\j in {1/2,1/3,2/4,2/5,3/6,3/7,5/8,5/9,9/10,9/11}
  \draw [style=edge] (\i) to (\j);
\node [below of=11] {\parbox{0.3\linewidth}{\subcaption*{$H_2$}}};
\end{scope}

\begin{scope}[xshift=9cm,scale=1]
\node [style=cotreenode] (1) at (1,1) {0};
\node [style=cotreenode] (2) at (0,0) {0};
\node [style=vertex] (3) at (2,0) {};
\node [style=cotreenode] (4) at (-0.5,-1) {1};
\node [style=cotreenode] (5) at (0.5,-1) {1};
\node [style=vertex] (8) at (0.125,-2) {};
\node [style=cotreenode] (9) at (0.875,-2) {1};
\node [style=vertex] (10) at (0.625,-3) {};
\node [style=vertex] (11) at (1.125,-3) {};
\node [style=vertex] (12) at (-0.75,-2) {};
\node [style=vertex] (13) at (-0.25,-2) {};
\foreach \i/\j in {1/2,1/3,2/4,2/5,5/8,5/9,9/10,9/11,4/12,4/13}
  \draw [style=edge] (\i) to (\j);
\node [below of=11] {\parbox{0.3\linewidth}{\subcaption*{$H_3$}}};
\end{scope}


\end{tikzpicture}
\end{subfigure}


\begin{subfigure}{0.6\textwidth}
\begin{tikzpicture}

\begin{scope}[xshift=0cm,scale=1]
\node [style=cotreenode] (1) at (1,1) {0};
\node [style=cotreenode] (2) at (0,0) {1};
\node [style=cotreenode] (3) at (2,0) {1};
\node [style=vertex] (4) at (-0.5,-1) {};
\node [style=vertex] (5) at (0.5,-1) {};
\node [style=vertex] (6) at (1.5,-1) {};
\node [style=cotreenode] (7) at (2.5,-1) {0};
\node [style=vertex] (8) at (2.125,-2) {};
\node [style=cotreenode] (9) at (2.875,-2) {1};
\node [style=vertex] (10) at (2.625,-3) {};
\node [style=vertex] (11) at (3.125,-3) {};

\foreach \i/\j in {1/2,1/3,2/4,2/5,3/6,3/7,7/8,7/9,9/10,9/11}
  \draw [style=edge] (\i) to (\j);
\node [below of=10,xshift=-1.625cm] {\parbox{0.3\linewidth}{\subcaption*{$I_1$}}};
\end{scope}

\begin{scope}[xshift=5cm,scale=1]
\node [style=cotreenode] (1) at (1,1) {0};
\node [style=vertex] (2) at (0,0) {};
\node [style=cotreenode] (3) at (2,0) {1};
\node [style=cotreenode] (4) at (1.25,-1) {0};
\node [style=cotreenode] (5) at (2.75,-1) {0};
\node [style=cotreenode] (6) at (0.825,-2) {1};
\node [style=vertex] (7) at (1.625,-2) {};
\node [style=vertex] (8) at (2.375,-2) {};
\node [style=cotreenode] (9) at (3.125,-2) {1};
\node [style=vertex] (10) at (0.575,-3) {};
\node [style=vertex] (11) at (1.075,-3) {};
\node [style=vertex] (12) at (2.875,-3) {};
\node [style=vertex] (13) at (3.375,-3) {};


\foreach \i/\j in {1/2,1/3,3/4,3/5,4/6,4/7,5/8,5/9,6/10,6/11,9/12,9/13}
  \draw [style=edge] (\i) to (\j);
\node [below of=12,xshift=-1.625cm] {\parbox{0.3\linewidth}{\subcaption*{$J_1$}}};
\end{scope}
\end{tikzpicture}
\end{subfigure}
\setlength{\abovecaptionskip}{5pt}
\caption{$H_1, H_2$ y $H_3$ son los coárboles binarios correspondientes a la obstrucción $H$. El coárbol binario $I_1$ es el único que corresponde a la obstrucción $I$. El coárbol binario $J_1$ es el único que corresponde a la obstrucción $J$.}\label{fig_obsts_bin}


\end{figure}


\begin{algorithm}[ht!]
\caption{Pertenece\_a\_M2}
\label{alg_decision}
\DontPrintSemicolon % Some LaTeX compilers require you to use \dontprintsemicolon instead
\KwIn{$g$, la raíz del coárbol de una cográfica $G$.}
\KwOut{$verdadero$ si $G$ pertenece a la clase hereditaria de cográficas $C$}

$g\_bin \gets \text{CrearÁrbolBinario}(g)$\;
$C\_bins \gets \{H_1, H_2, H_3, I_1, J_1\}$\;

\ForEach{$bin$ \textbf{\emph{en}} $C\_bins$}{
    $f = \text{Función\_de\_coasignación}(g\_bin,bin)$\;
    \If{$f(bin) = marcado$}{
        $\Return\ verdadero$\;
    }
}

$\Return\ falso$\;

\end{algorithm}




    \subsection{Algoritmo certificador}
        Si bien, el Algoritmo \ref{alg_decision} es capaz de identificar a las cográficas que pertenecen a la clase $M_2$, no es posible determinar a partir de éste cuáles son las dos partes en las que se puede dividir una gráfica de la clase. La presente subsección muestra un algoritmo (Algoritmo \ref{alg_cert_m2}) que, dada una cográfica $G$ representada por su coárbol $T$, devuelve una coloración de las hojas de este último. Si $G$ pertenece a $M_2$, cada una de las hojas de $T$ tendrá uno de dos colores, $verde$ o $azul$. Cada uno de estos cuales corresponde a una parte de la $M_2$-partición de $G$. En el caso contrario, las hojas de $T$ correspondientes a los vértices que forman una obstrucción mínima tendrán un color que indique de qué obstrucción mínima se trata ($amarillo$ para $H$, $anaranjado$ para $I$ y $rojo$ para $J$). El Algoritmo \ref{alg_cert_m2} hace uso de los Algoritmos \ref{alg_cert_caso1} y \ref{alg_cert_caso2}, que funcionan de la misma manera para casos específicos del problema. La correctitud de estos algoritmo se sigue de la demostración del Teorema \ref{teo_obsts_m2}

\subsubsection{Algoritmo para reconocer gráficas bipartitas completas conexas}

El Algoritmo \ref{alg_bpc} es un algoritmo que resulta útil para los algoritmos subsecuentes. Éste recibe la raíz de un coárbol, $g$, y devuelve $verdadero$ si la gráfica representada por dicho coárbol es una gráfica bipartita completa conexa, coloreando los vértices de una parte de color $azul$ y los de la otra parte de $verde$. En el caso contrario, se colorean con $amarillo$  tres hojas cuyo ancestro común más profundo sea un nodo con etiqueta 1. Es decir que se colorean los vértices que inducen un $K_3$ en la gráfica. El bloque de la línea 9 a la 28 se ejecuta sólo si $g$ tiene exactamente dos hijos. En las líneas 10 a 18 se busca un $K_3$ en el primer hijo de $g$ y en las líneas 19 a 27 se busca en el segundo hijo. La Figura \ref{fig_bipartita} muestra el resultado de la ejecución de este algoritmo para algunos coárboles.

\begin{algorithm}[!htbp]
\caption{Es\_bipartita_completa}
\label{alg_bpc}

\DontPrintSemicolon % Some LaTeX compilers require you to use \dontprintsemicolon instead
\KwIn{$g$, la raíz de un coárbol, $G$}
\KwOut{Verdadero si la gráfica representada por $G$ es bipartita completa. Falso en el caso contrario. Las hojas de $\acute{a}rbol(g)$ se colorean.}

\If{g \text{es una hoja}}{
    $g.color \gets azul$\;
    $\Return\ verdadero$\;
}
\ElseIf{$g.etiqueta = 0$}{
    $\Return\ falso$\;
}
\ElseIf{$g.hijos.tama\tilde{n}o > 2$}{
    Marcar con amarillo: una hoja en cada uno de tres hijos diferentes de $g$\;
    $\Return\ falso$\;
}
\Else(\tcp*[h]{Hay exactamente dos hijos}){
    \If{$g.hijos\emph{[0]}$ es una hoja}{
        $g.hijos[0].color \gets azul$\;
    }
    \Else{
        \ForEach{$gchild$ \textbf{\emph{en}} $g.hijos\emph{[0]}.hijos$}{
            \If{$gchild$ es una hoja}{
                $gchild.color \gets azul$\;
            }
            \Else{
                Marcar con amarillo: dos hojas que tengan como ancestro común más profundo a $gchild$ y una hoja descendiente de $g.hijos[1]$\;
                $\Return\ falso$\;
            }
        }
    }
    \If{$g.hijos\emph{[1]}$ es una hoja}{
        $g.hijos[1].color \gets verde$\;
    }
    \Else{
        \ForEach{$gchild$ \textbf{\emph{en}} $g.hijos\emph{[1]}.hijos$}{
            \If{gchild \text{es una hoja}}{
                $gchild.color \gets verde$\;
            }
            \Else{
                Marcar con amarillo: dos hojas que tengan como ancestro común más profundo a $gchild$ y una hoja descendiente de $g.hijos[0]$\;
                $\Return\ falso$\;
            }
        }
    }
}

$\Return\ verdadero$\;

\end{algorithm}

\begin{figure}[!htbp]
\begin{center}
\begin{tikzpicture}

\begin{scope}[xshift=0cm,scale=1]
\node [style=cotreenode] (1) at (1,3) {1};
\node [style=vertex, fill=blue] (2) at (0.5,2) {};
\node [style=vertex, fill=green] (3) at (1.5,2) {};
\foreach \i/\j in {1/2,1/3}
  \draw [style=edge] (\i) to (\j);
\end{scope}

\begin{scope}[xshift=2cm,scale=1]
\node [style=cotreenode] (1) at (1,3) {1};
\node [style=vertex, fill=blue] (2) at (0.5,2) {};
\node [style=cotreenode] (3) at (1.5,2) {0};
\node [style=vertex, fill=green] (4) at (1,1) {};
\node [style=vertex, fill=green] (5) at (1.5,1) {};
\node [style=vertex, fill=green] (6) at (2,1) {};
\foreach \i/\j in {1/2,1/3,3/4,3/5,3/6}
  \draw [style=edge] (\i) to (\j);
\end{scope}

\begin{scope}[xshift=4.75cm,scale=1]
\node [style=cotreenode] (1) at (1,3) {1};
\node [style=cotreenode] (2) at (0.5,2) {0};
\node [style=cotreenode] (3) at (1.5,2) {0};
\node [style=vertex, fill=blue] (4) at (0.25,1) {};
\node [style=vertex, fill=blue] (5) at (0.5,1) {};
\node [style=vertex, fill=blue] (6) at (0.75,1) {};
\node [style=vertex, fill=green] (7) at (1.25,1) {};
\node [style=vertex, fill=green] (8) at (1.5,1) {};
\node [style=vertex, fill=green] (9) at (1.75,1) {};

\foreach \i/\j in {1/2,1/3,2/4,2/5,2/6,3/7,3/8,3/9}
  \draw [style=edge] (\i) to (\j);
\end{scope}

\begin{scope}[xshift=7cm,scale=1]
\node [style=cotreenode] (1) at (1,3) {1};
\node [style=vertex, fill=yellow] (2) at (0.5,2) {};
\node [style=vertex, fill=yellow] (3) at (1,2) {};
\node [style=vertex, fill=yellow] (4) at (1.5,2) {};
\foreach \i/\j in {1/2,1/3,1/4}
  \draw [style=edge] (\i) to (\j);
\end{scope}

\begin{scope}[xshift=9.25cm,scale=1]
\node [style=cotreenode] (1) at (1,3) {1};
\node [style=cotreenode] (2) at (0.5,2) {0};
\node [style=cotreenode] (3) at (1.5,2) {0};
\node [style=vertex, fill=yellow] (4) at (0.25,1) {};
\node [style=vertex, fill=blue] (6) at (0.75,1) {};
\node [style=vertex, fill=green] (7) at (1.25,1) {};
\node [style=cotreenode] (9) at (1.75,1) {1};
\node [style=vertex, fill=yellow] (10) at (1.5,0) {};
\node [style=vertex, fill=yellow] (11) at (2,0) {};

\foreach \i/\j in {1/2,1/3,2/4,2/6,3/7,3/9,9/10,9/11}
  \draw [style=edge] (\i) to (\j);
\end{scope}

\end{tikzpicture}
\end{center}
\caption{Ejemplos del resultado de la ejecución del Algoritmo \ref{alg_bpc}.}
\label{fig_bipartita}
\end{figure}



\subsubsection{Caso 1}

El algoritmo \ref{alg_cert_caso1} corresponde al $Caso\ 1$ de la demostración del Teorema \ref{teo_obsts_m2}. Éste recibe como entrada la raíz de un coárbol que representa una cográfica inconexa que tiene al menos dos componentes conexas no triviales. En el bloque de las líneas 1 a 12 se aborda el caso en el que la gráfica tiene exactamente dos componentes conexas y se busca un $Paw$ que pueda formar la obstrucción $I$. En el bloque de las líneas 13 a 17 se aborda el caso en el que hay al menos 3 componente conexas y se busca un $K_3$ en cada componente para formar la obstrucción $H$. Si no se encuentra ninguna de las obstrucciones mínimas, se devuelve $verdadero$ y cada una de las hojas del coárbol tendrán color $azul$ o $verde$. Las Figuras \ref{fig_certificador_caso1_01} y \ref{fig_certificador_caso1_02} muestran la ejecución del algoritmo para gráficas sin ninguna de las obstrucciones. La Figura \ref{fig_certificador_caso1_03} muestra el resultado de la ejecución para tres gráficas, cada una de las cuales contiene una obstrucción.

\begin{algorithm}[!htbp]
\small
\caption{M2\_Caso\_1}
\label{alg_cert_caso1}

\DontPrintSemicolon % Some LaTeX compilers require you to use \dontprintsemicolon instead
\KwIn{$g$, la raíz de un coárbol con etiqueta 0 y al menos dos hijos que no son hojas}
\KwOut{Verdadero si $G$ pertenece a la clase $M_2$. Falso en el caso contrario. Las hojas de $\acute{a}rbol(g)$ se colorean.}

\If{$g.hijos.tama\tilde{n}o = 2$}{
    \For{gchild \textbf{\emph{en}} g.hijos\emph{[0]}}{
        \If{$gchild$ es una hoja}{
            $gchild.color \gets azul$\;
        }
        \Else{
            \ForEach{$ggchild$ \textbf{\emph{en}} $gchild.hijos$}{
                \If{$ggchild$ es una hoja}{
                    $gchild.color \gets azul$\;
                }
                \Else(\tcp*[h]{Se marca la obstrucción $I$}){
                    Marcar con anaranjado: una hoja en $ggchild.hijos[0]$, una hoja en $ggchild.hijos[1]$, una hoja en un hermano de $ggchild$, una hoja en un hermano de $gchild$ y dos hojas cuyo ancestro común más profundo sea el hermano de $g.hijos[0]$\;

                    $\Return\ falso$\;
                }
            }
        }
    }
    Repetir el procedimiento de las líneas 2 a 11 para $g.hijos[1]$, pero marcando con color $verde$ en vez de $azul$\;
}
\Else{
    \For{$child$ \textbf{\emph{en}} $g.hijos$}{
        \If{\emph{Es\_bipartita\_completa(}$child$\emph{)} $= falso$}{
             Marcar con amarillo: dos hojas cuyo ancestro común más profundo sea un hermano de $child$ que no sea una hoja y una hoja en un hermano diferente\;
                        $\Return\ falso$\;
        }
    }
}


$\Return\ verdadero$\;

\end{algorithm}


\begin{figure}[!htbp]
\centering

\begin{subfigure}{\textwidth}
\centering
\begin{tikzpicture}
\begin{scope}[xshift=0cm,scale=1]
\node [style=cotreenode, fill=lightgray] (1) at (1.5,4) {0};
\node [style=cotreenode] (2) at (0.5,3) {1};
\node [style=cotreenode] (3) at (2.5,3) {1};
\node [style=vertex] (4) at (0,2) {};
\node [style=vertex] (6) at (1,2) {};
\node [style=vertex] (7) at (2,2) {};
\node [style=cotreenode] (8) at (3,2) {0};
\node [style=vertex] (9) at (2.5,1) {};
\node [style=vertex] (10) at (3,1) {};
\node [style=vertex] (11) at (3.5,1) {};
\foreach \i/\j in {1/2,1/3,2/4,2/6,3/7,3/8,8/9,8/10,8/11}
  \draw [style=edge] (\i) to (\j);
\end{scope}
\begin{scope}[xshift=4cm,scale=1]
\node [style=cotreenode, fill=lightgray] (1) at (1.5,4) {0};
\node [style=cotreenode, fill=lightgray] (2) at (0.5,3) {1};
\node [style=cotreenode] (3) at (2.5,3) {1};
\node [style=vertex] (4) at (0,2) {};
\node [style=vertex] (6) at (1,2) {};
\node [style=vertex] (7) at (2,2) {};
\node [style=cotreenode] (8) at (3,2) {0};
\node [style=vertex] (9) at (2.5,1) {};
\node [style=vertex] (10) at (3,1) {};
\node [style=vertex] (11) at (3.5,1) {};
\foreach \i/\j in {1/2,1/3,2/4,2/6,3/7,3/8,8/9,8/10,8/11}
  \draw [style=edge] (\i) to (\j);
\end{scope}
\begin{scope}[xshift=8cm,scale=1]
\node [style=cotreenode, fill=lightgray] (1) at (1.5,4) {0};
\node [style=cotreenode, fill=lightgray] (2) at (0.5,3) {1};
\node [style=cotreenode] (3) at (2.5,3) {1};
\node [style=vertex, fill=lightgray] (4) at (0,2) {};
\node [style=vertex] (6) at (1,2) {};
\node [style=vertex] (7) at (2,2) {};
\node [style=cotreenode] (8) at (3,2) {0};
\node [style=vertex] (9) at (2.5,1) {};
\node [style=vertex] (10) at (3,1) {};
\node [style=vertex] (11) at (3.5,1) {};
\foreach \i/\j in {1/2,1/3,2/4,2/6,3/7,3/8,8/9,8/10,8/11}
  \draw [style=edge] (\i) to (\j);
\end{scope}
\end{tikzpicture}
\end{subfigure}

\begin{subfigure}{\textwidth}
\centering
\begin{tikzpicture}
\begin{scope}[xshift=0cm,scale=1]
\node [style=cotreenode, fill=lightgray] (1) at (1.5,4) {0};
\node [style=cotreenode, fill=lightgray] (2) at (0.5,3) {1};
\node [style=cotreenode] (3) at (2.5,3) {1};
\node [style=vertex, fill=blue] (4) at (0,2) {};
\node [style=vertex, fill=lightgray] (6) at (1,2) {};
\node [style=vertex] (7) at (2,2) {};
\node [style=cotreenode] (8) at (3,2) {0};
\node [style=vertex] (9) at (2.5,1) {};
\node [style=vertex] (10) at (3,1) {};
\node [style=vertex] (11) at (3.5,1) {};
\foreach \i/\j in {1/2,1/3,2/4,2/6,3/7,3/8,8/9,8/10,8/11}
  \draw [style=edge] (\i) to (\j);
\end{scope}
\begin{scope}[xshift=4cm,scale=1]
\node [style=cotreenode, fill=lightgray] (1) at (1.5,4) {0};
\node [style=cotreenode] (2) at (0.5,3) {1};
\node [style=cotreenode, fill=lightgray] (3) at (2.5,3) {1};
\node [style=vertex, fill=blue] (4) at (0,2) {};
\node [style=vertex, fill=blue] (6) at (1,2) {};
\node [style=vertex] (7) at (2,2) {};
\node [style=cotreenode] (8) at (3,2) {0};
\node [style=vertex] (9) at (2.5,1) {};
\node [style=vertex] (10) at (3,1) {};
\node [style=vertex] (11) at (3.5,1) {};
\foreach \i/\j in {1/2,1/3,2/4,2/6,3/7,3/8,8/9,8/10,8/11}
  \draw [style=edge] (\i) to (\j);
\end{scope}
\begin{scope}[xshift=8cm,scale=1]
\node [style=cotreenode, fill=lightgray] (1) at (1.5,4) {0};
\node [style=cotreenode] (2) at (0.5,3) {1};
\node [style=cotreenode, fill=lightgray] (3) at (2.5,3) {1};
\node [style=vertex, fill=blue] (4) at (0,2) {};
\node [style=vertex, fill=blue] (6) at (1,2) {};
\node [style=vertex, fill=lightgray] (7) at (2,2) {};
\node [style=cotreenode] (8) at (3,2) {0};
\node [style=vertex] (9) at (2.5,1) {};
\node [style=vertex] (10) at (3,1) {};
\node [style=vertex] (11) at (3.5,1) {};
\foreach \i/\j in {1/2,1/3,2/4,2/6,3/7,3/8,8/9,8/10,8/11}
  \draw [style=edge] (\i) to (\j);
\end{scope}
\end{tikzpicture}
\end{subfigure}

\begin{subfigure}{\textwidth}
\centering
\begin{tikzpicture}
\begin{scope}[xshift=0cm,scale=1]
\node [style=cotreenode, fill=lightgray] (1) at (1.5,4) {0};
\node [style=cotreenode] (2) at (0.5,3) {1};
\node [style=cotreenode, fill=lightgray] (3) at (2.5,3) {1};
\node [style=vertex, fill=blue] (4) at (0,2) {};
\node [style=vertex, fill=blue] (6) at (1,2) {};
\node [style=vertex, fill=green] (7) at (2,2) {};
\node [style=cotreenode, fill=lightgray] (8) at (3,2) {0};
\node [style=vertex] (9) at (2.5,1) {};
\node [style=vertex] (10) at (3,1) {};
\node [style=vertex] (11) at (3.5,1) {};
\foreach \i/\j in {1/2,1/3,2/4,2/6,3/7,3/8,8/9,8/10,8/11}
  \draw [style=edge] (\i) to (\j);
\end{scope}
\begin{scope}[xshift=4cm,scale=1]
\node [style=cotreenode, fill=lightgray] (1) at (1.5,4) {0};
\node [style=cotreenode] (2) at (0.5,3) {1};
\node [style=cotreenode, fill=lightgray] (3) at (2.5,3) {1};
\node [style=vertex, fill=blue] (4) at (0,2) {};
\node [style=vertex, fill=blue] (6) at (1,2) {};
\node [style=vertex, fill=green] (7) at (2,2) {};
\node [style=cotreenode, fill=lightgray] (8) at (3,2) {0};
\node [style=vertex, fill=lightgray] (9) at (2.5,1) {};
\node [style=vertex] (10) at (3,1) {};
\node [style=vertex] (11) at (3.5,1) {};
\foreach \i/\j in {1/2,1/3,2/4,2/6,3/7,3/8,8/9,8/10,8/11}
  \draw [style=edge] (\i) to (\j);
\end{scope}
\begin{scope}[xshift=8cm,scale=1]
\node [style=cotreenode, fill=lightgray] (1) at (1.5,4) {0};
\node [style=cotreenode] (2) at (0.5,3) {1};
\node [style=cotreenode, fill=lightgray] (3) at (2.5,3) {1};
\node [style=vertex, fill=blue] (4) at (0,2) {};
\node [style=vertex, fill=blue] (6) at (1,2) {};
\node [style=vertex, fill=green] (7) at (2,2) {};
\node [style=cotreenode, fill=lightgray] (8) at (3,2) {0};
\node [style=vertex, fill=green] (9) at (2.5,1) {};
\node [style=vertex, fill=lightgray] (10) at (3,1) {};
\node [style=vertex] (11) at (3.5,1) {};
\foreach \i/\j in {1/2,1/3,2/4,2/6,3/7,3/8,8/9,8/10,8/11}
  \draw [style=edge] (\i) to (\j);
\end{scope}
\end{tikzpicture}
\end{subfigure}

\begin{subfigure}{\textwidth}
\centering
\begin{tikzpicture}
\begin{scope}[xshift=0cm,scale=1]
\node [style=cotreenode, fill=lightgray] (1) at (1.5,4) {0};
\node [style=cotreenode] (2) at (0.5,3) {1};
\node [style=cotreenode, fill=lightgray] (3) at (2.5,3) {1};
\node [style=vertex, fill=blue] (4) at (0,2) {};
\node [style=vertex, fill=blue] (6) at (1,2) {};
\node [style=vertex, fill=green] (7) at (2,2) {};
\node [style=cotreenode, fill=lightgray] (8) at (3,2) {0};
\node [style=vertex, fill=green] (9) at (2.5,1) {};
\node [style=vertex, fill=green] (10) at (3,1) {};
\node [style=vertex, fill=lightgray] (11) at (3.5,1) {};
\foreach \i/\j in {1/2,1/3,2/4,2/6,3/7,3/8,8/9,8/10,8/11}
  \draw [style=edge] (\i) to (\j);
\end{scope}
\begin{scope}[xshift=4cm,scale=1]
\node [style=cotreenode, fill=lightgray] (1) at (1.5,4) {0};
\node [style=cotreenode] (2) at (0.5,3) {1};
\node [style=cotreenode, fill=lightgray] (3) at (2.5,3) {1};
\node [style=vertex, fill=blue] (4) at (0,2) {};
\node [style=vertex, fill=blue] (6) at (1,2) {};
\node [style=vertex, fill=green] (7) at (2,2) {};
\node [style=cotreenode, fill=lightgray] (8) at (3,2) {0};
\node [style=vertex, fill=green] (9) at (2.5,1) {};
\node [style=vertex, fill=green] (10) at (3,1) {};
\node [style=vertex, fill=green] (11) at (3.5,1) {};
\foreach \i/\j in {1/2,1/3,2/4,2/6,3/7,3/8,8/9,8/10,8/11}
  \draw [style=edge] (\i) to (\j);
\end{scope}
\end{tikzpicture}
\end{subfigure}

\caption{Ejemplo de la ejecución del Algoritmo \ref{alg_cert_caso1}. Se muestran en color gris los nodos del árbol que están siendo procesados. Los colores de las hojas corresponden a los colores que asigna el algoritmo.}
\label{fig_certificador_caso1_01}
\end{figure}

\begin{figure}[!htbp]
\centering
\begin{subfigure}{\textwidth}
\centering
\begin{tikzpicture}
\begin{scope}[xshift=0cm,scale=1]
\node [style=cotreenode, fill=lightgray] (1) at (2,4) {0};
\node [style=cotreenode] (2) at (0.5,3) {1};
\node [style=vertex] (3) at (2,3) {};
\node [style=cotreenode] (4) at (3.5,3) {1};
\node [style=vertex] (5) at (0,2) {};
\node [style=vertex] (6) at (1,2) {};
\node [style=cotreenode] (7) at (2.75,2) {0};
\node [style=cotreenode] (8) at (4.25,2) {0};
\node [style=vertex] (9) at (2.25,1) {};
\node [style=vertex] (10) at (2.75,1) {};
\node [style=vertex] (11) at (3.25,1) {};
\node [style=vertex] (12) at (3.75,1) {};
\node [style=vertex] (13) at (4.25,1) {};
\node [style=vertex] (14) at (4.75,1) {};
\foreach \i/\j in {1/2,1/3,1/4,2/5,2/6,4/7,4/8,7/9,7/10,7/11,8/12,8/13,8/14}
  \draw [style=edge] (\i) to (\j);
\end{scope}
\begin{scope}[xshift=6cm,scale=1]
\node [style=cotreenode, fill=lightgray] (1) at (2,4) {0};
\node [style=cotreenode, fill=lightgray] (2) at (0.5,3) {1};
\node [style=vertex] (3) at (2,3) {};
\node [style=cotreenode] (4) at (3.5,3) {1};
\node [style=vertex] (5) at (0,2) {};
\node [style=vertex] (6) at (1,2) {};
\node [style=cotreenode] (7) at (2.75,2) {0};
\node [style=cotreenode] (8) at (4.25,2) {0};
\node [style=vertex] (9) at (2.25,1) {};
\node [style=vertex] (10) at (2.75,1) {};
\node [style=vertex] (11) at (3.25,1) {};
\node [style=vertex] (12) at (3.75,1) {};
\node [style=vertex] (13) at (4.25,1) {};
\node [style=vertex] (14) at (4.75,1) {};
\foreach \i/\j in {1/2,1/3,1/4,2/5,2/6,4/7,4/8,7/9,7/10,7/11,8/12,8/13,8/14}
  \draw [style=edge] (\i) to (\j);
\end{scope}
\end{tikzpicture}
\end{subfigure}

\begin{subfigure}{\textwidth}
\centering
\begin{tikzpicture}
\begin{scope}[xshift=0cm,scale=1]
\node [style=cotreenode, fill=lightgray] (1) at (2,4) {0};
\node [style=cotreenode] (2) at (0.5,3) {1};
\node [style=vertex, fill=lightgray] (3) at (2,3) {};
\node [style=cotreenode] (4) at (3.5,3) {1};
\node [style=vertex, fill=blue] (5) at (0,2) {};
\node [style=vertex, fill=green] (6) at (1,2) {};
\node [style=cotreenode] (7) at (2.75,2) {0};
\node [style=cotreenode] (8) at (4.25,2) {0};
\node [style=vertex] (9) at (2.25,1) {};
\node [style=vertex] (10) at (2.75,1) {};
\node [style=vertex] (11) at (3.25,1) {};
\node [style=vertex] (12) at (3.75,1) {};
\node [style=vertex] (13) at (4.25,1) {};
\node [style=vertex] (14) at (4.75,1) {};
\foreach \i/\j in {1/2,1/3,1/4,2/5,2/6,4/7,4/8,7/9,7/10,7/11,8/12,8/13,8/14}
  \draw [style=edge] (\i) to (\j);
\end{scope}
\begin{scope}[xshift=6cm,scale=1]
\node [style=cotreenode, fill=lightgray] (1) at (2,4) {0};
\node [style=cotreenode] (2) at (0.5,3) {1};
\node [style=vertex, fill=blue] (3) at (2,3) {};
\node [style=cotreenode, fill=lightgray] (4) at (3.5,3) {1};
\node [style=vertex, fill=blue] (5) at (0,2) {};
\node [style=vertex, fill=green] (6) at (1,2) {};
\node [style=cotreenode] (7) at (2.75,2) {0};
\node [style=cotreenode] (8) at (4.25,2) {0};
\node [style=vertex] (9) at (2.25,1) {};
\node [style=vertex] (10) at (2.75,1) {};
\node [style=vertex] (11) at (3.25,1) {};
\node [style=vertex] (12) at (3.75,1) {};
\node [style=vertex] (13) at (4.25,1) {};
\node [style=vertex] (14) at (4.75,1) {};
\foreach \i/\j in {1/2,1/3,1/4,2/5,2/6,4/7,4/8,7/9,7/10,7/11,8/12,8/13,8/14}
  \draw [style=edge] (\i) to (\j);
\end{scope}
\end{tikzpicture}
\end{subfigure}

\begin{subfigure}{\textwidth}
\centering
\begin{tikzpicture}
\begin{scope}[xshift=0cm,scale=1]
\node [style=cotreenode, fill=lightgray] (1) at (2,4) {0};
\node [style=cotreenode] (2) at (0.5,3) {1};
\node [style=vertex, fill=blue] (3) at (2,3) {};
\node [style=cotreenode, fill=lightgray] (4) at (3.5,3) {1};
\node [style=vertex, fill=blue] (5) at (0,2) {};
\node [style=vertex, fill=green] (6) at (1,2) {};
\node [style=cotreenode] (7) at (2.75,2) {0};
\node [style=cotreenode] (8) at (4.25,2) {0};
\node [style=vertex, fill=blue] (9) at (2.25,1) {};
\node [style=vertex, fill=blue] (10) at (2.75,1) {};
\node [style=vertex, fill=blue] (11) at (3.25,1) {};
\node [style=vertex, fill=green] (12) at (3.75,1) {};
\node [style=vertex, fill=green] (13) at (4.25,1) {};
\node [style=vertex, fill=green] (14) at (4.75,1) {};
\foreach \i/\j in {1/2,1/3,1/4,2/5,2/6,4/7,4/8,7/9,7/10,7/11,8/12,8/13,8/14}
  \draw [style=edge] (\i) to (\j);
\end{scope}
\end{tikzpicture}
\end{subfigure}

\caption{Ejemplo de la ejecución del Algoritmo \ref{alg_cert_caso1}. Se muestran en color gris los nodos del árbol que están siendo procesados. Los colores de las hojas corresponden a los colores que asigna el algoritmo.}
\label{fig_certificador_caso1_02}
\end{figure}

\begin{figure}[!htbp]
\centering
\begin{subfigure}{\textwidth}
\centering
\begin{tikzpicture}
\begin{scope}[xshift=0cm,scale=1]
\node [style=cotreenode] (1) at (1.5,4) {0};
\node [style=cotreenode] (2) at (0.5,3) {1};
\node [style=cotreenode] (4) at (2.5,3) {1};
\node [style=vertex, fill=orange] (5) at (0,2) {};
\node [style=vertex, fill=blue] (6) at (1,2) {};
\node [style=vertex, fill=orange] (7) at (1.75,2) {};
\node [style=cotreenode] (8) at (3.25,2) {0};
\node [style=vertex, fill=orange] (12) at (2.75,1) {};
\node [style=cotreenode] (14) at (3.75,1) {1};
\node [style=vertex, fill=orange] (15) at (3.5,0) {};
\node [style=vertex, fill=orange] (16) at (4,0) {};
\foreach \i/\j in {1/2,1/4,2/5,2/6,4/7,4/8,8/12,8/14,14/15,14/16}
  \draw [style=edge] (\i) to (\j);
\end{scope}
\begin{scope}[xshift=4.5cm,scale=1]
\node [style=cotreenode] (1) at (1.5,4) {0};
\node [style=cotreenode] (2) at (0.5,3) {1};
\node [style=vertex, fill=yellow] (3) at (1.5,3) {};
\node [style=cotreenode] (4) at (2.5,3) {1};
\node [style=vertex, fill=yellow] (5) at (0,2) {};
\node [style=vertex, fill=yellow] (6) at (1,2) {};
\node [style=vertex, fill=yellow] (7) at (2,2) {};
\node [style=vertex, fill=yellow] (8) at (2.5,2) {};
\node [style=vertex, fill=yellow] (9) at (3,2) {};
\foreach \i/\j in {1/2,1/3,1/4,2/5,2/6,4/7,4/8,4/9}
  \draw [style=edge] (\i) to (\j);
\end{scope}
\begin{scope}[xshift=8.5cm,scale=1]
\node [style=cotreenode] (1) at (1.5,4) {0};
\node [style=cotreenode] (2) at (0.5,3) {1};
\node [style=vertex, fill=yellow] (3) at (1.5,3) {};
\node [style=cotreenode] (4) at (2.5,3) {1};
\node [style=vertex, fill=yellow] (5) at (0,2) {};
\node [style=vertex, fill=yellow] (6) at (1,2) {};
\node [style=vertex, fill=yellow] (7) at (1.75,2) {};
\node [style=cotreenode] (8) at (3.25,2) {0};
\node [style=vertex, fill=green] (12) at (2.75,1) {};
\node [style=cotreenode] (14) at (3.75,1) {1};
\node [style=vertex, fill=yellow] (15) at (3.5,0) {};
\node [style=vertex, fill=yellow] (16) at (4,0) {};
\foreach \i/\j in {1/2,1/3,1/4,2/5,2/6,4/7,4/8,8/12,8/14,14/15,14/16}
  \draw [style=edge] (\i) to (\j);
\end{scope}
\end{tikzpicture}
\end{subfigure}


\caption{Ejemplos del resultado de la ejecución del Algoritmo \ref{alg_cert_caso1} en los que se encuentra una obstrucción.}
\label{fig_certificador_caso1_03}
\end{figure}

\subsubsection{Caso 2}

El algoritmo \ref{alg_cert_caso2} corresponde al $Caso\ 2$ de la demostración del Teorema \ref{teo_obsts_m2}. Éste recibe como entrada la raíz, $g$, de un coárbol que representa una cográfica inconexa que tiene exactamente una componente conexa no trivial y al menos una trivial. En el bloque de las líneas 6 a 28 se procesa el hijo de $g$ que no es una hoja. En las líneas 7 a 19 se procesan los nietos de $g$ y se registra si alguno tiene un hijo que no sea una hoja (es decir que dicho nieto de $g$ corresponde a una gráfica no multipartita completa) en la variable $aux\_gchild$. La cantidad de hijos diferentes de una hoja de éste se registra en $ggchildren\_no\_hojas$. Si hay más de un nieto que tenga hijos que no son hojas, se marca la obstrucción $J$ (Línea 18). Una vez procesados los nietos de $g$, se decide cómo será procesado el nieto de $g$ que no corresponde a una gráfica multipartita completa. Si tal hijo no existe, la partición ya se habrá hecho (Líneas 20 y 21), esto corresponde a una parte del caso base del $Caso\_2$ de la demostración del Teorema \ref{teo_obsts_m2}. Si dicho nieto tiene un solo hijo que no es una hoja, se procesa recursivamente (Líneas 22 y 23), esto corresponde al paso inductivo del $Caso\_2$ de la demostración ya mencionada. Y finalmente, si tiene más de un hijo que no es una hoja, se busca que todos estos hijos sean bipartitas, esta es la otra parte del caso base del $Caso\_2$ de la demostración. La Figura \ref{fig_certificador_caso2_01} muestra un ejemplo de la ejecución del algoritmo para un coárbol cuya cográfica no contiene a ninguna de las obstrucciones mínimas de $M_2$. La Figura \ref{fig_certificador_caso2_02} muestra el resultado de la ejecución para coárboles que contienen una obstrucción.




\begin{algorithm}[!htbp]
\SetInd{1pt}{10pt}
\footnotesize
\caption{M2\_Caso\_2}
\label{alg_cert_caso2}

\DontPrintSemicolon % Some LaTeX compilers require you to use \dontprintsemicolon instead
\KwIn{$g$, la raíz de un coárbol con etiqueta 0 que tiene exactamente un hijo que no es una hoja y al menos uno que es una hoja}
\KwOut{Verdadero si $G$ peretenece a la clase $M_2$. Falso en el caso contrario. Las hojas de $G$ se colorean.}


    $aux\_gchild \gets null$\;
    $ggchildren\_no\_hojas \gets 0$\;

    \For{child \textbf{\emph{en}} $g.hijos$}{
        \If{child \text{es una hoja}}{
            $child.color \gets azul$\;
        }
        \Else(\tcp*[h]{Sólo se ejecuta una vez}){
            \For{gchild \textbf{\emph{en}} $child.hijos$}{
                \If{gchild \text{es una hoja}}{
                    $gchild.color \gets verde$\;
                }
                \Else{
                    \For{ggchild \textbf{\emph{en}} $gchild.hijos$}{
                        \If{ggchild \text{es una hoja}}{
                            $ggchild.color \gets verde$\;
                        }
                        \ElseIf{$aux\_gchild = null \emph{\textbf{ o }} aux\_gchild = gchild$}{
                            $aux\_gchild \gets gchild$\;
                            $ggchildren\_no\_hojas \gets ggchildren\_no\_hojas + 1$\;
                        }
                        \Else{
                            Marcar con rojo: Un hijo de $g$ que sea una hoja, dos hojas cuyo ancestro común más profundo sea $ggchild$, una hoja en un hermano de $ggchild$, dos hojas cuyo ancestro común más profundo sea un hijo de $aux\_gchild$ que no es una hoja y una hoja en un hijo de $aux\_gchild$ diferente del anterior\;
                            $\Return\ falso$\;
                        }
                    }
                }
            }

            \If{ggchildren\_no\_hojas = 0}{
                $\Return\ verdadero$\;
            }
            \ElseIf{ggchildren\_no\_hojas = 1}{
                $\Return$ M2\_Caso\_2($aux\_gchild$)\;
            }
            \Else{
                \For{ggchild \textbf{\emph{en}} aux\_gchild}{
                    \If{\emph{Es\_bipartita_completa(}$ggchild$\emph{)} = falso}{
                        Marcar con amarillo: dos hojas cuyo ancestro común más profundo sea un hermano de $ggchild$ que no sea una hoja y un hijo de $g$ que sea una hoja\;
                        $\Return\ falso$\;
                    }
                }
            }

        }
    }

    $\Return\ verdadero$\;


\end{algorithm}

\begin{figure}[!htbp]
\centering

\begin{subfigure}{\textwidth}
\centering
\begin{tikzpicture}
\begin{scope}[xshift=0cm,scale=1]
\node [style=cotreenode, fill=lightgray] (1) at (3.5,5) {0};
\node [style=vertex] (2) at (0.5,4) {};
\node [style=vertex] (3) at (1.5,4) {};
\node [style=vertex] (4) at (2.5,4) {};
\node [style=cotreenode] (5) at (3.5,4) {1};
\node [style=vertex] (6) at (0.5,3) {};
\node [style=cotreenode] (7) at (2,3) {0};
\node [style=cotreenode] (8) at (3.5,3) {0};
\node [style=vertex] (9) at (1.75,2) {};
\node [style=vertex] (10) at (2.25,2) {};
\node [style=vertex] (11) at (2.75,2) {};
\node [style=cotreenode] (12) at (3.5,2) {1};
\node [style=cotreenode] (13) at (4.5,2) {1};
\node [style=vertex] (14) at (3.25,1) {};
\node [style=vertex] (15) at (3.75,1) {};
\node [style=vertex] (16) at (4.25,1) {};
\node [style=vertex] (17) at (4.75,1) {};
\foreach \i/\j in {1/2,1/3,1/4,1/5,5/6,5/7,5/8,7/9,7/10,8/11,8/12,8/13,12/14,12/15,13/16,13/17}
  \draw [style=edge] (\i) to (\j);
\end{scope}
\begin{scope}[xshift=5cm,scale=1]
\node [style=cotreenode, fill=lightgray] (1) at (3.5,5) {0};
\node [style=vertex, fill=blue] (2) at (0.5,4) {};
\node [style=vertex, fill=blue] (3) at (1.5,4) {};
\node [style=vertex, fill=blue] (4) at (2.5,4) {};
\node [style=cotreenode, fill=lightgray] (5) at (3.5,4) {1};
\node [style=vertex] (6) at (0.5,3) {};
\node [style=cotreenode] (7) at (2,3) {0};
\node [style=cotreenode] (8) at (3.5,3) {0};
\node [style=vertex] (9) at (1.75,2) {};
\node [style=vertex] (10) at (2.25,2) {};
\node [style=vertex] (11) at (2.75,2) {};
\node [style=cotreenode] (12) at (3.5,2) {1};
\node [style=cotreenode] (13) at (4.5,2) {1};
\node [style=vertex] (14) at (3.25,1) {};
\node [style=vertex] (15) at (3.75,1) {};
\node [style=vertex] (16) at (4.25,1) {};
\node [style=vertex] (17) at (4.75,1) {};
\foreach \i/\j in {1/2,1/3,1/4,1/5,5/6,5/7,5/8,7/9,7/10,8/11,8/12,8/13,12/14,12/15,13/16,13/17}
  \draw [style=edge] (\i) to (\j);
\end{scope}
\begin{scope}[xshift=10cm,scale=1]
\node [style=cotreenode, fill=lightgray] (1) at (3.5,5) {0};
\node [style=vertex, fill=blue] (2) at (0.5,4) {};
\node [style=vertex, fill=blue] (3) at (1.5,4) {};
\node [style=vertex, fill=blue] (4) at (2.5,4) {};
\node [style=cotreenode, fill=lightgray] (5) at (3.5,4) {1};
\node [style=vertex, fill=green] (6) at (0.5,3) {};
\node [style=cotreenode, fill=lightgray] (7) at (2,3) {0};
\node [style=cotreenode] (8) at (3.5,3) {0};
\node [style=vertex] (9) at (1.75,2) {};
\node [style=vertex] (10) at (2.25,2) {};
\node [style=vertex] (11) at (2.75,2) {};
\node [style=cotreenode] (12) at (3.5,2) {1};
\node [style=cotreenode] (13) at (4.5,2) {1};
\node [style=vertex] (14) at (3.25,1) {};
\node [style=vertex] (15) at (3.75,1) {};
\node [style=vertex] (16) at (4.25,1) {};
\node [style=vertex] (17) at (4.75,1) {};
\foreach \i/\j in {1/2,1/3,1/4,1/5,5/6,5/7,5/8,7/9,7/10,8/11,8/12,8/13,12/14,12/15,13/16,13/17}
  \draw [style=edge] (\i) to (\j);
\end{scope}
\end{tikzpicture}
\end{subfigure}
\begin{subfigure}{\textwidth}
\centering
\begin{tikzpicture}
\begin{scope}[xshift=0cm,scale=1]
\node [style=cotreenode, fill=lightgray] (1) at (3.5,5) {0};
\node [style=vertex, fill=blue] (2) at (0.5,4) {};
\node [style=vertex, fill=blue] (3) at (1.5,4) {};
\node [style=vertex, fill=blue] (4) at (2.5,4) {};
\node [style=cotreenode, fill=lightgray] (5) at (3.5,4) {1};
\node [style=vertex, fill=green] (6) at (0.5,3) {};
\node [style=cotreenode] (7) at (2,3) {0};
\node [style=cotreenode, fill=lightgray] (8) at (3.5,3) {0};
\node [style=vertex, fill=green] (9) at (1.75,2) {};
\node [style=vertex, fill=green] (10) at (2.25,2) {};
\node [style=vertex] (11) at (2.75,2) {};
\node [style=cotreenode] (12) at (3.5,2) {1};
\node [style=cotreenode] (13) at (4.5,2) {1};
\node [style=vertex] (14) at (3.25,1) {};
\node [style=vertex] (15) at (3.75,1) {};
\node [style=vertex] (16) at (4.25,1) {};
\node [style=vertex] (17) at (4.75,1) {};
\foreach \i/\j in {1/2,1/3,1/4,1/5,5/6,5/7,5/8,7/9,7/10,8/11,8/12,8/13,12/14,12/15,13/16,13/17}
  \draw [style=edge] (\i) to (\j);
\end{scope}
\begin{scope}[xshift=5cm,scale=1]
\node [style=cotreenode, fill=lightgray] (1) at (3.5,5) {0};
\node [style=vertex, fill=blue] (2) at (0.5,4) {};
\node [style=vertex, fill=blue] (3) at (1.5,4) {};
\node [style=vertex, fill=blue] (4) at (2.5,4) {};
\node [style=cotreenode, fill=lightgray] (5) at (3.5,4) {1};
\node [style=vertex, fill=green] (6) at (0.5,3) {};
\node [style=cotreenode] (7) at (2,3) {0};
\node [style=cotreenode, fill=lightgray] (8) at (3.5,3) {0};
\node [style=vertex, fill=green] (9) at (1.75,2) {};
\node [style=vertex, fill=green] (10) at (2.25,2) {};
\node [style=vertex, fill=green] (11) at (2.75,2) {};
\node [style=cotreenode, fill=lightgray] (12) at (3.5,2) {1};
\node [style=cotreenode] (13) at (4.5,2) {1};
\node [style=vertex] (14) at (3.25,1) {};
\node [style=vertex] (15) at (3.75,1) {};
\node [style=vertex] (16) at (4.25,1) {};
\node [style=vertex] (17) at (4.75,1) {};
\foreach \i/\j in {1/2,1/3,1/4,1/5,5/6,5/7,5/8,7/9,7/10,8/11,8/12,8/13,12/14,12/15,13/16,13/17}
  \draw [style=edge] (\i) to (\j);
\end{scope}
\begin{scope}[xshift=10cm,scale=1]
\node [style=cotreenode, fill=lightgray] (1) at (3.5,5) {0};
\node [style=vertex, fill=blue] (2) at (0.5,4) {};
\node [style=vertex, fill=blue] (3) at (1.5,4) {};
\node [style=vertex, fill=blue] (4) at (2.5,4) {};
\node [style=cotreenode, fill=lightgray] (5) at (3.5,4) {1};
\node [style=vertex, fill=green] (6) at (0.5,3) {};
\node [style=cotreenode] (7) at (2,3) {0};
\node [style=cotreenode, fill=lightgray] (8) at (3.5,3) {0};
\node [style=vertex, fill=green] (9) at (1.75,2) {};
\node [style=vertex, fill=green] (10) at (2.25,2) {};
\node [style=vertex, fill=green] (11) at (2.75,2) {};
\node [style=cotreenode] (12) at (3.5,2) {1};
\node [style=cotreenode, fill=lightgray] (13) at (4.5,2) {1};
\node [style=vertex] (14) at (3.25,1) {};
\node [style=vertex] (15) at (3.75,1) {};
\node [style=vertex] (16) at (4.25,1) {};
\node [style=vertex] (17) at (4.75,1) {};
\foreach \i/\j in {1/2,1/3,1/4,1/5,5/6,5/7,5/8,7/9,7/10,8/11,8/12,8/13,12/14,12/15,13/16,13/17}
  \draw [style=edge] (\i) to (\j);
\end{scope}
\end{tikzpicture}
\end{subfigure}

\begin{subfigure}{\textwidth}
\centering
\begin{tikzpicture}
\begin{scope}[xshift=0cm,scale=1]
\node [style=cotreenode, fill=lightgray] (1) at (3.5,5) {0};
\node [style=vertex, fill=blue] (2) at (0.5,4) {};
\node [style=vertex, fill=blue] (3) at (1.5,4) {};
\node [style=vertex, fill=blue] (4) at (2.5,4) {};
\node [style=cotreenode, fill=lightgray] (5) at (3.5,4) {1};
\node [style=vertex, fill=green] (6) at (0.5,3) {};
\node [style=cotreenode] (7) at (2,3) {0};
\node [style=cotreenode, fill=lightgray] (8) at (3.5,3) {0};
\node [style=vertex, fill=green] (9) at (1.75,2) {};
\node [style=vertex, fill=green] (10) at (2.25,2) {};
\node [style=vertex, fill=green] (11) at (2.75,2) {};
\node [style=cotreenode, fill=lightgray] (12) at (3.5,2) {1};
\node [style=cotreenode] (13) at (4.5,2) {1};
\node [style=vertex] (14) at (3.25,1) {};
\node [style=vertex] (15) at (3.75,1) {};
\node [style=vertex] (16) at (4.25,1) {};
\node [style=vertex] (17) at (4.75,1) {};
\foreach \i/\j in {1/2,1/3,1/4,1/5,5/6,5/7,5/8,7/9,7/10,8/11,8/12,8/13,12/14,12/15,13/16,13/17}
  \draw [style=edge] (\i) to (\j);
\end{scope}
\begin{scope}[xshift=5cm,scale=1]
\node [style=cotreenode, fill=lightgray] (1) at (3.5,5) {0};
\node [style=vertex, fill=blue] (2) at (0.5,4) {};
\node [style=vertex, fill=blue] (3) at (1.5,4) {};
\node [style=vertex, fill=blue] (4) at (2.5,4) {};
\node [style=cotreenode, fill=lightgray] (5) at (3.5,4) {1};
\node [style=vertex, fill=green] (6) at (0.5,3) {};
\node [style=cotreenode] (7) at (2,3) {0};
\node [style=cotreenode, fill=lightgray] (8) at (3.5,3) {0};
\node [style=vertex, fill=green] (9) at (1.75,2) {};
\node [style=vertex, fill=green] (10) at (2.25,2) {};
\node [style=vertex, fill=green] (11) at (2.75,2) {};
\node [style=cotreenode] (12) at (3.5,2) {1};
\node [style=cotreenode, fill=lightgray] (13) at (4.5,2) {1};
\node [style=vertex, fill=blue] (14) at (3.25,1) {};
\node [style=vertex, fill=green] (15) at (3.75,1) {};
\node [style=vertex] (16) at (4.25,1) {};
\node [style=vertex] (17) at (4.75,1) {};
\foreach \i/\j in {1/2,1/3,1/4,1/5,5/6,5/7,5/8,7/9,7/10,8/11,8/12,8/13,12/14,12/15,13/16,13/17}
  \draw [style=edge] (\i) to (\j);
\end{scope}
\begin{scope}[xshift=10cm,scale=1]
\node [style=cotreenode, fill=lightgray] (1) at (3.5,5) {0};
\node [style=vertex, fill=blue] (2) at (0.5,4) {};
\node [style=vertex, fill=blue] (3) at (1.5,4) {};
\node [style=vertex, fill=blue] (4) at (2.5,4) {};
\node [style=cotreenode, fill=lightgray] (5) at (3.5,4) {1};
\node [style=vertex, fill=green] (6) at (0.5,3) {};
\node [style=cotreenode] (7) at (2,3) {0};
\node [style=cotreenode, fill=lightgray] (8) at (3.5,3) {0};
\node [style=vertex, fill=green] (9) at (1.75,2) {};
\node [style=vertex, fill=green] (10) at (2.25,2) {};
\node [style=vertex, fill=green] (11) at (2.75,2) {};
\node [style=cotreenode] (12) at (3.5,2) {1};
\node [style=cotreenode, fill=lightgray] (13) at (4.5,2) {1};
\node [style=vertex, fill=blue] (14) at (3.25,1) {};
\node [style=vertex, fill=green] (15) at (3.75,1) {};
\node [style=vertex, fill=blue] (16) at (4.25,1) {};
\node [style=vertex, fill=green] (17) at (4.75,1) {};
\foreach \i/\j in {1/2,1/3,1/4,1/5,5/6,5/7,5/8,7/9,7/10,8/11,8/12,8/13,12/14,12/15,13/16,13/17}
  \draw [style=edge] (\i) to (\j);
\end{scope}
\end{tikzpicture}
\end{subfigure}
\caption{Ejemplo de la ejecución del Algoritmo \ref{alg_cert_caso2}. Se muestran en color gris los nodos del árbol que están siendo procesados. El procesamiento de las hojas hermanas se realiza en una sola imagen. Los colores de las hojas corresponden a los colores que asigna el algoritmo.}
\label{fig_certificador_caso2_01}
\end{figure}


\begin{figure}[!htbp]
\centering

\begin{subfigure}{\textwidth}
\centering
\begin{tikzpicture}
\begin{scope}[xshift=0cm,scale=1]
\node [style=cotreenode] (1) at (3.5,5) {0};
\node [style=vertex, fill=yellow] (2) at (0.5,4) {};
\node [style=vertex, fill=blue] (3) at (1.5,4) {};
\node [style=vertex, fill=blue] (4) at (2.5,4) {};
\node [style=cotreenode] (5) at (3.5,4) {1};
\node [style=vertex, fill=green] (6) at (0.5,3) {};
\node [style=cotreenode] (7) at (2,3) {0};
\node [style=cotreenode] (8) at (3.5,3) {0};
\node [style=vertex, fill=green] (9) at (1.75,2) {};
\node [style=vertex, fill=green] (10) at (2.25,2) {};
\node [style=vertex, fill=green] (11) at (2.75,2) {};
\node [style=cotreenode] (12) at (3.5,2) {1};
\node [style=cotreenode] (13) at (4.5,2) {1};
\node [style=vertex, fill=yellow] (14) at (3.25,1) {};
\node [style=vertex, fill=yellow] (15) at (3.75,1) {};
\node [style=vertex, fill=yellow] (16) at (4.25,1) {};
\node [style=vertex, fill=yellow] (17) at (4.75,1) {};
\node [style=vertex, fill=yellow] (18) at (4.5,1) {};
\foreach \i/\j in {1/2,1/3,1/4,1/5,5/6,5/7,5/8,7/9,7/10,8/11,8/12,8/13,12/14,12/15,13/16,13/17,13/18}
  \draw [style=edge] (\i) to (\j);
\end{scope}
\begin{scope}[xshift=5cm,scale=1]
\node [style=cotreenode] (1) at (3.5,5) {0};
\node [style=vertex, fill=red] (2) at (0.5,4) {};
\node [style=vertex, fill=blue] (3) at (1.5,4) {};
\node [style=vertex, fill=blue] (4) at (2.5,4) {};
\node [style=cotreenode] (5) at (3.5,4) {1};
\node [style=vertex, fill=green] (6) at (0.5,3) {};
\node [style=vertex, fill=green] (7) at (1.5,3) {};
\node [style=cotreenode] (8) at (2.5,3) {0};
\node [style=cotreenode] (9) at (4.5,3) {0};
\node [style=cotreenode] (10) at (2,2) {1};
\node [style=vertex, fill=red] (11) at (3,2) {};
\node [style=cotreenode] (12) at (4,2) {1};
\node [style=vertex, fill=red] (13) at (5,2) {};
\node [style=vertex, fill=red] (14) at (1.5,1) {};
\node [style=vertex, fill=red] (15) at (2.5,1) {};
\node [style=vertex, fill=red] (16) at (3.5,1) {};
\node [style=vertex, fill=red] (17) at (4.5,1) {};
\foreach \i/\j in {1/2,1/3,1/4,1/5,5/6,5/7,5/8,5/9,8/10,8/11,9/12,9/13,10/14,10/15,12/16,12/17}
  \draw [style=edge] (\i) to (\j);
\end{scope}
\end{tikzpicture}
\end{subfigure}
\caption{Ejemplo del resultado de la ejecución del Algoritmo \ref{alg_cert_caso2} para coárboles que incluyen una obstrucción.}
\label{fig_certificador_caso2_02}
\end{figure}


\subsubsection{Algoritmo certificador}

Finalmente, el Algoritmo \ref{alg_cert_m2} utiliza los algoritmos anteriores para colorear las hojas del coárbol recibido como entrada, $g$. En el caso de que la gráfica sea conexa (líneas 4 a 8), simplemente se llama el algoritmo para cada una de los hijos de $g$. Esto no significa que sea un algoritmo recursivo, dado que, para las gráficas inconexas y las hojas, el algoritmo no vuelve a ser llamado. En el caso de que la gráfica sea inconexa, se ejecuta el bloque de las líneas 10 a 21. En las líneas 10 a 15 se cuenta el número de componentes conexas de la gráfica representada (es decir que se cuentan los hijos de $g$ que no son hojas). Y por último se toma la decisión de qué caso debe llamarse.


\begin{algorithm}[!htbp]
\caption{M2\_Certificador}
\label{alg_cert_m2}

\DontPrintSemicolon % Some LaTeX compilers require you to use \dontprintsemicolon instead
\KwIn{$g$, la raíz de un coárbol, $G$}
\KwOut{Verdadero si la gráfica representada por $G$ pertenece a la clase $M_2$. Falso en el caso contrario. Las hojas de $G$ se colorean.}

    \If{$g$ \text{es una hoja}}{
        $g.color \gets azul$\;
        $\Return\ verdadero$\;
    }
    \ElseIf{$g.etiqueta = 1$}{
        \For{child \textbf{\emph{en}} $g.hijos$}{
            \If{\emph{M2\_Certificador(}child\emph{)} = falso}{
                $\Return\ falso$\;
            }
            $\Return\ verdadero$\;
        }
    }
    \Else{
        $componentes\_no\_triviales \gets 0$\;
        \For{child \textbf{\emph{en}} $g.hijos$}{
            \If{$child$ \text{es una hoja}}{
                $child.color \gets azul$\;
            }
            \Else{
                $componentes\_no\_triviales \gets componentes\_no\_triviales + 1$\;
            }
        }
        \If{componentes\_no\_triviales = 0}{
            $\Return\ verdadero$\;
        }
        \ElseIf{componentes\_no\_triviales = 1}{
            $\Return$ M2\_Caso\_2($g$)\;
        }
        \Else{
            $\Return$ M2\_Caso\_1($g$)\;
        }
    }


$\Return\ verdadero$\;

\end{algorithm}

\section{Las clases $(\alpha, \beta)$-$M_2$}

Al igual que con las cográficas polares, podemos obtener subclases de $M_2$ al limitar el tamaño de sus partes. En la presente sección estudiamos un conjunto de subclases de $M_2$ a las que llamamos clases $(\alpha, \beta)$-$M_2$. El estudio de estas clases consolida un ejemplo de cómo los conjuntos de obstrucciones mínimas de un conjunto de clases $C_1,C_2, \dots$ tales que $C_i \subset C_{i+1}$ para todo entero $i \geq 1$ se relacionan de manera que podemos encontrar fórmulas y algoritmos para generar dichos conjuntos de obstrucciones mínimas. El resultado principal de esta sección es un algoritmo de fuerza bruta para encontrar algunas obstrucciones mínimas de cualquier clase $(\alpha, \beta)$-$M_2$. Este algoritmo es utilizado para generar los conjuntos de obstrucciones mínimas de varias clases.

    \subsection{Definición y clases conocidas}
        Sean $\alpha$ y $\beta$ enteros tales que $0 < \alpha \le \beta$, una gr\'afica $G$ est\'a en la \emph{\textbf{clase $(\alpha, \beta)-M_2$}}  si y sólo si su conjunto de vértices acepta una partición $(A,B)$ tal que $G[A]$ es una gráfica multipartita completa formada por a lo más $\alpha$ conjuntos independientes y $G[B]$ es una gráfica multipartita completa formada por a lo más $\beta$ conjuntos independientes. Decimos que $(A,B)$ es una $M_2$-partición de $G$ de tamaño $(\alpha, \beta)$.

Notemos que la clase $M_2$ es la clase $(\infty, \infty)$-$M_2$ y que la clase $(1,1)$-$M_2$ es la clase de todas las cográficas bipartitas. En la demostración del Teorema \ref{teo_obsts_m2} % No es común referir resultados futuros.   Quizá bastaría mencionar el contexto en el que esta observación será útil más adelante. Es un Teorema anterior. 
 hablamos de las cográficas que aceptan una partición en un conjunto independiente y una gráfica multipartita completa. Es decir, las gráficas de la clase $(1,\infty)$-$M_2$. A partir de esta demostración, podemos encontrar las obstrucciones mínimas de la clase. Éstas nos serán de ayuda para entender el comportamiento de las clases $(1,\beta)$-$M_2$.

\begin{lemma}

Sea $G$ una cográfica, $G\in(1,\infty)$-$M_2$ si y sólo si $G$ es libre de las gráficas de la Figura \ref{obsts_1infM2}.

\end{lemma}

\begin{figure}[ht!]
\begin{center}
\begin{tikzpicture}

\begin{scope}[xshift=0cm,scale=1]

\node [style=vertex] (1) at (0,0) {};
\node [style=vertex] (2) at (1,0) {};
\node [style=vertex] (3) at (0,0.5) {};
\node [style=vertex] (4) at (1,0.5) {};
\node [style=vertex] (5) at (0.5,1.25) {};
\foreach \i/\j in {1/2,3/4,3/5,4/5}
  \draw [style=edge] (\i) to (\j);
\node [below of=1,xshift=.5cm]
{\parbox{0.3\linewidth}{\subcaption*{$H'$}}};

\end{scope}

\begin{scope}[xshift=3cm,scale=1]

\node [style=vertex] (1) at (0,0) {};
\node [style=vertex] (2) at (0.5,0.5) {};
\node [style=vertex] (3) at (1.5,0.5) {};
\node [style=vertex] (4) at (0.5,1.5) {};
\node [style=vertex] (5) at (1.5,1.5) {};
\node [style=vertex] (6) at (0,2) {};

\foreach \i/\j in {1/2,1/3,1/6,2/3,2/4,2/5,3/4,3/5,4/5,4/6,5/6}
  \draw [style=edge] (\i) to (\j);
\node [below of=1,xshift=0.75cm] {\parbox{0.3\linewidth}{\subcaption*{$J'$}}};

\end{scope}
\end{tikzpicture}
\end{center}
\setlength{\abovecaptionskip}{-15pt}
\caption{Obstrucciones mínimas para la clase $(1,\infty)$-$M_2$.}
\label{obsts_1infM2}
\end{figure}

\begin{proof}

Notemos que $H'=K_2 + K_3$ y $J'=\overline{P_3} \oplus \overline{P_3}$.

Supongamos primero que $G \in (1,\infty)$-$M_2$. Veamos que $H'\notin (1,\infty)$-$M_2$. Procedamos por contradicción. Supongamos que $H' \in (1,\infty)$-$M_2$. Luego, $H$ (Figura \ref{obsts_M2}) también está en $(1,\infty)$-$M_2$ y por lo tanto $H\in M_2$, lo que es una contradicción. Así,  $H'\notin (1,\infty)$-$M_2$. Análogamente para $J'$. Como ni $H'$ ni $J'$ están en $(1,\infty)$-$M_2$ y toda subgráfica inducida de $G$ sí lo está, $G$ es libre de $H'$ y $J'$.

Recíprocamente, si $G$ es libre de $H'$ y $J'$, mostremos que $G\in (1,\infty)$-$M_2$. Sea $r$ la raíz del coárbol de $G$. Consideremos los siguientes casos que son exhaustivos.

Supongamos primero que $G$ es conexa.  Si $G$ es un vértice aislado, es claro que $G \in (1,\infty)$-$M_2$. En el caso contrario, como $G$ es libre de $J'$, todos los hijos de $r$ representan gráficas multipartitas completas. Luego, $G$ es la unión completa de varias gráficas multipartitas completas, por lo que $G$ es multipartita completa. Así, $G \in (1,\infty)$-$M_2$.
% Todos quizá excepto uno, ¿no?

Si $G$ es una gr\'afica vac\'ia, entoces es claro que $G \in (1,\infty)$-$M_2$.

Como tercer caso, consideremos que $G$ es inconexa y tiene exactamente una componente no trivial y al menos una trivial. Entonces, del Caso 2 de la segunda parte de la demostración del Teorema \ref{teo_obsts_m2}, se sigue que $G \in (1,\infty)$-$M_2$.
% ¿La referencia es correcta?   Creo que sí, pero en los warnings,
% que son más importantes de lo que parece, tienes etiquetas
% definidas en múltiples ocasiones.  Revísalas para que todo tenga
% sentido.

Finalmente, supongamos que $G$ tiene al menos dos componentes no triviales. Como $G$ es libre de $H'$, entonces cada componente de $G$ es libre de $K_3$. Es decir que cada componente de $G$ es bipartita. Luego, $G$ es bipartita. Así, $G \in (1,1)$-$M_2$, por lo que $G \in (1,\infty)$-$M_2$.

\end{proof}

    \subsection{Conjunto de parejas mínimas}
        Sean $\alpha$ y $\beta$ enteros tales que $0 < \alpha \le \beta$, una gr\'afica $G$ est\'a en la \emph{\textbf{clase $(\alpha, \beta)-M_2$}}  si y sólo si su conjunto de vértices acepta una partición $(A,B)$ tal que $G[A]$ es una gráfica multipartita completa formada por a lo más $\alpha$ conjuntos independientes y $G[B]$ es una gráfica multipartita completa formada por a lo más $\beta$ conjuntos independientes. Decimos que $(A,B)$ es una $M_2$-partición de $G$ de tamaño $(\alpha, \beta)$.

Notemos que la clase $M_2$ es la clase $(\infty, \infty)$-$M_2$ y que la clase $(1,1)$-$M_2$ es la clase de todas las cográficas bipartitas. En la demostración del Teorema \ref{teo_obsts_m2} % No es común referir resultados futuros.   Quizá bastaría mencionar el contexto en el que esta observación será útil más adelante. Es un Teorema anterior. 
 hablamos de las cográficas que aceptan una partición en un conjunto independiente y una gráfica multipartita completa. Es decir, las gráficas de la clase $(1,\infty)$-$M_2$. A partir de esta demostración, podemos encontrar las obstrucciones mínimas de la clase. Éstas nos serán de ayuda para entender el comportamiento de las clases $(1,\beta)$-$M_2$.

\begin{lemma}

Sea $G$ una cográfica, $G\in(1,\infty)$-$M_2$ si y sólo si $G$ es libre de las gráficas de la Figura \ref{obsts_1infM2}.

\end{lemma}

\begin{figure}[ht!]
\begin{center}
\begin{tikzpicture}

\begin{scope}[xshift=0cm,scale=1]

\node [style=vertex] (1) at (0,0) {};
\node [style=vertex] (2) at (1,0) {};
\node [style=vertex] (3) at (0,0.5) {};
\node [style=vertex] (4) at (1,0.5) {};
\node [style=vertex] (5) at (0.5,1.25) {};
\foreach \i/\j in {1/2,3/4,3/5,4/5}
  \draw [style=edge] (\i) to (\j);
\node [below of=1,xshift=.5cm]
{\parbox{0.3\linewidth}{\subcaption*{$H'$}}};

\end{scope}

\begin{scope}[xshift=3cm,scale=1]

\node [style=vertex] (1) at (0,0) {};
\node [style=vertex] (2) at (0.5,0.5) {};
\node [style=vertex] (3) at (1.5,0.5) {};
\node [style=vertex] (4) at (0.5,1.5) {};
\node [style=vertex] (5) at (1.5,1.5) {};
\node [style=vertex] (6) at (0,2) {};

\foreach \i/\j in {1/2,1/3,1/6,2/3,2/4,2/5,3/4,3/5,4/5,4/6,5/6}
  \draw [style=edge] (\i) to (\j);
\node [below of=1,xshift=0.75cm] {\parbox{0.3\linewidth}{\subcaption*{$J'$}}};

\end{scope}
\end{tikzpicture}
\end{center}
\setlength{\abovecaptionskip}{-15pt}
\caption{Obstrucciones mínimas para la clase $(1,\infty)$-$M_2$.}
\label{obsts_1infM2}
\end{figure}

\begin{proof}

Notemos que $H'=K_2 + K_3$ y $J'=\overline{P_3} \oplus \overline{P_3}$.

Supongamos primero que $G \in (1,\infty)$-$M_2$. Veamos que $H'\notin (1,\infty)$-$M_2$. Procedamos por contradicción. Supongamos que $H' \in (1,\infty)$-$M_2$. Luego, $H$ (Figura \ref{obsts_M2}) también está en $(1,\infty)$-$M_2$ y por lo tanto $H\in M_2$, lo que es una contradicción. Así,  $H'\notin (1,\infty)$-$M_2$. Análogamente para $J'$. Como ni $H'$ ni $J'$ están en $(1,\infty)$-$M_2$ y toda subgráfica inducida de $G$ sí lo está, $G$ es libre de $H'$ y $J'$.

Recíprocamente, si $G$ es libre de $H'$ y $J'$, mostremos que $G\in (1,\infty)$-$M_2$. Sea $r$ la raíz del coárbol de $G$. Consideremos los siguientes casos que son exhaustivos.

Supongamos primero que $G$ es conexa.  Si $G$ es un vértice aislado, es claro que $G \in (1,\infty)$-$M_2$. En el caso contrario, como $G$ es libre de $J'$, todos los hijos de $r$ representan gráficas multipartitas completas. Luego, $G$ es la unión completa de varias gráficas multipartitas completas, por lo que $G$ es multipartita completa. Así, $G \in (1,\infty)$-$M_2$.
% Todos quizá excepto uno, ¿no?

Si $G$ es una gr\'afica vac\'ia, entoces es claro que $G \in (1,\infty)$-$M_2$.

Como tercer caso, consideremos que $G$ es inconexa y tiene exactamente una componente no trivial y al menos una trivial. Entonces, del Caso 2 de la segunda parte de la demostración del Teorema \ref{teo_obsts_m2}, se sigue que $G \in (1,\infty)$-$M_2$.
% ¿La referencia es correcta?   Creo que sí, pero en los warnings,
% que son más importantes de lo que parece, tienes etiquetas
% definidas en múltiples ocasiones.  Revísalas para que todo tenga
% sentido.

Finalmente, supongamos que $G$ tiene al menos dos componentes no triviales. Como $G$ es libre de $H'$, entonces cada componente de $G$ es libre de $K_3$. Es decir que cada componente de $G$ es bipartita. Luego, $G$ es bipartita. Así, $G \in (1,1)$-$M_2$, por lo que $G \in (1,\infty)$-$M_2$.

\end{proof}


    \subsection{Cálculo del conjunto de parejas mínimas de una gráfica}
        A continuación presentamos una serie de lemas para encontrar el conjunto de parejas mínimas de una gráfica.

\begin{lemma}\label{lema_parejas_01}
Si $G$ es una gr\'afica vac\'ia, entonces $G$ es un elemento de $M_2$ y
$\mu(G) = \{(0,1)\}$.
\end{lemma}

\begin{proof}
Sea $(A,B)$ una $M_2$-partición de $G$. Si ni $A$ ni $B$ son vacíos, entonces $(A,B)$ es de tamaño $(1,1)$. Si uno de los dos es vacío, entonces $(A,B)$ es de tamaño $(0,1)$. Luego, $S(G)=\{(0,1),(1,1)\}$. Como $(1,1)$ domina a $(0,1)$ y $(0,1)$ no domina a ningún elemento de $S(G)$, entonces $\mu(G)=\{(0,1)\}$.
\end{proof}

\begin{lemma}\label{lema_parejas_02}
Si $G \in M_2$ es una gráfica inconexa con al menos una componente trivial y exactamente una componente no trivial que es una gráfica multipartita completa formada por $n$ conjuntos independientes, entonces $\mu(G)=\{(1,n-1)\}$.
\end{lemma}

\begin{proof}
Notemos que, como $G$ contiene vértices aislados, toda $M_2$-partición que $G$ acepta es una partición en un conjunto independiente y una gráfica multipartita completa. Sea $(A,B)$ una $M_2$-partición de $G$ y $G[A]$ un conjunto independiente, abordemos los siguientes dos casos que son exhaustivos.

\emph{Caso 1}: $n = 2$.

Notemos que la componente no trivial de $G$ es bipartita y por lo tanto $G$ también lo es. Si tanto $G[A]$ como $G[B]$ son conjuntos independientes, entonces $(A,B)$ es de tamaño $(1,1)$. En el caso contrario $B$ contiene al menos un par de vértices que son adyacentes. Luego, $A$ contiene a todos los vértices aislados de $G$ y $B$ contiene sólo vértices de la componente no trivial de $G$. Como ésta es bipartita, entonces $(A,B)$ es de tamaño $(1,2)$. Así, $S(G)=\{(1,1),(1,2)\}$. Luego, $\mu(G)=\{(1,1)\} = \{(1,n-1)\}$.

\emph{Caso 2}: $n > 2$.

Como $G$ no es bipartita, entonces $B$ debe contener al menos un par de vértices adyacentes. Luego, $A$ debe de contener a todos los vértices aislados de $G$, y $B$ sólo puede contener vértices de la componente no trivial de $G$. Sea $(B_1, B_2, \dots, B_n)$ una partición de la componente no trivial de $G$ tal que $G[B_i]$ es un conjunto independiente para todo $0\le i \le n$. Si $A$ contiene todos los vértices de $B_j$ para algún $0\le j \le n$, entonces $(A,B)$ es de tamaño $(1,n-1)$. En el caso contrario, $(A,B)$ es de tamaño $(1,n)$. Así, $S(G)=\{(1,n-1),(1,n)\}$. Luego, $\mu(G)=\{(1,n-1)\}$.

\end{proof}

\begin{lemma}\label{lema_parejas_03}
Sea $G \in M_2$ una gráfica inconexa con al menos una componente trivial y
exactamente una componente no trivial que no es una gráfica multipartita
completa.   Si $T$ es el coárbol de $G$ y $r$ la raíz de $T$, entonces
$\mu(G) = \{(1,B(r))\}$, en donde $B(x)$ se define de la siguiente manera
para cualquier nodo $x$ de $T$ tal que $x$ tiene etiqueta 0 y al menos un
hijo que no es una hoja.
\begin{enumerate}
    \item Si $x$ es el nodo más profundo con etiqueta 0 que tiene al menos un hijo que no es una hoja.
    \begin{enumerate}
        \item Si $x$ tiene al menos dos hijos que no son hojas, entonces $B(x) = 1$.
        \item Si $x$ tiene un solo hijo, $y$, que no es una hoja, tal que $G[y]$ es una gráfica multipartita completa formada por $n$ conjuntos independientes, entonces $B(x) = n-1$.
    \end{enumerate}
    \item En el caso contrario. Sea $y$ el único hijo de $x$ que no es una hoja, $n$ el número de hijos de $y$ y $x'$ el único hijo de $y$ tal que $G[x']$ no es una gráfica multipartita completa, entonces $B(x) = n-1 + B(x')$.
\end{enumerate}
\end{lemma}

\begin{proof}

Podemos verificar que estos casos son exhaustivos a través del \emph{Caso 2} de la demostración del Teorema \ref{teo_obsts_m2}.

Sea $(A,B)$ una $M_2$-partición de $G$, notemos que $(A,B)$ es de tamaño $(1,\beta)$ para algún entero $\beta \geq 2$. Luego $(1,\beta)\in \mu(G)$ si y sólo si $\beta$ es mínimo. Es decir que $\mu(G)$ tendrá un solo elemento, y éste será el tamaño de una $M_2$-partición de $G$ en un conjunto independiente y una gráfica multipartita completa con el menor número posible de partes.

Sea $x$ un nodo de $T$. Mostremos que si $x$ tiene etiqueta 0 y al menos un hijo que no es una hoja, entonces una $M_2$-partición de $G[x]$ en un conjunto independiente y una gráfica multipartita completa con el menor número posible de partes será de tamaño $(1,B(x))$. Sea $z$ el nodo más profundo de $T$ con etiqueta 0 y al menos un hijo que no es una hoja, y $d$ la distancia desde $x$ hasta $z$, probemos por inducción sobre $d$.

\emph{Caso base}: $d = 0$, es decir que $x = z$.

Por el \emph{Caso 2} de la demostración del Teorema \ref{teo_obsts_m2}, sabemos que $x$ tiene al menos dos hijos que no son hojas o $x$ tiene un solo hijo $y$ tal que $y$ no es una hoja y que $G[y]$ es una gráfica multipartita completa formada por $n$ conjuntos independientes.

Si $x$ tiene al menos dos hijos que no son hojas, cada uno de estos hijos es la raíz del coárbol de una gráfica bipartita. Luego, $G[x]$ es bipartita. Así, $G[x]$ acepta una $M_2$-partición $(A,B)$ de tamaño $(1,1)$. Como $G[x]$ no es un conjunto independiente, entonces $(A,B)$ es una $M_2$-partición de $G[x]$ en un conjunto independiente y una gráfica multipartita completa con el menor número posible de partes. Finalmente, notemos que $(A,B)$ es de tamaño $(1,B(x))$ dado que se cumple la primera condición \emph{1.a)} del lema.

Si $x$ tiene un solo hijo $y$ que no es una hoja tal que $G[y]$ es una gráfica multipartita completa formada por $n$ conjuntos independientes, por el Lema \ref{lema_parejas_02}, sabemos que $\mu(G[x]) = (1,n-1)$. Es claro que no existe una $M_2$-partición de $G[x]$ en un conjunto independiente y una gráfica multipartita completa con menos de $n-1$ partes. Así, dado que se cumple la condición \emph{1.b)} del lema, una $M_2$-partición de $G[x]$ en un conjunto independiente y una gráfica multipartita completa con el menor número posible de partes será de tamaño $(1,n-1)=(1,B(x))$.

\emph{Paso inductivo}: $d > 0$.

Sea $x'$ un nodo de $T$ con etiqueta 0 y al menos un hijo que no es una hoja. Supongamos como hipótesis inductiva (H.I.) que, si la distancia desde $x'$ hasta $z$ es menor a $d$, entonces una $M_2$-partición de $G[x']$ en un conjunto independiente y una gráfica multipartita completa con el menor número posible de partes es de tamaño $(1,B(x'))$.

Por el \emph{Caso 2} de la demostración del Teorema \ref{teo_obsts_m2}, sabemos que $x$ tiene un sólo hijo $y$ que no es una hoja mientras que el resto son hojas. Asimismo, sabemos que $y$ tiene un solo hijo $x'$ tal que $G[x']$ no es una gráfica multipartita completa. Es decir que $x'$, que tiene etiqueta 0, tiene al menos un hijo que no es una hoja. Por H.I., una $M_2$-partición de $G[x']$ en un conjunto independiente y una gráfica multipartita completa con el menor número posible de partes es de tamaño $(1,B(x'))$.

Sean $(A',B')$ una partición de $G[x']$ de tamaño $(1,B(x'))$ tal que $G[A']$ es un conjunto independiente y $(A,B)$ una partición de $G[x]$ tal que $G[A]$ es un conjunto independiente y $G[B]$ una gráfica multipartita completa. Como $G$ no es bipartita, entonces $A$ debe contener todos los vértices de $G$ que son hijos de $x$ (es decir, todos los hijos de $x$ que son hojas), mientras que $B$ sólo puede contener vértices de $G[y]$. Para que $G[B]$ sea una una gráfica multipartita completa, ésta debe de ser la unión completa de un conjunto de gráficas multipartitas completas. Como todos los hijos de $y$ representan gráficas multipartitas completas menos $x'$, entonces $A$ debe de contener un subconjunto $S$ de los vértices de $G[x']$ tal que $G[x']-S$ sea una gráfica multipartita completa.

Sea $\beta'$ un entero tal que $G[x']-S$ es una gráfica multipartita completa
formada por $\beta'$ conjuntos independientes y $n$ el número de hijos de $y$,
calculemos el tamaño $(\alpha,\beta)$ de $(A,B)$ en función de $\beta'$ y $n$.
Sabemos que $\alpha=1$ y que $\beta$ es el número de conjuntos independientes que
forman a la gráfica multipartita completa $G[x']-S$. Como $G[x']-S$ es la unión
completa de $n$ gráficas multipartitas completas, entonces $\beta$ es la suma del
número de conjuntos independientes que conforman a cada una de estas gráficas
multipartitas completas. Todos los hijos de $y$ que no son $x'$ son hojas o
representan gráficas multipartitas completas inconexas. Es decir, conjuntos independientes. En cualquier caso, cada uno de los $n-1$ hijos de $y$ representa un conjunto independiente. Como $G[x']-S$ es la unión completa de $\beta'$ conjuntos independientes, entonces $\beta = n-1+\beta'$.

Finalmente, encontremos el menor $\beta'$ posible. Si $S=A'$, entonces $\beta'=B(x')$. Como $B(x')$ es mínimo por H.I., entonces el mínimo valor que puede tomar $\beta'$ es $B(x')$. Luego, el valor mínimo de $\beta$ es $n-1+B(x')$ . Así, una $M_2$-partición de $G[x]$ en un conjunto independiente y una gráfica multipartita completa con el menor número posible de partes es de tamaño $(1,n-1+B(x'))$. Como $x$ cumple la condición 2 del lema, este tamaño es igual a $(1,B(x))$.

\end{proof}

\begin{lemma}\label{lema_parejas_04}
Sea $G\in M_2$ una gráfica inconexa no bipartita con dos componentes conexas no triviales. Si las componentes de $G$ son una la unión ajena de $\alpha$ conjuntos independientes y la otra la unión ajena de $\beta$ conjuntos independientes con $\alpha \le \beta$, entonces $\mu(G)=\{(\alpha,\beta)\}$.
\end{lemma}

\begin{proof}
Sea $(A,B)$ una $M_2$-partición de $G$. Como $G$ no es bipartita, entonces $A$ debe de contener todos los vértices de una de las componentes de $G$ y $B$ debe contener todos los vértices de la otra componente. Como ésta es la única posible $M_2$-partición de $G$ y es de tamaño $(\alpha, \beta)$, entonces $\mu(G)=\{(\alpha,\beta)\}$.
\end{proof}

\begin{lemma}\label{lema_parejas_05}
Si $G \in M_2$ es una gráfica inconexa bipartita con al menos dos componentes conexas no triviales, entonces $\mu(G)=\{(1,1)\}$.
\end{lemma}

\begin{proof}
Como $G$ es bipartita, entonces $G$ acepta una $M_2$-partición de tamaño
$(1,1)$. Como $G$ tiene un $\overline{P_3}$, entonces $G$ no es una gráfica
multipartita completa. Luego, $G$ no acepta $M_2$-particiones de tamaño
$(0,\beta)$ para cualquier entero $\beta \ge 1$. Así, cualquier
$M_2$-partición de $G$ será de tamaño $(\alpha,\beta')$ para algunos
$\alpha \ge 1$ y $\beta \ge 1$. Luego, $\mu(G)=\{(1,1)\}$.
\end{proof}

\begin{lemma}\label{lema_parejas_06}
Si $G$ es una gráfica isomorfa a $K_1$, entonces $\mu(G)=\{0,1\}$.
\end{lemma}

\begin{proof}
Es claro que la única $M_2$-partición posible de $G$ es de tamaño $(0,1)$. Luego, $\mu(G)=\{(0,1)\}$.
\end{proof}

\begin{lemma}\label{lema_parejas_07}
Sean $G_1, G_2 \in M_2$, $G=G_1 \oplus G_2$ y $S'(G)$ el conjunto de todas
las parejas $(\alpha, \beta)$ tales que para algunos $(\alpha_1,\beta_1)
\in \mu(G_1)$ y $(\alpha_2, \beta_2) \in \mu(G_2)$ alguna de las siguientes
condiciones se cumple
\begin{itemize}
    \item $\alpha = \alpha_1+\alpha_2$ y $\beta = \beta_1 + \beta_2$ o
    \item $\alpha = \emph{min}(\alpha_1+\beta_2, \alpha_2+\beta_1)$ y $\beta = \emph{max}(\alpha_1+\beta_2, \alpha_2+\beta_1)$.
\end{itemize}
Una pareja de enteros $P$ está en $\mu(G)$ si y sólo si $P\in S'(G)$ y $P$ no domina a ningún otro elemento de $S'(G)$.
\end{lemma}

\begin{proof}
Notemos que $(A,B)$ es una $M_2$-partición de $G$ si y sólo si existen $M_2$-particiones $(A_1,B_1)$ y $(A_2,B_2)$ de $G_1$ y $G_2$ respectivamente tales que $(A,B)=(A_1\cup A_2, B_1 \cup B_2)$.

Veamos que $S'(G)\subset S(G)$. Sea $(x,y)$ un elemento de $S'(G)$. Sabemos que existen $(\alpha_1,\beta_1)\in \mu(G_1)$ y $(\alpha_2,\beta_2)\in \mu(G_2)$ tales que se cumple alguna de las condiciones mencionadas en el lema. Como $(\alpha_1,\beta_1)\in \mu(G_1)$, entonces existe una $M_2$-partición $(A_1,B_1)$ de $G_1$ de tamaño $(\alpha_1, \beta_1)$. Análogamente, existe una $M_2$-partición $(A_2,B_2)$ de $G_2$ de tamaño $(\alpha_2, \beta_2)$. Luego, $(A_1\cup A_2, B_1 \cup B_2)$ es una $M_2$-partición de $G$ de tamaño $(x,y)$ o $(A_1\cup B_2, B_1 \cup A_2)$ es una $M_2$ partición de $G$ de tamaño $(x,y)$. En cualquiera de los casos, $(x,y)\in S(G)$. Así, $S'(G)\subset S(G)$.

Para la contención restante, sea $(x,y)$ un elemento de $S'(G)$ tal que $(x,y)$ no domina a ningún otro elemento de $S'(G)$. Como $S'(G)\subset S(G)$, entonces $(x,y)\in S(G)$. Luego, existe una $M_2$-partición $(A,B)$ de $G$ de tamaño $(x,y)$. Mostremos que para cualquier $M_2$-partición $(A',B')$ de $G$ de tamaño $(\alpha',\beta')\in S(G)$, se cumple que $(x,y)$ no domina a $(\alpha',\beta')$. Si $(\alpha',\beta')\in S'(G)$, entonces $(x,y)$ no domina a $(\alpha',\beta')$. En el caso contrario, para cualesquiera $M_2$-particiones $P_1=(A_1,B_1)$ y $P_2=(A_2,B_2)$ de $G_1$ y $G_2$ de tamaños $(\alpha_1,\beta_1)\in \mu(G_1)$ y $(\alpha_2,\beta_2)\in \mu(G_2)$, respectivamente, se cumplen las siguientes condiciones:
\begin{itemize}
    \item $\alpha' \neq \alpha_1+\alpha_2$ o $\beta' \neq \beta_1 + \beta_2$ y
    \item $\alpha' \neq \emph{min}(\alpha_1+\beta_2, \alpha_2+\beta_1)$ o  $\beta' \neq \emph{max}(\alpha_1+\beta_2, \alpha_2+\beta_1)$.
\end{itemize}
Luego, ni $A'$ ni $B'$ son el resultado de la unión de una parte de $P_1$ y una parte de $P_2$. Sea $(A'',B'')$ una $M_2$-partición de $G$ tal que $A''=A_1\cup A_2$ y $B''=B_1\cup B_2$ y $(\alpha'', \beta'')$ el tamaño de $(A'',B'')$, notemos que $(\alpha'', \beta'')\in S'(G)$ y que necesariamente $(\alpha', \beta')$ domina a $(\alpha'', \beta'')$. Luego, como $(\alpha'', \beta'')\in S'(G)$, entonces $(x,y)$ no domina a $(\alpha'', \beta'')$. Así, como la dominación es una relación transitiva, $(x,y)$ no domina a $(\alpha', \beta')$.

Recíprocamente. Sea $(x,y)\in\mu(G)$, entonces existe una $M_2$-partición $(A,B)$ de $G$ de tamaño $(x,y)$. Luego, existen $M_2$-particiones $(A_1,B_1)$ de tamaño $(\alpha_1, \beta_1)$ y $(A_2,B_2)$ de tamaño $(\alpha_2, \beta_2)$ de $G_1$ y $G_2$ respectivamente tales que $(A,B)=(A_1\cup A_2, B_1 \cup B_2)$. Como todos los vértices de $A_1$ son adyacentes a todos los vértices de $A_2$ y todos los vértices de $B_1$ son adyacentes a todos los vértices de $B_2$, se cumple que $\alpha = \alpha_1+\alpha_2$ y $\beta = \beta_1 + \beta_2$, o bien, se cumple que $\alpha = \emph{min}(\alpha_1+\beta_2, \alpha_2+\beta_1)$ y $\beta = \emph{max}(\alpha_1+\beta_2, \alpha_2+\beta_1)$.

Veamos que $(\alpha_1, \beta_1)\in\mu(G_1)$. Procedamos por contradicción. Supongamos que $(\alpha_1, \beta_1)\notin\mu(G_1)$. Como $(A_1,B_1)$ existe, entonces $(\alpha_1, \beta_1)\in S(G_1)$. Luego debe de existir otra pareja $(\alpha_1', \beta_1')\in\S(G_1)$ tal que $(\alpha_1, \beta_1)$ domina a $(\alpha_1', \beta_1')$. De esto se sigue que existe una $M_2$-partición $(A_1',B_1')$ de $G_1$ de tamaño $(\alpha_1', \beta_1')$. Como $(A_1'\cup A_2, B_1'\cup B_2)$ es una $M_2$-partición de $G$ cuyo tamaño es dominado por $(x,y)$, entonces $(x,y)\notin\mu(G)$. Lo cual es una contradicción. Así, $(\alpha_1, \beta_1)\in\mu(G_1)$. Análogamente, $(\alpha_2, \beta_2)\in\mu(G_2)$. Así, $(x,y)\in S'(G)$.

Como $S'(G)\subset S(G)$ y $(x,y)\in\mu(G)$, entonces $(x,y)$ no domina a ningún elemento de $S(G)$, y por lo tanto tampoco domina a ningún elemento de $S'(G)$.

\end{proof}


    \subsection{Algoritmo para generar obstrucciones mínimas}
        En esta subsección presentamos cuatro algoritmos. Los primeros dos (Algoritmos \ref{alg_func_B} y \ref{alg_parejas_min}) se utilizan para calcular el conjunto de parejas mínimas de una gráfica. El Algoritmo \ref{alg_decision_alfabeta} resuelve el problema de decisión de si una gráfica pertenece a una clase $(\alpha,\beta)$-$M_2$. El Algoritmo \ref{alg_obstrucciones_alfabeta} es capaz de generar las obstrucciones mínimas de una clase $(\alpha,\beta)$-$M_2$ con a lo más un número $n$ de vértices. Éste es utilizado en la siguiente sección para generar algunas obstrucciones mínimas para algunas clases $(\alpha,\beta)$-$M_2$.

\subsubsection{Algoritmos para el cálculo del conjunto de parejas mínimas}

El Algoritmo \ref{alg_func_B} evalúa la función $B(x)$ descrita en el Lema \ref{lema_parejas_03} para un nodo $x$ de un coárbol que representa a una gráfica $G$. Es decir que ésta encuentra el menor entero $n$ tal que $G[x]$ acepta una partición en un conjunto independiente y una gráfica multipartita completa formada por  $n$ conjuntos independientes. Dado que este algoritmo es una implementación directa del Lema \ref{lema_parejas_03}, la demostración de este lema funciona también para el algoritmo. 

\begin{algorithm}[ht!]
\caption{Función\_B}
\label{alg_func_B}
\DontPrintSemicolon % Some LaTeX compilers require you to use \dontprintsemicolon instead
\KwIn{$x$, el nodo de un coárbol con etiqueta 0 y al menos un hijo que no es una hoja. $x$ es la raíz del coárbol de una cográfica en $M_2$.}
\KwOut{El valor de $B(x)$, de acuerdo con la definición del Lema \ref{lema_parejas_03}.}

\If{$X$ tiene más de un hijo que no es una hoja}{
    \Return 1\;
}
\Else{
    $y\gets $ el hijo de $x$ que no es una hoja.\;
    \If{$y$ es la raíz del coárbol de una cográfica multipartita completa}{
        \Return $y.hijos.tama\tilde{n}o - 1$\;
    }
    \Else{
        $x' \gets $ el único hijo de $y$ que tiene al menos un hijo que no es una hoja\;
        \Return $y.hijos.tama\tilde{n}o - 1 + \text{Función\_B}(x')$\;
    }
}

\end{algorithm}

El algoritmo \ref{alg_parejas_min} recibe como entrada una gráfica $G$ representada a través de la raíz de su coárbol y devuelve su conjunto de parejas mínimas $\mu(G)$. Éste funciona aplicando los Lemas \ref{lema_parejas_01}, \ref{lema_parejas_03}, \ref{lema_parejas_04}, \ref{lema_parejas_05}, \ref{lema_parejas_06} y \ref{lema_parejas_07}. 

\begin{algorithm}[ht!]
\caption{Conjunto\_De\_Parejas\_Mínimas}
\label{alg_parejas_min}
\DontPrintSemicolon % Some LaTeX compilers require you to use \dontprintsemicolon instead
\KwIn{$g$, la raíz del coárbol de una cográfica $G$.}
\KwOut{$\mu(G)$}

$\mu(G)\gets\varnothing$ \;

\If{$g$ es una hoja}{
    Agregar $(0,1)$ a $\mu(G)$\;
}
\ElseIf{$g$ tiene etiqueta 1}{
    $\mu(G)\gets\text{Conjunto\_De\_Parejas\_Mínimas}(g.hijos[0])$\;
    \ForEach{$h$ \emph{\textbf{en}} $g.hijos$ con excepción de $g.hijos[0]$}{
        $T\gets\text{Conjunto\_De\_Parejas\_Mínimas}(h)$\;
        $S'\gets \varnothing$\;
        \ForEach{$(\alpha_1,\beta_1)\in \mu(G)$ \emph{\textbf{y cada}} $(\alpha_1,\beta_1)\in T$}{
            Agregar $(\alpha_1+\alpha_2,\beta_1+\beta_2)$ a $S'$\;
            Agregar $(\text{min}(\alpha_1+\beta_2,\alpha_2+\beta_1), \text{max}(\alpha_1+\beta_2,\alpha_2+\beta_1))$ a $S'$\;
        }
        $\mu(G)\gets\varnothing$ \;
        Agregar a $\mu(G)$ todos los elementos de $S'$ que no dominan a ningún otro elemento de $S'$\;
    }
}
\Else{
    $n\gets $ el número de hijos de $g$ que no son hojas\;
    \If{$n = 0$}{
        Agregar $(0,1)$ a $\mu(G)$\;
    }
    \ElseIf{$n=1$}{
        Agregar $(0,\text{Función\_B}(g))$ a $\mu(G)$\;
    }
    \ElseIf{$n=2$ y $G$ no es bipartita}{
        $\alpha \gets g.hijos[0].hijos.tama\tilde{n}o$\;
        $\beta \gets g.hijos[1].hijos.tama\tilde{n}o$\;
        Agregar $(\text{min}(\alpha,\beta),\text{max}(\alpha,\beta))$ a $\mu(G)$\;
    }
    \Else{
        Agregar $(1,1)$ a $\mu(G)$\;
    }
}

\Return $\mu(G)$\;

\end{algorithm}

Notemos que los casos considerados son exhaustivos. Las condiciones de las líneas 2 y 4 cubren todos los casos en los que la gráfica $G$ es conexa, mientras que las condiciones de las líneas 16, 18, 20 y 24 cubren todos los casos en los que la gráfica $G$ es inconexa.

En la línea 3 del algoritmo se aplica el Lema \ref{lema_parejas_06}. En la línea 17 se aplica el Lema \ref{lema_parejas_01}. En la línea 19 se aplica el Lema \ref{lema_parejas_03}. En la línea 23 se aplica el Lema \ref{lema_parejas_04} y en la línea 25 se aplica el Lema \ref{lema_parejas_05}. 

El único caso en el que no se aplica directamente un lema de la subsección anterior es en el caso en el que $G$ es una gráfica conexa no trivial. Éste corresponde al bloque de líneas 5 a 13 del algoritmo. En este bloque se calcula el conjunto de parejas mínimas de $G$ de forma incremental aplicando el Lema \ref{lema_parejas_07}. Sea $G=G_1\oplus G_2 \oplus \dots \oplus G_n$ para un entero $n>1$, primero se calcula $\mu(G_1)$, posteriormente $\mu(G_1\oplus G_2)$ y así sucesivamente hasta encontrar $\mu(G)$. En la línea 5 se inicializa $\mu(G)$ con los elementos de $G_1$. En el bloque de las líneas 6 a 13 se procesan el resto de las $G_i$ con $1<i\le n$. En cada iteración $i$ se calcula $\mu(G_i)$ y se guarda en $T$ (línea 7). Luego, en las líneas 10 y 11 se aplica el Lema \ref{lema_parejas_07} para calcular $S'(G_1 \oplus \dots \oplus G_i)$. Finalmente, en la línea 13 se actualiza $\mu(G)$ con los elementos de $S'(G_1 \oplus \dots \oplus G_i)$ que no dominan a ningún elemento del mismo conjunto. Así, al final de cada iteración, $\mu(G)$ tiene asignado el valor de $\mu(G_1 \oplus \dots \oplus G_i)$, y al final de la última iteración, $\mu(G)=\mu(G_1 \oplus \dots \oplus G_n)$.


\subsubsection{Algoritmo para el problema de decisión}

El Algoritmo \ref{alg_decision_alfabeta} es una implementación del Lema \ref{lema_parejas_principal}. Éste recibe como entrada la raíz del coárbol de una gráfica $G$, elemento de $M_2$, y dos enteros $\alpha$ y $\beta$ tales que $\alpha \le \beta$. Luego, calcula $\mu(G)$ y determina si $G\in(\alpha,\beta)$-$M_2$.

\begin{algorithm}[ht!]
\caption{Pertenece\_a\_Alfa\_Beta\_M2}
\label{alg_decision_alfabeta}
\DontPrintSemicolon % Some LaTeX compilers require you to use \dontprintsemicolon instead
\KwIn{$g$, la raíz del coárbol de una gráfica $G\in M_2$; $\alpha$, un entero mayor o igual a uno; $\beta$, un entero mayor o igual a $\alpha$}
\KwOut{$verdadero$ si $G\in(\alpha,\beta)$-$M_2$. $falso$ en el caso contrario}

$\mu(G) \gets \text{Conjunto\_De\_Parejas\_Mínimas}(g)$\;

\ForEach{$(\alpha',\beta')$ \emph{\textbf{en}} $\mu(G)$}{
    \If{$(\alpha',\beta')$ domina a $(\alpha,\beta)$}{
        \Return $verdadero$\;
    }
}

\Return $falso$\;

\end{algorithm} 

\subsubsection{Algoritmo para generar las obstrucciones mínimas de una clase $(\alpha,\beta)$-$M_2$}

El Algoritmo \ref{alg_obstrucciones_alfabeta} recibe como entrada tres enteros $\alpha$, $\beta$ y $n$ mayores a uno tales que $\alpha \le \beta$, y devuelve $S$, un conjunto con los coárboles de las obstrucciones mínimas de la clase $(\alpha,\beta)$-$M_2$ con a lo más $n$ vértices. Para ello, el algoritmo genera todos los coárboles con $i$ hojas para cada $1\le i \le n$ \cite{Jones}. Éste determina si cada uno de estos coárboles representa una gráfica en $(\alpha,\beta)$-$M_2$. Si la gráfica no pertenece a la clase, esto significa que es una obstrucción de la misma, por lo que evalúa cada una de sus subgráficas inducidas para determinar si es una obstrucción mínima.

\begin{algorithm}[ht!]
\caption{Generar\_obstrucciones}
\label{alg_obstrucciones_alfabeta}
\DontPrintSemicolon % Some LaTeX compilers require you to use \dontprintsemicolon instead
\KwIn{$\alpha$, $\beta$ y $n$, enteros mayores a uno}
\KwOut{$S$, un conjunto cuyos elementos son coárboles de gráficas de a lo más $n$ vértices que son obstrucciones mínimas de la clase $(\alpha,\beta)$-$M_2$}

\ForEach{$1\le i \le n$}{
    $\tau\gets$ todos los coárboles con $i$ hojas\;
    \ForEach{coárbol $T$ en $\tau$ con raíz $r$ que representa a una gráfica $G$}{
        \If{$\emph{Pertenece\_a\_Alfa\_Beta\_M2}(r,\alpha,\beta) = falso$}{
            \If{Toda subgráfica inducida de $G$ está en $(\alpha,\beta)$-$M_2$}{
                Agregar $T$ a $S$\;
            }
        }    
    }
}

\Return $S$\;

\end{algorithm} 

    \subsection{Resultados del algoritmo}
        En esta subsección mostramos las obstrucciones mínimas de algunas clases $(\alpha, \beta)$-$M_2$ encontradas con el Algoritmo \ref{alg_obstrucciones_alfabeta}. Las obstrucciones se buscaron en coárboles de hasta 15 hojas.

\subsubsection{Clases $(1,\beta)$-$M_2$}

Las Figuras \ref{obsts_1_1_M2},\ref{obsts_1_2_M2},\ref{obsts_1_3_M2},\ref{obsts_1_4_M2} y \ref{obsts_1_5_M2} muestran los conjuntos de obstrucciones mínimas encontradas con el Algoritmo \ref{alg_obstrucciones_alfabeta} para las clases $(1,1)$-$M_2$, $(1,2)$-$M_2$, $(1,3)$-$M_2$, $(1,4)$-$M_2$ y $(1,5)$-$M_2$ respectivamente.

\begin{figure}[ht!]

\begin{subfigure}{\textwidth}
\begin{center}
\begin{tikzpicture}
\begin{scope}[xshift=0cm,scale=1]

\node [style=vertex] (1) at (0,0.5) {};
\node [style=vertex] (2) at (1,0.5) {};
\node [style=vertex] (3) at (0.5,1.25) {};

\foreach \i/\j in {1/2,1/3,2/3} \draw [style=edge] (\i) to (\j);
\node at (0.5,0) {\parbox{0.3\linewidth}{\subcaption*{$o_{(1,1),1}$}}};
\end{scope}

\end{tikzpicture}
\end{center}
\end{subfigure}

%\setlength{\abovecaptionskip}{-15pt}
\caption{Obstrucción mínima de la clase $(1,1)$-$M_2$.}
\label{obsts_1_1_M2}
\end{figure}
\begin{figure}[ht!]

\begin{subfigure}{\textwidth}
\begin{center}
\begin{tikzpicture}

\begin{scope}[xshift=0cm,scale=1]

\node [style=vertex] (1) at (0,0.5) {};
\node [style=vertex] (2) at (1,0.5) {};
\node [style=vertex] (3) at (0,1) {};
\node [style=vertex] (4) at (1,1) {};
\node [style=vertex] (5) at (0.5,1.75) {};

\foreach \i/\j in {1/2,3/4,3/5,4/5} \draw [style=edge] (\i) to (\j);
\node at (0.5,0) {\parbox{0.3\linewidth}{\subcaption*{$o_{(1,2),1}$}}};
\end{scope}

\begin{scope}[xshift=2.5cm,scale=1]

\node [style=vertex] (1) at (0,0.5) {};
\node [style=vertex] (2) at (1,0.5) {};
\node [style=vertex] (3) at (0.5,0.85) {};
\node [style=vertex] (4) at (0.5,1.5) {};

\foreach \i/\j in {1/2,1/3,1/4,2/3,2/4,3/4} \draw [style=edge] (\i) to (\j);
\node at (0.5,0) {\parbox{0.3\linewidth}{\subcaption*{$o_{(1,2),2}$}}};
\end{scope}

\end{tikzpicture}
\end{center}
\end{subfigure}

%\setlength{\abovecaptionskip}{-15pt}
\caption{Algunas obstrucciones mínimas de la clase $(1,2)$-$M_2$.}
\label{obsts_1_2_M2}
\end{figure}
\begin{figure}[ht!]

\begin{subfigure}{\textwidth}
\begin{center}
\begin{tikzpicture}

\begin{scope}[xshift=0cm,scale=1]

\node [style=vertex] (1) at (0,0.5) {};
\node [style=vertex] (2) at (1,0.5) {};
\node [style=vertex] (3) at (0,1) {};
\node [style=vertex] (4) at (1,1) {};
\node [style=vertex] (5) at (0.5,1.75) {};

\foreach \i/\j in {1/2,3/4,3/5,4/5} \draw [style=edge] (\i) to (\j);
\node at (0.5,0) {\parbox{0.3\linewidth}{\subcaption*{$o_{(1,3),1}$}}};
\end{scope}

\begin{scope}[xshift=2.5cm,scale=1]

\node [style=vertex] (1) at (0.5,0.5) {};
\node [style=vertex] (2) at (1.5,0.5) {};
\node [style=vertex] (3) at (0.5,1.5) {};
\node [style=vertex] (4) at (1.5,1.5) {};
\node [style=vertex] (5) at (0,2) {};
\node [style=vertex] (6) at (2,2) {};

\foreach \i/\j in {1/2,1/3,1/4,2/3,2/4,3/4,5/1,5/3,5/6,6/2,6/4} \draw [style=edge] (\i) to (\j);
\node at (1,0) {\parbox{0.3\linewidth}{\subcaption*{$o_{(1,3),2}$}}};
\end{scope}

\begin{scope}[xshift=6cm,scale=1]

\node [style=vertex] (1) at (0.25,0.5) {};
\node [style=vertex] (2) at (1.25,0.5) {};
\node [style=vertex] (3) at (0,1.15) {};
\node [style=vertex] (4) at (1.5,1.15) {};
\node [style=vertex] (5) at (0.75,1.75) {};

\foreach \i/\j in {1/2,1/3,1/4,1/5,2/3,2/4,2/5,3/4,3/5,4/5} \draw [style=edge] (\i) to (\j);
\node at (0.75,0) {\parbox{0.3\linewidth}{\subcaption*{$o_{(1,3),3}$}}};
\end{scope}

\end{tikzpicture}
\end{center}
\end{subfigure}

%\setlength{\abovecaptionskip}{-15pt}
\caption{Algunas obstrucciones mínimas de la clase $(1,3)$-$M_2$.}
\label{obsts_1_3_M2}
\end{figure}
\begin{figure}[ht!]

\begin{subfigure}{\textwidth}
\begin{center}
\begin{tikzpicture}

\begin{scope}[xshift=0cm,scale=1]

\node [style=vertex] (1) at (0,0.5) {};
\node [style=vertex] (2) at (1,0.5) {};
\node [style=vertex] (3) at (0,1) {};
\node [style=vertex] (4) at (1,1) {};
\node [style=vertex] (5) at (0.5,1.75) {};

\foreach \i/\j in {1/2,3/4,3/5,4/5} \draw [style=edge] (\i) to (\j);
\node at (0.5,0) {\parbox{0.3\linewidth}{\subcaption*{$o_{(1,4),1}$}}};
\end{scope}

\begin{scope}[xshift=2.5cm,scale=1]

\node [style=vertex] (1) at (0.5,0.5) {};
\node [style=vertex] (2) at (1.5,0.5) {};
\node [style=vertex] (3) at (0.5,1.5) {};
\node [style=vertex] (4) at (1.5,1.5) {};
\node [style=vertex] (5) at (0,2) {};
\node [style=vertex] (6) at (2,2) {};

\foreach \i/\j in {1/2,1/3,1/4,2/3,2/4,3/4,5/1,5/3,5/6,6/2,6/4} \draw [style=edge] (\i) to (\j);
\node at (1,0) {\parbox{0.3\linewidth}{\subcaption*{$o_{(1,4),2}$}}};
\end{scope}

\begin{scope}[xshift=6cm,scale=1]

\node [style=vertex] (1) at (0.5,0.5) {};
\node [style=vertex] (2) at (1.5,0.5) {};
\node [style=vertex] (3) at (0,1.25) {};
\node [style=vertex] (4) at (2,1.25) {};
\node [style=vertex] (5) at (0.5,2) {};
\node [style=vertex] (6) at (1.5,2) {};

\foreach \i/\j in {1/2,1/3,1/4,1/5,1/6,2/3,2/4,2/5,2/6,3/4,3/5,3/6,4/5,4/6,5/6} \draw [style=edge] (\i) to (\j);
\node at (1,0) {\parbox{0.3\linewidth}{\subcaption*{$o_{(1,4),3}$}}};
\end{scope}

\end{tikzpicture}
\end{center}
\end{subfigure}

%\setlength{\abovecaptionskip}{-15pt}
\caption{Algunas obstrucciones mínimas de la clase $(1,4)$-$M_2$.}
\label{obsts_1_4_M2}
\end{figure}
\begin{figure}[ht!]

\begin{subfigure}{\textwidth}
\begin{center}
\begin{tikzpicture}

\begin{scope}[xshift=0cm,scale=1]

\node [style=vertex] (1) at (0,0.5) {};
\node [style=vertex] (2) at (1,0.5) {};
\node [style=vertex] (3) at (0,1) {};
\node [style=vertex] (4) at (1,1) {};
\node [style=vertex] (5) at (0.5,1.75) {};

\foreach \i/\j in {1/2,3/4,3/5,4/5} \draw [style=edge] (\i) to (\j);
\node at (0.5,0) {\parbox{0.3\linewidth}{\subcaption*{$o_{(1,5),1}$}}};
\end{scope}

\begin{scope}[xshift=2.5cm,scale=1]

\node [style=vertex] (1) at (0.5,0.5) {};
\node [style=vertex] (2) at (1.5,0.5) {};
\node [style=vertex] (3) at (0.5,1.5) {};
\node [style=vertex] (4) at (1.5,1.5) {};
\node [style=vertex] (5) at (0,2) {};
\node [style=vertex] (6) at (2,2) {};

\foreach \i/\j in {1/2,1/3,1/4,2/3,2/4,3/4,5/1,5/3,5/6,6/2,6/4} \draw [style=edge] (\i) to (\j);
\node at (1,0) {\parbox{0.3\linewidth}{\subcaption*{$o_{(1,5),2}$}}};
\end{scope}

\begin{scope}[xshift=6cm,scale=1]

\node [style=vertex] (1) at (0.5,0.5) {};
\node [style=vertex] (2) at (1.5,0.5) {};
\node [style=vertex] (3) at (0,1) {};
\node [style=vertex] (4) at (2,1) {};
\node [style=vertex] (5) at (0,2) {};
\node [style=vertex] (6) at (2,2) {};
\node [style=vertex] (7) at (1,2.5) {};

\foreach \i/\j in {1/2,1/3,1/4,1/5,1/6,1/7,2/3,2/4,2/5,2/6,2/7,3/4,3/5,3/6,3/7,4/5,4/6,4/7,5/6,5/7,6/7} \draw [style=edge] (\i) to (\j);
\node at (1,0) {\parbox{0.3\linewidth}{\subcaption*{$o_{(1,5),3}$}}};
\end{scope}

\end{tikzpicture}
\end{center}
\end{subfigure}

%\setlength{\abovecaptionskip}{-15pt}
\caption{Algunas obstrucciones mínimas de la clase $(1,5)$-$M_2$.}
\label{obsts_1_5_M2}
\end{figure}

Podemos observar que el algoritmo encuentra que $K_3$ es una obstrucción mínima de la clase $(1,1)$-$M_2$, lo cuál sabemos que es correcto, dado que ésta es la clase de las cográficas bipartitas. 

Sea $\beta$ un entero mayor o igual a uno. Si $2<\beta\le 5$, notemos que el algoritmo encuentra tres obstrucciones mínimas para la clase $(1,\beta)$-$M_2$. Dos de éstas son las obstrucciones mínimas de la clase $(1,\infty)$-$M_2$. Por otra parte, si $0\le\beta\le 5$, se cumple que $K_{\beta+2}$ es una obstrucción  mínima de la clase $(1,\beta)$-$M_2$. Esto nos lleva a pensar en la clase $(1,\infty)$-$M_2$ como el límite cuando $\beta$ tiende a infinito de las clases $(1,\beta)$-$M_2$. Es decir que esperamos que, si una gráfica $G$ es una obstrucción mínima de cada una de las clases $(\alpha,\beta)$-$M_2$ con $\alpha$ fijo y $\beta>n$ para un entero $n$, entonces $G$ sea una obstrucción mínima de la clase $(\alpha,\infty)$-$M_2$.

\subsubsection{Clases $(2,\beta)$-$M_2$}

Las Figuras \ref{obsts_2_2_M2},\ref{obsts_2_3_M2} y \ref{obsts_2_4_M2} muestran los conjuntos de obstrucciones mínimas encontradas con el Algoritmo \ref{alg_obstrucciones_alfabeta} para las clases $(2,2)$-$M_2$, $(2,3)$-$M_2$ y $(2,4)$-$M_2$ respectivamente. Además de éstas, enlistamos las obstrucciones mínimas de las clases $(2,5)$-$M_2$, $(2,6)$-$M_2$ y $(2,7)$-$M_2$.

\begin{figure}[ht!]

\begin{subfigure}{\textwidth}
\begin{center}
\begin{tikzpicture}

\begin{scope}[xshift=0cm,scale=1]

\node [style=vertex] (1) at (0.25,0.5) {};
\node [style=vertex] (2) at (1.25,0.5) {};
\node [style=vertex] (3) at (0,1.15) {};
\node [style=vertex] (4) at (1.5,1.15) {};
\node [style=vertex] (5) at (0.75,1.75) {};

\foreach \i/\j in {1/2,1/3,1/4,1/5,2/3,2/4,2/5,3/4,3/5,4/5} \draw [style=edge] (\i) to (\j);
\node at (0.75,0) {\parbox{0.3\linewidth}{\subcaption*{$o_{(2,2),1}$}}};
\end{scope}

\begin{scope}[xshift=3cm,scale=1]

\node [style=vertex] (1) at (0,0.5) {};
\node [style=vertex] (2) at (1,0.5) {};
\node [style=vertex] (3) at (0.5,0.85) {};
\node [style=vertex] (4) at (0.5,1.5) {};
\node [style=vertex] (5) at (0.5,2) {};

\foreach \i/\j in {1/2,1/3,1/4,2/3,2/4,3/4} \draw [style=edge] (\i) to (\j);
\node at (0.5,0) {\parbox{0.3\linewidth}{\subcaption*{$o_{(2,2),2}$}}};
\end{scope}

\begin{scope}[xshift=5.5cm,scale=1]

\node [style=vertex] (1) at (0,0.5) {};
\node [style=vertex] (2) at (1,0.5) {};
\node [style=vertex] (3) at (0,1) {};
\node [style=vertex] (4) at (1,1) {};
\node [style=vertex] (5) at (0.5,1.75) {};

\foreach \i/\j in {1/2,3/4,3/5,4/5} \draw [style=edge] (\i) to (\j);
\node at (0.5,0) {\parbox{0.3\linewidth}{\subcaption*{$o_{(2,2),3}$}}};
\end{scope}

\end{tikzpicture}
\end{center}
\end{subfigure}

%\setlength{\abovecaptionskip}{-15pt}
\caption{Algunas obstrucciones mínimas de la clase $(2,2)$-$M_2$.}
\label{obsts_2_2_M2}
\end{figure}

Las obstrucciones mínimas de la clase $(2,2)$-$M_2$, ilustradas en la Figura \ref{obsts_2_2_M2} se pueden expresar de la siguiente manera:
\begin{itemize}
    \item $o_{(2,2),1}=K_5$.
    \item $o_{(2,2),2}=K_1+K_4$.
    \item $o_{(2,2),3}=K_2+K_3$.
\end{itemize}

\begin{figure}[ht!]

\begin{subfigure}{\textwidth}
\begin{center}
\begin{tikzpicture}

\begin{scope}[xshift=0cm,scale=1]

\node [style=vertex] (1) at (0,0.5) {};
\node [style=vertex] (2) at (1,0.5) {};
\node [style=vertex] (3) at (0,1.5) {};
\node [style=vertex] (4) at (1,1.5) {};
\node [style=vertex] (5) at (0,2.5) {};
\node [style=vertex] (6) at (1,2.5) {};

\foreach \i/\j in {1/2,1/3,2/3,4/5,4/6,5/6} \draw [style=edge] (\i) to (\j);
\node at (0.5,0) {\parbox{0.3\linewidth}{\subcaption*{$o_{(2,3),1}$}}};
\end{scope}

\begin{scope}[xshift=2cm,scale=1]

\node [style=vertex] (1) at (0.5,0.5) {};
\node [style=vertex] (2) at (1.5,0.5) {};
\node [style=vertex] (3) at (0,1.25) {};
\node [style=vertex] (4) at (2,1.25) {};
\node [style=vertex] (5) at (0.5,2) {};
\node [style=vertex] (6) at (1.5,2) {};

\foreach \i/\j in {1/2,1/3,1/4,1/5,1/6,2/3,2/4,2/5,2/6,3/4,3/5,3/6,4/5,4/6,5/6} \draw [style=edge] (\i) to (\j);
\node at (1,0) {\parbox{0.3\linewidth}{\subcaption*{$o_{(2,3),2}$}}};
\end{scope}

\begin{scope}[xshift=5cm,scale=1]

\node [style=vertex] (1) at (0.25,0.5) {};
\node [style=vertex] (2) at (1.25,0.5) {};
\node [style=vertex] (3) at (0,1.15) {};
\node [style=vertex] (4) at (1.5,1.15) {};
\node [style=vertex] (5) at (0.75,1.75) {};
\node [style=vertex] (6) at (0.75,2.25) {};

\foreach \i/\j in {1/2,1/3,1/4,1/5,2/3,2/4,2/5,3/4,3/5,4/5} \draw [style=edge] (\i) to (\j);
\node at (0.75,0) {\parbox{0.3\linewidth}{\subcaption*{$o_{(2,3),3}$}}};
\end{scope}

\begin{scope}[xshift=7.5cm,scale=1]

\node [style=vertex] (1) at (0,0.5) {};
\node [style=vertex] (2) at (1,0.5) {};
\node [style=vertex] (3) at (0.5,0.85) {};
\node [style=vertex] (4) at (0.5,1.5) {};
\node [style=vertex] (5) at (0,2) {};
\node [style=vertex] (6) at (1,2) {};

\foreach \i/\j in {1/2,1/3,1/4,2/3,2/4,3/4,4/5,4/6,5/6} \draw [style=edge] (\i) to (\j);
\node at (0.5,0) {\parbox{0.3\linewidth}{\subcaption*{$o_{(2,3),4}$}}};
\end{scope}

\begin{scope}[xshift=9.5cm,scale=1]

\node [style=vertex] (1) at (0,0.5) {};
\node [style=vertex] (2) at (1,0.5) {};
\node [style=vertex] (3) at (0.5,0.85) {};
\node [style=vertex] (4) at (0.5,1.5) {};
\node [style=vertex] (5) at (0,2) {};
\node [style=vertex] (6) at (1,2) {};

\foreach \i/\j in {1/2,1/3,1/4,2/3,2/4,3/4,5/6} \draw [style=edge] (\i) to (\j);
\node at (0.5,0) {\parbox{0.3\linewidth}{\subcaption*{$o_{(2,3),5}$}}};
\end{scope}

\end{tikzpicture}
\end{center}
\end{subfigure}

%\setlength{\abovecaptionskip}{-15pt}
\caption{Algunas obstrucciones mínimas de la clase $(2,3)$-$M_2$.}
\label{obsts_2_3_M2}
\end{figure}

Las obstrucciones mínimas de la clase $(2,3)$-$M_2$, ilustradas en la Figura \ref{obsts_2_3_M2}, se pueden expresar de la siguiente manera:
\begin{itemize}
    \item $o_{(2,3),1}=2K_3$.
    \item $o_{(2,3),2}=K_6$.
    \item $o_{(2,3),3}=K_1+K_5$.
    \item $o_{(2,3),4}=K_1\oplus(K_2+K_3)$.
    \item $o_{(2,3),5}=K_2+K_4$.
\end{itemize}

\begin{figure}[ht!]

\begin{subfigure}{\textwidth}
\begin{center}
\begin{tikzpicture}

\begin{scope}[xshift=0cm,scale=1]

\node [style=vertex] (1) at (0,0.5) {};
\node [style=vertex] (2) at (1,0.5) {};
\node [style=vertex] (3) at (0,1.5) {};
\node [style=vertex] (4) at (1,1.5) {};
\node [style=vertex] (5) at (0,2.5) {};
\node [style=vertex] (6) at (1,2.5) {};

\foreach \i/\j in {1/2,1/3,2/3,4/5,4/6,5/6} \draw [style=edge] (\i) to (\j);
\node at (0.5,0) {\parbox{0.3\linewidth}{\subcaption*{$o_{(2,4),1}$}}};
\end{scope}

\begin{scope}[xshift=2.5cm,scale=1]

\node [style=vertex] (1) at (0.625,0.5) {};
\node [style=vertex] (2) at (1.375,0.5) {};
\node [style=vertex] (3) at (0,1) {};
\node [style=vertex] (4) at (2,1) {};
\node [style=vertex] (5) at (0,1.75) {};
\node [style=vertex] (6) at (2,1.75) {};
\node [style=vertex] (7) at (0.25,2.5) {};
\node [style=vertex] (8) at (1.75,2.5) {};
\node [style=vertex] (9) at (1,2.75) {};

\foreach \i/\j in {1/3,1/4,1/5,1/6,1/7,1/8,1/9,2/4,2/5,2/6,2/7,2/8,2/9,3/4,3/5,3/6,3/7,3/8,3/9,4/5,4/6,4/7,4/9,5/6,5/8,6/7,6/9,7/8,7/9,8/9} \draw [style=edge] (\i) to (\j);
\node at (1,0) {\parbox{0.3\linewidth}{\subcaption*{$o_{(2,4),2}$}}};
\end{scope}

\begin{scope}[xshift=6cm,scale=1]

\node [style=vertex] (1) at (0.5,0.5) {};
\node [style=vertex] (2) at (1.5,0.5) {};
\node [style=vertex] (3) at (0,1) {};
\node [style=vertex] (4) at (2,1) {};
\node [style=vertex] (5) at (0,2) {};
\node [style=vertex] (6) at (2,2) {};
\node [style=vertex] (7) at (1,2.5) {};

\foreach \i/\j in {1/2,1/3,1/4,1/5,1/6,1/7,2/3,2/4,2/5,2/6,2/7,3/4,3/5,3/6,3/7,4/5,4/6,4/7,5/6,5/7,6/7} \draw [style=edge] (\i) to (\j);
\node at (1,0) {\parbox{0.3\linewidth}{\subcaption*{$o_{(2,4),3}$}}};
\end{scope}

\begin{scope}[xshift=9.5cm,scale=1]

\node [style=vertex] (4) at (0.25,1.75) {};
\node [style=vertex] (5) at (1.25,1.75) {};
\node [style=vertex] (2) at (0,1.10) {};
\node [style=vertex] (3) at (1.5,1.10) {};
\node [style=vertex] (1) at (0.75,0.5) {};
\node [style=vertex] (6) at (0.25,2.75) {};
\node [style=vertex] (7) at (1.25,2.75) {};

\foreach \i/\j in {1/2,1/3,1/4,1/5,2/3,2/4,2/5,3/4,3/5,4/5,4/6,4/7,5/6,5/7,6/7} \draw [style=edge] (\i) to (\j);
\node at (0.75,0) {\parbox{0.3\linewidth}{\subcaption*{$o_{(2,4),4}$}}};
\end{scope}

\end{tikzpicture}
\end{center}
\end{subfigure}

\begin{subfigure}{\textwidth}
\begin{center}
\begin{tikzpicture}

\begin{scope}[xshift=0cm,scale=1]

\node [style=vertex] (1) at (0.5,0.5) {};
\node [style=vertex] (2) at (1.5,0.5) {};
\node [style=vertex] (3) at (0,1.25) {};
\node [style=vertex] (4) at (2,1.25) {};
\node [style=vertex] (5) at (0.5,2) {};
\node [style=vertex] (6) at (1.5,2) {};
\node [style=vertex] (7) at (1,2.5) {};

\foreach \i/\j in {1/2,1/3,1/4,1/5,1/6,2/3,2/4,2/5,2/6,3/4,3/5,3/6,4/5,4/6,5/6} \draw [style=edge] (\i) to (\j);
\node at (1,0) {\parbox{0.3\linewidth}{\subcaption*{$o_{(2,4),5}$}}};
\end{scope}

\begin{scope}[xshift=3.5cm,scale=1]

\node [style=vertex] (1) at (0.25,0.5) {};
\node [style=vertex] (2) at (1.25,0.5) {};
\node [style=vertex] (3) at (0,1.15) {};
\node [style=vertex] (4) at (1.5,1.15) {};
\node [style=vertex] (5) at (0.75,1.75) {};
\node [style=vertex] (6) at (0.25,2.5) {};
\node [style=vertex] (7) at (1.25,2.5) {};

\foreach \i/\j in {1/2,1/3,1/4,1/5,2/3,2/4,2/5,3/4,3/5,4/5,5/6,5/7,6/7} \draw [style=edge] (\i) to (\j);
\node at (0.75,0) {\parbox{0.3\linewidth}{\subcaption*{$o_{(2,4),6}$}}};
\end{scope}

\begin{scope}[xshift=6.5cm,scale=1]

\node [style=vertex] (1) at (0.25,0.5) {};
\node [style=vertex] (2) at (1.25,0.5) {};
\node [style=vertex] (3) at (0,1.15) {};
\node [style=vertex] (4) at (1.5,1.15) {};
\node [style=vertex] (5) at (0.75,1.75) {};
\node [style=vertex] (6) at (0.25,2.5) {};
\node [style=vertex] (7) at (1.25,2.5) {};

\foreach \i/\j in {1/2,1/3,1/4,1/5,2/3,2/4,2/5,3/4,3/5,4/5,6/7} \draw [style=edge] (\i) to (\j);
\node at (0.75,0) {\parbox{0.3\linewidth}{\subcaption*{$o_{(2,4),7}$}}};
\end{scope}

\end{tikzpicture}
\end{center}
\end{subfigure}

%\setlength{\abovecaptionskip}{-15pt}
\caption{Algunas obstrucciones mínimas de la clase $(2,4)$-$M_2$.}
\label{obsts_2_4_M2}
\end{figure}

Las obstrucciones mínimas de la clase $(2,4)$-$M_2$, ilustradas en la Figura \ref{obsts_2_4_M2}, se pueden expresar de la siguiente manera:
\begin{itemize}
    \item $o_{(2,4),1}=2K_3$.
    \item $o_{(2,4),2}=\overline{P_3}\oplus\overline{P_3}\oplus\overline{P_3}$.
    \item $o_{(2,4),3}=K_7$.
    \item $o_{(2,4),4}=K_2\oplus(K_2+K_3)$.
    \item $o_{(2,4),5}=K_1+K_6$.
    \item $o_{(2,4),6}=K_1\oplus(K_2+K_4)$.
    \item $o_{(2,4),7}=K_2+K_5$.
\end{itemize}


Las obstrucciones mínimas de la clase $(2,5)$-$M_2$ encontradas con el Algoritmo \ref{alg_obstrucciones_alfabeta} se pueden expresar de la siguiente manera:
\begin{itemize}
    \item $o_{(2,5),1}=2K_3$.
    \item $o_{(2,5),2}=\overline{P_3}\oplus\overline{P_3}\oplus\overline{P_3}$.
    \item $o_{(2,5),3}=\overline{P_3}\oplus(K_2+K_3)=(K_1+K_2)\oplus(K_2+K_3)$.
    \item $o_{(2,5),4}=K_8$.
    \item $o_{(2,5),5}=K_3\oplus(K_2+K_3)$.
    \item $o_{(2,5),6}=K_2\oplus(K_2+K_4)$.
    \item $o_{(2,5),7}=K_1+K_7$.
    \item $o_{(2,5),8}=K_1\oplus(K_2+K_5)$.
    \item $o_{(2,5),9}=K_2+K_6$.
\end{itemize}

Las obstrucciones mínimas de la clase $(2,6)$-$M_2$ encontradas con el Algoritmo \ref{alg_obstrucciones_alfabeta} se pueden expresar de la siguiente manera:
\begin{itemize}
    \item $o_{(2,6),1}=2K_3$.
    \item $o_{(2,6),2}=\overline{P_3}\oplus\overline{P_3}\oplus\overline{P_3}$.
    \item $o_{(2,6),3}=\overline{P_3}\oplus(K_2+K_3)=(K_1+K_2)\oplus(K_2+K_3)$.
    \item $o_{(2,6),4}=K_9$.
    \item $o_{(2,6),5}=K_4\oplus(K_2+K_3)$.
    \item $o_{(2,6),6}=K_3\oplus(K_2+K_4)$.
    \item $o_{(2,6),7}=K_2\oplus(K_2+K_5)$.
    \item $o_{(2,6),8}=K_1+K_8$.
    \item $o_{(2,6),9}=K_1\oplus(K_2+K_6)$.
    \item $o_{(2,6),10}=K_2+K_7$.
\end{itemize}

Las obstrucciones mínimas de la clase $(2,7)$-$M_2$ encontradas con el Algoritmo \ref{alg_obstrucciones_alfabeta} se pueden expresar de la siguiente manera:
\begin{itemize}
    \item $o_{(2,7),1}=2K_3$.
    \item $o_{(2,7),2}=\overline{P_3}\oplus\overline{P_3}\oplus\overline{P_3}$.
    \item $o_{(2,7),3}=\overline{P_3}\oplus(K_2+K_3)=(K_1+K_2)\oplus(K_2+K_3)$.
    \item $o_{(2,7),4}=K_{10}$.
    \item $o_{(2,7),5}=K_5\oplus(K_2+K_3)$.
    \item $o_{(2,7),6}=K_4\oplus(K_2+K_4)$.
    \item $o_{(2,7),7}=K_3\oplus(K_2+K_5)$.
    \item $o_{(2,7),8}=K_2\oplus(K_2+K_6)$.
    \item $o_{(2,7),9}=K_1+K_9$.
    \item $o_{(2,7),10}=K_1\oplus(K_2+K_7)$.
    \item $o_{(2,7),11}=K_2+K_8$.
\end{itemize}

Como podemos ver, la gráfica $2K_3$ es una obstrucción mínima de las clases $(2,\beta)$-$M_2$ para $3\le\beta\le7$, la gráfica $\overline{P_3}\oplus\overline{P_3}\oplus\overline{P_3}$ es una obstrucción mínima de las clases $(2,\beta)$-$M_2$ para $4\le\beta\le7$ y la gráfica $\overline{P_3}\oplus(K_2+K_3)$ es una obstrucción mínima de las clases $(2,\beta)$-$M_2$ para $5\le\beta\le7$. Esto nos lleva a pensar que estas tres gráficas pueden ser obstrucciones mínimas de la clase $(2,\infty)$-$M_2$. Además, podemos identificar algunas posibles familias de obstrucciones para cualquier clase $(2,\beta)$-$M_2$.  

%Éstas familias son bastante fáciles de identificar. También es fácil demostrar que son familias de obstrucciones mínimas. Tal vez podamos hacerlo al final si nos queda tiempo.

\subsubsection{Obstrucciones de algunas otras clases  $(\alpha,\beta)$-$M_2$}

%Tengo las obstrucciones mínimas de las clases (3,3), (3,4), (3,5) y (4,4). Se pueden generar más. Tal vez podemos incluirlas en los apéndices.

\section{Particiones en más de dos partes}
    
En esta sección tomamos como base los resultados obtenidos en nuestro estudio de la clase $M_2$ para estudiar a las clases $M_i$ para enteros $i$ mayores a dos. Empezamos por caracterizar a la clase $M_3$ a través de su conjunto de obstrucciones mínimas. El conjunto de obstrucciones mínimas de la clase $M_2$ y el de la clase $M_3$ nos ayudan a distinguir dos familias de obstrucciones mínimas para cualquier clase $M_i$.  Posteriormente presentamos dos familias de obstrucciones mínimas para cualquier clase $M_i$.

    \subsection{Obstrucciones mínimas de la clase $M_3$}
        \begin{theorem} \label{teo_obsts_m2}

    Para una cográfica $G$, las siguientes afirmaciones son equivalentes.
    \begin{enumerate}[(a)]
        \item $G \in M_3$.
        \item $G$ no contiene a ninguna de las gráficas de las Figuras \ref{obsts_O_M3} como subgráfica inducida.
    \end{enumerate}

\end{theorem}

\begin{figure}[ht!]
\begin{subfigure}{\textwidth}
\begin{center}
\begin{tikzpicture}
\begin{scope}[xshift=0cm,scale=1]
%K4
\node [style=vertex] (1) at (0,1) {};
\node [style=vertex] (2) at (1,1) {};
\node [style=vertex] (3) at (0.5,1.3) {};
\node [style=vertex] (4) at (0.5,1.75) {};
%K3
\node [style=vertex] (5) at (0,2.25) {};
\node [style=vertex] (6) at (1,2.25) {};
\node [style=vertex] (7) at (0.5,3) {};
%K2
\node [style=vertex] (8) at (0,3.5) {};
\node [style=vertex] (9) at (1,3.5) {};
%K1
\node [style=vertex] (10) at (0.5,4) {};

\foreach \i/\j in {1/2,1/3,1/4,2/3,2/4,3/4,5/6,5/7,6/7,8/9} \draw [style=edge] (\i) to (\j);
\node at (0.5,0) {\parbox{0.3\linewidth}{\subcaption*{$O_{3,1}$}}};
\end{scope}

\begin{scope}[xshift=2.5cm,scale=1]
%K4
\node [style=vertex] (1) at (0,1) {};
\node [style=vertex] (2) at (1,1) {};
\node [style=vertex] (3) at (0.5,1.3) {};
\node [style=vertex] (4) at (0.5,1.75) {};
%K3
\node [style=vertex] (5) at (0,2.25) {};
\node [style=vertex] (6) at (1,2.25) {};
\node [style=vertex] (7) at (0.5,3) {};
%K2
\node [style=vertex] (8) at (0,4) {};
\node [style=vertex] (9) at (1,4) {};
%K1
\node [style=vertex] (10) at (0.5,3.5) {};

\foreach \i/\j in {1/2,1/3,1/4,2/3,2/4,3/4,5/6,5/7,6/7,8/9} \draw [style=edge] (\i) to (\j);
\foreach \i/\j in {7/10} \draw [style=edge] (\i) to (\j);
\node at (0.5,0) {\parbox{0.3\linewidth}{\subcaption*{$O_{3,2}$}}};
\end{scope}

\begin{scope}[xshift=5cm,scale=1]
%K4
\node [style=vertex] (1) at (0,1) {};
\node [style=vertex] (2) at (1,1) {};
\node [style=vertex] (3) at (0.5,1.3) {};
\node [style=vertex] (4) at (0.5,1.75) {};
%K3
\node [style=vertex] (5) at (0,2.75) {};
\node [style=vertex] (6) at (1,2.75) {};
\node [style=vertex] (7) at (0.5,3.5) {};
%K2
\node [style=vertex] (8) at (0,4) {};
\node [style=vertex] (9) at (1,4) {};
%K1
\node [style=vertex] (10) at (0.5,2.25) {};

\foreach \i/\j in {1/2,1/3,1/4,2/3,2/4,3/4,5/6,5/7,6/7,8/9} \draw [style=edge] (\i) to (\j);
\foreach \i/\j in {4/10} \draw [style=edge] (\i) to (\j);
\node at (0.5,0) {\parbox{0.3\linewidth}{\subcaption*{$O_{3,3}$}}};
\end{scope}

\begin{scope}[xshift=7.5cm,scale=1]
%K4
\node [style=vertex] (1) at (0,1) {};
\node [style=vertex] (2) at (1,1) {};
\node [style=vertex] (3) at (0.5,1.3) {};
\node [style=vertex] (4) at (0.5,1.75) {};
%K3
\node [style=vertex] (5) at (0,2.75) {};
\node [style=vertex] (6) at (1,2.75) {};
\node [style=vertex] (7) at (0.5,3.5) {};
%K2
\node [style=vertex] (8) at (0,4) {};
\node [style=vertex] (9) at (1,4) {};
%K1
\node [style=vertex] (10) at (0.5,2.25) {};

\foreach \i/\j in {1/2,1/3,1/4,2/3,2/4,3/4,5/6,5/7,6/7,8/9} \draw [style=edge] (\i) to (\j);
\foreach \i/\j in {4/10,1/10} \draw [style=edge] (\i) to (\j);
\node at (0.5,0) {\parbox{0.3\linewidth}{\subcaption*{$O_{3,4}$}}};
\end{scope}

\begin{scope}[xshift=10cm,scale=1]
%K4
\node [style=vertex] (1) at (0,1) {};
\node [style=vertex] (2) at (1,1) {};
\node [style=vertex] (3) at (0.5,1.3) {};
\node [style=vertex] (4) at (0.5,1.75) {};
%K3
\node [style=vertex] (5) at (0,2.75) {};
\node [style=vertex] (6) at (1,2.75) {};
\node [style=vertex] (7) at (0.5,3.5) {};
%K2
\node [style=vertex] (8) at (0,2.25) {};
\node [style=vertex] (9) at (1,2.25) {};
%K1
\node [style=vertex] (10) at (0.5,4) {};

\foreach \i/\j in {1/2,1/3,1/4,2/3,2/4,3/4,5/6,5/7,6/7,8/9} \draw [style=edge] (\i) to (\j);
\foreach \i/\j in {4/8,4/9} \draw [style=edge] (\i) to (\j);
\node at (0.5,0) {\parbox{0.3\linewidth}{\subcaption*{$O_{3,5}$}}};
\end{scope}
\end{tikzpicture}
\end{center}
\end{subfigure}

\begin{subfigure}{\textwidth}
\begin{center}
\begin{tikzpicture}

\begin{scope}[xshift=0cm,scale=1]
%K4
\node [style=vertex] (1) at (0,1) {};
\node [style=vertex] (2) at (1,1) {};
\node [style=vertex] (3) at (0.5,1.3) {};
\node [style=vertex] (4) at (0.5,1.75) {};
%K3
\node [style=vertex] (5) at (0,3.25) {};
\node [style=vertex] (6) at (1,3.25) {};
\node [style=vertex] (7) at (0.5,4) {};
%K2
\node [style=vertex] (8) at (0.5,2.5) {};
\node [style=vertex] (9) at (1,2.5) {};
%K1
\node [style=vertex] (10) at (0,2.5) {};

\foreach \i/\j in {1/2,1/3,1/4,2/3,2/4,3/4,5/6,5/7,6/7,8/9} \draw [style=edge] (\i) to (\j);
\foreach \i/\j in {4/8,4/9,4/10} \draw [style=edge] (\i) to (\j);
\node at (0.5,0) {\parbox{0.3\linewidth}{\subcaption*{$O_{3,6}$}}};
\end{scope}

\begin{scope}[xshift=2.5cm,scale=1]
%K4
\node [style=vertex] (1) at (0,1) {};
\node [style=vertex] (2) at (1,1) {};
\node [style=vertex] (3) at (0.5,1.3) {};
\node [style=vertex] (4) at (0.5,1.75) {};
%K3
\node [style=vertex] (5) at (0,3.25) {};
\node [style=vertex] (6) at (1,3.25) {};
\node [style=vertex] (7) at (0.5,4) {};
%K2
\node [style=vertex] (8) at (0.5,2.5) {};
\node [style=vertex] (9) at (1,2.5) {};
%K1
\node [style=vertex] (10) at (0,2.5) {};

\foreach \i/\j in {1/2,1/3,1/4,2/3,2/4,3/4,5/6,5/7,6/7,8/9} \draw [style=edge] (\i) to (\j);
\foreach \i/\j in {4/8,4/9,4/10,1/10} \draw [style=edge] (\i) to (\j);
\node at (0.5,0) {\parbox{0.3\linewidth}{\subcaption*{$O_{3,7}$}}};
\end{scope}

\begin{scope}[xshift=5cm,scale=1]
%K4
\node [style=vertex] (1) at (0,1) {};
\node [style=vertex] (2) at (1,1) {};
\node [style=vertex] (3) at (0.5,1.3) {};
\node [style=vertex] (4) at (0.5,1.75) {};
%K3
\node [style=vertex] (5) at (0,2.75) {};
\node [style=vertex] (6) at (1,2.75) {};
\node [style=vertex] (7) at (0.5,3.5) {};
%K2
\node [style=vertex] (8) at (0,2.25) {};
\node [style=vertex] (9) at (1,2.25) {};
%K1
\node [style=vertex] (10) at (0.5,4) {};

\foreach \i/\j in {1/2,1/3,1/4,2/3,2/4,3/4,5/6,5/7,6/7,8/9} \draw [style=edge] (\i) to (\j);
\foreach \i/\j in {4/8,4/9,7/10} \draw [style=edge] (\i) to (\j);
\node at (0.5,0) {\parbox{0.3\linewidth}{\subcaption*{$O_{3,5}$}}};
\end{scope}

\end{tikzpicture}
\end{center}
\end{subfigure}

\setlength{\abovecaptionskip}{-15pt}
\caption{Obstrucciones mínimas para la clase $M_2$.}
\label{obsts_O_M3}
\end{figure}

\begin{proof}

\end{proof}


    \subsection{Familia $O$ de obstrucciones}
        En esta subsección presentamos a la familia $O$ de obstrucciones mínimas para cualquier clase $M_i$.

\subsubsection{$O$-componentes}

Sean $G$ una cográfica y $l$ un entero mayor o igual a cero, decimos que $G$ es una \textbf{\emph{$O$-componente}} si su conjunto de vértices acepta una partición $(A,B_1,B_2,\dots,B_l)$ tal que, para cualesquiera dos enteros $1\le i,j \le l$, se cumplen las siguientes condiciones :

\begin{itemize}
    \item $G[A]$ es una gráfica completa con $m_0$ vértices.
    \item $G[B_i]$ es una gráfica completa con $m_i$ vértices en donde $1\le m_i<m_0-1$.
    \item Si $i\neq j$, entonces $m_i\neq m_j$.
    \item Cada uno de los vértices de $B_i$ es adyacente a los mismos $n_i$ vértices de $A$, con $0<n_i<m_0-m_i$.
    \item Si $i\neq j$, entonces ningún vértice de $B_i$ es adyacente a algún vértice de $B_j$.
\end{itemize}

Decimos que $(A,B_1,B_2,\dots,B_l)$ es una $O$-partición de $G$ y que $G$ incluye a la gráfica completa $K_m$ si $m$ es igual a $m_k$ para algún $0\le k\le l$. Decimos que $G[A]$ es el cuerpo de $G$ y que, para cualquier $0\le k\le l$, $G[B_k]$ es una extremidad de $G$.

Sea $G$ una $O$-componente y $(A,B_1,B_2,\dots,B_l)$ una $O$-partición de $G$, notemos que el cuerpo de $G$ es necesariamente el clan más grande en $G$.

En la Figura \ref{fig_ejemplos_O} podemos ver algunos ejemplos de gráficas que son $O$-componentes y otros ejemplos de gráficas que no son $O$-componentes. Las gráficas $G_1$, $G_2$, $G_3$ y $G_4$ son $O$-componentes. La gráfica $G_1$ acepta la $O$-partición $(\{a,b,c\},\{d\})$. La gráfica $G_2$ acepta la $O$-partición $(\{a,b,c,d\},\{e\})$. La gráfica $G_3$ acepta la $O$-partición $(\{a,b,c,d\},\{e\})$. La gráfica $G_4$ acepta la $O$-partición $(\{a,b,c,d\},\{e\},\{f,g\})$. 

Por el contrario, las gráficas $G_5$, $G_6$, $G_7$ y $G_8$ no son $O$-componentes. La gráfica $G_5$ tiene dos extremidades del mismo orden. El cuerpo de la gráfica $G_6$ es un $K_3$, pero ésta tiene una extremidad de orden mayor a  1. La gráfica $G_7$ tiene una extremidad que se conecta a más de 2 vértices del cuerpo de $G$. La gráfica $G_8$ cumple con las características listadas, pero los vértices $e$, $a$, $d$ y $f$ forman un $P_4$, por lo que $G_8$ no es una cográfica, y por lo tanto, tampoco es una $O$-componente. Esto nos lleva a realizar la siguiente observación.

Sean $G$ una $O$-componente, $l$ un  entero mayor o igual a 0, $i,j$ enteros diferentes entre sí tales que $1\le i,j \le l$ , $(A,B_2,\dots,B_l)$ una $O$-partición de $G$, $x$ un vértice de $B_i$ y $y$ un vértice de $B_j$, notemos que no existen vértices diferentes $v,w\in A$ tales que $v$ es adyacente a los vértices en $B_i$, pero no a los vértices en $B_j$ y $w$ es adyacente a los vértices en $B_j$, pero no a los vértices en $B_i$, ya que, de ser así, $x$, $v$, $w$ y $y$ formarían un $P_4$. De esto se sigue que si $|B_i| < |B_j|$, los vértices en $B_i$ serán adyacentes a cada uno de los vértices de $A$ a los que sean adyacentes los vértices de $B_j$. 

\begin{figure}[ht!]
\begin{subfigure}{\textwidth}
\begin{center}
\begin{tikzpicture}

\begin{scope}[xshift=0cm,scale=1]

\node [style=vertex] (1) at (0,0.5) {};
\node [style=vertex] (2) at (1.5,0.5) {};
\node [style=vertex] (3) at (0.75,1.75) {};
\node [style=vertex] (4) at (0.75,2.5) {};

\node at (-0.25,0.5) {$a$};
\node at (1.75,0.5) {$b$};
\node at (1,1.75) {$c$};
\node at (1,2.5) {$d$};

\foreach \i/\j in {1/2,1/3,2/3,3/4} \draw [style=edge] (\i) to (\j);
\node at (0.75,0) {\parbox{0.3\linewidth}{\subcaption*{$G_1$}}};
\end{scope}

\begin{scope}[xshift=3cm,scale=1]

\node [style=vertex] (1) at (0,0.5) {};
\node [style=vertex] (2) at (1.5,0.5) {};
\node [style=vertex] (3) at (0.75,1) {};
\node [style=vertex] (4) at (0.75,1.75) {};
\node [style=vertex] (5) at (0.75,2.5) {};

\node at (-0.25,0.5) {$a$};
\node at (1.75,0.5) {$b$};
\node at (0.75,0.75) {$c$};
\node at (1,1.75) {$d$};
\node at (1,2.5) {$e$};

\foreach \i/\j in {1/2,1/3,1/4,2/3,2/4,3/4,4/5} \draw [style=edge] (\i) to (\j);
\node at (0.75,0) {\parbox{0.3\linewidth}{\subcaption*{$G_2$}}};
\end{scope}

\begin{scope}[xshift=6cm,scale=1]

\node [style=vertex] (1) at (0,0.5) {};
\node [style=vertex] (2) at (1.5,0.5) {};
\node [style=vertex] (3) at (0.75,1) {};
\node [style=vertex] (4) at (0.75,1.75) {};
\node [style=vertex] (5) at (0.75,2.5) {};

\node at (-0.25,0.5) {$a$};
\node at (1.75,0.5) {$b$};
\node at (0.75,0.75) {$c$};
\node at (1,1.75) {$d$};
\node at (1,2.5) {$e$};

\foreach \i/\j in {1/2,1/3,1/4,1/5,2/3,2/4,3/4,4/5} \draw [style=edge] (\i) to (\j);
\node at (0.75,0) {\parbox{0.3\linewidth}{\subcaption*{$G_2$}}};
\end{scope}

\begin{scope}[xshift=9cm,scale=1]

\node [style=vertex] (1) at (0,0.5) {};
\node [style=vertex] (2) at (1.5,0.5) {};
\node [style=vertex] (3) at (0.75,1) {};
\node [style=vertex] (4) at (0.75,1.75) {};
\node [style=vertex] (5) at (0,2.5) {};
\node [style=vertex] (6) at (0.75,2.5) {};
\node [style=vertex] (7) at (1.5,2.5) {};

\node at (-0.25,0.5) {$a$};
\node at (1.75,0.5) {$b$};
\node at (0.75,0.75) {$c$};
\node at (1,1.75) {$d$};
\node at (-0.25,2.5) {$e$};
\node at (0.5,2.5) {$f$};
\node at (1.75,2.5) {$g$};

\foreach \i/\j in {1/2,1/3,1/4,2/3,2/4,3/4,4/5,4/6,4/7,6/7} \draw [style=edge] (\i) to (\j);
\node at (0.75,0) {\parbox{0.3\linewidth}{\subcaption*{$G_4$}}};
\end{scope}

\end{tikzpicture}
\end{center}
\end{subfigure}

\begin{subfigure}{\textwidth}
\begin{center}
\begin{tikzpicture}

\begin{scope}[xshift=0cm,scale=1]

\node [style=vertex] (1) at (0,0.5) {};
\node [style=vertex] (2) at (1.5,0.5) {};
\node [style=vertex] (3) at (0.75,1.75) {};
\node [style=vertex] (4) at (0,2.5) {};
\node [style=vertex] (5) at (1.5,2.5) {};

\node at (-0.25,0.5) {$a$};
\node at (1.75,0.5) {$b$};
\node at (1,1.75) {$c$};
\node at (-0.25,2.5) {$d$};
\node at (1.75,2.5) {$e$};

\foreach \i/\j in {1/2,1/3,2/3,3/4,3/5} \draw [style=edge] (\i) to (\j);
\node at (0.75,0) {\parbox{0.3\linewidth}{\subcaption*{$G_5$}}};
\end{scope}

\begin{scope}[xshift=3cm,scale=1]

\node [style=vertex] (1) at (0,0.5) {};
\node [style=vertex] (2) at (1.5,0.5) {};
\node [style=vertex] (3) at (0.75,1.75) {};
\node [style=vertex] (4) at (0,2.5) {};
\node [style=vertex] (5) at (1.5,2.5) {};

\node at (-0.25,0.5) {$a$};
\node at (1.75,0.5) {$b$};
\node at (1,1.75) {$c$};
\node at (-0.25,2.5) {$d$};
\node at (1.75,2.5) {$e$};

\foreach \i/\j in {1/2,1/3,2/3,3/4,3/5,4/5} \draw [style=edge] (\i) to (\j);
\node at (0.75,0) {\parbox{0.3\linewidth}{\subcaption*{$G_6$}}};
\end{scope}

\begin{scope}[xshift=6cm,scale=1]

\node [style=vertex] (1) at (0,0.5) {};
\node [style=vertex] (2) at (1.5,0.5) {};
\node [style=vertex] (3) at (0.75,1) {};
\node [style=vertex] (4) at (0.75,1.75) {};
\node [style=vertex] (5) at (0.75,2.5) {};

\node at (-0.25,0.5) {$a$};
\node at (1.75,0.5) {$b$};
\node at (0.75,0.75) {$c$};
\node at (1.25,1.75) {$d$};
\node at (1,2.5) {$e$};

\foreach \i/\j in {1/2,1/3,1/4,1/5,2/3,2/4,2/5,3/4,4/5} \draw [style=edge] (\i) to (\j);
\node at (0.75,0) {\parbox{0.3\linewidth}{\subcaption*{$G_7$}}};
\end{scope}

\begin{scope}[xshift=9cm,scale=1]

\node [style=vertex] (1) at (0,0.5) {};
\node [style=vertex] (2) at (1.5,0.5) {};
\node [style=vertex] (3) at (0.75,1) {};
\node [style=vertex] (4) at (0.75,1.75) {};
\node [style=vertex] (5) at (0,2.5) {};
\node [style=vertex] (6) at (0.75,2.5) {};
\node [style=vertex] (7) at (1.5,2.5) {};

\node at (-0.25,0.5) {$a$};
\node at (1.75,0.5) {$b$};
\node at (0.75,0.75) {$c$};
\node at (1,1.75) {$d$};
\node at (-0.25,2.5) {$e$};
\node at (0.5,2.5) {$f$};
\node at (1.75,2.5) {$g$};

\foreach \i/\j in {1/2,1/3,1/4,1/5,2/3,2/4,3/4,4/6,4/7,6/7} \draw [style=edge] (\i) to (\j);
\node at (0.75,0) {\parbox{0.3\linewidth}{\subcaption*{$G_8$}}};
\end{scope}

\end{tikzpicture}
\end{center}
\end{subfigure}

%\setlength{\abovecaptionskip}{-15pt}
\caption{Ejemplos de $O$-componentes ($G_1$,$G_2$,$G_3$ y $G_4$) y de gráficas que no son $O$-componentes ($G_5$,$G_6$,$G_7$ y $G_8$).}
\label{fig_ejemplos_O}
\end{figure}  

\begin{lemma}
Sea $G$ una $O$-componente, $G$ es una gráfica completa o $G$ no es una gráfica multipartita completa.
\end{lemma}

\begin{proof}
Sean $l$ un entero mayor o igual a cero y $(A,B_1,B_2,\dots,B_l)$ una $O$-partición de $G$. Si $l=0$, entonces $G$ es igual a $G[a]$, y por lo tanto es una gráfica completa. En el caso contrario, veamos que, dado un entero $1\le i\le l$, $A$ tiene al menos dos vértices que no son adyacentes a ningún $B_i$. Sea $v\in B_i$, sabemos por la definición de $O$-componente que $v$ es adyacente a máximo $|A|-|B_i|-1$ vértices de $A$. Como $i$ es mayor o igual a 1, entonces $v$ es adyacente a lo más a $|A|-2$ vértices de $A$. Luego, existen dos vértices $w,x\in A$ tales que no son adyacentes a $v$, pero que sí son adyacentes el uno al otro. Así, $v$, $w$ y $x$ forman un $\overline{P_3}$. De esto se sigue que $G$ no es una gráfica multipartita completa.
\end{proof}

\subsubsection{$O$-obstrucciones}

Sean $G$ una cográfica y $n,m$ enteros mayores o iguales a uno, decimos que $G$ es una \emph{\textbf{$O_n$-obstrucción}} si $G$ es la unión ajena de algunas $O$-componentes $G_1,G_2,\dots G_m$ tales que, para cualesquiera enteros $1\le i \le m$ y $1\le j \le n+1$, existe una única $O$-componente $G_i$ que contiene a $K_j$. 

Sea $n$ un entero mayor o igual a uno, la \emph{\textbf{familia $O_n$ de obstrucciones}}, denotada simplemente por $O_n$, es el conjunto de todas las $O_n$-obstrucciones.

\begin{theorem}
Sean $n$ un entero mayor o igual a uno y $G$ una cográfica. Si $G$ es una $O_n$ obstrucción, entonces $G$ es una obstruccción mínima de la clase $M_n$.
\end{theorem}

\begin{proof}
Mostremos primero que $G$ no está en la clase $M_n$. Procedamos por inducción sobre $n$.

\emph{Caso base}: $n=1$.

Tenemos que la única $O_1$-obstrucción que hay es $K_1+K_2$. Luego, $G=K_1+K_2=\overline{P_3}$. Como $\overline{P_3}$ es la única obstrucción de las gráficas multipartitas completas, se sigue que $G\notin M_1$.

\emph{Paso inductivo}: $n>1$.

Supongamos como hipótesis inductiva que si una cográfica $G'$ es una $O_{n-1}$-obstrucción, entonces $G'$ es una obstrucción mínima de la clase $M_{n-1}$.

Sea $P = (A_1, A_2, \dots, A_n)$ una partición de los vértices de $G$, veamos que $P$ no es una $M_n$-partición. Recordemos que, como $G$ es una $O_n$-obstrucción, ésta es la unión ajena de hasta $n+1$ $O$-componentes. Y que exactamente una de estas $O$-componentes $H$ contiene a $K_{n+1}$. Como $G$ tiene a $K_{n+1}$ como subgráfica inducida, $G$ no es $n$-coloreable, y por lo tanto al menos dos vértices de dicho $K_{n+1}$ están en la misma parte. Supongamos sin pérdida de generalidad que $A_1$ contiene al menos dos vértices del $K_{n+1}$ de $G$. 

Como $H$ es una $O$-componente, ésta acepta una $O$-partición $Q=(C,B_1,B_2,\dots,B_i)$ para algún entero $0\le i \le n-2$. Notemos que $C$ es el conjunto de los vértices del $K_{n+1}$ de $H$. Denotemos por $m_j$ la cardinalidad de $B_j$ para todo $1\le j\le < i$. Supongamos sin pérdida de generalidad que $m_k < m_{k+1}$ para todo $1\le k\ < i$. Recordemos que , $A_1$ contiene al menos dos vértices que están en $C$ y que hay al menos dos vértices de $C$ que no son adyacentes a ningún vértice de $B_l$ para todo $1\le l\ \le i$. 

Si $A_1$ contiene sólo vértices de $C$, los vértices de $H-A_1$ inducen en $G$ la unión ajena de un conjunto de gráficas que tienen como subgráficas inducidas $O$-componentes que, en conjunto, contienen a las gráficas completas $G[B_1], G[B_2],\dots,G[B_i]$. Los vértices de $H-A_1$ y las $O$-componentes de $G$ diferentes de $H$ deben de repartirse entre las $n-1$ partes de $P$ diferentes de $A_1$. Notemos que $G-A_1$ tiene como subgráfica inducida a un conjunto de $O$-componentes que contienen todas las gráficas completas desde $K_1$ hasta $K_{n}$. Es decir que $G-A_1$ tiene como subgráfica inducida una $O_{n-1}$-obstrucción que, por hipótesis inductiva, no está en la clase $M_{n-1}$. Luego, $G$ no está en la clase $M_n$. En la Figura \ref{fig_dem_O_01} se muestra un ejemplo de esto.

\begin{figure}[ht!]

\begin{subfigure}{\textwidth}
\begin{center}
\begin{tikzpicture}
\begin{scope}[xshift=0cm,scale=1]
%K6
\node [style=vertex, fill=red] (1) at (1,0.5) {};
\node [style=vertex, fill=red] (2) at (2,0.5) {};
\node [style=vertex, fill=red] (3) at (0,1) {};
\node [style=vertex, fill=red] (4) at (3,1) {};
\node [style=vertex, fill=red] (5) at (1,1.5) {};
\node [style=vertex, fill=red] (6) at (2,1.5) {};

%K1
\node [style=vertex] (7) at (0,3) {};

%K2
\node [style=vertex] (8) at (1,3) {};
\node [style=vertex] (9) at (2,3) {};

%K3
\node [style=vertex] (10) at (2.5,2.5) {};
\node [style=vertex] (11) at (3.5,2) {};
\node [style=vertex] (12) at (3.25,2.5) {};

%K4
\node [style=vertex] (13) at (4,0.625) {};
\node [style=vertex] (14) at (4.75,0.625) {};
\node [style=vertex] (15) at (4,1.375) {};
\node [style=vertex] (16) at (4.75,1.375) {};

\foreach \i/\j in {1/2,1/3,1/4,1/5,1/6,2/3,2/4,2/5,2/6,3/4,3/5,3/6,4/5,4/6,5/6} \draw [style=edge] (\i) to (\j);
\foreach \i/\j in {7/3,7/4,7/5,7/6} \draw [style=edge] (\i) to (\j);
\foreach \i/\j in {8/4,8/5,8/6,8/9,9/4,9/5,9/6} \draw [style=edge] (\i) to (\j);
\foreach \i/\j in {10/11,10/12,11/12,10/4,10/6,11/4,11/6,12/4,12/6} \draw [style=edge] (\i) to (\j);
\foreach \i/\j in {13/14,13/15,13/16,14/15,14/16,15/16,13/4,14/4,15/4,16/4} \draw [style=edge] (\i) to (\j);

\node at (2.375,0) {\parbox{0.3\linewidth}{\subcaption*{$G$}}};
\end{scope}

\end{tikzpicture}
\end{center}
\end{subfigure}

%\setlength{\abovecaptionskip}{-15pt}
\caption{La gráfica $G$ es una $O$-componente que contiene a $K_1$, $K_2$, $K_3$, $K_4$ y $K_6$. En rojo se marcan los vértices que se agregan a $A_1$. Se puede apreciar que el resto de los vértices inducen la gráfica $K_1+K_2+K_3+K_4$.}
\label{fig_dem_O_01}
\end{figure}


Abordemos el caso en el que $A_1$ contiene, además de dos vértices de $C$, a lo más un vértice de al menos una parte de $Q$ diferente de $C$. Sean $j,k$ enteros tales que $A_1$ contiene un vértice de $B_j$ y un vértice de $B_k$. Si $A_1$ contiene un vértice $v$ de $C$ tal que $v$ es adyacente a los vértices de $B_j$ pero no es adyacente a los vértices de $B_k$, entonces el subconjunto de $V(A_1)$ que contiene a $v$, a un vértice de $B_j$ y a un vértice de $B_k$ induce un $\overline{P_3}$, con lo que $G[A_1]$ no es una gráfica multipartita completa y $P$ no es una $O$-partición. 

En el caso contrario, un elemento de $C$ está en $A_1$ si y sólo si es adyacente a todos los elementos de $A_1-C$ o, en su defecto, no es adyacente a ningún elemento de $A_1-C$. Si $A_1$ contiene dos elementos de $C$ que no son adyacentes a ningún elemento de $A_1-C$, entonces $G[A_1]$ no es una gráfica multipartita completa y por lo tanto $P$ no es una $M_n$-partición. 

Si todos los elementos de $C$ que están en $A_1$ menos a lo más uno son adyacentes a todos los vértices de $A_1-C$, veamos que $H-A_1$ tiene como subgráfica inducida a la unión ajena de algunas $O$-componentes que, en conjunto, contienen a $K_{m_l}$ para cualquier entero $1\le l\le i$ . Sea $m$ el máximo entero tal que $A_1$ tiene un vértice de $B_m$. Notemos que para cualquier entero $m < l \le i$, se cumple que $B_{l}\subset V(H)-A_1$, ya que no hay vértices de $B_l$ en $A_1$. Por otra parte, para cualquier entero $1 < l \le m$, se cumple que $B_{l}-A_1$ tiene como subgráfica inducida a la gráfica completa con $m_{l-1}$ vértices, ya que $m_l>m_{l-1}$ y a lo más uno de los vértices de $B_l$ está en $A_1$. Como ya hemos verificado que todas las gráficas completas con $m_1$, $m_2$, \dots, $m_i$ vértices menos la que tiene $|B_m|$ vértices son subgráficas inducidas de $H-A_1$ y no tienen vértices en común, sólo nos falta verificar que la gráfica completa con $|B_m|$ vértices también está en $H-A_1$, que no tiene vértices en común con las otras gráficas completas mencionadas y que éstas forman $O$-obstrucciones.

Dado que todos los vértices de $C$ que están en $A_1$ menos a lo más uno son adyacentes a todos los vértices de $A_1-C$ y $m$ es el mayor entero tal que $A_1$ tiene un vértice de $B_m$, se cumple que $A_1$ contiene a lo más $n-|B_m|+1$ vértices de $C$. Luego, $C-A_1$ debe de tener al menos $|B_m|$ vértices, por lo que $H-A_1$ contiene a la gráfica completa con $|B_m|$ vértices como subgráfica inducida. Como $C$ es ajeno a $B_l$ para cualquier entero $1\le l\le i$, entonces esta gráfica completa con $|B_m|$ vértices no tiene vértices en común con ninguna de las otras gráficas completas encontradas. Ahora veamos que estas gráficas completas forman un conjunto ajeno de $O$-obstrucciones.

Sean $C'$ un subconjunto de $C$ tal que contiene $|B_m|$ elementos de $C$ que no son adyacentes a elemento alguno de $B_m$, $B_l'$ un subconjunto de $B_{l+1}$ con $|B_l|$ elementos para cada entero $1\le l < m$, veamos que $Q'=(C',B_1', B_2', \dots, B_{m-1}')$ es una $O$-partición. Sabemos que $G[C']$ es una gráfica completa con $m$ vértices. De igual manera, sabemos que, para cualquier entero $1\le l < m$, $G[B_l']$ es una gráfica completa con $m_l$ vértices con $1\le m_l < m$. De igual manera, sabemos que ninguna de las partes de $Q$ tiene la misma cardinalidad y que vértices de diferentes partes diferentes de $C'$ no son adyacentes. Sólo falta por mostrar que, para cualquier entero $1\le m_l < m$, cada uno de los vértices de $B_l'$ es adyacente a los mismos $a_l$ vértices de $C'$ con $0< a_l < m - m_l$. Como $H$ es una $O$-componente, sabemos que cada uno de los vértices de $B_l'$ es adyacente a los mismos $a_l$ vértices de $C'$. Como cada vértice de $C$ en $A_1$ es adyacente a cada uno de los vértices de $B_{l+1}$, entonces se mantiene la propiedad de que $0< a_l < m - m_l$. Luego, $Q'$ es una $O$-partición. Notemos que para cualquier entero $m<l\le 1$ se cumple que $B_l$ no tiene vértices adyacentes a vértice alguno de $C'$. Así, $H-A_1$ es la unión ajena de una $O$-componente y varias gráficas completas que, en conjunto, contienen a todas las gráficas completas con $B_1$, $B_2$, $\dots$ y $B_i$ vértices. En la Figura \ref{fig_dem_O_02} se muestra un ejemplo de esto.

Finalmente, $G-A_1$ tiene como subgráfica inducida a la unión de algunas $O$-componentes que, en conjunto, contienen a todas las gráficas completas desde 1 hasta $n$ vértices. Luego $G-A_1$ tiene como subgráfica inducida una $O_{n-1}$-onstrucción, que, por hipótesis inductiva, no está en la clase $M_{n-1}$. Así, $P$ no es una $M_n$-partición.

\begin{figure}[ht!]

\begin{subfigure}{\textwidth}
\begin{center}
\begin{tikzpicture}
\begin{scope}[xshift=0cm,scale=1]
%K6
\node [style=vertex] (1) at (1,0.5) {};
\node [style=vertex, fill=red] (2) at (2,0.5) {};
\node [style=vertex] (3) at (0,1) {};
\node [style=vertex, fill=red] (4) at (3,1) {};
\node [style=vertex] (5) at (1,1.5) {};
\node [style=vertex, fill=red] (6) at (2,1.5) {};

%K1
\node [style=vertex, fill=red] (7) at (0,3) {};

%K2
\node [style=vertex] (8) at (1,3) {};
\node [style=vertex] (9) at (2,3) {};

%K3
\node [style=vertex, fill=red] (10) at (2.5,2.5) {};
\node [style=vertex] (11) at (3.5,2) {};
\node [style=vertex] (12) at (3.25,2.5) {};

%K4
\node [style=vertex] (13) at (4,0.625) {};
\node [style=vertex] (14) at (4.75,0.625) {};
\node [style=vertex] (15) at (4,1.375) {};
\node [style=vertex] (16) at (4.75,1.375) {};

\foreach \i/\j in {1/2,1/3,1/4,1/5,1/6,2/3,2/4,2/5,2/6,3/4,3/5,3/6,4/5,4/6,5/6} \draw [style=edge] (\i) to (\j);
\foreach \i/\j in {7/3,7/4,7/5,7/6} \draw [style=edge] (\i) to (\j);
\foreach \i/\j in {8/4,8/5,8/6,8/9,9/4,9/5,9/6} \draw [style=edge] (\i) to (\j);
\foreach \i/\j in {10/11,10/12,11/12,10/4,10/6,11/4,11/6,12/4,12/6} \draw [style=edge] (\i) to (\j);
\foreach \i/\j in {13/14,13/15,13/16,14/15,14/16,15/16,13/4,14/4,15/4,16/4} \draw [style=edge] (\i) to (\j);

\node at (2.375,0) {\parbox{0.3\linewidth}{\subcaption*{$G$}}};
\end{scope}

\begin{scope}[xshift=6.5cm,scale=1]
%K6
\node [style=vertex, fill=blue] (1) at (0,0.5) {};
\node [style=vertex, fill=blue] (3) at (1.5,0.5) {};
\node [style=vertex, fill=blue] (5) at (0.75,1.75) {};

%K2
\node [style=vertex, fill=blue] (8) at (0,3) {};
\node [style=vertex] (9) at (1.5,3) {};

%K3
\node [style=vertex, fill=blue] (11) at (2,3) {};
\node [style=vertex, fill=blue] (12) at (3.5,3) {};

%K4
\node [style=vertex, fill=blue] (13) at (2,0.5) {};
\node [style=vertex, fill=blue] (14) at (3.5,0.5) {};
\node [style=vertex, fill=blue] (15) at (2.75,1) {};
\node [style=vertex, fill=blue] (16) at (2.75,1.75) {};

\foreach \i/\j in {1/3,1/5,3/5} \draw [style=edge] (\i) to (\j);

\foreach \i/\j in {8/5,8/9,9/5} \draw [style=edge] (\i) to (\j);
\foreach \i/\j in {11/12} \draw [style=edge] (\i) to (\j);
\foreach \i/\j in {13/14,13/15,13/16,14/15,14/16,15/16} \draw [style=edge] (\i) to (\j);

\node at (1.75,0) {\parbox{0.3\linewidth}{\subcaption*{$G-A_1$}}};
\end{scope}

\end{tikzpicture}
\end{center}
\end{subfigure}

%\setlength{\abovecaptionskip}{-15pt}
\caption{La gráfica $G$ es una $O$-componente que contiene a $K_1$, $K_2$, $K_3$, $K_4$ y a $K_6$. En rojo se marcan los vértices que se agregan a $A_1$. A la derecha se muestra $G-A_1$. En azul se muestran los vértices de una $O$-obstrucción que contiene a $K_1$, $K_2$, $K_3$ y a $K_4$.}
\label{fig_dem_O_02}
\end{figure}

Abordemos el caso en el que $A_1$ contiene, además de dos vértices de $C$, al menos dos vértices de $B_j$ para algún $1\le j \le i$. Si $A_1$ contiene un vértice de $C$ que no es adyacente a los vértices de $B_j$ o algún vértice que no es elemento ni de $C$ ni de $B_j$, entonces $G[A_1]$ no es una grafica multipartita completa. Y por lo tanto, $P$ no es una $M_n$-partición. Si $A_1$ contiene sólo vértices de $B_j$ y vértices de $C$ adyacentes a los vértices de $B_j$, como los vértices del $B_j$ son adyacentes a máximo $n-m_j$ vértices de $C$, entonces al menos $m_j+1$ vértices de $C$ no son adyacentes a los vértices de $B_j$. Luego, al menos $m_j$ vértices de $C$ no están en $A_1$. Así, los vértices de $H$ que no están en $A_1$ inducen una gráfica que tiene como subgráfica inducida la unión ajena de algunas $O$-componentes que, en conjunto, contienen a las gráficas completas con $m_1, m_2,\dots$ y $m_{i-1}$ vértices. Luego, $G-A_1$ tiene como subgráfica inducida una $O_{n-1}$-obstrucción que, por hipótesis inductiva, no está en la clase $M_{n-1}$. Luego, $G$ no está en la clase $M_n$. En la Figura \ref{fig_dem_O_03} se muestra un ejemplo de esto.

\begin{figure}[ht!]

\begin{subfigure}{\textwidth}
\begin{center}
\begin{tikzpicture}
\begin{scope}[xshift=0cm,scale=1]
%K6
\node [style=vertex] (1) at (1,0.5) {};
\node [style=vertex] (2) at (2,0.5) {};
\node [style=vertex] (3) at (0,1) {};
\node [style=vertex, fill=red] (4) at (3,1) {};
\node [style=vertex] (5) at (1,1.5) {};
\node [style=vertex,fill=red] (6) at (2,1.5) {};

%K1
\node [style=vertex] (7) at (0,3) {};

%K2
\node [style=vertex] (8) at (1,3) {};
\node [style=vertex] (9) at (2,3) {};

%K3
\node [style=vertex, fill=red] (10) at (2.5,2.5) {};
\node [style=vertex, fill=red] (11) at (3.5,2) {};
\node [style=vertex, fill=red] (12) at (3.25,2.5) {};

%K4
\node [style=vertex] (13) at (4,0.625) {};
\node [style=vertex] (14) at (4.75,0.625) {};
\node [style=vertex] (15) at (4,1.375) {};
\node [style=vertex] (16) at (4.75,1.375) {};

\foreach \i/\j in {1/2,1/3,1/4,1/5,1/6,2/3,2/4,2/5,2/6,3/4,3/5,3/6,4/5,4/6,5/6} \draw [style=edge] (\i) to (\j);
\foreach \i/\j in {7/3,7/4,7/5,7/6} \draw [style=edge] (\i) to (\j);
\foreach \i/\j in {8/4,8/5,8/6,8/9,9/4,9/5,9/6} \draw [style=edge] (\i) to (\j);
\foreach \i/\j in {10/11,10/12,11/12,10/4,10/6,11/4,11/6,12/4,12/6} \draw [style=edge] (\i) to (\j);
\foreach \i/\j in {13/14,13/15,13/16,14/15,14/16,15/16,13/4,14/4,15/4,16/4} \draw [style=edge] (\i) to (\j);

\node at (2.375,0) {\parbox{0.3\linewidth}{\subcaption*{$G$}}};
\end{scope}

\begin{scope}[xshift=6.5cm,scale=1]
%K6
\node [style=vertex, fill=blue] (1) at (1.5,0.5) {};
\node [style=vertex, fill=blue] (3) at (0,0.5) {};
\node [style=vertex] (5) at (0.75,1.75) {};
\node [style=vertex, fill=blue] (6) at (0.75,1) {};

%K1
\node [style=vertex, fill=blue] (7) at (0,3) {};

%K2
\node [style=vertex, fill=blue] (8) at (0.75,3) {};
\node [style=vertex, fill=blue] (9) at (1.75,3) {};

%K4
\node [style=vertex, fill=blue] (13) at (3.5,0.5) {};
\node [style=vertex, fill=blue] (14) at (2,0.5) {};
\node [style=vertex, fill=blue] (15) at (2.75,1.75) {};
\node [style=vertex, fill=blue] (16) at (2.75,1) {};

\foreach \i/\j in {1/3,1/5,1/6,3/5,3/6,5/6} \draw [style=edge] (\i) to (\j);
\foreach \i/\j in {7/3,7/5} \draw [style=edge] (\i) to (\j);
\foreach \i/\j in {8/5,8/9,9/5} \draw [style=edge] (\i) to (\j);
\foreach \i/\j in {13/14,13/15,13/16,14/15,14/16,15/16} \draw [style=edge] (\i) to (\j);

\node at (1.75,0) {\parbox{0.3\linewidth}{\subcaption*{$G-A_1$}}};
\end{scope}

\end{tikzpicture}
\end{center}
\end{subfigure}

%\setlength{\abovecaptionskip}{-15pt}
\caption{La gráfica $G$ es una $O$-componente que contiene a $K_1$, $K_2$, $K_3$, $K_4$ y a $K_6$. En rojo se marcan los vértices que se agregan a $A_1$. A la derecha se muestra $G-A_1$. En azul se muestran los vértices de una $O$-obstrucción que contiene a $K_1$, $K_2$, $K_3$ y a $K_4$.}
\label{fig_dem_O_03}
\end{figure}

Para mostrar que $G$ es una obstrucción mínima, basta con notar que en cada uno de los casos anteriores, si se sustrae un vértice de $G$, entonces la gráfica resultante está en $M_n$.

\end{proof}

\subsubsection{$O$-obstrucciones conocidas}

A lo largo de este documento, podemos identificar varias $O$-obstrucciones. Notemos que $\overline{P_3}$ es una $O$-obstrucción. En la Figura \ref{obsts_M2}, las gráficas $H$ e $I$ son $O$-obstrucciones. Todas las gráficas de la Figura \ref{obsts_M3_O} son $O$-obstrucciones. En el apéndice \ref{apéndice_O} podemos observar a la familia $O_4$ de obstrucciones.

    \subsection{Familia $P$ de obstrucciones}
        En esta subsección presentamos a la familia $P$ de obstrucciones mínimas para cualquier clase $M_i$.

Sean $G$ una cográfica y $n$ un entero mayor a uno, decimos que $G$ es una \emph{\textbf{$P_n$-obstrucción}} si, para algunas obstrucciones mínimas $H_1, H_2$ de la clase $M_{n-1}$, $G=K_1+(H_1\oplus H_2)$.

Sea $n$ un entero mayor a uno, la \emph{\textbf{familia $P_n$ de obstrucciones}}, denotada simplemente por $P_n$, eses el conjunto de todas las $P_n$-obstrucciones.

\begin{theorem}
Sea $G$ una cográfica y $n$ un entero mayor a uno. Si $G$ es una $P_n$-obstrucción, entonces $G$ es una obstrucción mínima de la clase $M_n$.
\end{theorem}



\begin{proof}
Como $G$ es una $P_n$-obstrucción, entonces existen dos obstrucciones mínimas $H_1$ y $H_2$ de la clase $M_{n-1}$ tales que $G=K_1+(H_1\oplus H_2)$.

Veamos primero que $G$ no está en la clase $M_n$. Sea $P=(A_1, A_2, \dots, A_n)$ una partición de los vértices de $G$, veamos que $P$ no es una $M_n$-partición. Supongamos sin pérdida de generalidad que el $K_1$ de $G$ está en $A_1$. Si $A_1$ contiene un par de vértices adyacentes entre sí, entonces $G[A_1]$ no es una gráfica multipartita completa y $P$ no es una $M_n$-partición. En el caso contrario, $A_1$ debe de ser un conjunto independiente. Como $A_1$ es un conjunto independiente y cada vértice de $H_1$ es adyacente a cualquier vértice de $H_2$, entonces $A_1$ no puede contener vértices tanto de $H_1$ como de $H_2$. Supongamos sin pérdida de generalidad que ningún vértice de $H_1$ está en $A_1$. Luego, los vértices de $H_1$ se deben de repartir en las $n-1$ partes de $P$ distintas de $A_1$. Como $H_1$ es una obstrucción mínima de la clase $M_{n-1}$, cualquier partición de sus vértices en $n-1$ partes no es una $M_{n-1}$-partición. De esto se sigue que $P$ no es una $M_n$-partición. Así, $G$ no está en la clase $M_n$.

Sea $x$ un vértice de $G$, veamos que $G-x$ sí está en la clase $M_n$. Si $x$ es el vértice que conforma al $K_1$ de $G$, entonces existe una partición $P=(A_1, A_2, \dots, A_n)$ tal que $A_1$ contiene un vértice $y_1$ de $H_1$ y un vértice $y_2$ de $H_2$. Como $H_1$ y $H_2$ son obstrucciones mínimas de la clase $M_{n-1}$, entonces $H_1-\{y_1\}$ y $H_2-\{y_2\}$ están en la clase $M_{n-1}$. Luego, $((H_1-\{y_1\})\oplus(H_2-\{y_2\}))$ está en $M_{n-1}$, por lo que sus vértices se pueden repartir entre las partes de $P$ diferentes de $A_1$ de forma que cada parte induzca una gráfica multipartita completa. Notemos que $G[A_1]=K_2$ también es una gráfica multipartita completa. Así $G-x$ está en $M_n$. 

Si $x$ no es el vértice que conforma al $K_1$ de $G$, supongamos sin pérdida de generalidad que $x$ es un vértice de $H_1$. Luego, existe una partición $Q=(B_1, B_2, \dots, B_n)$ tal que $B_1$ contiene al vértice aislado de $G$ y a un vértice de $H_2$. Por el argumento anterior, sabemos que $G-B_1$ está en la clase $M_{n-1}$. Por este motivo, y porque $G[B_1]=2K_1$ es una gráfica multipartita completa, se sigue que $G-x$ está en $M_n$. Así, $G$ es una obstrucción mínima de la clase $M_n$.

\end{proof}

\subsubsection{$P$-obstrucciones conocidas}

A lo largo de este documento podemos identificar algunas $P$-obstrucciones. En la Figura \ref{obsts_M2}, la gráfica $J$ es una $P$-obstrucción. En la lista de obstrucciones mínimas de la clase $M_3$, la gráfica $P_{3,i}$ es una $P$-obstrucción para todo $1\le i \le 6$.
