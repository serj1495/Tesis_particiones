

\section{Términos y algoritmos generales}
    El presente capítulo tiene como objetivo exponer un conjunto de conceptos y algoritmos cuya utilidad, aunque acotada a las cográficas, no se limita a la clase de cográficas cuyo conjunto de vértices acepta una partición en dos gráficas multipartitas completas.
    \subsection{Coárbol binario}
        Tomando como base el concepto de coárbol, podemos imaginar otra estructura de tipo árbol para la representación de las cográficas en la que cada nodo tenga a lo más un número $k$ de hijos. Esta limitante resulta útil para formular algoritmos rápidos en cográficas. El menor valor que puede tomar $k$ es de 2. Este valor será utilizado a lo largo de este capítulo para representar a las cográficas a través de árboles binarios.


\begin{definition}{}

    Sean $G=(V,E)$ una cográfica, $C = c_1, c_2, \dots, c_n$ el conjunto de las componentes conexas de $G$, $D = {d_1, d_2, \dots, d_m}$ el conjunto de las componentes conexas de $\overline{G}$, $(C_1, C_2)$ una partición en dos partes de $C$ y $(D_1, D_2)$ una partición en dos partes de $D$, decimos que el árbol binario arraigado etiquetado, $(T,r)$, es un \textbf{\emph{coárbol binario}} de $G$ si se puede construir de la siguiente manera: Si $G$ consta de un sólo vértice, entonces $T$ sólo contiene a $r$. De lo contrario, si $G$ es conexa, entonces $r$ tiene la etiqueta $1$, uno de los hijos de $r$ es el coárbol binario de $G-D_1$ y el otro es el coárbol binario de $G-D_2$. Y finalmente, si $G$ es inconexa, entonces $r$ tiene la etiqueta $0$, uno de sus hijos es el coárbol binario de $G-C_1$ y el otro el coárbol binario de $G-C_2$.


\end{definition}

Claramente, una cográfica puede ser representada por más de un coárbol binario diferente como se muestra en la Figura \ref{fig_coar_bin01}. Sin embargo, la propiedad de que dos vértices son adyacentes si y sólo si su ancestro común más profundo tiene la etiqueta 1 se mantiene.

\begin{figure}[ht!]
\begin{center}
\begin{tikzpicture}

\begin{scope}[xshift=0cm,scale=1]

\node [vertex] (1) at (0,0) {};
\node [vertex] (2) at (1,0) {};
\node [vertex] (3) at (0,1) {};
\node [vertex] (4) at (1,1) {};
\foreach \i/\j in {1/2,1/3,1/4,2/3,2/4,3/4}
  \draw [edge] (\i) to (\j);
\node [below of=1,xshift=.5cm] {\parbox{0.3\linewidth}{\subcaption{}}};

\end{scope}

\begin{scope}[xshift=3.5cm,scale=1]

\node [cotreenode] (1) at (1,1) {1};
\node [cotreenode] (2) at (0,0) {1};
\node [cotreenode] (3) at (2,0) {1};
\node [vertex] (4) at (-0.5,-1) {};
\node [vertex] (5) at (0.5,-1) {};
\node [vertex] (6) at (1.5,-1) {};
\node [vertex] (7) at (2.5,-1) {};
\foreach \i/\j in {1/2,1/3,2/4,2/5,2/4,3/6,3/7}
  \draw [edge] (\i) to (\j);
\node [below of=5,xshift=.5cm] {\parbox{0.3\linewidth}{\subcaption{}}};

\end{scope}

\begin{scope}[xshift=7.5cm,scale=1]

\node [cotreenode] (1) at (1,1) {1};
\node [vertex] (2) at (0,0) {};
\node [cotreenode] (3) at (2,0) {1};
\node [vertex] (6) at (1,-1) {};
\node [cotreenode] (7) at (3,-1) {1};
\node [vertex] (8) at (2,-2) {};
\node [vertex] (9) at (4,-2) {};

\foreach \i/\j in {1/2,1/3,3/6,3/7,7/8,7/9}
  \draw [edge] (\i) to (\j);
\node [below of=8] {\parbox{0.3\linewidth}{\subcaption{}}};

\end{scope}


\end{tikzpicture}
\end{center}
\setlength{\abovecaptionskip}{-10pt}
\caption{(b) y (c) son dos coárboles binarios diferentes que representan a la cográfica (a).}\label{fig_coar_bin01}
\end{figure}

    \subsection{Algoritmo para generar un coárbol binario}
        Podemos obtener un coárbol binario a partir de un coárbol con el Algoritmo \ref{alg_coa_bin01}. En este algoritmo un nodo interno con al menos tres hijos, $r$, de un coárbol, se procesa creando un nuevo coárbol binario de la siguiente forma: La raíz del coárbol binario tiene como primer hijo al coárbol binario resultante de procesar al primer hijo de $r$ y como segundo hijo un nodo con la misma etiqueta de $r$ que a su vez tiene como primer hijo al árbol binario resultante de procesar al segundo hijo de $r$ y como segundo hijo un nuevo nodo con la misma etiqueta y así sucesivamente. Cuando sólo quedan los últimos dos hijos de $r$, estos se procesan y los árboles binarios resultantes son los hijos del último nodo creado. El árbol binario resultante es un árbol cargado a la derecha. La Figura \ref{fig_alg_coa_bin01} muestra una ejecución ilustrativa del algoritmo.

\begin{algorithm}[ht!]
\caption{CrearArbolBinario}
\label{alg_coa_bin01}
\DontPrintSemicolon % Some LaTeX compilers require you to use \dontprintsemicolon instead
\KwIn{$r$ la raíz del coárbol}
\KwOut{$r'$ la raíz del coárbol binario}

$r' \gets \text{nuevo nodo de árbol binario}$\;

\If{$r\ \emph{es un nodo interno} $}{
    $r'.etiqueta = r.etiqueta$\;
    $s \gets r'$\;
    $i \gets 0$\;
    \While{$i < r.children.size - 2$}{
        $s.primerHijo \gets \text{CrearArbolBinario}(r.hijos[i])$\;
        $s.segundoHijo \gets \text{nuevo nodo de árbol binario}$\;
        $s \gets s.segundoHijo$\;
        $s.etiqueta \gets r.etiqueta$\;
        $i \gets i+1$\;
    }
    $s.primerHijo \gets \text{CrearArbolBinario}(r.hijos[i])$\;
    $s.segundoHijo \gets \text{CrearArbolBinario}(r.hijos[i+1])$\;
}
\Return $r'$\;

\end{algorithm}

\begin{figure}[ht!]
\centering

\begin{subfigure}{0.7\textwidth}
\begin{tikzpicture}
\begin{scope}[xshift=0cm,scale=1]
\node [style=cotreenode, fill=lightgray] (1) at (2,1) {1};
\node [style=vertex] (2) at (0.5,0) {};
\node [style=vertex] (3) at (1.5,0) {};
\node [style=vertex] (4) at (2.5,0) {};
\node [style=cotreenode] (5) at (3.5,0) {0};
\node [style=vertex] (6) at (3,-1) {};
\node [style=vertex] (7) at (4,-1) {};
\foreach \i/\j in {1/2,1/3,1/4,1/5,5/6,5/7}
  \draw [style=edge] (\i) to (\j);
\end{scope}
\begin{scope}[xshift=6.5cm,scale=1]
\node [style=cotreenode] (1) at (1,1) {1};
\node [style=vertex] (2) at (0.5,0) {};
\node [style=cotreenode] (3) at (1.5,0) {1};
\foreach \i/\j in {1/2,1/3}
  \draw [style=edge] (\i) to (\j);
\end{scope}
\end{tikzpicture}
\end{subfigure}

\par\bigskip

\begin{subfigure}{0.7\textwidth}
\begin{tikzpicture}
\begin{scope}[xshift=0cm,scale=1]
\node [style=cotreenode, fill=lightgray] (1) at (2,1) {1};
\node [style=vertex] (2) at (0.5,0) {};
\node [style=vertex] (3) at (1.5,0) {};
\node [style=vertex] (4) at (2.5,0) {};
\node [style=cotreenode] (5) at (3.5,0) {0};
\node [style=vertex] (6) at (3,-1) {};
\node [style=vertex] (7) at (4,-1) {};
\foreach \i/\j in {1/2,1/3,1/4,1/5,5/6,5/7}
  \draw [style=edge] (\i) to (\j);
\end{scope}
\begin{scope}[xshift=6.5cm,scale=1]
\node [style=cotreenode] (1) at (1,1) {1};
\node [style=vertex] (2) at (0.5,0) {};
\node [style=cotreenode] (3) at (1.5,0) {1};
\node [style=vertex] (4) at (1,-1) {};
\node [style=cotreenode] (5) at (2,-1) {1};
\foreach \i/\j in {1/2,1/3,3/4,3/5}
  \draw [style=edge] (\i) to (\j);
\end{scope}
\end{tikzpicture}
\end{subfigure}

\par\bigskip

\begin{subfigure}{0.7\textwidth}
\begin{tikzpicture}
\begin{scope}[xshift=0cm,scale=1]
\node [style=cotreenode, fill=lightgray] (1) at (2,1) {1};
\node [style=vertex] (2) at (0.5,0) {};
\node [style=vertex] (3) at (1.5,0) {};
\node [style=vertex] (4) at (2.5,0) {};
\node [style=cotreenode] (5) at (3.5,0) {0};
\node [style=vertex] (6) at (3,-1) {};
\node [style=vertex] (7) at (4,-1) {};
\foreach \i/\j in {1/2,1/3,1/4,1/5,5/6,5/7}
  \draw [style=edge] (\i) to (\j);
\end{scope}
\begin{scope}[xshift=6.5cm,scale=1]
\node [style=cotreenode] (1) at (1,1) {1};
\node [style=vertex] (2) at (0.5,0) {};
\node [style=cotreenode] (3) at (1.5,0) {1};
\node [style=vertex] (4) at (1,-1) {};
\node [style=cotreenode] (5) at (2,-1) {1};
\node [style=vertex] (6) at (1.5,-2) {};
\node [style=cotreenode] (7) at (2.5,-2) {0};
\foreach \i/\j in {1/2,1/3,3/4,3/5,5/6,5/7}
  \draw [style=edge] (\i) to (\j);
\end{scope}
\end{tikzpicture}
\end{subfigure}

\par\bigskip

\begin{subfigure}{0.7\textwidth}
\begin{tikzpicture}
\begin{scope}[xshift=0cm,scale=1]
\node [style=cotreenode] (1) at (2,1) {1};
\node [style=vertex] (2) at (0.5,0) {};
\node [style=vertex] (3) at (1.5,0) {};
\node [style=vertex] (4) at (2.5,0) {};
\node [style=cotreenode, fill=lightgray] (5) at (3.5,0) {0};
\node [style=vertex] (6) at (3,-1) {};
\node [style=vertex] (7) at (4,-1) {};
\foreach \i/\j in {1/2,1/3,1/4,1/5,5/6,5/7}
  \draw [style=edge] (\i) to (\j);
\end{scope}
\begin{scope}[xshift=6.5cm,scale=1]
\node [style=cotreenode] (1) at (1,1) {1};
\node [style=vertex] (2) at (0.5,0) {};
\node [style=cotreenode] (3) at (1.5,0) {1};
\node [style=vertex] (4) at (1,-1) {};
\node [style=cotreenode] (5) at (2,-1) {1};
\node [style=vertex] (6) at (1.5,-2) {};
\node [style=cotreenode] (7) at (2.5,-2) {0};
\node [style=vertex] (8) at (2,-3) {};
\node [style=vertex] (9) at (3,-3) {};
\foreach \i/\j in {1/2,1/3,3/4,3/5,5/6,5/7,7/8,7/9}
  \draw [style=edge] (\i) to (\j);
\end{scope}
\end{tikzpicture}
\end{subfigure}


\caption{Ejemplo de la ejecución del Algoritmo \ref{alg_coa_bin01}. A la izquierda se muestra el coárbol original, mienrtras se marca con gris el nodo que se está procesando. A la derecha aparece el coárbol binario que se va construyendo.}\label{fig_alg_coa_bin01}


\end{figure}


En términos de las particiones de las componentes conexas de la gráfica, el algoritmo realiza lo siguiente. Si la etiqueta de $r$ es $0$, entonces el coárbol con raíz en $r$ representa una cográfica inconexa y se elige la partición de sus vértices en la que la primera parte es una componente conexa y la segunda parte es el resto. Sucede lo mismo si la etiqueta de $r$ es uno, pero como la cográfica representada es conexa, en su lugar se toman una componente conexa del complemento de la cográfica representada en la primera parte y el resto en la segunda.

Dado que el Algoritmo \ref{alg_coa_bin01} recorre a lo más una vez cada nodo de $r$, su tiempo de ejecución es $O(n)$ en donde $n$ es el número total de nodos del árbol con raíz $r$.

    \subsection{Algoritmo para generar todos los coárboles binarios}
        \iffalse

Podemos obtener todos los coárboles binarios correspondientes a un coárbol haciendo uso del Algoritmo \ref{alg_coa_bin02}. Este algoritmo recibe como entrada la raíz del coárbol, $r$, y devuelve un conjunto de nodos, $S$, cada uno de cuyos elementos es la raíz de un coárbol binario. Los nodos internos son procesados creando un nuevo coárbol binario para cada posible partición del conjunto de hijos de dicho nodo. Al procesar las hojas, simplemente se crea un nuevo nodo que será una hoja en los árboles binarios. 


\begin{algorithm}[h]
\caption{CrearÁrbolesBinarios}
\label{alg_coa_bin02}
\DontPrintSemicolon % Some LaTeX compilers require you to use \dontprintsemicolon instead
\KwIn{$r$ la raíz de un coárbol}
\KwOut{$S = \{r'_1, r'_2, \dots, r'_n\}$ con $r_i$ la raíz de un coárbol binario}

$r' \gets \text{nuevo nodo de árbol binario}$\;

\If{$r\ \emph{es un nodo interno} $}{
    $r'.etiqueta = r.etiqueta$\;
    $s \gets r'$\;
    $i \gets 0\;
    \While{$i < r.children.size - 2$}{
        $s.primerHijo = \text{CrearArbolBinario}(r.hijos[i])$\;
        $s.segundoHijo \gets \text{nuevo nodo de árbol binario}$\;
        $s \gets s.segundoHijo$\;
        $s.etiqueta = r.etiqueta$\;
        $i = i+1$\;
    }
    $s.primerHijo = \text{CrearArbolBinario}(r.hijos[i])$\;
    $s.segundoHijo = \text{CrearArbolBinario}(r.hijos[i+1])$\;
}
\Return $r'$\;
    
\end{algorithm}

\fi
    \subsection{Subcoárbol}
        A continuación presentamos los conceptos de subcoárbol y subcoárbol binario
que serán utilizados para determinar si una cográfica $H$ es subgráfica
inducida de una cográfica $G$ en el Algoritmo \ref{alg_subgraph}.

Sean $T$ y $U$ dos coárboles y $u_1$, $u_2$ y $u_3$ nodos de $U$, decimos
que $U$ es un \emph{\textbf{subcoárbol}} de $T$ si existe una función
inyectiva $f:V(U)\rightarrow V(T)$ tal que, si $u_1$ es una hoja, entonces
$f(u_1)$ tambi\'en es una hoja; si no, entonces $u_1$ y $f(u_1)$ tienen la
misma etiqueta y, si $u_3$ es el ancestro común más profundo de $u_1$ y
$u_2$, entonces $f(u_3)$ es el ancestro común más profundo de $f(u_1)$ y
$f(u_2)$. Llamamos a $f$ la \textbf{\emph{función de coasignación}} de $U$ a
$T$.

El concepto de subcoárbol es diferente del de subárbol dado que, si $T$ y $U$
son coárboles con $U$ subcoárbol de $T$, entonces tenemos que los nodos de $U$
se pueden encontrar dispersos entre los nodos de $T$ a diferencia de lo que se
tendría si $U$ fuera subárbol de $T$. Esto se puede apreciar en la Figura
\ref{fig_subcoarbol01}. N\'otese que esta definición funciona también para
coárboles binarios.

\begin{figure}[h!]
\begin{center}
\begin{tikzpicture}

\begin{scope}[xshift=0cm,scale=1]
\node [style=cotreenode] (1) at (1,1) {0};
\node [style=cotreenode] (2) at (-0.5,0) {1};
\node [style=cotreenode] (3) at (2.5,0) {1};
\node [style=cotreenode] (4) at (-1.25,-1) {0};
\node [style=cotreenode] (5) at (0.25,-1) {0};
\node [style=cotreenode] (6) at (1.75,-1) {0};
\node [style=cotreenode] (7) at (3.25,-1) {0};
\node [style=vertex] (8) at (-1.5,-2) {};
\node [style=vertex] (9) at (-1,-2) {};
\node [style=vertex] (10) at (0,-2) {};
\node [style=vertex] (11) at (0.5,-2) {};
\node [style=vertex] (12) at (1.5,-2) {};
\node [style=vertex] (13) at (2,-2) {};
\node [style=vertex] (14) at (3,-2) {};
\node [style=vertex] (15) at (3.5,-2) {};

\node (16) at (0.25,1) {$f(a)$};
\node (17) at (-1.6,-2.4) {$f(b)$};
\node (18) at (3.25,0) {$f(c)$};
\node (19) at (1.4,-2.4) {$f(d)$};
\node (20) at (3.6,-2.4) {$f(e)$};

\foreach \i/\j in {1/2,1/3,2/4,2/5,3/6,3/7,4/8,4/9,5/10,5/11,6/12,6/13,7/14,7/15}
  \draw [style=edge] (\i) to (\j);
\node [below of=19,xshift=-0.25cm] {\parbox{0.3\linewidth}{\subcaption{}}};
\end{scope}

\begin{scope}[xshift=6cm,scale=1]
\node [style=cotreenode] (1) at (1,1) {0};
\node [style=vertex] (2) at (0,0) {};
\node [style=cotreenode] (3) at (2,0) {1};
\node [style=vertex] (4) at (1.5,-1) {};
\node [style=vertex] (5) at (2.5,-1) {};

\node (6) at (0.5,1) {$a$};
\node (7) at (-0.3,0) {$b$};
\node (8) at (2.5,0) {$c$};
\node (9) at (1.5,-1.3) {$d$};
\node (10) at (2.5,-1.3) {$e$};

\foreach \i/\j in {1/2,1/3,3/4,3/5}
  \draw [style=edge] (\i) to (\j);
\node [below of=9,xshift=-0.25cm] {\parbox{0.3\linewidth}{\subcaption{}}};
\end{scope}

\end{tikzpicture}
\end{center}
\setlength{\abovecaptionskip}{-10pt}
\caption{El coárbol (b) es subcoárbol del coárbol (a). Las etiquetas en los nodos de ambos coárboles indican la asignación de los nodos de (b) a los nodos de (a).}\label{fig_subcoarbol01}
\end{figure}

\begin{lemma}\label{lema_subcoa_01}
    Sean $G$ y $H$ cográficas y $T_G$ y $T_H$ sus coárboles correspondientes, entonces $H$ es una subgráfica inducida de $G$ si y sólo si $T_H$ es subcoárbol de $T_G$.
\end{lemma}

\begin{proof}
    {\color{red}
    Revisa esta demostraci\'on.   En particular, al principio, cuando
    est'as declarando variables, declaras a $f$ como si ya existiera,
    pero justamente quieres demostrar su existencia.   En lugar de
    declararla con las dem\'as variables, puedes decir que vamos a
    definir una funci\'on $f$ con $f \colon V(T_H)\rightarrow V(T_G)$,
    tal que, etc. y verificaremos que es una coasignaci\'on.   De
    entrada, no es clara la definici\'on de $f$; parece ser una
    definici\'on recursiva, de abajo hacia arriba, pero esto tampoco
    es claro.
    }
    Supongamos primero que $H$ es una subgráfica inducida de $G$. Sabemos que
    $V(H)\subset V(G)$ y que dos vértices son adyacentes en $H$ si y sólo si
    también son adyacentes en $G$. Sean $v$ y $w$ vértices de $H$, $n_H$ el
    ancestro común más profundo de $v$ y $w$ en $T_H$, $n_G$ el ancestro común
    más profundo de $v$ y $w$ en $T_G$ y $f:V(T_H)\rightarrow V(T_G)$ una
    función tal que $f(v)=v$ y $f(n_H) = n_G$, mostremos que $f$ es una función
    de coasignación de $T_H$ a $T_G$. Sean $x,y,z\in V(T_H)$. Si $x$ es una
    hoja de $T_H$, es claro que $f(x) = x$ es una hoja de $T_G$. Si no,
    entonces $x$ es el ancestro común más profundo de algún par de hojas de
    $T_H$ y tiene etiqueta 1 si y sólo si éstas son adyacentes en $H$, lo que
    ocurre si y sólo si son adyacentes en $G$, que a su vez ocurre si y sólo si
    su ancestro común más profundo en $T_G$ tiene etiqueta etiqueta 1. Así $x$
    y $f(x)$ tienen necesariamente la misma etiqueta. Finalmente mostremos que
    si $x$ es el ancestro común más profundo de $y$ y $z$, entonces $f(x)$ es
    el ancestro común más profundo de $f(y)$ y $f(z)$. Si $y$ y $z$ son hojas,
    esto se cumple trivialmente por la definición de $f$. Si $y$ es un nodo
    interno y $z$ es una hoja, entonces existen dos hojas $y'$ y $y''$ de $T_H$
    tales que $y$ es su ancestro común más profundo. Tenemos que $f(y)$ es el
    ancestro común más profundo de $f(y')$ y $f(y'')$. A su vez, dado que $x$
    es el ancestro común más profundo de $y'$ y $z$, entonces $f(x)$ es el
    ancestro común mas profundo de $f(y')$ y $f(z)$. De igual manera $f(x)$ es
    el ancestro común mas profundo de $f(y'')$ y $f(z)$ Luego, $f(y)$ se
    encuentra en el camino desde $f(y')$ hasta $f(x)$ y por lo tanto $f(x)$ es
    el nodo común más profundo de $f(y)$ y $f(z)$. Análogamente si $z$ es un
    nodo interno y $y$ es una hoja. Si tanto $y$ como $z$ son nodos internos,
    el argumento es prácticamente el mismo, pero utilizando dos hojas $y'$ y
    $y''$ cuyo ancestro común más profundo es $y$ y otras dos hojas $z'$ y
    $z''$ cuyo ancestro común más profundo es $z$. Así, $f$ es una función de
    coasignación de $T_H$ a $T_G$ y $T_H$ es subcoárbol de $T_G$.

    Rec\'iprocamente, si $T_H$ es subcoárbol de $T_G$, entonces existe
    una función de coasignación, $f$, de $T_H$ a $T_G$. Luego, sean $h_1$ y
    $h_2$ hojas de $T_H$ y $h_3$ el ancestro común más profundo de $h_1$ y
    $h_2$, tenemos que $f(h_1)$ y $f(h_2)$ son hojas de $T_G$ y que
    $h_3.etiqueta = f(h_3).etiqueta$. Así, $h_1$ y $h_2$ son adyacentes
    en $H$ si y sólo si $f(h_1)$ y $f(h_2)$ son adyacentes en $G$. Luego,
    $G[f[V(H)]]$ es una subgráfica de $G$ que es isomorfa a $H$. Así, $H$
    es subgráfica de $G$.

\end{proof}

Notemos que esta demostración funciona únicamente para los coárboles y no para los coárboles binarios. En la Figura \ref{fig_coar_bin01} se observan dos coárboles binarios que representan a la misma cográfica. Sin embargo ninguno de los dos es subcoárbol del otro.

\begin{lemma}
    Sean $G$ y $H$ cográficas y $T_G$ y $T_H$ coárboles binarios de $G$ y $H$ respectivamente. Si $T_H$ es subcoárbol de $T_G$, entonces $H$ es subgráfica de $G$.
\end{lemma}

\begin{proof}
    La demostración es igual a la segunda parte de la demostración del Lema \ref{lema_subcoa_01}.
\end{proof}

    \subsection{Algoritmo para encontrar una subgráfica en un coárbol} \label{sec_AlgoSub}
        La presente sección aborda el problema de determinar si una cográfica $G$ tiene a otra cográfica $H$ como subgráfica inducida haciendo uso de los conceptos de coárbol binario y subcoárbol. Se proporciona un algoritmo (Algoritmo \ref{alg_subgraph}) para resolver este problema tal que, si se fija el tamaño de $H$, su tiempo de ejecución crece de forma lineal con respecto al tamaño de $G$. Este algoritmo es útil para identificar a las gráficas pertenecientes a una clase caracterizada a través de su conjunto de obstrucciones mínimas de forma rápida. 

\subsubsection{Algoritmo para determinar si un coárbol binario es subcoárbol binario de otro}

\begin{definition}
    Sean $T$ y $U$ coárboles (binarios) y $u$ un nodo de $U$, decimos que $f:V(U)\rightarrow\{marcado, no\_marcado\}$ es una \textbf{\emph{función de verificación}} de $T$ para $U$ si $f(u) = marcado$ si y sólo si el coárbol (binario) con raíz en $u$ es subcoárbol (binario) de $T$. Si $f(u) = marcado$, decimos que $f$ \textbf{\emph{marca}} a $u$.
\end{definition}

El Algoritmo \ref{alg_subcoarbol} recibe como entradas dos coárboles binarios, $G$ y $H$ representados por sus raíces $g$ y $h$ respectivamente, y devuelve una función de verificación, $f_g$, de $G$ para $H$. 

Este algoritmo funciona creando la función de verificación de cada subárbol de $G$ para $H$, empezando por los más profundos. De esta manera, si la función de verificación de $G$ para $H$ evaluada en $h$ es $marcado$, entonces  $H$ es subcoárbol de $G$.

\begin{algorithm}[h]
\caption{Función\_de\_coasignación}
\label{alg_subcoarbol}
\DontPrintSemicolon % Some LaTeX compilers require you to use \dontprintsemicolon instead
\KwIn{$g$ y $h$, las raíces de dos coárboles binarios para las gráficas $G$ y $H$ respectivamente}
\KwOut{$func$, la función de verificación de $G$ para $H$}

 $func \gets \text{nueva función de coasignación tal que} func(x)=no\_marcado \text{ para todo } x\in V(H)$\;
 
 \If{g \emph{es una hoja}}{
    $func \text{ marca a todas las hojas de } H$\;
 }
 \Else{
    $v_{izq} \gets \text{Función\_de\_coasignación}(g.izquierda, h)$\;
    $v_{der} \gets \text{Función\_de\_coasignación}(g.derecha, h)$\;
    
    \ForEach{nodo \textbf{\emph{de}} H}{
        \If{$v_{izq}(nodo) = marcado \emph{ \textbf{o} } v_{der}(nodo) = marcado$}{
            $func(nodo) \gets marcado$\;
        }
        \ElseIf{nodo.etiqueta = g.etiqueta \emph{\textbf{y}} $v_{izq}$ \emph{marca a uno de los hijos de} nodo \emph{y} $v_{der}$ \emph{al otro}}{
            $func(nodo) \gets marcado$\;
        }
    }
    
 }

$\Return func$
    
\end{algorithm}
    
\begin{theorem}
    La ejecución del Algoritmo \ref{alg_subcoarbol}, Función\_de\_coasignación $(g, h)$ regresa una función, $func$, tal que $func$ es una función de verificación de $árbol(g)$ para $árbol(h)$.
\end{theorem}

\begin{proof}
    
    Sea $n$ un nodo de $árbol(h)$. Para probar que $func$ es una función de verificación de $árbol(g)$ para $árbol(h)$, tenemos que probar que $func(n) = marcado$ si y sólo si $árbol(n)$ es subcoárbol de $árbol(g)$.

    \textbf{Necesidad}: En esta parte de la demostración, se supone que el algoritmo ha sido ejecutado y que $func$ marca a $n$. Procedamos por inducción sobre la altura de $g$.
    
    \emph{Caso base:} Si $g$ tiene altura 0, entonces $g$ es una hoja, por lo que $func$ marca únicamente a las hojas de $árbol(h)$. Como $func$ marca a $n$, entonces $n$ es una hoja. Luego, la función $f=\{(n,g)\}$ es una función de coasignación de $árbol(n)$ a $árbol(g)$, por lo que $árbol(n)$ es subcoárbol de $árbol(n)$.
    
    \emph{Paso inductivo:} Si $g$ tiene altura $k > 0$. Supongamos como hipotesis inductiva (H.I.) que, para todo nodo de un coárbol binario, $g'$, de altura $k' < k$ se cumple que, si $func' = $ Función\_de\_coasignación$(g',h)$ marca a un nodo $n'$ de $árbol(h)$, entonces $árbol(n')$ es subcoárbol de $árbol(g')$. Como $g$ no es una hoja, el algoritmo debió de entrar al bloque de instrucciones de las líneas 5 a 11. En las líneas 5 y 6 se crean dos funciones que cumplen con la H.I., ya que $g.izquierda$ y $g.derecha$ tienen ambas una altura menor a $k$. Como $func$ marca a $n$, entonces $n$ debe de cumplir la condición de la línea 8 o la condición de la línea 10. Si se cumple la condición de la línea 8, entonces $v_{izq}(n) = marcado$ o $v_{der}(n) = marcado$, por lo que $árbol(n)$ es subcoárbol de $árbol(g.izquierda)$ o de $árbol(g.derecha)$, y por lo tanto es subcoárbol de $árbol(g)$. De lo contrario, se cumple la condición de la línea 10, entonces $v_{izq}$ marca a $n.izquierda$ o a $n.derecha$ y $v_{der}$ marca al otro. Supongamos sin pérdida de generalidad que $v_{izq}$ marca a $n.izquierda$ y $v_{der}$ marca a $n.derecha$. Sean $f_i:V(árbol(n.izquierda))\rightarrow V(árbol(g.izquierda))$ la función de coasignación de $árbol(n.izquierda)$ a $árbol(g.izquierda)$ y $f_d:V(árbol(n.derecha))\rightarrow V(árbol(g.derecha))$ la función de coasignación de $árbol(n.derecha)$ a $árbol(g.derecha)$, mostremos que la función $f = f_i \cup f_d \cup \{(n,g)\}$ es una función de coasignación de $árbol(n)$ a $árbol(g)$. Como los dominios de $f_i$ y $f_d$ son ajenos y ninguno contiene a $n$, entonces $f$ es una función. Como los rangos de $f_i$ y $f_d$ son ajenos, ninguno contiene a $g$ y tanto $f_i$ como $f_d$ son inyectivas, entonces $f$ es inyectiva. Por otra parte, por la condición de la línea 10, sabemos que $n.etiqueta = g.etiqueta$. También sabemos que, sea $x \in V(árbol(n.izquierda))$, si $x$ es una hoja, entonces $f(x) = f_i(x)$ es una hoja y si no, entonces $x.etiqueta = f_i(x).etiqueta = f(x).etiqueta$. Análogamente para un $y \in V(árbol(n.derecha))$ y $f_d$. Finalmente, si $n$ es el ancestro común más profundo de dos nodos $z_1$ y $z_2$, entonces $z_1$ es descendiente de $n.derecha$ y $z_2$ es descendiente de $n.izquierda$ o viceversa. Supongamos lo primero sin pérdida de generalidad. Luego, por la condición de la línea 10, $v_{izq}$ marca a uno y $v_{der}$ marca al otro. Supongamos sin pérdida de generalidad que $v_{izq}$ marca a $z_1$ y $v_{der}$ marca a $z_2$. Entonces, $f(z_1) = f_i(z_1) \in V(árbol(g.izquierda))$ y $f(z_2) = f_i(z_2) \in V(árbol(g.derecha))$, por lo que el ancestro común más profundo de $f(z_1)$ y $f(z_2)$ es $g = f(n)$. Así, $f$ es una función de coasignación de $árbol(n)$ a $árbol(g)$ y $árbol(n)$ es subcoárbol de $árbol(g)$.
    
     \textbf{Suficiencia}: En esta parte de la demostración se supone que $árbol(n)$ es subcoárbol de $árbol(g)$ y se sigue la ejecución del algoritmo para mostrar que, al final de la misma, $func$ marcará a $n$. Sea $f$ la función de cosignación de $árbol(n)$ a $árbol(g)$, procedamos por inducción sobre la altura de $g$.
    
    \emph{Caso base:} Si la altura de $g$ es 0, entonces $g$ es una hoja, por lo que se cumple con la condición de la línea 2 y se ejecuta la línea 3, haciendo que $func$ marque todas las hojas de $H$. Como $árbol(n)$ es subcoárbol de $árbol(g)$ y $árbol(g)$ sólo tiene un nodo, entonces $n$ debe de ser una hoja. Luego, $func$ marca a $n$.
    
    \emph{Paso inductivo:} Si $g$ tiene altura $k > 0$. Supongamos como H.I. que todo coárbol, $g'$, con altura $k' < k$ cumple con que, siendo $n'$ un nodo de $árbol(h)$, si $árbol(n')$ es subcoárbol de $árbol(g')$, entonces $func'=$Función\_de\_coasignación $(g',h)$ marca a $n'$. Como $g$ no es una hoja, el algoritmo ejecuta las líneas 5 y 6 y posteriormente el bloque de las líneas 8 a 11 para cada nodo de $H$. Si $n$ es marcada por $v_{izq}$ o $v_{der}$, entonces se ejecuta la línea 9 y $func$ marca a $n$. En el caso contrario, probemos que se cumple la condición de la línea 10. Mostremos primero que $f(n) = g$ procediendo por contradicción. Supongamos que $f(n) = x$ para algún $x\in V(árbol(g))-\{g\}$. Como $x$ es descendiente de $g$, tiene altura menor a $k$. También sabemos que $f$ es una función de coasignación de $árbol(n)$ a $árbol(x)$, por lo que, por H.I., $n$ debería de ser marcado ya sea por $v_{izq}$ o por $v_{der}$, lo que es una contradicción. Luego, $f(n) = g$, y por lo tanto $n$ no es una hoja y $f(n).etiqueta = g.etiqueta$. Mostremos ahora que tanto $n.izquierda$ como $n.derecha$ son marcados cada uno ya sea por $v_{izq}$ o por $v_{der}$. Sabemos que $f(n.izquierda)$ y $f(n.derecha)$ son descendientes de $r$. Como $f\mid_{V(árbol(n.izquierda))}$ es una función de coasignación de $árbol(n.izquierda)$ a $árbol(f(n.izquierda))$ y $f(n.izquierda) \neq g$ ya que $f$ es inyectiva, entonces $árbol(n.izquierda)$ es subcoárbol de algún descendiente de $g$, al que llamaremos $y$. Como $y$ tiene altura menor a $k$, su función de verificación correspondiente marca a $n.izquierda$ (por H.I.), y por la condición de la línea 8, sus ancestros también lo marcan. Luego $v_{izq}$ o $v_{der}$ marcan a $n.izquierda$. Análogamente para $n.derecha$. Así, tanto $n.izquierda$ como $n.derecha$ están marcados cada uno ya sea en $v_{izq}$ o en $v_{der}$. Mostremos, por último, que uno es marcado po $v_{izq}$ y el otro es marcado por $v_{der}$. Como el ancestro común más profundo de $n.izquierda$ y $n.derecha$ es $n$, y $f(n)=g$, entonces el ancestro común más profundo de $f(n.izquierda)$ y $f(n.derecha)$ debe de ser $g$. Luego, $f(n.izquierda)$ está en una rama de $g$ y $f(n.derecha)$ está en la otra. Supongamos sin pérdida de generalidad que $f(n.izquierda)$ está en la rama izquierda de $g$ y $f(n.derecha)$ está en la rama derecha. Como $f\mid_{V(árbol(n.izquierda))}$ es una función de coasignación de $árbol(n.izquierda)$ a $árbol(g.izquierda)$ y por H.I., entonces $v_{izq}$ marca a $n.izquierda$. De forma análoga, $v_{der}$ marca a $n.derecha$. Concluyendo, como $n.etiqueta = r.etiqueta$ y tanto $n.izquierda$ como $n.derecha$ son marcados uno por $v_{izq}$ y el otro por $v_{der}$, se cumple la condición de la línea 10 y $func$ marca a $n$. Así, al final de la ejecución del algoritmo, $n$ estará marcado.
    
\end{proof}

Dado que, para cada nodo de $G$, se crea una función de verificación cuyo dominio es el conjunto de los nodos de $H$, el tiempo de ejecución del algoritmo crece de la forma $O(\mid V(G) \mid \mid V(H) \mid)$. 

\subsubsection{Determinar si una cográfica es subcográfica de otra}

Haciendo uso del Algoritmo \ref{alg_subcoarbol}, se puede idear otro algoritmo para determinar si una cográfica, $H$ es subgráfica de otra cográfica, $G$, al buscar todas las formas del coárbol binario de $H$ en un solo coárbol binario de $G$. 

\begin{algorithm}[h]
\caption{Es\_subgráfica}
\label{alg_subgraph}
\DontPrintSemicolon % Some LaTeX compilers require you to use \dontprintsemicolon instead
\KwIn{$g$ y $h$, las raíces de dos coárboles, $G$ y $H$ respectivamente.}
\KwOut{$verdadero$ si la cográfica representada por $H$ es subgráfica de la cográfica representada por $G$. $falso$ en el caso contrario.}

$g\_bin \gets \text{CrearÁrbolBinario}(g)$\;
$h\_bins \gets \text{las raíces de todos los coárboles binarios correspondientes a } H$\;

\ForEach{bin \textbf{\emph{en}} h\_bins}{
    $f = \text{Función\_de\_coasignación}(g\_bin,bin)$\;
    \If{f(bin) = marcado}{
        $\Return\ verdadero$\;
    }
}

$\Return\ falso$\;
    
\end{algorithm}

Como la línea 1 Algoritmo \ref{alg_subgraph} se ejecuta en tiempo $O(\mid V(G) \mid)$, la complejidad temporal de éste depende del número de coárboles binarios correspondientes a $H$ (que crece con mayor rapidez). Sin embargo, si se fija $H$, la complejidad temporal de éste es simplemente  $O(\mid V(G) \mid)$. Fijar $H$ resultará útil cuando se esté resolviendo un problema específico como el de encontrar una obstrucción mínima en una gráfica.


\section{La clase $M_2$}

    En el presente capítulo se desarrolla el tema principal de la tesis, del cual se desprenden el resto de los resultados.

    \begin{definition}
        La \textbf{\emph{clase $M_2$}} es la clase de cográficas cuyo conjunto de vértices acepta una partición en dos partes tal que cada parte induce una gráfica multipartita completa. %Es decir que $M_2 = \{G \mid G $ es una cográfica y existe una partición de $ V = (A,B) $ tal que $ G(A) $ y $ G(B) $  son gráficas multipartitas completas$\}$.
    \end{definition}

    Claramente, $M_2$ es una clase hereditaria, pues si tomamos una gráfica de esta clase y sustraemos uno de sus vértices, de cualquiera de sus dos partes, dicha parte seguirá siendo una gráfica multipartita completa. Al ser una clase hereditaria, $M_2$ puede ser caracterizada por un conjunto de obstrucciones mínimas. También se puede decidir si una cográfica pertenece a ésta en tiempo lineal \cite{unknown}.

    \subsection{Obstrucciones mínimas}
        El primer paso en nuestro estudio de la clase $M_2$ es caracterizar a la
misma a través de su conjunto de obstrucciones mínimas, lo cual se realiza 
en el Teorema \ref{teo_obsts_m2}, cuya demostración requiere del Teorema
\ref{teo_paw} y del Lema \ref{lema_bipartitas}. Recordemos que
la gr\'afica conocida como \textbf{\emph{Paw}} es la gr\'afica obtenida de
$K_3$ al agregar un v\'ertice nuevo y hacerlo adyacente a exactamente un
v\'ertice de $K_3$, o bien $K_1 \oplus (K_1 + K_2)$.

\begin{theorem}[\cite{Olariu}] \label{teo_paw}
	Sea $G$ una gráfica perfecta, $G$ es libre de $Paw$ si y sólo si cada componente de $G$ es libre de $K_3$ o multipartita completa.
\end{theorem}

Aplicando este teorema, podemos concluir lo siguiente. Dado que las cográficas son gráficas perfectas y toda cográfica libre de $K_3$ es bipartita, si una cográfica $G$ es libre de $Paw$, entonces $G$ es bipartita o multipartita completa.

\begin{lemma} \label{lema_bipartitas}
Sea $G$ una cográfica conexa. Si $G$ es bipartita, entonces $G$ es bipartita completa.
\end{lemma}

\begin{proof}
Sea $r$ la raíz del coárbol de $G$. Si $G$ es trivial, es claro que $G$ es bipartita completa. En el caso contrario, $r$ tiene etiqueta 1. Como $G$ es bipartita, entonces es libre de $K_3$. Luego, $r$ tiene exactamente dos hijos, ninguno de los cuales puede contener un $K_2$. Así cada uno de los hijos de $r$ representa a un conjunto independiente. Luego, $G$ es la unión completa de dos conjuntos independientes. Es decir, $G$ es una gráfica bipartita completa.
\end{proof}

\begin{theorem} \label{teo_obsts_m2}

    Para una cográfica $G$, las siguientes afirmaciones son equivalentes.
    \begin{enumerate}[(a)]
        \item $G \in M_2$.
        \item $G$ no contiene a ninguna de las gráficas de las Figuras \ref{obsts_O_M3} como subgráficas inducidas.
    \end{enumerate}

\end{theorem}

\begin{figure}[ht!]
\begin{center}
\begin{tikzpicture}

\begin{scope}[xshift=0cm,scale=1]

\node [style=vertex] (1) at (0,0) {};
\node [style=vertex] (2) at (1,0) {};
\node [style=vertex] (3) at (0,0.5) {};
\node [style=vertex] (4) at (1,0.5) {};
\node [style=vertex] (5) at (0.5,1.25) {};
\node [style=vertex] (6) at (0.5,2) {};
\foreach \i/\j in {1/2,3/4,3/5,4/5}
  \draw [style=edge] (\i) to (\j);
\node [below of=1,xshift=.5cm]
{\parbox{0.3\linewidth}{\subcaption*{$H$}}};

\end{scope}

\begin{scope}[xshift=3cm,scale=1]

\node [style=vertex] (1) at (0,0) {};
\node [style=vertex] (2) at (1,0) {};
\node [style=vertex] (3) at (0,0.5) {};
\node [style=vertex] (4) at (1,0.5) {};
\node [style=vertex] (5) at (0.5,1.25) {};
\node [style=vertex] (6) at (0.5,2) {};
\foreach \i/\j in {1/2,3/4,3/5,4/5,5/6}
  \draw [style=edge] (\i) to (\j);
\node [below of=1,xshift=.5cm]  {\parbox{0.3\linewidth}{\subcaption*{$I$}}};

\end{scope}

\begin{scope}[xshift=6cm,scale=1]

\node [style=vertex] (1) at (0,0) {};
\node [style=vertex] (2) at (0.5,0.5) {};
\node [style=vertex] (3) at (1.5,0.5) {};
\node [style=vertex] (4) at (0.5,1.5) {};
\node [style=vertex] (5) at (1.5,1.5) {};
\node [style=vertex] (6) at (0,2) {};
\node [style=vertex] (7) at (2,1) {};

\foreach \i/\j in {1/2,1/3,1/6,2/3,2/4,2/5,3/4,3/5,4/5,4/6,5/6}
  \draw [style=edge] (\i) to (\j);
\node [below of=1,xshift=1cm] {\parbox{0.3\linewidth}{\subcaption*{$J$}}};

\end{scope}
\end{tikzpicture}
\end{center}
\setlength{\abovecaptionskip}{-15pt}
\caption{Obstrucciones mínimas para la clase $M_2$.}
\label{obsts_O_M3}
\end{figure}

\begin{proof}

  Notemos que las gráficas $H, I$ y $J$ pueden ser construidas de la siguiente manera:

  \begin{enumerate}[(1)]
      \item $H = K_1 + K_2 + K_3$.
      \item $I = Paw + K_2$.
      \item $J = (\overline{P_3} \oplus \overline{P_3}) + K_1$.
  \end{enumerate}

  Supongamos primero que $G \in M_2$ y procedamos probando la contrapositiva
  de la afirmación. Es decir, probemos que que si $G$ tiene a $H$, a $I$
  o a $J$ como subgráficas inducidas, entonces $G \notin M_2$. Para ello basta
  con que probemos que ninguna de estas gráficas es un elemento de $M_2$.
  Veamos que ninguna partici\'on en dos partes es una $M_2$-partici\'on.
  Sea $(X,Y)$ una partici\'on de $G$. Si ambos vértices de $K_2$ en $H$
  se encuentran en $X$, entonces la existencia de cualquier vértice adicional
  en $X$, implicar\'ia que $G[X]$ contiene un $\overline{P_3}$. Como los
  vértices restantes inducen una gráfica que no es multipartita completa,
  esta partici\'on no es una $M_2$-partici\'on. Supongamos entonces que un
  v\'ertice de $K_2$ est\'a en $X$ y el otro en $Y$. Como la gráfica inducida
  por los vértices restantes de $H$ contiene un $K_3$, podemos suponer sin
  p\'erdida de generalidad que dos de sus v\'ertices se encuentran en $X$,
  por lo que $X$ contiene una copia inducida de $\overline{P_3}$. Como la
  partici\'on fue elegida arbitrariamente, tenemos que $H$ no pertenece a
  $M_2$. El argumento para probar que $I$ no está en $M_2$ es an\'algo al
  anterior.

  Por otra parte, como $J$ tiene un vértice aislado, para que admitiera
  una $M_2$-partici\'on, el resto de sus vértices (que inducen un
  $\overline{P_3} \oplus \overline{P_3}$) deben de poder dividirse en dos
  partes de manera tales que una induzca un conjunto independiente y la otra
  una gráfica multipartita completa. Siempre que tomamos uno de los vértices
  de uno de los dos $\overline{P_3}$ para formar el conjunto independiente,
  ninguno los vértices del otro $\overline{P_3}$ puede ser agregado al mismo,
  pues es adyacente al vértice que agregamos primero. Así, la subgráfica
  inducida $\overline{P_3} \oplus \overline{P_3}$ no acepta una partición en
  un conjunto independiente y una gráfica multipartita completa. Luego, $J$
  no está en $M_2$.

  Recíprocamente, supongamos que $G$ es libre de $H$, $I$ y $J$ y probemos
  que es elemento de $M_2$. Para ello consideramos
  cuatro casos que son exhaustivos. Los primeros tres casos cubren toda
  situación en la que $G$ es inconexa, mientras que el último caso se cumple
  si $G$ es conexa.

  \emph{Caso 1:} $G$ tiene al menos dos componentes conexas no triviales.

  Consideremos la partición de $V$ en dos partes $(A,B)$ tal que $A$ contiene
  únicamente una componente no trivial y $B$ el resto. Como $G[A]$ y $G[B]$
  contienen ambas componentes no triviales, las dos poseen un $K_2$. Luego,
  ni $G[B]$ ni $G[A]$ pueden contener un $Paw$, o $G$ tendría a $I$ como
  subgráfica inducida. Dado que $G[A]$ y $G[B]$ son cográficas, son también
  gráficas perfectas, y al ninguna tener un $Paw$ como subgráfica inducida,
  cada una es bipartita o multipartita completa \cite{Olariu}.

  % Realmente no se acostumbra a argumentar así cuando usas un resultado
  % de alguien más.  Lo común es escribir el teorema, citando la fuente
  % (algo como \begin{theorem}[\cite{Olariu}] ...), etiquetarlo, y referirte
  % al teorema dentro de tu trabajo.   De otra forma, le estás dejando al
  % lector el trabajo de buscar el artículo, y tratar de adivinar qué
  % resultado estás aplicando.

  Si tanto $G[A]$ como $G[B]$ son gráficas multipartitas completas,
  entonces $G \in M_2$. Si ambas son bipartitas, entonces $G$ es bipartita
  también y acepta una partición en dos conjuntos independientes, cada uno
  de los cuales es una gráfica multipartita completa, por lo que $G \in M_2$.
  Si $G[A]$ es bipartita y $G[B]$ es multipartita completa, como $G[A]$ es
  una cográfica conexa, entonces es una gráfica multipartita completa y $G
  \in M_2$.

  % El último enunciado necesita una referencia al resultado de que toda
  % cográfica bipartita es bipartita completa.   Que por cierto, no recuerdo
  % haber visto.   Creo que eso es algo que podría ir en el capítulo de
  % antecedentes, sección de cográficas.

  Finalmente, si $G[A]$ es multipartita completa y $G[B]$ es bipartita. Si
  $G[B]$ tiene una sola componente, $G[B]$ es bipartita completa y $G \in
  M_2$. Si $G[B]$ tiene más de una componente, como al menos una es no
  trivial, debe tener a $\overline{P_3}$ como subgráfica inducida. Luego,
  $G[A]$ debe ser libre de $K_3$ o $G$ tendría a $H$ como subráfica inducida.
  Así, $G[A]$ es bipartita. Como ambas son bipartitas, $G \in M_2$.


  \emph{Caso 2:} $G$ tiene exactamente una componente conexa no trivial y
  al menos una trivial.

  Como $G$ contiene al menos una componente trivial, la única partición que
  puede aceptar en dos gráficas multipartitas completas es una partición en
  un conjunto independiente y una gráfica multipartita completa. Luego, la
  componente no trivial de $G$, a la que llamaremos $G'$, debe de aceptar una
  partición en un conjunto independiente y una gráfica multipartita completa.

  Si $G'$ es bipartita, entonces acepta una partición en dos conjuntos
  independientes, y por lo tanto $G \in M_2$. Si $G'$ es una gráfica
  multipartita completa, entonces $G \in M_2$. Si $G$ no es una gráfica
  bipartita ni multipartita completa, dado que es una cográfica, y por lo
  tanto una gráfica perfecta, $G$ contiene un $Paw$. Sea $y$ la raíz del
  coárbol de $G'$ y sea $z$, descendiente de $y$, el nodo más profundo que
  tiene un $Paw$ como subgráfica inducida, probemos por inducción sobre la
  distancia desde $y$ hasta $z$, denotada por $d$, que $G'[y]$ acepta una
  partición en un conjunto independiente y una gráfica multipartita completa.

  \textbf{Caso base}: $d = 0$, o bien, $y = z$.

  Notemos que $z$ tiene etiqueta 1, pues $Paw$ es una gráfica conexa. Dado
  que $z$ tiene etiqueta 1, todos sus hijos inducen gráficas multipartitas
  completas menos uno, $w$, que tiene etiqueta 0. Mostremos por contradicción
  que todos los hijos de $w$ inducen gráficas multipartitas completas.
  Supongamos que alguno de los hijos de $w$ contiene un $\overline{P_3}$.
  Como el nodo más profundo que contiene un $\overline{P_3}$ debe tener
  etiqueta 0, $w$ tiene un hijo de etiqueta 1 y éste a su vez tiene al menos
  2 hijos, uno de los cuales contiene a $\overline{P_3}$ y el otro que tiene
  al menos un $K_1$. Luego, dicho hijo contiene un $Paw$, lo que es una
  contradicción.

  Si $w$ tiene un sólo hijo que no es un vértice, el resto de sus hijos
  forman un conjunto independiente, $C$. Si eliminamos este conjunto
  independiente, como el único hijo de $w$ que queda induce una gráfica
  multipartita completa, entonces $G'[w] - C$ es una gráfica multipartita
  completa. Luego, $G'[z] - C$ es la unión completa de varias gráficas
  multipartitas completas y por lo tanto es una gráfica multipartita
  completa. De esto se sigue que $G[z]$ acepta una partición en un conjunto
  independiente, $C$, y una gráfica multipartita completa $G'[z] - C$.

  Si $w$ tiene al menos dos hijos no triviales, notemos que ninguno de
  ellos puede contener a $K_3$, o de lo contrario $w$ contendría a $K_2+K_3$
  y $G$ no sería libre de $I$. Luego, todos los hijos de $w$ inducen gráficas
  bipartitas, es decir que $w$ induce también una gráfica bipartita. En otras
  palabras, $G'[w]$ acepta una partición en dos conjuntos independientes. Si
  sustraemos uno de estos conjuntos independientes, denotado por $D$, entonces
  $G'[w]-D$ es un conjunto independiente. Luego $G'[z]-D$ es la unión completa
  de al menos una gráfica multipartita completa y un conjunto independiente.
  Así, $G'[z]-D$ es una gráfica multipartita completa. Luego, $G'[z]$ acepta
  una partición en un conjunto independiente, $D$ y una gráfica multipartita
  completa, $G'[z] - D$.

  Como en todos los casos $z$ acepta una partición en un conjunto
  independiente y una gráfica multipartita completa y $y = z$, entonces $y$
  acepta la misma partición.

  \textbf{Paso inductivo}: $d \ge 2$.

  Notemos que $d$ siempre será par, ya que tanto $y$ como $z$ son nodos con
  etiqueta 1. Sea $k$ un entero tal que $k \ge 2$. Supongamos, como hipótesis
  inductiva, que si $G''$ es una cográfica conexa libre de $H, I$ y $J$ tal
  que la distancia, $d'$, entre la raíz, $y'$ de su coárbol y el nodo más
  profundo que contiene un $Paw$ es igual a $k-2$, entonces $G''$ acepta una
  partición en un conjunto independiente y una gráfica multipartita completa.

  Dado que $G'$ es libre de $J$, todos los hijos de $y$, menos uno, inducen
  gráficas multipartitas completas. Dicho hijo, $v$, tiene etiqueta 0 y al
  menos uno de sus hijos debe de contener un $Paw$. Denotemos a dicho hijo
  como $u$. El resto de los hijos de $v$ deben de ser vértices, o de lo
  contrario, $G'[v]$ contendría a $K_2 + K_3$ como subgráfica inducida, por
  lo que $G$ contendría a $I$. Denotemos a este conjunto de vértices como
  $E$. Luego, $G'[u]$  es una cográfica que cumple con las condiciones de
  la hipótesis inductiva, por lo que acepta una partición en un conjunto
  independiente, $D$ y una gráfica multipartita completa. Tenemos entonces
  que $G'[u] - D$ es una gráfica multipartita completa. Se sigue que
  $(G'[v]-D)-E$ es una gráfica multipartita completa. Luego, $(G'-D)-E$ es
  una unión completa de gráficas multipartitas completas por lo que también
  es una gráfica multipartita completa. Notemos que, dado que $v$ tiene
  etiqueta 0, no existen aristas entre los vértices en $D$ y los vértices en
  $E$, es decir que $D \cup E$ es un conjunto independiente. Así, $G'$ acepta
  una partición en un conjunto independiente, $D \cup E$ y una gráfica
  multipartita completa, $(G' - D) - E$.

  Como $G'$ acepta una partición en un conjunto independiente y una gráfica
  multipartita completa, entonces $G \in M_2$.


  \emph{Caso 3:} $G$ es un conjunto independiente con al menos dos vértices.

  Dado que $G$ es una gráfica multipartita completa, se sigue inmediatamente
  que está en $M_2$.

  \emph{Caso 4:} $G$ es conexa.

  Dado que toda cográfica inconexa libre de $H$, $I$ y $J$ acepta una
  partición en dos gráficas multipartitas completas, y como una cográfica
  conexa es o un vértice aislado o una unión completa de cográficas
  inconexas se sigue de los casos anteriores y del hecho de que la clase
  $M_2$ es cerrada bajo uniones completas (Lema
  \ref{lema_union_completa}), que $G \in M_2$.

\end{proof}


    \subsection{Reconocimiento de la clase $M_2$}
        Haciendo uso  del Algoritmo \ref{alg_esta_en_clase} se puede determinar si una cográfica pertenece o no a la clase $M_2$. Como se especificó en la Sección \ref{sec_AlgoSub}, el tiempo de este algoritmo crece de forma lineal de acuerdo con el tamaño de la gráfica de entrada si encontramos primero todos los coárboles binarios de las obstrucciones de la clase. Como conocemos las obstrucciones mínimas de la clase $M_2$, que son finitas, se puede buscar cada una en tiempo lineal y por lo tanto se puede reconocer si una cográfica pertenece a la clase $M_2$ en tiempo lineal. El Algoritmo \ref{alg_decision}, que es una instancia del Algoritmo \ref{alg_esta_en_clase} corresponde a este proceso. Los árboles binarios de cada una de las obstrucciones de la clase $M_2$ se muestran en la Figura \ref{fig_obsts_bin}.

\begin{figure}[ht!]
\centering

\begin{subfigure}{0.85\textwidth}
\begin{tikzpicture}

\begin{scope}[xshift=0cm,scale=1]
\node [style=cotreenode] (1) at (1,1) {0};
\node [style=cotreenode] (2) at (0,0) {0};
\node [style=cotreenode] (3) at (2,0) {1};
\node [style=vertex] (4) at (-0.5,-1) {};
\node [style=cotreenode] (5) at (0.5,-1) {1};
\node [style=vertex] (6) at (1.5,-1) {};
\node [style=cotreenode] (7) at (2.5,-1) {1};
\node [style=vertex] (8) at (0.25,-2) {};
\node [style=vertex] (9) at (0.75,-2) {};
\node [style=vertex] (10) at (2.25,-2) {};
\node [style=vertex] (11) at (2.75,-2) {};
\foreach \i/\j in {1/2,1/3,2/4,2/5,3/6,3/7,5/8,5/9,7/10,7/11}
  \draw [style=edge] (\i) to (\j);
\node [below of=9,xshift=0.25cm] {\parbox{0.3\linewidth}{\subcaption*{$H_1$}}};
\end{scope}

\begin{scope}[xshift=4.5cm,scale=1]
\node [style=cotreenode] (1) at (1,1) {0};
\node [style=cotreenode] (2) at (0,0) {0};
\node [style=cotreenode] (3) at (2,0) {1};
\node [style=vertex] (4) at (-0.5,-1) {};
\node [style=cotreenode] (5) at (0.5,-1) {1};
\node [style=vertex] (6) at (1.5,-1) {};
\node [style=vertex] (7) at (2.5,-1) {};
\node [style=vertex] (8) at (0.125,-2) {};
\node [style=cotreenode] (9) at (0.875,-2) {1};
\node [style=vertex] (10) at (0.625,-3) {};
\node [style=vertex] (11) at (1.125,-3) {};
\foreach \i/\j in {1/2,1/3,2/4,2/5,3/6,3/7,5/8,5/9,9/10,9/11}
  \draw [style=edge] (\i) to (\j);
\node [below of=11] {\parbox{0.3\linewidth}{\subcaption*{$H_2$}}};
\end{scope}

\begin{scope}[xshift=9cm,scale=1]
\node [style=cotreenode] (1) at (1,1) {0};
\node [style=cotreenode] (2) at (0,0) {0};
\node [style=vertex] (3) at (2,0) {};
\node [style=cotreenode] (4) at (-0.5,-1) {1};
\node [style=cotreenode] (5) at (0.5,-1) {1};
\node [style=vertex] (8) at (0.125,-2) {};
\node [style=cotreenode] (9) at (0.875,-2) {1};
\node [style=vertex] (10) at (0.625,-3) {};
\node [style=vertex] (11) at (1.125,-3) {};
\node [style=vertex] (12) at (-0.75,-2) {};
\node [style=vertex] (13) at (-0.25,-2) {};
\foreach \i/\j in {1/2,1/3,2/4,2/5,5/8,5/9,9/10,9/11,4/12,4/13}
  \draw [style=edge] (\i) to (\j);
\node [below of=11] {\parbox{0.3\linewidth}{\subcaption*{$H_3$}}};
\end{scope}


\end{tikzpicture}
\end{subfigure}


\begin{subfigure}{0.6\textwidth}
\begin{tikzpicture}

\begin{scope}[xshift=0cm,scale=1]
\node [style=cotreenode] (1) at (1,1) {0};
\node [style=cotreenode] (2) at (0,0) {1};
\node [style=cotreenode] (3) at (2,0) {1};
\node [style=vertex] (4) at (-0.5,-1) {};
\node [style=vertex] (5) at (0.5,-1) {};
\node [style=vertex] (6) at (1.5,-1) {};
\node [style=cotreenode] (7) at (2.5,-1) {0};
\node [style=vertex] (8) at (2.125,-2) {};
\node [style=cotreenode] (9) at (2.875,-2) {1};
\node [style=vertex] (10) at (2.625,-3) {};
\node [style=vertex] (11) at (3.125,-3) {};

\foreach \i/\j in {1/2,1/3,2/4,2/5,3/6,3/7,7/8,7/9,9/10,9/11}
  \draw [style=edge] (\i) to (\j);
\node [below of=10,xshift=-1.625cm] {\parbox{0.3\linewidth}{\subcaption*{$I_1$}}};
\end{scope}

\begin{scope}[xshift=5cm,scale=1]
\node [style=cotreenode] (1) at (1,1) {0};
\node [style=vertex] (2) at (0,0) {};
\node [style=cotreenode] (3) at (2,0) {1};
\node [style=cotreenode] (4) at (1.25,-1) {0};
\node [style=cotreenode] (5) at (2.75,-1) {0};
\node [style=cotreenode] (6) at (0.825,-2) {1};
\node [style=vertex] (7) at (1.625,-2) {};
\node [style=vertex] (8) at (2.375,-2) {};
\node [style=cotreenode] (9) at (3.125,-2) {1};
\node [style=vertex] (10) at (0.575,-3) {};
\node [style=vertex] (11) at (1.075,-3) {};
\node [style=vertex] (12) at (2.875,-3) {};
\node [style=vertex] (13) at (3.375,-3) {};


\foreach \i/\j in {1/2,1/3,3/4,3/5,4/6,4/7,5/8,5/9,6/10,6/11,9/12,9/13}
  \draw [style=edge] (\i) to (\j);
\node [below of=12,xshift=-1.625cm] {\parbox{0.3\linewidth}{\subcaption*{$J_1$}}};
\end{scope}
\end{tikzpicture}
\end{subfigure}
\setlength{\abovecaptionskip}{5pt}
\caption{$H_1, H_2$ y $H_3$ son los coárboles binarios correspondientes a la obstrucción $H$. El coárbol binario $I_1$ es el único que corresponde a la obstrucción $I$. El coárbol binario $J_1$ es el único que corresponde a la obstrucción $J$.}\label{fig_obsts_bin}


\end{figure}


\begin{algorithm}[ht!]
\caption{Pertenece\_a\_M2}
\label{alg_decision}
\DontPrintSemicolon % Some LaTeX compilers require you to use \dontprintsemicolon instead
\KwIn{$g$, la raíz del coárbol de una cográfica $G$.}
\KwOut{$verdadero$ si $G$ pertenece a la clase hereditaria de cográficas $C$}

$g\_bin \gets \text{CrearÁrbolBinario}(g)$\;
$C\_bins \gets \{H_1, H_2, H_3, I_1, J_1\}$\;

\ForEach{$bin$ \textbf{\emph{en}} $C\_bins$}{
    $f = \text{Función\_de\_coasignación}(g\_bin,bin)$\;
    \If{$f(bin) = marcado$}{
        $\Return\ verdadero$\;
    }
}

$\Return\ falso$\;

\end{algorithm}




    \subsection{Algoritmo certificador}
        Si bien, el Algoritmo \ref{alg_decision} es capaz de identificar a las cográficas que pertenecen a la clase $M_2$, no es posible determinar a partir de éste cuáles son las dos partes en las que se puede dividir una gráfica de la clase. La presente subsección muestra un algoritmo (Algoritmo \ref{alg_cert_m2}) que, dada una cográfica $G$ representada por su coárbol $T$, devuelve una coloración de las hojas de este último. Si $G$ pertenece a $M_2$, cada una de las hojas de $T$ tendrá uno de dos colores, $verde$ o $azul$. Cada uno de estos cuales corresponde a una parte de la $M_2$-partición de $G$. En el caso contrario, las hojas de $T$ correspondientes a los vértices que forman una obstrucción mínima tendrán un color que indique de qué obstrucción mínima se trata ($amarillo$ para $H$, $anaranjado$ para $I$ y $rojo$ para $J$). El Algoritmo \ref{alg_cert_m2} hace uso de los Algoritmos \ref{alg_cert_caso1} y \ref{alg_cert_caso2}, que funcionan de la misma manera para casos específicos del problema. La correctitud de estos algoritmo se sigue de la demostración del Teorema \ref{teo_obsts_m2}

\subsubsection{Algoritmo para reconocer gráficas bipartitas completas conexas}

El Algoritmo \ref{alg_bpc} es un algoritmo que resulta útil para los algoritmos subsecuentes. Éste recibe la raíz de un coárbol, $g$, y devuelve $verdadero$ si la gráfica representada por dicho coárbol es una gráfica bipartita completa conexa, coloreando los vértices de una parte de color $azul$ y los de la otra parte de $verde$. En el caso contrario, se colorean con $amarillo$  tres hojas cuyo ancestro común más profundo sea un nodo con etiqueta 1. Es decir que se colorean los vértices que inducen un $K_3$ en la gráfica. El bloque de la línea 9 a la 28 se ejecuta sólo si $g$ tiene exactamente dos hijos. En las líneas 10 a 18 se busca un $K_3$ en el primer hijo de $g$ y en las líneas 19 a 27 se busca en el segundo hijo. La Figura \ref{fig_bipartita} muestra el resultado de la ejecución de este algoritmo para algunos coárboles.

\begin{algorithm}[!htbp]
\caption{Es\_bipartita_completa}
\label{alg_bpc}

\DontPrintSemicolon % Some LaTeX compilers require you to use \dontprintsemicolon instead
\KwIn{$g$, la raíz de un coárbol, $G$}
\KwOut{Verdadero si la gráfica representada por $G$ es bipartita completa. Falso en el caso contrario. Las hojas de $\acute{a}rbol(g)$ se colorean.}

\If{g \text{es una hoja}}{
    $g.color \gets azul$\;
    $\Return\ verdadero$\;
}
\ElseIf{$g.etiqueta = 0$}{
    $\Return\ falso$\;
}
\ElseIf{$g.hijos.tama\tilde{n}o > 2$}{
    Marcar con amarillo: una hoja en cada uno de tres hijos diferentes de $g$\;
    $\Return\ falso$\;
}
\Else(\tcp*[h]{Hay exactamente dos hijos}){
    \If{$g.hijos\emph{[0]}$ es una hoja}{
        $g.hijos[0].color \gets azul$\;
    }
    \Else{
        \ForEach{$gchild$ \textbf{\emph{en}} $g.hijos\emph{[0]}.hijos$}{
            \If{$gchild$ es una hoja}{
                $gchild.color \gets azul$\;
            }
            \Else{
                Marcar con amarillo: dos hojas que tengan como ancestro común más profundo a $gchild$ y una hoja descendiente de $g.hijos[1]$\;
                $\Return\ falso$\;
            }
        }
    }
    \If{$g.hijos\emph{[1]}$ es una hoja}{
        $g.hijos[1].color \gets verde$\;
    }
    \Else{
        \ForEach{$gchild$ \textbf{\emph{en}} $g.hijos\emph{[1]}.hijos$}{
            \If{gchild \text{es una hoja}}{
                $gchild.color \gets verde$\;
            }
            \Else{
                Marcar con amarillo: dos hojas que tengan como ancestro común más profundo a $gchild$ y una hoja descendiente de $g.hijos[0]$\;
                $\Return\ falso$\;
            }
        }
    }
}

$\Return\ verdadero$\;

\end{algorithm}

\begin{figure}[!htbp]
\begin{center}
\begin{tikzpicture}

\begin{scope}[xshift=0cm,scale=1]
\node [style=cotreenode] (1) at (1,3) {1};
\node [style=vertex, fill=blue] (2) at (0.5,2) {};
\node [style=vertex, fill=green] (3) at (1.5,2) {};
\foreach \i/\j in {1/2,1/3}
  \draw [style=edge] (\i) to (\j);
\end{scope}

\begin{scope}[xshift=2cm,scale=1]
\node [style=cotreenode] (1) at (1,3) {1};
\node [style=vertex, fill=blue] (2) at (0.5,2) {};
\node [style=cotreenode] (3) at (1.5,2) {0};
\node [style=vertex, fill=green] (4) at (1,1) {};
\node [style=vertex, fill=green] (5) at (1.5,1) {};
\node [style=vertex, fill=green] (6) at (2,1) {};
\foreach \i/\j in {1/2,1/3,3/4,3/5,3/6}
  \draw [style=edge] (\i) to (\j);
\end{scope}

\begin{scope}[xshift=4.75cm,scale=1]
\node [style=cotreenode] (1) at (1,3) {1};
\node [style=cotreenode] (2) at (0.5,2) {0};
\node [style=cotreenode] (3) at (1.5,2) {0};
\node [style=vertex, fill=blue] (4) at (0.25,1) {};
\node [style=vertex, fill=blue] (5) at (0.5,1) {};
\node [style=vertex, fill=blue] (6) at (0.75,1) {};
\node [style=vertex, fill=green] (7) at (1.25,1) {};
\node [style=vertex, fill=green] (8) at (1.5,1) {};
\node [style=vertex, fill=green] (9) at (1.75,1) {};

\foreach \i/\j in {1/2,1/3,2/4,2/5,2/6,3/7,3/8,3/9}
  \draw [style=edge] (\i) to (\j);
\end{scope}

\begin{scope}[xshift=7cm,scale=1]
\node [style=cotreenode] (1) at (1,3) {1};
\node [style=vertex, fill=yellow] (2) at (0.5,2) {};
\node [style=vertex, fill=yellow] (3) at (1,2) {};
\node [style=vertex, fill=yellow] (4) at (1.5,2) {};
\foreach \i/\j in {1/2,1/3,1/4}
  \draw [style=edge] (\i) to (\j);
\end{scope}

\begin{scope}[xshift=9.25cm,scale=1]
\node [style=cotreenode] (1) at (1,3) {1};
\node [style=cotreenode] (2) at (0.5,2) {0};
\node [style=cotreenode] (3) at (1.5,2) {0};
\node [style=vertex, fill=yellow] (4) at (0.25,1) {};
\node [style=vertex, fill=blue] (6) at (0.75,1) {};
\node [style=vertex, fill=green] (7) at (1.25,1) {};
\node [style=cotreenode] (9) at (1.75,1) {1};
\node [style=vertex, fill=yellow] (10) at (1.5,0) {};
\node [style=vertex, fill=yellow] (11) at (2,0) {};

\foreach \i/\j in {1/2,1/3,2/4,2/6,3/7,3/9,9/10,9/11}
  \draw [style=edge] (\i) to (\j);
\end{scope}

\end{tikzpicture}
\end{center}
\caption{Ejemplos del resultado de la ejecución del Algoritmo \ref{alg_bpc}.}
\label{fig_bipartita}
\end{figure}



\subsubsection{Caso 1}

El algoritmo \ref{alg_cert_caso1} corresponde al $Caso\ 1$ de la demostración del Teorema \ref{teo_obsts_m2}. Éste recibe como entrada la raíz de un coárbol que representa una cográfica inconexa que tiene al menos dos componentes conexas no triviales. En el bloque de las líneas 1 a 12 se aborda el caso en el que la gráfica tiene exactamente dos componentes conexas y se busca un $Paw$ que pueda formar la obstrucción $I$. En el bloque de las líneas 13 a 17 se aborda el caso en el que hay al menos 3 componente conexas y se busca un $K_3$ en cada componente para formar la obstrucción $H$. Si no se encuentra ninguna de las obstrucciones mínimas, se devuelve $verdadero$ y cada una de las hojas del coárbol tendrán color $azul$ o $verde$. Las Figuras \ref{fig_certificador_caso1_01} y \ref{fig_certificador_caso1_02} muestran la ejecución del algoritmo para gráficas sin ninguna de las obstrucciones. La Figura \ref{fig_certificador_caso1_03} muestra el resultado de la ejecución para tres gráficas, cada una de las cuales contiene una obstrucción.

\begin{algorithm}[!htbp]
\small
\caption{M2\_Caso\_1}
\label{alg_cert_caso1}

\DontPrintSemicolon % Some LaTeX compilers require you to use \dontprintsemicolon instead
\KwIn{$g$, la raíz de un coárbol con etiqueta 0 y al menos dos hijos que no son hojas}
\KwOut{Verdadero si $G$ pertenece a la clase $M_2$. Falso en el caso contrario. Las hojas de $\acute{a}rbol(g)$ se colorean.}

\If{$g.hijos.tama\tilde{n}o = 2$}{
    \For{gchild \textbf{\emph{en}} g.hijos\emph{[0]}}{
        \If{$gchild$ es una hoja}{
            $gchild.color \gets azul$\;
        }
        \Else{
            \ForEach{$ggchild$ \textbf{\emph{en}} $gchild.hijos$}{
                \If{$ggchild$ es una hoja}{
                    $gchild.color \gets azul$\;
                }
                \Else(\tcp*[h]{Se marca la obstrucción $I$}){
                    Marcar con anaranjado: una hoja en $ggchild.hijos[0]$, una hoja en $ggchild.hijos[1]$, una hoja en un hermano de $ggchild$, una hoja en un hermano de $gchild$ y dos hojas cuyo ancestro común más profundo sea el hermano de $g.hijos[0]$\;

                    $\Return\ falso$\;
                }
            }
        }
    }
    Repetir el procedimiento de las líneas 2 a 11 para $g.hijos[1]$, pero marcando con color $verde$ en vez de $azul$\;
}
\Else{
    \For{$child$ \textbf{\emph{en}} $g.hijos$}{
        \If{\emph{Es\_bipartita\_completa(}$child$\emph{)} $= falso$}{
             Marcar con amarillo: dos hojas cuyo ancestro común más profundo sea un hermano de $child$ que no sea una hoja y una hoja en un hermano diferente\;
                        $\Return\ falso$\;
        }
    }
}


$\Return\ verdadero$\;

\end{algorithm}


\begin{figure}[!htbp]
\centering

\begin{subfigure}{\textwidth}
\centering
\begin{tikzpicture}
\begin{scope}[xshift=0cm,scale=1]
\node [style=cotreenode, fill=lightgray] (1) at (1.5,4) {0};
\node [style=cotreenode] (2) at (0.5,3) {1};
\node [style=cotreenode] (3) at (2.5,3) {1};
\node [style=vertex] (4) at (0,2) {};
\node [style=vertex] (6) at (1,2) {};
\node [style=vertex] (7) at (2,2) {};
\node [style=cotreenode] (8) at (3,2) {0};
\node [style=vertex] (9) at (2.5,1) {};
\node [style=vertex] (10) at (3,1) {};
\node [style=vertex] (11) at (3.5,1) {};
\foreach \i/\j in {1/2,1/3,2/4,2/6,3/7,3/8,8/9,8/10,8/11}
  \draw [style=edge] (\i) to (\j);
\end{scope}
\begin{scope}[xshift=4cm,scale=1]
\node [style=cotreenode, fill=lightgray] (1) at (1.5,4) {0};
\node [style=cotreenode, fill=lightgray] (2) at (0.5,3) {1};
\node [style=cotreenode] (3) at (2.5,3) {1};
\node [style=vertex] (4) at (0,2) {};
\node [style=vertex] (6) at (1,2) {};
\node [style=vertex] (7) at (2,2) {};
\node [style=cotreenode] (8) at (3,2) {0};
\node [style=vertex] (9) at (2.5,1) {};
\node [style=vertex] (10) at (3,1) {};
\node [style=vertex] (11) at (3.5,1) {};
\foreach \i/\j in {1/2,1/3,2/4,2/6,3/7,3/8,8/9,8/10,8/11}
  \draw [style=edge] (\i) to (\j);
\end{scope}
\begin{scope}[xshift=8cm,scale=1]
\node [style=cotreenode, fill=lightgray] (1) at (1.5,4) {0};
\node [style=cotreenode, fill=lightgray] (2) at (0.5,3) {1};
\node [style=cotreenode] (3) at (2.5,3) {1};
\node [style=vertex, fill=lightgray] (4) at (0,2) {};
\node [style=vertex] (6) at (1,2) {};
\node [style=vertex] (7) at (2,2) {};
\node [style=cotreenode] (8) at (3,2) {0};
\node [style=vertex] (9) at (2.5,1) {};
\node [style=vertex] (10) at (3,1) {};
\node [style=vertex] (11) at (3.5,1) {};
\foreach \i/\j in {1/2,1/3,2/4,2/6,3/7,3/8,8/9,8/10,8/11}
  \draw [style=edge] (\i) to (\j);
\end{scope}
\end{tikzpicture}
\end{subfigure}

\begin{subfigure}{\textwidth}
\centering
\begin{tikzpicture}
\begin{scope}[xshift=0cm,scale=1]
\node [style=cotreenode, fill=lightgray] (1) at (1.5,4) {0};
\node [style=cotreenode, fill=lightgray] (2) at (0.5,3) {1};
\node [style=cotreenode] (3) at (2.5,3) {1};
\node [style=vertex, fill=blue] (4) at (0,2) {};
\node [style=vertex, fill=lightgray] (6) at (1,2) {};
\node [style=vertex] (7) at (2,2) {};
\node [style=cotreenode] (8) at (3,2) {0};
\node [style=vertex] (9) at (2.5,1) {};
\node [style=vertex] (10) at (3,1) {};
\node [style=vertex] (11) at (3.5,1) {};
\foreach \i/\j in {1/2,1/3,2/4,2/6,3/7,3/8,8/9,8/10,8/11}
  \draw [style=edge] (\i) to (\j);
\end{scope}
\begin{scope}[xshift=4cm,scale=1]
\node [style=cotreenode, fill=lightgray] (1) at (1.5,4) {0};
\node [style=cotreenode] (2) at (0.5,3) {1};
\node [style=cotreenode, fill=lightgray] (3) at (2.5,3) {1};
\node [style=vertex, fill=blue] (4) at (0,2) {};
\node [style=vertex, fill=blue] (6) at (1,2) {};
\node [style=vertex] (7) at (2,2) {};
\node [style=cotreenode] (8) at (3,2) {0};
\node [style=vertex] (9) at (2.5,1) {};
\node [style=vertex] (10) at (3,1) {};
\node [style=vertex] (11) at (3.5,1) {};
\foreach \i/\j in {1/2,1/3,2/4,2/6,3/7,3/8,8/9,8/10,8/11}
  \draw [style=edge] (\i) to (\j);
\end{scope}
\begin{scope}[xshift=8cm,scale=1]
\node [style=cotreenode, fill=lightgray] (1) at (1.5,4) {0};
\node [style=cotreenode] (2) at (0.5,3) {1};
\node [style=cotreenode, fill=lightgray] (3) at (2.5,3) {1};
\node [style=vertex, fill=blue] (4) at (0,2) {};
\node [style=vertex, fill=blue] (6) at (1,2) {};
\node [style=vertex, fill=lightgray] (7) at (2,2) {};
\node [style=cotreenode] (8) at (3,2) {0};
\node [style=vertex] (9) at (2.5,1) {};
\node [style=vertex] (10) at (3,1) {};
\node [style=vertex] (11) at (3.5,1) {};
\foreach \i/\j in {1/2,1/3,2/4,2/6,3/7,3/8,8/9,8/10,8/11}
  \draw [style=edge] (\i) to (\j);
\end{scope}
\end{tikzpicture}
\end{subfigure}

\begin{subfigure}{\textwidth}
\centering
\begin{tikzpicture}
\begin{scope}[xshift=0cm,scale=1]
\node [style=cotreenode, fill=lightgray] (1) at (1.5,4) {0};
\node [style=cotreenode] (2) at (0.5,3) {1};
\node [style=cotreenode, fill=lightgray] (3) at (2.5,3) {1};
\node [style=vertex, fill=blue] (4) at (0,2) {};
\node [style=vertex, fill=blue] (6) at (1,2) {};
\node [style=vertex, fill=green] (7) at (2,2) {};
\node [style=cotreenode, fill=lightgray] (8) at (3,2) {0};
\node [style=vertex] (9) at (2.5,1) {};
\node [style=vertex] (10) at (3,1) {};
\node [style=vertex] (11) at (3.5,1) {};
\foreach \i/\j in {1/2,1/3,2/4,2/6,3/7,3/8,8/9,8/10,8/11}
  \draw [style=edge] (\i) to (\j);
\end{scope}
\begin{scope}[xshift=4cm,scale=1]
\node [style=cotreenode, fill=lightgray] (1) at (1.5,4) {0};
\node [style=cotreenode] (2) at (0.5,3) {1};
\node [style=cotreenode, fill=lightgray] (3) at (2.5,3) {1};
\node [style=vertex, fill=blue] (4) at (0,2) {};
\node [style=vertex, fill=blue] (6) at (1,2) {};
\node [style=vertex, fill=green] (7) at (2,2) {};
\node [style=cotreenode, fill=lightgray] (8) at (3,2) {0};
\node [style=vertex, fill=lightgray] (9) at (2.5,1) {};
\node [style=vertex] (10) at (3,1) {};
\node [style=vertex] (11) at (3.5,1) {};
\foreach \i/\j in {1/2,1/3,2/4,2/6,3/7,3/8,8/9,8/10,8/11}
  \draw [style=edge] (\i) to (\j);
\end{scope}
\begin{scope}[xshift=8cm,scale=1]
\node [style=cotreenode, fill=lightgray] (1) at (1.5,4) {0};
\node [style=cotreenode] (2) at (0.5,3) {1};
\node [style=cotreenode, fill=lightgray] (3) at (2.5,3) {1};
\node [style=vertex, fill=blue] (4) at (0,2) {};
\node [style=vertex, fill=blue] (6) at (1,2) {};
\node [style=vertex, fill=green] (7) at (2,2) {};
\node [style=cotreenode, fill=lightgray] (8) at (3,2) {0};
\node [style=vertex, fill=green] (9) at (2.5,1) {};
\node [style=vertex, fill=lightgray] (10) at (3,1) {};
\node [style=vertex] (11) at (3.5,1) {};
\foreach \i/\j in {1/2,1/3,2/4,2/6,3/7,3/8,8/9,8/10,8/11}
  \draw [style=edge] (\i) to (\j);
\end{scope}
\end{tikzpicture}
\end{subfigure}

\begin{subfigure}{\textwidth}
\centering
\begin{tikzpicture}
\begin{scope}[xshift=0cm,scale=1]
\node [style=cotreenode, fill=lightgray] (1) at (1.5,4) {0};
\node [style=cotreenode] (2) at (0.5,3) {1};
\node [style=cotreenode, fill=lightgray] (3) at (2.5,3) {1};
\node [style=vertex, fill=blue] (4) at (0,2) {};
\node [style=vertex, fill=blue] (6) at (1,2) {};
\node [style=vertex, fill=green] (7) at (2,2) {};
\node [style=cotreenode, fill=lightgray] (8) at (3,2) {0};
\node [style=vertex, fill=green] (9) at (2.5,1) {};
\node [style=vertex, fill=green] (10) at (3,1) {};
\node [style=vertex, fill=lightgray] (11) at (3.5,1) {};
\foreach \i/\j in {1/2,1/3,2/4,2/6,3/7,3/8,8/9,8/10,8/11}
  \draw [style=edge] (\i) to (\j);
\end{scope}
\begin{scope}[xshift=4cm,scale=1]
\node [style=cotreenode, fill=lightgray] (1) at (1.5,4) {0};
\node [style=cotreenode] (2) at (0.5,3) {1};
\node [style=cotreenode, fill=lightgray] (3) at (2.5,3) {1};
\node [style=vertex, fill=blue] (4) at (0,2) {};
\node [style=vertex, fill=blue] (6) at (1,2) {};
\node [style=vertex, fill=green] (7) at (2,2) {};
\node [style=cotreenode, fill=lightgray] (8) at (3,2) {0};
\node [style=vertex, fill=green] (9) at (2.5,1) {};
\node [style=vertex, fill=green] (10) at (3,1) {};
\node [style=vertex, fill=green] (11) at (3.5,1) {};
\foreach \i/\j in {1/2,1/3,2/4,2/6,3/7,3/8,8/9,8/10,8/11}
  \draw [style=edge] (\i) to (\j);
\end{scope}
\end{tikzpicture}
\end{subfigure}

\caption{Ejemplo de la ejecución del Algoritmo \ref{alg_cert_caso1}. Se muestran en color gris los nodos del árbol que están siendo procesados. Los colores de las hojas corresponden a los colores que asigna el algoritmo.}
\label{fig_certificador_caso1_01}
\end{figure}

\begin{figure}[!htbp]
\centering
\begin{subfigure}{\textwidth}
\centering
\begin{tikzpicture}
\begin{scope}[xshift=0cm,scale=1]
\node [style=cotreenode, fill=lightgray] (1) at (2,4) {0};
\node [style=cotreenode] (2) at (0.5,3) {1};
\node [style=vertex] (3) at (2,3) {};
\node [style=cotreenode] (4) at (3.5,3) {1};
\node [style=vertex] (5) at (0,2) {};
\node [style=vertex] (6) at (1,2) {};
\node [style=cotreenode] (7) at (2.75,2) {0};
\node [style=cotreenode] (8) at (4.25,2) {0};
\node [style=vertex] (9) at (2.25,1) {};
\node [style=vertex] (10) at (2.75,1) {};
\node [style=vertex] (11) at (3.25,1) {};
\node [style=vertex] (12) at (3.75,1) {};
\node [style=vertex] (13) at (4.25,1) {};
\node [style=vertex] (14) at (4.75,1) {};
\foreach \i/\j in {1/2,1/3,1/4,2/5,2/6,4/7,4/8,7/9,7/10,7/11,8/12,8/13,8/14}
  \draw [style=edge] (\i) to (\j);
\end{scope}
\begin{scope}[xshift=6cm,scale=1]
\node [style=cotreenode, fill=lightgray] (1) at (2,4) {0};
\node [style=cotreenode, fill=lightgray] (2) at (0.5,3) {1};
\node [style=vertex] (3) at (2,3) {};
\node [style=cotreenode] (4) at (3.5,3) {1};
\node [style=vertex] (5) at (0,2) {};
\node [style=vertex] (6) at (1,2) {};
\node [style=cotreenode] (7) at (2.75,2) {0};
\node [style=cotreenode] (8) at (4.25,2) {0};
\node [style=vertex] (9) at (2.25,1) {};
\node [style=vertex] (10) at (2.75,1) {};
\node [style=vertex] (11) at (3.25,1) {};
\node [style=vertex] (12) at (3.75,1) {};
\node [style=vertex] (13) at (4.25,1) {};
\node [style=vertex] (14) at (4.75,1) {};
\foreach \i/\j in {1/2,1/3,1/4,2/5,2/6,4/7,4/8,7/9,7/10,7/11,8/12,8/13,8/14}
  \draw [style=edge] (\i) to (\j);
\end{scope}
\end{tikzpicture}
\end{subfigure}

\begin{subfigure}{\textwidth}
\centering
\begin{tikzpicture}
\begin{scope}[xshift=0cm,scale=1]
\node [style=cotreenode, fill=lightgray] (1) at (2,4) {0};
\node [style=cotreenode] (2) at (0.5,3) {1};
\node [style=vertex, fill=lightgray] (3) at (2,3) {};
\node [style=cotreenode] (4) at (3.5,3) {1};
\node [style=vertex, fill=blue] (5) at (0,2) {};
\node [style=vertex, fill=green] (6) at (1,2) {};
\node [style=cotreenode] (7) at (2.75,2) {0};
\node [style=cotreenode] (8) at (4.25,2) {0};
\node [style=vertex] (9) at (2.25,1) {};
\node [style=vertex] (10) at (2.75,1) {};
\node [style=vertex] (11) at (3.25,1) {};
\node [style=vertex] (12) at (3.75,1) {};
\node [style=vertex] (13) at (4.25,1) {};
\node [style=vertex] (14) at (4.75,1) {};
\foreach \i/\j in {1/2,1/3,1/4,2/5,2/6,4/7,4/8,7/9,7/10,7/11,8/12,8/13,8/14}
  \draw [style=edge] (\i) to (\j);
\end{scope}
\begin{scope}[xshift=6cm,scale=1]
\node [style=cotreenode, fill=lightgray] (1) at (2,4) {0};
\node [style=cotreenode] (2) at (0.5,3) {1};
\node [style=vertex, fill=blue] (3) at (2,3) {};
\node [style=cotreenode, fill=lightgray] (4) at (3.5,3) {1};
\node [style=vertex, fill=blue] (5) at (0,2) {};
\node [style=vertex, fill=green] (6) at (1,2) {};
\node [style=cotreenode] (7) at (2.75,2) {0};
\node [style=cotreenode] (8) at (4.25,2) {0};
\node [style=vertex] (9) at (2.25,1) {};
\node [style=vertex] (10) at (2.75,1) {};
\node [style=vertex] (11) at (3.25,1) {};
\node [style=vertex] (12) at (3.75,1) {};
\node [style=vertex] (13) at (4.25,1) {};
\node [style=vertex] (14) at (4.75,1) {};
\foreach \i/\j in {1/2,1/3,1/4,2/5,2/6,4/7,4/8,7/9,7/10,7/11,8/12,8/13,8/14}
  \draw [style=edge] (\i) to (\j);
\end{scope}
\end{tikzpicture}
\end{subfigure}

\begin{subfigure}{\textwidth}
\centering
\begin{tikzpicture}
\begin{scope}[xshift=0cm,scale=1]
\node [style=cotreenode, fill=lightgray] (1) at (2,4) {0};
\node [style=cotreenode] (2) at (0.5,3) {1};
\node [style=vertex, fill=blue] (3) at (2,3) {};
\node [style=cotreenode, fill=lightgray] (4) at (3.5,3) {1};
\node [style=vertex, fill=blue] (5) at (0,2) {};
\node [style=vertex, fill=green] (6) at (1,2) {};
\node [style=cotreenode] (7) at (2.75,2) {0};
\node [style=cotreenode] (8) at (4.25,2) {0};
\node [style=vertex, fill=blue] (9) at (2.25,1) {};
\node [style=vertex, fill=blue] (10) at (2.75,1) {};
\node [style=vertex, fill=blue] (11) at (3.25,1) {};
\node [style=vertex, fill=green] (12) at (3.75,1) {};
\node [style=vertex, fill=green] (13) at (4.25,1) {};
\node [style=vertex, fill=green] (14) at (4.75,1) {};
\foreach \i/\j in {1/2,1/3,1/4,2/5,2/6,4/7,4/8,7/9,7/10,7/11,8/12,8/13,8/14}
  \draw [style=edge] (\i) to (\j);
\end{scope}
\end{tikzpicture}
\end{subfigure}

\caption{Ejemplo de la ejecución del Algoritmo \ref{alg_cert_caso1}. Se muestran en color gris los nodos del árbol que están siendo procesados. Los colores de las hojas corresponden a los colores que asigna el algoritmo.}
\label{fig_certificador_caso1_02}
\end{figure}

\begin{figure}[!htbp]
\centering
\begin{subfigure}{\textwidth}
\centering
\begin{tikzpicture}
\begin{scope}[xshift=0cm,scale=1]
\node [style=cotreenode] (1) at (1.5,4) {0};
\node [style=cotreenode] (2) at (0.5,3) {1};
\node [style=cotreenode] (4) at (2.5,3) {1};
\node [style=vertex, fill=orange] (5) at (0,2) {};
\node [style=vertex, fill=blue] (6) at (1,2) {};
\node [style=vertex, fill=orange] (7) at (1.75,2) {};
\node [style=cotreenode] (8) at (3.25,2) {0};
\node [style=vertex, fill=orange] (12) at (2.75,1) {};
\node [style=cotreenode] (14) at (3.75,1) {1};
\node [style=vertex, fill=orange] (15) at (3.5,0) {};
\node [style=vertex, fill=orange] (16) at (4,0) {};
\foreach \i/\j in {1/2,1/4,2/5,2/6,4/7,4/8,8/12,8/14,14/15,14/16}
  \draw [style=edge] (\i) to (\j);
\end{scope}
\begin{scope}[xshift=4.5cm,scale=1]
\node [style=cotreenode] (1) at (1.5,4) {0};
\node [style=cotreenode] (2) at (0.5,3) {1};
\node [style=vertex, fill=yellow] (3) at (1.5,3) {};
\node [style=cotreenode] (4) at (2.5,3) {1};
\node [style=vertex, fill=yellow] (5) at (0,2) {};
\node [style=vertex, fill=yellow] (6) at (1,2) {};
\node [style=vertex, fill=yellow] (7) at (2,2) {};
\node [style=vertex, fill=yellow] (8) at (2.5,2) {};
\node [style=vertex, fill=yellow] (9) at (3,2) {};
\foreach \i/\j in {1/2,1/3,1/4,2/5,2/6,4/7,4/8,4/9}
  \draw [style=edge] (\i) to (\j);
\end{scope}
\begin{scope}[xshift=8.5cm,scale=1]
\node [style=cotreenode] (1) at (1.5,4) {0};
\node [style=cotreenode] (2) at (0.5,3) {1};
\node [style=vertex, fill=yellow] (3) at (1.5,3) {};
\node [style=cotreenode] (4) at (2.5,3) {1};
\node [style=vertex, fill=yellow] (5) at (0,2) {};
\node [style=vertex, fill=yellow] (6) at (1,2) {};
\node [style=vertex, fill=yellow] (7) at (1.75,2) {};
\node [style=cotreenode] (8) at (3.25,2) {0};
\node [style=vertex, fill=green] (12) at (2.75,1) {};
\node [style=cotreenode] (14) at (3.75,1) {1};
\node [style=vertex, fill=yellow] (15) at (3.5,0) {};
\node [style=vertex, fill=yellow] (16) at (4,0) {};
\foreach \i/\j in {1/2,1/3,1/4,2/5,2/6,4/7,4/8,8/12,8/14,14/15,14/16}
  \draw [style=edge] (\i) to (\j);
\end{scope}
\end{tikzpicture}
\end{subfigure}


\caption{Ejemplos del resultado de la ejecución del Algoritmo \ref{alg_cert_caso1} en los que se encuentra una obstrucción.}
\label{fig_certificador_caso1_03}
\end{figure}

\subsubsection{Caso 2}

El algoritmo \ref{alg_cert_caso2} corresponde al $Caso\ 2$ de la demostración del Teorema \ref{teo_obsts_m2}. Éste recibe como entrada la raíz, $g$, de un coárbol que representa una cográfica inconexa que tiene exactamente una componente conexa no trivial y al menos una trivial. En el bloque de las líneas 6 a 28 se procesa el hijo de $g$ que no es una hoja. En las líneas 7 a 19 se procesan los nietos de $g$ y se registra si alguno tiene un hijo que no sea una hoja (es decir que dicho nieto de $g$ corresponde a una gráfica no multipartita completa) en la variable $aux\_gchild$. La cantidad de hijos diferentes de una hoja de éste se registra en $ggchildren\_no\_hojas$. Si hay más de un nieto que tenga hijos que no son hojas, se marca la obstrucción $J$ (Línea 18). Una vez procesados los nietos de $g$, se decide cómo será procesado el nieto de $g$ que no corresponde a una gráfica multipartita completa. Si tal hijo no existe, la partición ya se habrá hecho (Líneas 20 y 21), esto corresponde a una parte del caso base del $Caso\_2$ de la demostración del Teorema \ref{teo_obsts_m2}. Si dicho nieto tiene un solo hijo que no es una hoja, se procesa recursivamente (Líneas 22 y 23), esto corresponde al paso inductivo del $Caso\_2$ de la demostración ya mencionada. Y finalmente, si tiene más de un hijo que no es una hoja, se busca que todos estos hijos sean bipartitas, esta es la otra parte del caso base del $Caso\_2$ de la demostración. La Figura \ref{fig_certificador_caso2_01} muestra un ejemplo de la ejecución del algoritmo para un coárbol cuya cográfica no contiene a ninguna de las obstrucciones mínimas de $M_2$. La Figura \ref{fig_certificador_caso2_02} muestra el resultado de la ejecución para coárboles que contienen una obstrucción.




\begin{algorithm}[!htbp]
\SetInd{1pt}{10pt}
\footnotesize
\caption{M2\_Caso\_2}
\label{alg_cert_caso2}

\DontPrintSemicolon % Some LaTeX compilers require you to use \dontprintsemicolon instead
\KwIn{$g$, la raíz de un coárbol con etiqueta 0 que tiene exactamente un hijo que no es una hoja y al menos uno que es una hoja}
\KwOut{Verdadero si $G$ peretenece a la clase $M_2$. Falso en el caso contrario. Las hojas de $G$ se colorean.}


    $aux\_gchild \gets null$\;
    $ggchildren\_no\_hojas \gets 0$\;

    \For{child \textbf{\emph{en}} $g.hijos$}{
        \If{child \text{es una hoja}}{
            $child.color \gets azul$\;
        }
        \Else(\tcp*[h]{Sólo se ejecuta una vez}){
            \For{gchild \textbf{\emph{en}} $child.hijos$}{
                \If{gchild \text{es una hoja}}{
                    $gchild.color \gets verde$\;
                }
                \Else{
                    \For{ggchild \textbf{\emph{en}} $gchild.hijos$}{
                        \If{ggchild \text{es una hoja}}{
                            $ggchild.color \gets verde$\;
                        }
                        \ElseIf{$aux\_gchild = null \emph{\textbf{ o }} aux\_gchild = gchild$}{
                            $aux\_gchild \gets gchild$\;
                            $ggchildren\_no\_hojas \gets ggchildren\_no\_hojas + 1$\;
                        }
                        \Else{
                            Marcar con rojo: Un hijo de $g$ que sea una hoja, dos hojas cuyo ancestro común más profundo sea $ggchild$, una hoja en un hermano de $ggchild$, dos hojas cuyo ancestro común más profundo sea un hijo de $aux\_gchild$ que no es una hoja y una hoja en un hijo de $aux\_gchild$ diferente del anterior\;
                            $\Return\ falso$\;
                        }
                    }
                }
            }

            \If{ggchildren\_no\_hojas = 0}{
                $\Return\ verdadero$\;
            }
            \ElseIf{ggchildren\_no\_hojas = 1}{
                $\Return$ M2\_Caso\_2($aux\_gchild$)\;
            }
            \Else{
                \For{ggchild \textbf{\emph{en}} aux\_gchild}{
                    \If{\emph{Es\_bipartita_completa(}$ggchild$\emph{)} = falso}{
                        Marcar con amarillo: dos hojas cuyo ancestro común más profundo sea un hermano de $ggchild$ que no sea una hoja y un hijo de $g$ que sea una hoja\;
                        $\Return\ falso$\;
                    }
                }
            }

        }
    }

    $\Return\ verdadero$\;


\end{algorithm}

\begin{figure}[!htbp]
\centering

\begin{subfigure}{\textwidth}
\centering
\begin{tikzpicture}
\begin{scope}[xshift=0cm,scale=1]
\node [style=cotreenode, fill=lightgray] (1) at (3.5,5) {0};
\node [style=vertex] (2) at (0.5,4) {};
\node [style=vertex] (3) at (1.5,4) {};
\node [style=vertex] (4) at (2.5,4) {};
\node [style=cotreenode] (5) at (3.5,4) {1};
\node [style=vertex] (6) at (0.5,3) {};
\node [style=cotreenode] (7) at (2,3) {0};
\node [style=cotreenode] (8) at (3.5,3) {0};
\node [style=vertex] (9) at (1.75,2) {};
\node [style=vertex] (10) at (2.25,2) {};
\node [style=vertex] (11) at (2.75,2) {};
\node [style=cotreenode] (12) at (3.5,2) {1};
\node [style=cotreenode] (13) at (4.5,2) {1};
\node [style=vertex] (14) at (3.25,1) {};
\node [style=vertex] (15) at (3.75,1) {};
\node [style=vertex] (16) at (4.25,1) {};
\node [style=vertex] (17) at (4.75,1) {};
\foreach \i/\j in {1/2,1/3,1/4,1/5,5/6,5/7,5/8,7/9,7/10,8/11,8/12,8/13,12/14,12/15,13/16,13/17}
  \draw [style=edge] (\i) to (\j);
\end{scope}
\begin{scope}[xshift=5cm,scale=1]
\node [style=cotreenode, fill=lightgray] (1) at (3.5,5) {0};
\node [style=vertex, fill=blue] (2) at (0.5,4) {};
\node [style=vertex, fill=blue] (3) at (1.5,4) {};
\node [style=vertex, fill=blue] (4) at (2.5,4) {};
\node [style=cotreenode, fill=lightgray] (5) at (3.5,4) {1};
\node [style=vertex] (6) at (0.5,3) {};
\node [style=cotreenode] (7) at (2,3) {0};
\node [style=cotreenode] (8) at (3.5,3) {0};
\node [style=vertex] (9) at (1.75,2) {};
\node [style=vertex] (10) at (2.25,2) {};
\node [style=vertex] (11) at (2.75,2) {};
\node [style=cotreenode] (12) at (3.5,2) {1};
\node [style=cotreenode] (13) at (4.5,2) {1};
\node [style=vertex] (14) at (3.25,1) {};
\node [style=vertex] (15) at (3.75,1) {};
\node [style=vertex] (16) at (4.25,1) {};
\node [style=vertex] (17) at (4.75,1) {};
\foreach \i/\j in {1/2,1/3,1/4,1/5,5/6,5/7,5/8,7/9,7/10,8/11,8/12,8/13,12/14,12/15,13/16,13/17}
  \draw [style=edge] (\i) to (\j);
\end{scope}
\begin{scope}[xshift=10cm,scale=1]
\node [style=cotreenode, fill=lightgray] (1) at (3.5,5) {0};
\node [style=vertex, fill=blue] (2) at (0.5,4) {};
\node [style=vertex, fill=blue] (3) at (1.5,4) {};
\node [style=vertex, fill=blue] (4) at (2.5,4) {};
\node [style=cotreenode, fill=lightgray] (5) at (3.5,4) {1};
\node [style=vertex, fill=green] (6) at (0.5,3) {};
\node [style=cotreenode, fill=lightgray] (7) at (2,3) {0};
\node [style=cotreenode] (8) at (3.5,3) {0};
\node [style=vertex] (9) at (1.75,2) {};
\node [style=vertex] (10) at (2.25,2) {};
\node [style=vertex] (11) at (2.75,2) {};
\node [style=cotreenode] (12) at (3.5,2) {1};
\node [style=cotreenode] (13) at (4.5,2) {1};
\node [style=vertex] (14) at (3.25,1) {};
\node [style=vertex] (15) at (3.75,1) {};
\node [style=vertex] (16) at (4.25,1) {};
\node [style=vertex] (17) at (4.75,1) {};
\foreach \i/\j in {1/2,1/3,1/4,1/5,5/6,5/7,5/8,7/9,7/10,8/11,8/12,8/13,12/14,12/15,13/16,13/17}
  \draw [style=edge] (\i) to (\j);
\end{scope}
\end{tikzpicture}
\end{subfigure}
\begin{subfigure}{\textwidth}
\centering
\begin{tikzpicture}
\begin{scope}[xshift=0cm,scale=1]
\node [style=cotreenode, fill=lightgray] (1) at (3.5,5) {0};
\node [style=vertex, fill=blue] (2) at (0.5,4) {};
\node [style=vertex, fill=blue] (3) at (1.5,4) {};
\node [style=vertex, fill=blue] (4) at (2.5,4) {};
\node [style=cotreenode, fill=lightgray] (5) at (3.5,4) {1};
\node [style=vertex, fill=green] (6) at (0.5,3) {};
\node [style=cotreenode] (7) at (2,3) {0};
\node [style=cotreenode, fill=lightgray] (8) at (3.5,3) {0};
\node [style=vertex, fill=green] (9) at (1.75,2) {};
\node [style=vertex, fill=green] (10) at (2.25,2) {};
\node [style=vertex] (11) at (2.75,2) {};
\node [style=cotreenode] (12) at (3.5,2) {1};
\node [style=cotreenode] (13) at (4.5,2) {1};
\node [style=vertex] (14) at (3.25,1) {};
\node [style=vertex] (15) at (3.75,1) {};
\node [style=vertex] (16) at (4.25,1) {};
\node [style=vertex] (17) at (4.75,1) {};
\foreach \i/\j in {1/2,1/3,1/4,1/5,5/6,5/7,5/8,7/9,7/10,8/11,8/12,8/13,12/14,12/15,13/16,13/17}
  \draw [style=edge] (\i) to (\j);
\end{scope}
\begin{scope}[xshift=5cm,scale=1]
\node [style=cotreenode, fill=lightgray] (1) at (3.5,5) {0};
\node [style=vertex, fill=blue] (2) at (0.5,4) {};
\node [style=vertex, fill=blue] (3) at (1.5,4) {};
\node [style=vertex, fill=blue] (4) at (2.5,4) {};
\node [style=cotreenode, fill=lightgray] (5) at (3.5,4) {1};
\node [style=vertex, fill=green] (6) at (0.5,3) {};
\node [style=cotreenode] (7) at (2,3) {0};
\node [style=cotreenode, fill=lightgray] (8) at (3.5,3) {0};
\node [style=vertex, fill=green] (9) at (1.75,2) {};
\node [style=vertex, fill=green] (10) at (2.25,2) {};
\node [style=vertex, fill=green] (11) at (2.75,2) {};
\node [style=cotreenode, fill=lightgray] (12) at (3.5,2) {1};
\node [style=cotreenode] (13) at (4.5,2) {1};
\node [style=vertex] (14) at (3.25,1) {};
\node [style=vertex] (15) at (3.75,1) {};
\node [style=vertex] (16) at (4.25,1) {};
\node [style=vertex] (17) at (4.75,1) {};
\foreach \i/\j in {1/2,1/3,1/4,1/5,5/6,5/7,5/8,7/9,7/10,8/11,8/12,8/13,12/14,12/15,13/16,13/17}
  \draw [style=edge] (\i) to (\j);
\end{scope}
\begin{scope}[xshift=10cm,scale=1]
\node [style=cotreenode, fill=lightgray] (1) at (3.5,5) {0};
\node [style=vertex, fill=blue] (2) at (0.5,4) {};
\node [style=vertex, fill=blue] (3) at (1.5,4) {};
\node [style=vertex, fill=blue] (4) at (2.5,4) {};
\node [style=cotreenode, fill=lightgray] (5) at (3.5,4) {1};
\node [style=vertex, fill=green] (6) at (0.5,3) {};
\node [style=cotreenode] (7) at (2,3) {0};
\node [style=cotreenode, fill=lightgray] (8) at (3.5,3) {0};
\node [style=vertex, fill=green] (9) at (1.75,2) {};
\node [style=vertex, fill=green] (10) at (2.25,2) {};
\node [style=vertex, fill=green] (11) at (2.75,2) {};
\node [style=cotreenode] (12) at (3.5,2) {1};
\node [style=cotreenode, fill=lightgray] (13) at (4.5,2) {1};
\node [style=vertex] (14) at (3.25,1) {};
\node [style=vertex] (15) at (3.75,1) {};
\node [style=vertex] (16) at (4.25,1) {};
\node [style=vertex] (17) at (4.75,1) {};
\foreach \i/\j in {1/2,1/3,1/4,1/5,5/6,5/7,5/8,7/9,7/10,8/11,8/12,8/13,12/14,12/15,13/16,13/17}
  \draw [style=edge] (\i) to (\j);
\end{scope}
\end{tikzpicture}
\end{subfigure}

\begin{subfigure}{\textwidth}
\centering
\begin{tikzpicture}
\begin{scope}[xshift=0cm,scale=1]
\node [style=cotreenode, fill=lightgray] (1) at (3.5,5) {0};
\node [style=vertex, fill=blue] (2) at (0.5,4) {};
\node [style=vertex, fill=blue] (3) at (1.5,4) {};
\node [style=vertex, fill=blue] (4) at (2.5,4) {};
\node [style=cotreenode, fill=lightgray] (5) at (3.5,4) {1};
\node [style=vertex, fill=green] (6) at (0.5,3) {};
\node [style=cotreenode] (7) at (2,3) {0};
\node [style=cotreenode, fill=lightgray] (8) at (3.5,3) {0};
\node [style=vertex, fill=green] (9) at (1.75,2) {};
\node [style=vertex, fill=green] (10) at (2.25,2) {};
\node [style=vertex, fill=green] (11) at (2.75,2) {};
\node [style=cotreenode, fill=lightgray] (12) at (3.5,2) {1};
\node [style=cotreenode] (13) at (4.5,2) {1};
\node [style=vertex] (14) at (3.25,1) {};
\node [style=vertex] (15) at (3.75,1) {};
\node [style=vertex] (16) at (4.25,1) {};
\node [style=vertex] (17) at (4.75,1) {};
\foreach \i/\j in {1/2,1/3,1/4,1/5,5/6,5/7,5/8,7/9,7/10,8/11,8/12,8/13,12/14,12/15,13/16,13/17}
  \draw [style=edge] (\i) to (\j);
\end{scope}
\begin{scope}[xshift=5cm,scale=1]
\node [style=cotreenode, fill=lightgray] (1) at (3.5,5) {0};
\node [style=vertex, fill=blue] (2) at (0.5,4) {};
\node [style=vertex, fill=blue] (3) at (1.5,4) {};
\node [style=vertex, fill=blue] (4) at (2.5,4) {};
\node [style=cotreenode, fill=lightgray] (5) at (3.5,4) {1};
\node [style=vertex, fill=green] (6) at (0.5,3) {};
\node [style=cotreenode] (7) at (2,3) {0};
\node [style=cotreenode, fill=lightgray] (8) at (3.5,3) {0};
\node [style=vertex, fill=green] (9) at (1.75,2) {};
\node [style=vertex, fill=green] (10) at (2.25,2) {};
\node [style=vertex, fill=green] (11) at (2.75,2) {};
\node [style=cotreenode] (12) at (3.5,2) {1};
\node [style=cotreenode, fill=lightgray] (13) at (4.5,2) {1};
\node [style=vertex, fill=blue] (14) at (3.25,1) {};
\node [style=vertex, fill=green] (15) at (3.75,1) {};
\node [style=vertex] (16) at (4.25,1) {};
\node [style=vertex] (17) at (4.75,1) {};
\foreach \i/\j in {1/2,1/3,1/4,1/5,5/6,5/7,5/8,7/9,7/10,8/11,8/12,8/13,12/14,12/15,13/16,13/17}
  \draw [style=edge] (\i) to (\j);
\end{scope}
\begin{scope}[xshift=10cm,scale=1]
\node [style=cotreenode, fill=lightgray] (1) at (3.5,5) {0};
\node [style=vertex, fill=blue] (2) at (0.5,4) {};
\node [style=vertex, fill=blue] (3) at (1.5,4) {};
\node [style=vertex, fill=blue] (4) at (2.5,4) {};
\node [style=cotreenode, fill=lightgray] (5) at (3.5,4) {1};
\node [style=vertex, fill=green] (6) at (0.5,3) {};
\node [style=cotreenode] (7) at (2,3) {0};
\node [style=cotreenode, fill=lightgray] (8) at (3.5,3) {0};
\node [style=vertex, fill=green] (9) at (1.75,2) {};
\node [style=vertex, fill=green] (10) at (2.25,2) {};
\node [style=vertex, fill=green] (11) at (2.75,2) {};
\node [style=cotreenode] (12) at (3.5,2) {1};
\node [style=cotreenode, fill=lightgray] (13) at (4.5,2) {1};
\node [style=vertex, fill=blue] (14) at (3.25,1) {};
\node [style=vertex, fill=green] (15) at (3.75,1) {};
\node [style=vertex, fill=blue] (16) at (4.25,1) {};
\node [style=vertex, fill=green] (17) at (4.75,1) {};
\foreach \i/\j in {1/2,1/3,1/4,1/5,5/6,5/7,5/8,7/9,7/10,8/11,8/12,8/13,12/14,12/15,13/16,13/17}
  \draw [style=edge] (\i) to (\j);
\end{scope}
\end{tikzpicture}
\end{subfigure}
\caption{Ejemplo de la ejecución del Algoritmo \ref{alg_cert_caso2}. Se muestran en color gris los nodos del árbol que están siendo procesados. El procesamiento de las hojas hermanas se realiza en una sola imagen. Los colores de las hojas corresponden a los colores que asigna el algoritmo.}
\label{fig_certificador_caso2_01}
\end{figure}


\begin{figure}[!htbp]
\centering

\begin{subfigure}{\textwidth}
\centering
\begin{tikzpicture}
\begin{scope}[xshift=0cm,scale=1]
\node [style=cotreenode] (1) at (3.5,5) {0};
\node [style=vertex, fill=yellow] (2) at (0.5,4) {};
\node [style=vertex, fill=blue] (3) at (1.5,4) {};
\node [style=vertex, fill=blue] (4) at (2.5,4) {};
\node [style=cotreenode] (5) at (3.5,4) {1};
\node [style=vertex, fill=green] (6) at (0.5,3) {};
\node [style=cotreenode] (7) at (2,3) {0};
\node [style=cotreenode] (8) at (3.5,3) {0};
\node [style=vertex, fill=green] (9) at (1.75,2) {};
\node [style=vertex, fill=green] (10) at (2.25,2) {};
\node [style=vertex, fill=green] (11) at (2.75,2) {};
\node [style=cotreenode] (12) at (3.5,2) {1};
\node [style=cotreenode] (13) at (4.5,2) {1};
\node [style=vertex, fill=yellow] (14) at (3.25,1) {};
\node [style=vertex, fill=yellow] (15) at (3.75,1) {};
\node [style=vertex, fill=yellow] (16) at (4.25,1) {};
\node [style=vertex, fill=yellow] (17) at (4.75,1) {};
\node [style=vertex, fill=yellow] (18) at (4.5,1) {};
\foreach \i/\j in {1/2,1/3,1/4,1/5,5/6,5/7,5/8,7/9,7/10,8/11,8/12,8/13,12/14,12/15,13/16,13/17,13/18}
  \draw [style=edge] (\i) to (\j);
\end{scope}
\begin{scope}[xshift=5cm,scale=1]
\node [style=cotreenode] (1) at (3.5,5) {0};
\node [style=vertex, fill=red] (2) at (0.5,4) {};
\node [style=vertex, fill=blue] (3) at (1.5,4) {};
\node [style=vertex, fill=blue] (4) at (2.5,4) {};
\node [style=cotreenode] (5) at (3.5,4) {1};
\node [style=vertex, fill=green] (6) at (0.5,3) {};
\node [style=vertex, fill=green] (7) at (1.5,3) {};
\node [style=cotreenode] (8) at (2.5,3) {0};
\node [style=cotreenode] (9) at (4.5,3) {0};
\node [style=cotreenode] (10) at (2,2) {1};
\node [style=vertex, fill=red] (11) at (3,2) {};
\node [style=cotreenode] (12) at (4,2) {1};
\node [style=vertex, fill=red] (13) at (5,2) {};
\node [style=vertex, fill=red] (14) at (1.5,1) {};
\node [style=vertex, fill=red] (15) at (2.5,1) {};
\node [style=vertex, fill=red] (16) at (3.5,1) {};
\node [style=vertex, fill=red] (17) at (4.5,1) {};
\foreach \i/\j in {1/2,1/3,1/4,1/5,5/6,5/7,5/8,5/9,8/10,8/11,9/12,9/13,10/14,10/15,12/16,12/17}
  \draw [style=edge] (\i) to (\j);
\end{scope}
\end{tikzpicture}
\end{subfigure}
\caption{Ejemplo del resultado de la ejecución del Algoritmo \ref{alg_cert_caso2} para coárboles que incluyen una obstrucción.}
\label{fig_certificador_caso2_02}
\end{figure}


\subsubsection{Algoritmo certificador}

Finalmente, el Algoritmo \ref{alg_cert_m2} utiliza los algoritmos anteriores para colorear las hojas del coárbol recibido como entrada, $g$. En el caso de que la gráfica sea conexa (líneas 4 a 8), simplemente se llama el algoritmo para cada una de los hijos de $g$. Esto no significa que sea un algoritmo recursivo, dado que, para las gráficas inconexas y las hojas, el algoritmo no vuelve a ser llamado. En el caso de que la gráfica sea inconexa, se ejecuta el bloque de las líneas 10 a 21. En las líneas 10 a 15 se cuenta el número de componentes conexas de la gráfica representada (es decir que se cuentan los hijos de $g$ que no son hojas). Y por último se toma la decisión de qué caso debe llamarse.


\begin{algorithm}[!htbp]
\caption{M2\_Certificador}
\label{alg_cert_m2}

\DontPrintSemicolon % Some LaTeX compilers require you to use \dontprintsemicolon instead
\KwIn{$g$, la raíz de un coárbol, $G$}
\KwOut{Verdadero si la gráfica representada por $G$ pertenece a la clase $M_2$. Falso en el caso contrario. Las hojas de $G$ se colorean.}

    \If{$g$ \text{es una hoja}}{
        $g.color \gets azul$\;
        $\Return\ verdadero$\;
    }
    \ElseIf{$g.etiqueta = 1$}{
        \For{child \textbf{\emph{en}} $g.hijos$}{
            \If{\emph{M2\_Certificador(}child\emph{)} = falso}{
                $\Return\ falso$\;
            }
            $\Return\ verdadero$\;
        }
    }
    \Else{
        $componentes\_no\_triviales \gets 0$\;
        \For{child \textbf{\emph{en}} $g.hijos$}{
            \If{$child$ \text{es una hoja}}{
                $child.color \gets azul$\;
            }
            \Else{
                $componentes\_no\_triviales \gets componentes\_no\_triviales + 1$\;
            }
        }
        \If{componentes\_no\_triviales = 0}{
            $\Return\ verdadero$\;
        }
        \ElseIf{componentes\_no\_triviales = 1}{
            $\Return$ M2\_Caso\_2($g$)\;
        }
        \Else{
            $\Return$ M2\_Caso\_1($g$)\;
        }
    }


$\Return\ verdadero$\;

\end{algorithm}

\section{Las clases $(\alpha, \beta)$-$M_2$}

    Al igual que con las cográficas polares, podemos obtener subclases de $M_2$ al limitar el tamaño de sus partes. En la presente sección estudiamos un conjunto de subclases de $M_2$ a las que llamamos clases $(\alpha, \beta)$-$M_2$. El estudio de estas clases consolida un ejemplo de cómo los conjuntos de obstrucciones mínimas de un conjunto de clases $C_1,C_2, \dots$ tales que $C_i \subset C_{i+1}$ para todo entero $i \geq 1$ se relacionan de manera que podemos encontrar fórmulas y algoritmos para generar dichos conjuntos de obstrucciones mínimas. El resultado principal de esta sección es un algoritmo de fuerza bruta para encontrar algunas obstrucciones mínimas de cualquier clase $(\alpha, \beta)$-$M_2$. Éste es utilizado para generar los conjuntos de obstrucciones mínimas de varias clases. 
    
    \subsection{Definición y clases conocidas}
        Sean $\alpha$ y $\beta$ enteros tales que $0 < \alpha \le \beta$, una gr\'afica $G$ est\'a en la \emph{\textbf{clase $(\alpha, \beta)-M_2$}}  si y sólo si su conjunto de vértices acepta una partición $(A,B)$ tal que $G[A]$ es una gráfica multipartita completa formada por a lo más $\alpha$ conjuntos independientes y $G[B]$ es una gráfica multipartita completa formada por a lo más $\beta$ conjuntos independientes. Decimos que $(A,B)$ es una $M_2$-partición de $G$ de tamaño $(\alpha, \beta)$.

Notemos que la clase $M_2$ es la clase $(\infty, \infty)$-$M_2$ y que la clase $(1,1)$-$M_2$ es la clase de todas las cográficas bipartitas. En la demostración del Teorema \ref{teo_obsts_m2} % No es común referir resultados futuros.   Quizá bastaría mencionar el contexto en el que esta observación será útil más adelante. Es un Teorema anterior. 
 hablamos de las cográficas que aceptan una partición en un conjunto independiente y una gráfica multipartita completa. Es decir, las gráficas de la clase $(1,\infty)$-$M_2$. A partir de esta demostración, podemos encontrar las obstrucciones mínimas de la clase. Éstas nos serán de ayuda para entender el comportamiento de las clases $(1,\beta)$-$M_2$.

\begin{lemma}

Sea $G$ una cográfica, $G\in(1,\infty)$-$M_2$ si y sólo si $G$ es libre de las gráficas de la Figura \ref{obsts_1infM2}.

\end{lemma}

\begin{figure}[ht!]
\begin{center}
\begin{tikzpicture}

\begin{scope}[xshift=0cm,scale=1]

\node [style=vertex] (1) at (0,0) {};
\node [style=vertex] (2) at (1,0) {};
\node [style=vertex] (3) at (0,0.5) {};
\node [style=vertex] (4) at (1,0.5) {};
\node [style=vertex] (5) at (0.5,1.25) {};
\foreach \i/\j in {1/2,3/4,3/5,4/5}
  \draw [style=edge] (\i) to (\j);
\node [below of=1,xshift=.5cm]
{\parbox{0.3\linewidth}{\subcaption*{$H'$}}};

\end{scope}

\begin{scope}[xshift=3cm,scale=1]

\node [style=vertex] (1) at (0,0) {};
\node [style=vertex] (2) at (0.5,0.5) {};
\node [style=vertex] (3) at (1.5,0.5) {};
\node [style=vertex] (4) at (0.5,1.5) {};
\node [style=vertex] (5) at (1.5,1.5) {};
\node [style=vertex] (6) at (0,2) {};

\foreach \i/\j in {1/2,1/3,1/6,2/3,2/4,2/5,3/4,3/5,4/5,4/6,5/6}
  \draw [style=edge] (\i) to (\j);
\node [below of=1,xshift=0.75cm] {\parbox{0.3\linewidth}{\subcaption*{$J'$}}};

\end{scope}
\end{tikzpicture}
\end{center}
\setlength{\abovecaptionskip}{-15pt}
\caption{Obstrucciones mínimas para la clase $(1,\infty)$-$M_2$.}
\label{obsts_1infM2}
\end{figure}

\begin{proof}

Notemos que $H'=K_2 + K_3$ y $J'=\overline{P_3} \oplus \overline{P_3}$.

Supongamos primero que $G \in (1,\infty)$-$M_2$. Veamos que $H'\notin (1,\infty)$-$M_2$. Procedamos por contradicción. Supongamos que $H' \in (1,\infty)$-$M_2$. Luego, $H$ (Figura \ref{obsts_M2}) también está en $(1,\infty)$-$M_2$ y por lo tanto $H\in M_2$, lo que es una contradicción. Así,  $H'\notin (1,\infty)$-$M_2$. Análogamente para $J'$. Como ni $H'$ ni $J'$ están en $(1,\infty)$-$M_2$ y toda subgráfica inducida de $G$ sí lo está, $G$ es libre de $H'$ y $J'$.

Recíprocamente, si $G$ es libre de $H'$ y $J'$, mostremos que $G\in (1,\infty)$-$M_2$. Sea $r$ la raíz del coárbol de $G$. Consideremos los siguientes casos que son exhaustivos.

Supongamos primero que $G$ es conexa.  Si $G$ es un vértice aislado, es claro que $G \in (1,\infty)$-$M_2$. En el caso contrario, como $G$ es libre de $J'$, todos los hijos de $r$ representan gráficas multipartitas completas. Luego, $G$ es la unión completa de varias gráficas multipartitas completas, por lo que $G$ es multipartita completa. Así, $G \in (1,\infty)$-$M_2$.
% Todos quizá excepto uno, ¿no?

Si $G$ es una gr\'afica vac\'ia, entoces es claro que $G \in (1,\infty)$-$M_2$.

Como tercer caso, consideremos que $G$ es inconexa y tiene exactamente una componente no trivial y al menos una trivial. Entonces, del Caso 2 de la segunda parte de la demostración del Teorema \ref{teo_obsts_m2}, se sigue que $G \in (1,\infty)$-$M_2$.
% ¿La referencia es correcta?   Creo que sí, pero en los warnings,
% que son más importantes de lo que parece, tienes etiquetas
% definidas en múltiples ocasiones.  Revísalas para que todo tenga
% sentido.

Finalmente, supongamos que $G$ tiene al menos dos componentes no triviales. Como $G$ es libre de $H'$, entonces cada componente de $G$ es libre de $K_3$. Es decir que cada componente de $G$ es bipartita. Luego, $G$ es bipartita. Así, $G \in (1,1)$-$M_2$, por lo que $G \in (1,\infty)$-$M_2$.

\end{proof}

    \subsection{Conjunto de parejas mínimas}
        Sean $\alpha$ y $\beta$ enteros tales que $0 < \alpha \le \beta$, una gr\'afica $G$ est\'a en la \emph{\textbf{clase $(\alpha, \beta)-M_2$}}  si y sólo si su conjunto de vértices acepta una partición $(A,B)$ tal que $G[A]$ es una gráfica multipartita completa formada por a lo más $\alpha$ conjuntos independientes y $G[B]$ es una gráfica multipartita completa formada por a lo más $\beta$ conjuntos independientes. Decimos que $(A,B)$ es una $M_2$-partición de $G$ de tamaño $(\alpha, \beta)$.

Notemos que la clase $M_2$ es la clase $(\infty, \infty)$-$M_2$ y que la clase $(1,1)$-$M_2$ es la clase de todas las cográficas bipartitas. En la demostración del Teorema \ref{teo_obsts_m2} % No es común referir resultados futuros.   Quizá bastaría mencionar el contexto en el que esta observación será útil más adelante. Es un Teorema anterior. 
 hablamos de las cográficas que aceptan una partición en un conjunto independiente y una gráfica multipartita completa. Es decir, las gráficas de la clase $(1,\infty)$-$M_2$. A partir de esta demostración, podemos encontrar las obstrucciones mínimas de la clase. Éstas nos serán de ayuda para entender el comportamiento de las clases $(1,\beta)$-$M_2$.

\begin{lemma}

Sea $G$ una cográfica, $G\in(1,\infty)$-$M_2$ si y sólo si $G$ es libre de las gráficas de la Figura \ref{obsts_1infM2}.

\end{lemma}

\begin{figure}[ht!]
\begin{center}
\begin{tikzpicture}

\begin{scope}[xshift=0cm,scale=1]

\node [style=vertex] (1) at (0,0) {};
\node [style=vertex] (2) at (1,0) {};
\node [style=vertex] (3) at (0,0.5) {};
\node [style=vertex] (4) at (1,0.5) {};
\node [style=vertex] (5) at (0.5,1.25) {};
\foreach \i/\j in {1/2,3/4,3/5,4/5}
  \draw [style=edge] (\i) to (\j);
\node [below of=1,xshift=.5cm]
{\parbox{0.3\linewidth}{\subcaption*{$H'$}}};

\end{scope}

\begin{scope}[xshift=3cm,scale=1]

\node [style=vertex] (1) at (0,0) {};
\node [style=vertex] (2) at (0.5,0.5) {};
\node [style=vertex] (3) at (1.5,0.5) {};
\node [style=vertex] (4) at (0.5,1.5) {};
\node [style=vertex] (5) at (1.5,1.5) {};
\node [style=vertex] (6) at (0,2) {};

\foreach \i/\j in {1/2,1/3,1/6,2/3,2/4,2/5,3/4,3/5,4/5,4/6,5/6}
  \draw [style=edge] (\i) to (\j);
\node [below of=1,xshift=0.75cm] {\parbox{0.3\linewidth}{\subcaption*{$J'$}}};

\end{scope}
\end{tikzpicture}
\end{center}
\setlength{\abovecaptionskip}{-15pt}
\caption{Obstrucciones mínimas para la clase $(1,\infty)$-$M_2$.}
\label{obsts_1infM2}
\end{figure}

\begin{proof}

Notemos que $H'=K_2 + K_3$ y $J'=\overline{P_3} \oplus \overline{P_3}$.

Supongamos primero que $G \in (1,\infty)$-$M_2$. Veamos que $H'\notin (1,\infty)$-$M_2$. Procedamos por contradicción. Supongamos que $H' \in (1,\infty)$-$M_2$. Luego, $H$ (Figura \ref{obsts_M2}) también está en $(1,\infty)$-$M_2$ y por lo tanto $H\in M_2$, lo que es una contradicción. Así,  $H'\notin (1,\infty)$-$M_2$. Análogamente para $J'$. Como ni $H'$ ni $J'$ están en $(1,\infty)$-$M_2$ y toda subgráfica inducida de $G$ sí lo está, $G$ es libre de $H'$ y $J'$.

Recíprocamente, si $G$ es libre de $H'$ y $J'$, mostremos que $G\in (1,\infty)$-$M_2$. Sea $r$ la raíz del coárbol de $G$. Consideremos los siguientes casos que son exhaustivos.

Supongamos primero que $G$ es conexa.  Si $G$ es un vértice aislado, es claro que $G \in (1,\infty)$-$M_2$. En el caso contrario, como $G$ es libre de $J'$, todos los hijos de $r$ representan gráficas multipartitas completas. Luego, $G$ es la unión completa de varias gráficas multipartitas completas, por lo que $G$ es multipartita completa. Así, $G \in (1,\infty)$-$M_2$.
% Todos quizá excepto uno, ¿no?

Si $G$ es una gr\'afica vac\'ia, entoces es claro que $G \in (1,\infty)$-$M_2$.

Como tercer caso, consideremos que $G$ es inconexa y tiene exactamente una componente no trivial y al menos una trivial. Entonces, del Caso 2 de la segunda parte de la demostración del Teorema \ref{teo_obsts_m2}, se sigue que $G \in (1,\infty)$-$M_2$.
% ¿La referencia es correcta?   Creo que sí, pero en los warnings,
% que son más importantes de lo que parece, tienes etiquetas
% definidas en múltiples ocasiones.  Revísalas para que todo tenga
% sentido.

Finalmente, supongamos que $G$ tiene al menos dos componentes no triviales. Como $G$ es libre de $H'$, entonces cada componente de $G$ es libre de $K_3$. Es decir que cada componente de $G$ es bipartita. Luego, $G$ es bipartita. Así, $G \in (1,1)$-$M_2$, por lo que $G \in (1,\infty)$-$M_2$.

\end{proof}


    \subsection{Reconocimiento de las clases $(\alpha, \beta)-M_2$}

    \subsection{Algoritmo para generar obstrucciones mínimas}

\section{Particiones en más de dos partes}
    \subsection{Las Clases $M_i$}

    \subsection{Obstrucciones mínimas de la clase $M_3$}
        \begin{theorem} \label{teo_obsts_m2}

    Para una cográfica $G$, las siguientes afirmaciones son equivalentes.
    \begin{enumerate}[(a)]
        \item $G \in M_3$.
        \item $G$ no contiene a ninguna de las gráficas de las Figuras \ref{obsts_O_M3} como subgráfica inducida.
    \end{enumerate}

\end{theorem}

\begin{figure}[ht!]
\begin{subfigure}{\textwidth}
\begin{center}
\begin{tikzpicture}
\begin{scope}[xshift=0cm,scale=1]
%K4
\node [style=vertex] (1) at (0,1) {};
\node [style=vertex] (2) at (1,1) {};
\node [style=vertex] (3) at (0.5,1.3) {};
\node [style=vertex] (4) at (0.5,1.75) {};
%K3
\node [style=vertex] (5) at (0,2.25) {};
\node [style=vertex] (6) at (1,2.25) {};
\node [style=vertex] (7) at (0.5,3) {};
%K2
\node [style=vertex] (8) at (0,3.5) {};
\node [style=vertex] (9) at (1,3.5) {};
%K1
\node [style=vertex] (10) at (0.5,4) {};

\foreach \i/\j in {1/2,1/3,1/4,2/3,2/4,3/4,5/6,5/7,6/7,8/9} \draw [style=edge] (\i) to (\j);
\node at (0.5,0) {\parbox{0.3\linewidth}{\subcaption*{$O_{3,1}$}}};
\end{scope}

\begin{scope}[xshift=2.5cm,scale=1]
%K4
\node [style=vertex] (1) at (0,1) {};
\node [style=vertex] (2) at (1,1) {};
\node [style=vertex] (3) at (0.5,1.3) {};
\node [style=vertex] (4) at (0.5,1.75) {};
%K3
\node [style=vertex] (5) at (0,2.25) {};
\node [style=vertex] (6) at (1,2.25) {};
\node [style=vertex] (7) at (0.5,3) {};
%K2
\node [style=vertex] (8) at (0,4) {};
\node [style=vertex] (9) at (1,4) {};
%K1
\node [style=vertex] (10) at (0.5,3.5) {};

\foreach \i/\j in {1/2,1/3,1/4,2/3,2/4,3/4,5/6,5/7,6/7,8/9} \draw [style=edge] (\i) to (\j);
\foreach \i/\j in {7/10} \draw [style=edge] (\i) to (\j);
\node at (0.5,0) {\parbox{0.3\linewidth}{\subcaption*{$O_{3,2}$}}};
\end{scope}

\begin{scope}[xshift=5cm,scale=1]
%K4
\node [style=vertex] (1) at (0,1) {};
\node [style=vertex] (2) at (1,1) {};
\node [style=vertex] (3) at (0.5,1.3) {};
\node [style=vertex] (4) at (0.5,1.75) {};
%K3
\node [style=vertex] (5) at (0,2.75) {};
\node [style=vertex] (6) at (1,2.75) {};
\node [style=vertex] (7) at (0.5,3.5) {};
%K2
\node [style=vertex] (8) at (0,4) {};
\node [style=vertex] (9) at (1,4) {};
%K1
\node [style=vertex] (10) at (0.5,2.25) {};

\foreach \i/\j in {1/2,1/3,1/4,2/3,2/4,3/4,5/6,5/7,6/7,8/9} \draw [style=edge] (\i) to (\j);
\foreach \i/\j in {4/10} \draw [style=edge] (\i) to (\j);
\node at (0.5,0) {\parbox{0.3\linewidth}{\subcaption*{$O_{3,3}$}}};
\end{scope}

\begin{scope}[xshift=7.5cm,scale=1]
%K4
\node [style=vertex] (1) at (0,1) {};
\node [style=vertex] (2) at (1,1) {};
\node [style=vertex] (3) at (0.5,1.3) {};
\node [style=vertex] (4) at (0.5,1.75) {};
%K3
\node [style=vertex] (5) at (0,2.75) {};
\node [style=vertex] (6) at (1,2.75) {};
\node [style=vertex] (7) at (0.5,3.5) {};
%K2
\node [style=vertex] (8) at (0,4) {};
\node [style=vertex] (9) at (1,4) {};
%K1
\node [style=vertex] (10) at (0.5,2.25) {};

\foreach \i/\j in {1/2,1/3,1/4,2/3,2/4,3/4,5/6,5/7,6/7,8/9} \draw [style=edge] (\i) to (\j);
\foreach \i/\j in {4/10,1/10} \draw [style=edge] (\i) to (\j);
\node at (0.5,0) {\parbox{0.3\linewidth}{\subcaption*{$O_{3,4}$}}};
\end{scope}

\begin{scope}[xshift=10cm,scale=1]
%K4
\node [style=vertex] (1) at (0,1) {};
\node [style=vertex] (2) at (1,1) {};
\node [style=vertex] (3) at (0.5,1.3) {};
\node [style=vertex] (4) at (0.5,1.75) {};
%K3
\node [style=vertex] (5) at (0,2.75) {};
\node [style=vertex] (6) at (1,2.75) {};
\node [style=vertex] (7) at (0.5,3.5) {};
%K2
\node [style=vertex] (8) at (0,2.25) {};
\node [style=vertex] (9) at (1,2.25) {};
%K1
\node [style=vertex] (10) at (0.5,4) {};

\foreach \i/\j in {1/2,1/3,1/4,2/3,2/4,3/4,5/6,5/7,6/7,8/9} \draw [style=edge] (\i) to (\j);
\foreach \i/\j in {4/8,4/9} \draw [style=edge] (\i) to (\j);
\node at (0.5,0) {\parbox{0.3\linewidth}{\subcaption*{$O_{3,5}$}}};
\end{scope}
\end{tikzpicture}
\end{center}
\end{subfigure}

\begin{subfigure}{\textwidth}
\begin{center}
\begin{tikzpicture}

\begin{scope}[xshift=0cm,scale=1]
%K4
\node [style=vertex] (1) at (0,1) {};
\node [style=vertex] (2) at (1,1) {};
\node [style=vertex] (3) at (0.5,1.3) {};
\node [style=vertex] (4) at (0.5,1.75) {};
%K3
\node [style=vertex] (5) at (0,3.25) {};
\node [style=vertex] (6) at (1,3.25) {};
\node [style=vertex] (7) at (0.5,4) {};
%K2
\node [style=vertex] (8) at (0.5,2.5) {};
\node [style=vertex] (9) at (1,2.5) {};
%K1
\node [style=vertex] (10) at (0,2.5) {};

\foreach \i/\j in {1/2,1/3,1/4,2/3,2/4,3/4,5/6,5/7,6/7,8/9} \draw [style=edge] (\i) to (\j);
\foreach \i/\j in {4/8,4/9,4/10} \draw [style=edge] (\i) to (\j);
\node at (0.5,0) {\parbox{0.3\linewidth}{\subcaption*{$O_{3,6}$}}};
\end{scope}

\begin{scope}[xshift=2.5cm,scale=1]
%K4
\node [style=vertex] (1) at (0,1) {};
\node [style=vertex] (2) at (1,1) {};
\node [style=vertex] (3) at (0.5,1.3) {};
\node [style=vertex] (4) at (0.5,1.75) {};
%K3
\node [style=vertex] (5) at (0,3.25) {};
\node [style=vertex] (6) at (1,3.25) {};
\node [style=vertex] (7) at (0.5,4) {};
%K2
\node [style=vertex] (8) at (0.5,2.5) {};
\node [style=vertex] (9) at (1,2.5) {};
%K1
\node [style=vertex] (10) at (0,2.5) {};

\foreach \i/\j in {1/2,1/3,1/4,2/3,2/4,3/4,5/6,5/7,6/7,8/9} \draw [style=edge] (\i) to (\j);
\foreach \i/\j in {4/8,4/9,4/10,1/10} \draw [style=edge] (\i) to (\j);
\node at (0.5,0) {\parbox{0.3\linewidth}{\subcaption*{$O_{3,7}$}}};
\end{scope}

\begin{scope}[xshift=5cm,scale=1]
%K4
\node [style=vertex] (1) at (0,1) {};
\node [style=vertex] (2) at (1,1) {};
\node [style=vertex] (3) at (0.5,1.3) {};
\node [style=vertex] (4) at (0.5,1.75) {};
%K3
\node [style=vertex] (5) at (0,2.75) {};
\node [style=vertex] (6) at (1,2.75) {};
\node [style=vertex] (7) at (0.5,3.5) {};
%K2
\node [style=vertex] (8) at (0,2.25) {};
\node [style=vertex] (9) at (1,2.25) {};
%K1
\node [style=vertex] (10) at (0.5,4) {};

\foreach \i/\j in {1/2,1/3,1/4,2/3,2/4,3/4,5/6,5/7,6/7,8/9} \draw [style=edge] (\i) to (\j);
\foreach \i/\j in {4/8,4/9,7/10} \draw [style=edge] (\i) to (\j);
\node at (0.5,0) {\parbox{0.3\linewidth}{\subcaption*{$O_{3,5}$}}};
\end{scope}

\end{tikzpicture}
\end{center}
\end{subfigure}

\setlength{\abovecaptionskip}{-15pt}
\caption{Obstrucciones mínimas para la clase $M_2$.}
\label{obsts_O_M3}
\end{figure}

\begin{proof}

\end{proof}


    \subsection{Familia $O$ de obstrucciones}

    \subsection{Familia $P$ de obstrucciones}
